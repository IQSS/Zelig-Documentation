\section{Introduction}
\label{section:param-intro}

Several general features - sampling distribution, link function,
systematic component, ancillary parameters, etc. - comprise
statistical models.  These features, while vastly differing between
any two given specific models, share features that are easily
classifiable, and usually necessary in the simulation of
\emph{quantities of interest}.  That is, all statistical models have
similar traits, and can be simulated using similar methods.  Using
this fact, the \emph{parameters} class provides a set of functions
and data-structures to aid in the planning and implementation of 
the statistical model.

\section{Method Signature of \code{param}}

The signature of the \code{param} method is straightforward and does not vary between differ Zelig models.

\begin{Code}
param.logit <- function (obj, num, ...) {
  # ...
}
\end{Code}

\section{Return Value of \code{param}}

The return value of a \code{param} method is simply a list containing several entries:

\begin{description}
	\item[simulations] A vector or matrix of random draws taken from
		the model's distribution function.  For example, a logit model
		will take random draws from a Multivariate Normal distribution.
		
	\item[alpha] A vector specifying parameters to be passed into
		the distribution function.  Values for this range from scaling
		factors to statistical means.
	
	\item[fam] An optional parameter.  \emph{fam} must be an object
		of class ``family''.  This allows for the implicit specification
		of the link and link-inverse function.  It is recommend that the
		developer set either this, the link, or the linkinv parameter
		explicitly.  Setting the family object implicitly defines
		\emph{link} and \emph{linkinv}.
		
	\item[link] An optional parameter.  \emph{link} must be a function.
		Setting the link function explicitly is useful for defining
		arbitrary statistical models.  \emph{link} is used primarily to
		numerically approximate its inverse - a necessary step for
		simulating \emph{quantities of interest}.
		
	\item[linkinv] An optional parameter.  \emph{linkinv} must be a
		function.  Setting the link's inverse explicitly allows for faster
		computations than a numerical approximation provides.  If the
		inverse function is known, it is recommended that this function
		is explicitly defined.
		
\end{description}

%\section{Methods of the \code{parameters} Object}
%\begin{description}
%	\item[alpha] Extracts the contents of alpha
%	\item[coef] Extracts the simulated parameters specified in the key \code{simulations}
%	\item[link] Extracts the link function from the parameter.  This value exists as long as \emph{fam} or \emph{link} are explicitly set
%	\item[linkinv] Extracts the link-inverse function from the parameters. This value exists as long as \emph{fam}, \emph{link} or \emph{linkinv} are explicitly set.  If \emph{linkinv} is not explicitly set, then a numerical approximation is used based on \emph{fam} or \emph{link}
%\end{description}


\section{Writing the \emph{param} Method}

The ``param'' function of an arbitrary Zelig model draws samples from the
model, and describes the statistical model.  In practice, this may be done
in a variety of fashions, depending upon the complexity of the model


% LIST METHODS
% ------------
% LOGIT EXAMPLE
\subsection{List Method: Returning an Indexed List of Parameters}

While the simple method of returning a vector or matrix from a \emph{param} function is extremely simple, it has no method for setting link or link-inverse functions for use within the actual simulation process.  That is, it does not provide a clear, easy-to-read method for simulating \emph{quantities of interest}.  By returning an indexed list - or a parameters object - the developer can provide clearly labeled and stored link and link-inverse functions, as well as, ancillary parameters.


\subsubsection{Example of Indexed List Method with \emph{fam} Object Set}

\begin{verbatim}
param.logit <- function(z, x, x1=NULL, num=num)
  list(
       coef  = mvrnorm(n=num, mu=coef(z), Sigma=vcov(z)),
       alpha = NULL,
       fam   = binomial(link="logit")
       )
\end{verbatim}

% warum kommst du nicht herueber?

\subsubsection{Explanation of Indexed List with \emph{fam} Object Set Example}

The above example shows how link and link-inverse functions (for a ``logit'' model) can be set using a ``family'' object.  Family objects exist for most statistical models - logit, probit, normal, Gaussian, et cetera - and come preset with values for link and link-inverses.  This method does not differ immensely from the simple, vector-only method; however, it allows for the use of several API functions - \emph{link}, \emph{linkinv}, \emph{coef}, \emph{alpha} - that improve the readability and simplicity of the model's implementation.

The \emph{param} function and the \emph{parameters} class offer methods for automating and simplifying a large amount of repetitive and cumbersome code that may come with building the arbitrary statistical model.  While both are in principle entirely optional - so long as the \emph{qi} function is well-written - they serve as a means to quickly and elegantly implement Zelig models.


% POISSON EXAMPLE
\subsubsection{Example of Indexed List Method (with \emph{link} Function) Set}

\begin{verbatim}
param.poisson <- function(z, x, x1=NULL, num=num) {
  list(
       coef = mvrnorm(n=num, mu=coef(z), Sigma=vcov(z)),
       link = log,
             
       # because ``link'' is set,
       # the next line is purely optional
       linkinv = exp
       )
}
\end{verbatim}

\subsubsection{Explanation of Indexed List (with \emph{link} Function) Example}

The above example shows how a \emph{parameters} object can be created with by explicitly setting the statistical model's link function.  The \emph{linkinv} parameter is purely optional, since Zelig will create a numerical inverse if it is undefined.  However, the computation of the inverse is typically slower than non-iterative methods.  As a result of this, if the link-inverse is known, it should be set, using the \emph{linkinv} parameter.

The above example can also contain an \emph{alpha} parameter, in order to store important ancillary parameters - mean, standard deviation, gamma-scale, etc. - that would be necessary in the computation of \emph{quantities of interest}.


%
\section{Using a \emph{parameters} Object}

Typically, a \emph{parameters} object is used within a model's \emph{qi} function.  While the developer can typically omit the \emph{param} function and the \emph{parameters} object, it is not recommended.  This is because making use of this function can vastly improve readability and functionality of a Zelig model.  That is, \emph{param} and \emph{parameters} automate a large amount of repetitive, cumbersome code, and offer allow access to an easy-to-use API.

\subsection{Example \emph{param} Function}

\begin{Code}
qi.logit <- function(z, x, x1=NULL, sim.param=NULL, num=1000) {
  coef <- coef(sim.param)
  inverse <- linkinv(sim.param)

  eta <- coef %*% t(x)
  theta <- link.inverse(eta)

  # et cetera...
}

\end{Code}


\subsection{Explanation of Above \emph{qi} Code}

The above is a portion of the actual code used to simulate \emph{quantities of interest} for a ``logit'' model.  By using the sim.par object, which is automatically passed into the function if a \emph{param} function is written, \emph{quantities of interest} can be computed extremely generically.  The step-by-step process of the above function is as follows:

\begin{itemize}
	\item{Assign the simulations from \emph{param.logit} to the variable ``coef''}
	\item{Assign the link-inverse from \emph{param.logit} to the variable ``inverse''}
	\item{Compute $\eta$ (eta) by matrix-multiplying our simulations with our explanatory results}
	\item{Conclude ``simulating'' the \emph{quantities of interest} by applying the inverse of the link function.  The result is a vector whose median is an approximate value of the \emph{quantity of interest} and has a standard deviation that will define the confidence interval around this value}
	
\end{itemize}


\section{Future Improvements}

In future releases of Zelig, \emph{parameters} will have more API functions to facilitate common operations - sample drawing, matrix-multiplication, et cetera - so that the developer's focus can be exclusively on implementing important components of the model.