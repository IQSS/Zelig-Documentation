%
%

\section{Introduction}
\label{section:describe-introduction}

When developing a Zelig model, developers have the ability to specify citation and parameter information directly into their model's code. This allows external API's, such as those provided by the Dataverse\footnote{\url{http://thedata.org/}}, to make use of Zelig models without any additional programming on the part of the developer. In this capacity, the \code{describe} method has two purposes:

\begin{enumerate}
	\item{to give correct citation information for Zelig models}
	\item{to declare the type of data that can be processed by the Zelig model}
\end{enumerate}


% METHOD SIGNATURE
\section{\code{describe} Method Signature}
\label{section:describe-signature}

The \code{describe} method takes a single parameter - the dots argument. This does not vary between Zelig models. As an example, the \code{logit} model's method signature is:

\begin{verbatim}
describe.logit <- function (...) {
  # ...
}

\end{verbatim}



% METHOD RETURN VALUE
\section{\code{describe} Method Return Value}
\label{section:describe-return-value}

The \code{describe} method's signature is simply a list containing several required key-value pairs:

\begin{itemize}
  \item {\bf \tt author}: a vector of character-strings specifying author names
  \item {\bf \tt text}: a character-string specifying a longer, more specific title to the Zelig model. For example, the \code{logit} models \code{text} field contains the value: "Logistic Regression for Dichotomous Dependent Variables"
  \item {\bf \tt year}: an integer specifying the year that the Zelig model was originally developed
\end{itemize}

{\noindent}In addition to these key-value pairs, several optional parameters may be included:

\begin{itemize}
  \item {\bf \tt category}: a character-string specifying the category of statistical regression to which the given model belongs
  \item {\bf \tt package}: a list-style object specifying information for integration with the Dataverse\footnote{The Dataverse (\url{http://thedata.org}) is a platform for extracting, sharing, and storing research data. With the correct implementation of the \code{describe} method, any Zelig model can interact directly and seamlessly with data stored on the Dataverse.}
\end{itemize}



% NOTABLE FEATURES
\section{\code{describe} Notable Features}
\label{section:describe-notable}

The \code{describe} method notably does very little computation. That is, while it allows the developer to specify a large amount of citation information, it does not necessarily compute anything. Typically, this equates to the \code{describe} method consisting entirely of a return-value.



% DEATAILS
% \section{\code{describe} Details}
% \label{section:describe-details}




% EXAMPLE
\section{Example of a {\tt describe} Function}

The following sections detail the \code{describe} method of the \code{logit} Zelig model.

\subsection{\code{describe.logit.R}}
The following example is an excerpt from Zelig's \code{logit} model:

\begin{verbatim}
describe.logit <- function(...)
  list(
       # [1]
       authors = c("Kosuke Imai", "Olivia Lau", "Gary King"),
       # [2]
       year    = 2008,
       # [3]
       text    = "Logistic Regression for Dichotomous Dependent Variables"
       )
\end{verbatim}


\subsection{Explanation of \code{describe.logit.R}}

The above code corresponds to the following ideas:

\begin{description}
  \item[{[1]}] Specify that the model had three contributors - Kosuke Imai, Olivia Lau and Gary King
  \item[{[2]}] Specify the original year of authorship as 2008
  \item[{[3]}] Specify the complete name of the model, that should be used for citation purposes
\end{description}


%
\subsection{Results of \code{describe.logit.R}}

The citation resulting from the above code example

\begin{verbatim}
How to cite this model in Zelig:
  Kosuke Imai, Gary King, and Olivia Lau. 2008.
  "logit: Logistic Regression for Dichotomous Dependent Variables"
  in Kosuke Imai, Gary King, and Olivia Lau,
  "Zelig: Everyone's Statistical Software,"
  http://gking.harvard.edu/zelig
\end{verbatim}

\section{Summary and Conclusion of the \code{describe} Method}

The \code{describe} method provides a simple mechanism for specifying citation information. As a result, this method can be written extremely tersely.