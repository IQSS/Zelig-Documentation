\documentclass{article}

\title{Workshop 1: Fitting Statistical Models}

\begin{document}

\maketitle

\section{Introduction}

This workshop focuses on using model-fitting functions to determine several
important values in the process of statistical simulation.

\section{Information}

Statistical simulation requires several components from a fitted linear model:

\begin{itemize}
  \item The parameter \emph{coefficients}, used to predict the outcome variable,
    (typically extracted via the {\tt coef} method)
  \item The \emph{variance-covariance} matrix (typically extracted via the 
    {\tt vcov} method)
  \item The link and inverse-link functions, used to fit the model
\end{itemize}


\section{Problem}

\begin{enumerate}
  \item Select a linear regression to implement (options are given below).
    
  \item Create the Zelig Object
    \begin{itemize}
      \item Create the zelig2 function signature
      \item Return a list with a model fitting function specified
    \end{itemize}

\end{enumerate}

\section{Models to Choose From}

Several common linear models are available to implement:

\begin{itemize}
  \item Logitistic Regression (glm, family = binomial())
  \item Gamma Regression (glm, family = Gamma())
  \item Poisson - Useful for event counts
  \item Two-stage least-squares
  \item Ordinal Logistic
\end{itemize}


\end{document}
