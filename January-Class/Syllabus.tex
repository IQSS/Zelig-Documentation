\documentclass{article}


\title{Programming Statistical Models with Zelig 4}
\author{Matt Owen}

\usepackage{hyperref}


\begin{document}


\maketitle


% Introduction
\section{Introduction}
\label{intro}

In this course, students will develop statistical packages in the R programming 
language using the Zelig 4 software suite and API. Emphasis is placed on using 
generalized modeling techniques - linear regressions, simulations - to create 
predictive models that work with a large array of data-sets. Students will also 
learn the basic of developing R packages for distribution via CRAN 
(The Comprehensive R Network), a platform for sharing open-source statistical 
software on the internet.

Registration can be found on the \href{http://www.iq.harvard.edu/events/node/2710}{event site}
(\url{http://www.iq.harvard.edu/events/node/2710}).



% Prequisites
\section{Prerequisites}
\label{prereq}

Due to the brevity of the course, several requirements are imposed on registered
students:

\begin{itemize}
  \item {\bf Basic Programming}: conditional statements, loops and functions
  \item {\bf R Programming}: Sampling distributions and Linear Regressions
  \item {\bf Statistics}: Descriptive statistics, linear regressions and bootstrapping
  \item {\bf Math}: Matrix algebra
\end{itemize}



% Goals
\section{Goals}
\label{goals}

Upon completion of this course, students should have an understanding of the processes
involved in:

\begin{itemize}
	\item Developing statistical packages in R
	\item Programming Statistical and Mathematical ideas using Zelig
	\item Submitting packages to CRAN
\end{itemize}



% Goals
\section{Project}
\label{project}

Throughout this course, students will be developing one of several well known statistical models
\footnote{Students will divide themselves into groups and work together to develop a full-featured
statistical package based on one of several well-known model fitting functions.}.
That is, students will develop a statistical package based on a well-known statistical technique, 
so that their work can be compared to well-known results in inferential statistics. Additionally,
this offers students a hands-on experience with the procedures and development cycle associated
with converting Statistical concepts to actual computer software.




% Schedule
\section{Schedule}
\label{schedule}

This course will be divided between 4 days, each with a lecture and workshop component.
Lectures will focus on introducing new concept and programming techniques. Workshops will
focus on the pragmatically implementing the concepts and techniques introduced during the
lecture. During the workshop, emphasis will be placed on the problem-solving required to 
translate a Mathematical idea into a full-featured R program.



% January 17th
\subsection{January 17\textsuperscript{th} \\ Creating R packages and Fitting Statistical Models}
  
This session focuses on the creation of R packages, presents a brief overview of statistical simulation
as it relates to creating R packages, and explains the use of model-fitting functions in this role.
  
\begin{itemize}

	% lecture
	\item {\bf Lecture}
		\begin{itemize}
		  \item Overview of programming statistical simulation
			\item The {\tt zelig} function
		  \item Interfacing Zelig with external statistical packages
		  \item Model Description and Citing Your Model
		\end{itemize}
		
	% workshop
	\item {\bf Workshop}
		\begin{itemize}
			\item Structuring a statistical package
			\item Choosing a statistical model to create
			\item Writing {\tt zelig2} function
		\end{itemize}

\end{itemize}



% January 18th
\subsection{January 18\textsuperscript{th} \\ Working with Data and Counterfactuals}

This section focuses on the role of data and parameter simulation in the overall
process of statistical simulation.

\begin{itemize}

	% lecture
	\item {\bf Lecture}
		\begin{itemize}
		  \item The {\tt setx} function
		  \item Simulating parameters of an external model
		  \item Programming statistical bootstrapping
		\end{itemize}

	% workshop
	\item {\bf Workshop}
		\begin{itemize}
			\item Specifying Explanatory Variables
	    \item Writing with the {\tt param} method
		  \item Programming Counterfactuals
		\end{itemize}

\end{itemize}



% January 19th
\subsection{January 19\textsuperscript{th} \\ Simulating Quantities of Interest}

This session focuses on the role of the actual simulation of \emph{quantities of interest}.
That is, through the combined work of {\bf fitting statistical models} and {\bf parameter
simulation}, we are able to produce values with significant and particular meaning.
Emphasis will be placed on developing both the qualitative and quantitative aspects of the
process as well as the general applicability of the procedure.

\begin{itemize}

	% lecture
	\item {\bf Lecture}
		\begin{itemize}
		  \item Understanding statistical simulation
			\item The {\tt sim} function
			\item Simulating quantities of interest
		\end{itemize}
	
	% workshop
	\item {\bf Workshop}
		\begin{itemize}
			\item Writing the {\tt qi} method
		  \item Making predictions
		\end{itemize}

\end{itemize}



% January 20th
\subsection{January 20\textsuperscript{th} \\ Overview and Problem-Solving Workshop}

This session is devoted to the final aspects of R package creation: solving problems, 
beautifying-output and submitting packages to CRAN. As a result, students will spend most of this
day completing their statistical package and working out solutions to the inevitable pitfalls
that accompany developing statistical packages.

\begin{itemize}

	% lecture
	\item {\bf Lecture}
		\begin{itemize}
		  \item How to submit packages to CRAN
		  \item Review of the complete process of developing statistical packages
		  \item What's next?
		\end{itemize}
		
	% workshop
	\item {\bf Workshop}
		\begin{itemize}
		  \item Problem-solving workshop
		  \item Putting it all together
		\end{itemize}
	
\end{itemize}

\end{document}
