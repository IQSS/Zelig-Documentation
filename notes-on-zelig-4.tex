\documentclass{article}
\title{Notes on Zelig 4 Time}
\author{Matt Owen}
\begin{document}

This is a hastily written document used to serve as a timetable, etc. for the fixes proposed in JH's "Notes on Zelig 4" document. Sorry for the informality in the doc.

\section{Plots}

\subsection{Bar Plots}

\begin{enumerate}
  \item Barplots should use percentages instead of counts for their measure on the Y-axis \begin{itemize}
      \item Plots of factor data should not display arbitrary measures such as counts
      \item Fix: Scale counts with against their total count
      \item Time: Within a day
    \end{itemize}

  \item Barplots do not acknowledge data missing from a simulation  \begin{itemize}
      \item Plots of factor data tend to not include potential entries that are missing from a simualtion. Consider the case where possible values are a range from 1 to 3, however
        the simulation only produces the results {1, 3}. 2 will be absent from the barplot
      \item Fix: If appropriate labels are included in the ``qi'' function of a model, this isn't a problem. Otherwise, we can perhaps look at the passed data.frame
      \item Time: Within a day
    \end{itemize}

  \item Sort integer factor data (particularly that which is categorical) properly \begin{itemize}
      \item In short, ensure that 1 < 10 < 0100, even if data is character strings
      \item Fix: Test the data.frame and categorize it by "numeric" "integer" "factor" in advance. This may require us to cast the plotted data as a different data-type than is returned from the simulations
      \item Time: Needs some research. 1-2 days (guess-timate).
      \item Note: This is tricky. And requires use to know something about what kind of data the user expects.
    \end{itemize}
\end{enumerate}


\subsection{Density Plots}

\begin{enumerate}
  \item Omit superflous information from density plot \begin{itemize}
      \item Show densities without bandwidth (and count information [???])
      \item Fix: Use a different type of plot. plot(density(...)) defaults to display bandwidth. There are other options
      \item Time: Within a day
    \end{itemize}

  \item First Difference should show density on the same plot \begin{itemize}
      \item First differences should combine the E(Y|X) and E(Y|X1) plots. This shows the use exactly what's going on.
      \item Fix: There are plotting tools made for this.
      \item Time: Within a day
      \item Note: I really like this change
    \end{itemize}
\end{enumerate}

\subsection{General}

\begin{enumerate}
  \item Liven up color palette \begin{itemize}
      \item More interesting colors
      \item Fix: Use different colors... I think a more gentle, modern color palette would work nicely
      \item Time: Within a day... or years if we can't agree on colors
    \end{itemize}

  \item Iterate Throught Colors Palette Differently \begin{itemize}
      \item Change how we iterate through palette?
      \item Fix: ?
      \item Time: Within a day
      \item Note: This is minor and questionable. A warning should be thrown if our palette is fewer than the number of qi's simulated, which was the original intent of the if-statement. Additionally, 
        replacing code with multiple operations in place is somewhat misleading in and of itself. It's a trend avoiding perl-style one-liners. The k <- k + 1 \%\% LENGTH is common enough that most people
        can read it without a second thought though. It should be noted though that it is typically used to sweep improperly sized array issues under the rug.
    \end{itemize}
\end{enumerate}

\subsection{Plots over a Range}

These are currently omitted from Zelig 4, due to there essentially being no example code to test against.

Fix: Examine how paraments are passed into the setx function. Ensure that the model-frame is constructed correctly, and write the proper type-casting functions for setx objects -> matrices
Time:5 days with debugging/plotting/etc
Note: This is a feature that I've been waiting to put back in

\section{Backwards Compatibilty}

Paricularly for the qi-methods. This can be discussed. There have been several places backwards compatible code has been placed (param for example), however maintaining the style of coding
where you return a list of simualtions and names
\begin{verbatim}
list(
  qi = list(ev1, ev2),
  qi.name = list("EV #1", "EV #2")
  )
\end{verbatim}

vs. focusing on having people use a different style, that is not dependent upon pairing up two arranged lists:

\begin{verbatim}
list(
  "Expected Value #1" = ev1,
  "Expected Value #2" = ev2
  )
\end{verbatim}

seemed not worth pursuing at the time.

It should be noted that in the above example the qi object returned can have its values extracted via:
obj\$ev1
obj\$ev2

That is, the qi object encode ever qi title into an acronym, then numbers is uniquely. This was to avoid having to fully write out:
obj["Expected Value \#1"]

Time (to resupport the old method): Hard to tell. 3-5 days is my best guess.
Fix: Write a function that detects this sort of return-value, and appropriately casts the result. Then create an object that acts as interface for the returned qi object. This would simply add another layer of abstraction to the qi object.
Note: I don't think this change should go through. In discussions during the developmental phase of Zelig 4, we thought it was better to lean towards simplifying the API instead of
backwards compatibility.

Either way, it's a worthwhile discussion to have.

\section{Style}

Using dot arguments \begin{itemize}
  \item the \texttt{eval.in} function uses the dot arguments ..1 and ..2
  \item Fix: use named arguments
  \item Issue with the fix: using named arguments can cause masking in the namespace, because attach is used within this function. Imagine if we correspondingly name that parameters
    expr and envir. Then if "expr" and "envir" are already somewhere in the namespace, they get masked.
  \item Motivation for ..1 and ..2: Attach has be used in this case, the \texttt{with}, \texttt{eval}, etc. function on their own cannot always evaluate symbolic name appropriately
\end{itemize}

Heavy dependence on environments \begin{itemize}
  \item Zelig stores \emph{a lot} of information in environments
  \item Simple Fix: None, really. This is somewhat of the idea behind Zelig. Alternatively, we can use object-literals, but that mucks up things in a different way.
  \item Good fix: There is no real necesssity that Zelig has to have a zelig2 function. That is, statistical models can be fit using the model fitting functino directly ({\tt glm}, {\tt lm}, etc.)
    And simply these models can be passed directly into the setx and sim functions. I'm all for at least supporting this concept.
  \item Note: I've wanted to implement the good-fix for a while.
\end{itemize}

Hidden files \begin{itemize}
  \item Function should be less hidden
  \item Fix: Do some house cleaning
  \item Time: <= 2 days
  \item Note: It'd be good to do. The R folder is getting mucked up a bit. Additionally, there are unused functions in the package... I think.
\end{itemize}

\section{Zelig Skeleton}

This is good feedback. I can make the appropriate error/documentation changes this week.

\section{Error Message}

The error messages are really brutal. This is a good list, and there are probably considerably more places we can improve error-notification than just those listed in the document. This can be done next week.

\end{document}
