\documentclass{article}

\begin{document}

\section{R-Build Review}

This document is intended to serve as a set of talking points on the
R-build review.

\begin{itemize}
\item
  Summarize the goals of R-build
\item
  Explain current issues in R-build
\item
  Specify available tools for R
\item
  Specify available tools for git/cvs
\item
  Showcase some basic implementations
\end{itemize}
\subsection{Goals of Rbuild}

R-build's intention is to do the following:

\begin{itemize}
\item
  Check packages for major warnings and errors
\item
  Build documentation for the specified R package
  \begin{itemize}
    \item
      PDF
    \item
      HTML
  \end{itemize}
\item
  Build the R package's tarball
\item
  Add this tarball to http://r.iq.harvard.edu/ repository
\item
  Update repository package listing (for installs via an interactive R
  session)
\item
  Update software pages on related sites:
  \begin{itemize}
    \item
      http://gking.harvard.edu/
    \item
      http://projects.iq.harvard.edu/
  \end{itemize}
\item
  Notify builder/relevent recipients of success of failure
\end{itemize}
\subsection{Issues with Current R-Build}

This is a short list of issues with R-build

\begin{enumerate}
\item
  Does not use the ``vignettes/'' folder for storing Rnw/Sweave
  documents
\item
  Requires technical of linux to force rebuilds/read log files
\item
  A bit hard to maintain
\item
  Some features seemed to have disappeared:
\item
  Updating software release page
\item
  Updating documentation page
\item
  This was noticed in particular with the \emph{CEM} package
\end{enumerate}
\subsection{R-build Implementation Details}

Here is an outline for the R-build script:

\begin{enumerate}
\item
  Download source code from repository
  \begin{itemize}
    \item
      CVS: Use the same method currently employed. This is good for private
      repositories
    \item
      GIT: Has a JSON API, making it really easy to clone repositories from
      GitHub. This is good for public repositories
  \end{itemize}
\item
  Build the source into a package via ``R CMD build pkg-name''. This
  should be done in a temporary directory
\item
  Check the resulting tarball via ``R CMD check
  pkg-name\_version.tar.gz''
  \begin{itemize}
    \item
      If there's an error, print to the log and notify relevant people
    \item
      Otherwise, send a success notification
  \end{itemize}
\item
  Move tarball to repository
\item
  Update ``PACKAGES'' and ``PACKAGES.gz'' files via R's
  ``tools:::write\_PACKAGES()''
\item
  In the current implementation, this is done a bit circuitously
\item
  Fin.
\end{enumerate}
Here's a script I wrote that partially implements these ideas. This has
no support for working with scholars:
http://github.com/zeligdev/build\_zelig/blob/master/pkg-build

\subsection{R-build Future Features}

This is a wish-list of features in R-build.

\begin{enumerate}
\item
  Dry-run option, that allows users to see the results of R-build
  operation \emph{without} sending notifications or making changes to
  the server. This is essentially an rbuild preview
\item
  Run R-build as a script rather than the current method of: ``touch
  /nfs/projects/r/rbuild/lock/force-''
\item
  Terser log file? Right now it's a bit verbose, and hard to get
  relevant information from
\item
  Reduce build notification spam. There should be a limit to how many
  error emails you can get in one day. That is, it's good to notify
  everyone when something fails/succeeds, but there's a threshold before
  it just all becomes white noise.
\item
  Notification emails for failures should be a bit more informative.
  Simply stating at what stage the R-build script failed would be very
  useful.
\end{enumerate}
\end{document}
