% @author Matt Owen
% @date   2/22/2012
% This document serves as a specification - and talking point - to clarifying
% and elaborating on the role, function and definition of model formulae in
% the Zelig software suite (version 4.0+)
\documentclass{article}

% Packages
\usepackage{hyperref}

\title{Zelig Formula Specification}
\author{Matt Owen}

\newcommand{\tweedly}[0]{$\sim${ }}

\begin{document}

\maketitle



%
%
%
\section{Introduction and Motivation}
\label{sec:intro}

The following is a technical specification for the types of model formula
objects that are, should be, and could be officially supported by Zelig Core.

The primary motivation for this document is to create a non-rigid guidelone the
types of model formulae that can be used in the Zelig software suite
\emph{without} the need to write additional support functions. That is, so long
as model formula follow the specification later defined in this document, there
should be little need to create or modify any code aside from the model's
{\tt zelig2}, {\tt param} and {\tt qi} methods.

It should be noted that any issue stemming from use of exotic or unsupported
formulae can be resolved with use of {\bf hook} functions or overloading the
{\bf \tt setx} or {\bf \tt sim} method. This is purely a result of the
unexpected results that originate from constructing design matrices from
non-standard formulae.



%
%
%
\section{Categories of Formulae}
\label{sec:req}

The current implementation of Zelig allows for the direct support of four
categories of models:

\begin{itemize}

  \item {\bf A single response term specified within a single model equation.}
    This is the case where users are studying a single response with a single
    set of predictor terms. This is the most basic model formula.

  \item {\bf A single response term specified across multiple model equations.}
    This is the case where users are studying a single response across multiple
    sets of predictor terms.

  \item {\bf Multiple response terms specified within a single model equation.}
    This is the case where users are studying the effects of a single set of
    predictor terms on multiple response terms.

  \item {\bf Multiple response terms specified within multiple model equations.}
    This is the extreme case where users are studying multiple predictor terms
    dependent on different sets of predictor terms. Additionally, this type of
    specification is the most general.

\end{itemize}



%
%
%
\section{Existing Conventions in Zelig}
\label{sec:existing-zelig}

The categories specified in section \ref{sec:req}, have implementations in
Zelig (versions 3.5 and 4.0). It should be noted that the {\tt list}-method
supported by Zelig (see the ``Multiple Equations'' row of the following table)
is the only non-standard formula specification.


\subsection{Table of Conventions}
\label{subsec:table-conventions}

% Table
{\noindent}\begin{tabular}{|l|l|l|}

  % Top Border
  \hline

  % Column Names (Row #1)
  & Single Response Term & Multiple Response Terms \\ \hline

  % Row #2
  Single Equation &
  {\tt y \tweedly a + b + c} & {\tt cbind(x, y) \tweedly a + b + c}
  \\ \hline

  % Row #3
  Multiple Equations & 
  {\tt list(x \tweedly a, y \tweedly b)} &
  {\tt list(x \tweedly a, y \tweedly b)} \\ \hline


\end{tabular}


\subsection{User-end Specification}
\label{subsec:user-spec-formula}

The four types of formulae in section \ref{subsec:table-conventions} are created
using three basic ingredients:

\begin{itemize}

  \item {\bf ``formula'' objects}, which are the building block of any model
    formula

  \item {\bf the ``cbind'' function}, which - by convention - is used in simple
    linear models to specify multiple response terms. This, in general, does not
    work for every statistical model.

  \item {\bf ``list'' objects}, which allow users to specify multiple
    model equations in an intuitive fashion. This method is primarily used for
    the bivariate, dichotomous regressions.

\end{itemize}


\pagebreak


%
%
%
\section{Technical Specification for Basic Formula}

The basic structure of a Zelig formula conforms to the following specifications:

\subsection{Left-Hand Side of a Simple Formula}
\label{subsec:lhs}

\begin{enumerate}

  \item The left-hand side of the formula specifies the response terms.

  \item W

  \item The left-hand side only supports the following operators and functions:
    +, *, :, | and cbind.

  % \item The \verb|+| operator specifies multiple response terms

  \item Mathematical operators (\verb|+|, \verb|*|, \verb|-|, \verb|/|) are
    interpretted arithematically. In particular, \verb|*| differs in meaning
    between the right and left sides of model formula.

  \item \verb|cbind| can only be used as the first operation on the left-hand side
    of an equation. That is, the formula
    {\tt cbind(x, y) + cbind(a, b) \tweedly 1} is considered illegal.

\end{enumerate}


\subsection{Right-Hand Side of a Simple Formula}
\label{subsec:rhs}

\begin{enumerate}

  \item The right-hand side of the formula specifies the predictor terms.

\end{enumerate}

\subsection{Restrictions on Simple Formulae}
\label{subsec:restriction-simple-formula}

\begin{enumerate}

  \item The formula must only specify a single left-hand-side and a single
    right-hand-side. That is, nested formula are considered illegal. For
    example, the formula \begin{verbatim}y ~ a ~ b\end{verbatim} is considered 
    illegal.

\end{enumerate}


\subsection{Formulae with Multiple Equations}

\begin{enumerate}

  \item The formula may be a list of model formulae, if it is describing
    multiple outcome variables. In this situation, each formulae must follow
    all of the above restrictions. Additionally, each element of the list may
    only specify a single response term.

\end{enumerate}



%
%
%
\section{Existing Conventions Outside of Zelig}
\label{sec:existing-elsewhere}

Before continuing onto the Zelig spec
There are several existing conventions that specify model formula from packages
outside of Zelig.

\begin{itemize}

  \item Basic model formula, designated using the \tweedly operator, specifya single
    model equation with a single response variable. Example: 
    {\tt y \tweedly a + b}

  \item Model formula designated using the \verb+cbind+ function in conjunction
    with the \tweedly operator specify multiple response terms for a single
    equation or set of predictor terms. Example: 
    {\tt cbind(x, y) \tweedly a + b + c}

  \item The ``Formula'', object found at
    \url{http://cran.r-project.org/web/packages/Formula/}, encapsulates the
    features of a standard ``formula'' object while extending its functionality.
    In particular, ``Formula'' objects allow for the specification of multiple
    response terms and equations without using a ``list'' object or a call to
    ``cbind''. Examples:
    \begin{itemize}
      \item {\tt Formula(y \tweedly a + b)}
      \item {\tt Formula(x | y \tweedly a + b)}
      \item {\tt Formula(y \tweedly a | b)}
      \item {\tt Formula(x | y \tweedly a | b)}
   \end{itemize}

   More on the ``Formula'' object specification can be found here:\\
   \url{http://cran.r-project.org/web/packages/Formula/index.html}

\end{itemize}



\end{document}
