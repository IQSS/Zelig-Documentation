\documentclass{article}

\usepackage{hyperref}

\title{Zelig Formula Specification}
\author{Matt Owen}

\begin{document}

\maketitle

\section{Introduction}

The following is a technical specification for allowable types of ``formula''
objects that are supported by Zelig.



\section{Requirements}

Zelig formulae need to be written generally enough to handle several types of
models:

\begin{itemize}

  \item A single response term within a single model equation\footnote{asdas}

  \item A single response term with multiple model equations (sets of predictor
    terms)\footnote{}

  \item Multiple response terms with a single set of predictor terms\footnote{asd}

  \item Multiple response terms with multiple sets\footnote{Don't drink the water}

\end{itemize}



\section{Existing Conventions}

There are several existing conventions that extend the functionality of standard
formula objects:

\begin{itemize}

  \item Basic model formula designated using the \verb+~+ operator specifies a single
    model equation with a single response variable. Example: \verb|y ~ a + b + c|

  \item Model formula designated using the \verb+cbind+ function in conjunction
    with the \verb+~+ operator specify multiple response terms for a single
    equation or set of predictor terms. Example: \verb|cbind(x, y) ~ a + b + c|

  \item The ``Formula'', object found at
    \url{http://cran.r-project.org/web/packages/Formula/}, encapsulates the
    features of a standard ``formula'' object while extending its  

\end{itemize}









\section{Current Support}

Currently, Zelig has support for two types of formula:

\begin{itemize}
  \item {\bf ``formula'' objects}, which 
  \item {\bf ``list'' objects}, which contain multiple model equations
\end{itemize}

Additionally, there is support for several tags within an


\section{Conceptual Requirements}
\label{Conceptual-Requirements}

\begin{enumerate}

  \item The formulae used to articulate models should be written generally
    enough to describe most conceivable statistical models.
    In particular, Zelig needs specifications to desribe the following types
    of dependent variables:

    \begin{itemize}
      \item Continuous,
      \item Ordered and
      \item Multinomial
    \end{itemize}

  \item Additionally, the Zelig formula specification needs to be able to
    express multivariate regressions.

\end{enumerate}


\section{Basic Formula}

The basic structure of a Zelig formula conforms to the following specifications:

\begin{enumerate}
  \item The left-hand side must be written in one of two formats:
    \begin{itemize}
      \item It must evaluate to a singular vector, or
      \item It must evaluate to a matrix with one column per outcome variable.
    \end{itemize}
  \item The left-hand side of the formula specifies the outcome (dependent) variables.
  \item The formula must be flat. That is, nested formula are considered
    illegal. For example, the formula \begin{verbatim}y ~ a ~ b\end{verbatim} is considered illegal.
  \item The left-hand side only supports the following operators: ``+'', ``*''. ``:'', and ``cbind'',
  \item ``cbind'' can only be used as the first operation on the left-hand side of an equation.
  \item The right-hand side of the formula specifies the explanatory (independent) variables.
  \item The formula may be a list if it is describing multiple outcome variables.
\end{enumerate}



\end{document}
