\section{{\tt chopit}: Compound Hierarchical Ordered Probit for Survey
Vignettes}
\label{chopit}

The Compound Hierarchical Ordered Probit ({\sc chopit}) model corrects
for ``differential item functioning'' or ``interpersonal
comparability'' in ordinal survey responses.  Given a self-assessment
question (such as, ``How healthy are you?  Excellent, good, fair, or
poor.''), different respondents may interpret the response categories
in different ways, such that excellent health to one individual may be
fair health to a hypochondriac.  For each ordinal self-assessment to be
corrected, the {\sc chopit} model requires one or more vignette
question (such as a description of a hypothetical person's health,
followed by the same response categories as the self-assessment), and
a set of associated explanatory variables for the respondent.  The key
assumption of the approach is that the thresholds (which determine how
respondents translate their views into the response categories) have
the same effect for different questions asked of the same respondent,
but may differ across respondents; the model uses
a parametric specification to predict the thresholds associated with an
individual.  The self-assessment and vignette questions may be taken
from different surveys, so long as both surveys include the same
explanatory variable questions to predict the thresholds.  For ordinal
data (without vignettes), see \Sref{oprobit}, \Sref{ologit}, and
\Sref{oprobit.bayes}.  

\subsubsection{Syntax}

\begin{verbatim}
> fml <- list(self = Y ~ X1 + X2, 
              vign = cbind(Z1, Z2, Z3) ~ 1, 
              tau  = ~ X1 + X2)
> z.out <- zelig(fml, data = list(self = data1, vign = data2), 
                 model = "chopit")
> x.out <- setx(z.out)
> s.out <- sim(z.out, x = x.out, x1 = NULL)
\end{verbatim}

\subsubsection{Inputs}

In this hierarchical model, the {\tt formula} and {\tt data} inputs to
{\tt zelig()} are lists with the following structure:  
\begin{itemize}
\item The {\tt formula} is a list with three {\tt formula} objects
corresponding to:  
\begin{itemize}
\item {\tt self}:  Specifies the self-response question ({\tt Y}) as a function
of a set of explanatory variables. 
\item {\tt vign}:  Specifies the vignette questions on the left-hand
side as a matrix in the form {\tt cbind(Z1, Z2, Z3)}.  
\item {\tt tau}:  Specifies explanatory variables that constrain the cut points
across both the vignette and self-response questions.  These
explanatory variables do not necessarily need to overlap with the set
of explanatory variables specified in the {\tt self} formula, but must
be observed in both the {\tt vign} and {\tt self} data frames,
described below. 
\end{itemize}
\item The {\tt data} argument is a list of two data frames with
\begin{itemize}
\item {\tt self}:  A data frame containing the self-response
question(s) specified in the {\tt self} formula and associated
explanatory variables listed in the {\tt self} and {\tt tau} formulas.
\item {\tt vign}:  A data frame containing the vignette questions
specified in the {\tt vign} formula and associated explanatory
variables listed in the {\tt tau} formula.
\end{itemize}
\end{itemize}  

\subsubsection{Additional Inputs} 

In addition to the standard inputs, {\tt zelig()} takes many
additional options for compound hierarchical ordered probit
regression, see {\tt help(chopit)} and \cite{WanKinLau07} for details.

\subsubsection{Examples}
\begin{enumerate}
\item {Basic Example}

Setting up the formula as a list for the self-response, vignettes, and
the cut points (drawn from both the self-response and vignette data sets).
\begin{Schunk}
\begin{Sinput}
RRR> formula <- list(self = y ~ sex + age + educ + factor(country),
+                 vign = cbind(v1, v2, v3, v4, v5) ~ 1,
+                 tau  = ~ sex + age + educ + factor(country))
\end{Sinput}
\end{Schunk}
Attaching the sample data sets.  The {\tt free1} data correspond to
the self-response data, and the {\tt free2} data correspond to the
vignette subset.  Note that the variables specified in the {\tt tau}
formula must be in both data sets.
\begin{Schunk}
\begin{Sinput}
RRR> data(free1, free2)
\end{Sinput}
\end{Schunk}
\begin{Schunk}
\begin{Sinput}
RRR> data <- list(self = free1, vign = free2)
\end{Sinput}
\end{Schunk}
Estimating parameter values for the {\sc chopit} regression:
\begin{Schunk}
\begin{Sinput}
RRR>  z.out <- zelig(formula,  data = data, model = "chopit") 
\end{Sinput}
\end{Schunk}
Setting values for the explanatory variables to their default values:
\begin{Schunk}
\begin{Sinput}
RRR>  x.out1 <- setx(z.out)
\end{Sinput}
\end{Schunk}
Simulating quantities of interest from the sampling distribution.
\begin{Schunk}
\begin{Sinput}
RRR>  s.out1 <- sim(z.out, x = x.out1)
\end{Sinput}
\end{Schunk}
\begin{Schunk}
\begin{Sinput}
RRR>  summary(s.out1)
\end{Sinput}
\end{Schunk}

\item {Simulating First Differences}

Estimate the first difference in expected values between the average
age (about 40 years old) and a 25 year old individual, with the other
explanatory variables held at their default values:
\begin{Schunk}
\begin{Sinput}
RRR>  x.out2 <- setx(z.out, age = 25)
\end{Sinput}
\end{Schunk}
\begin{Schunk}
\begin{Sinput}
RRR>  s.out2 <- sim(z.out, x = x.out1, x1 = x.out2)
\end{Sinput}
\end{Schunk}
\begin{Schunk}
\begin{Sinput}
RRR>  summary(s.out2)
\end{Sinput}
\end{Schunk}

\item {Conditional prediction}  

Conditional prediction generates expected values that are conditional
on the observed self-response.  
\begin{Schunk}
\begin{Sinput}
RRR> x.out3 <- setx(z.out, cond = TRUE)
\end{Sinput}
\end{Schunk}
Since conditional prediction involves numeric integration, the
procedure takes approximately one second per observation in {\tt
x.out3} on 64-bit R. 
\begin{Schunk}
\begin{Sinput}
RRR> s.out3 <- sim(z.out, x = x.out3)
\end{Sinput}
\end{Schunk}
\begin{Schunk}
\begin{Sinput}
RRR> summary(s.out3)
\end{Sinput}
\end{Schunk}
\end{enumerate}

\subsubsection{Model}

This model has two sets of response variables, one for the
self-assessment and one for the vignettes.  Let $Y_i$ be the observed
ordinal self-assessment for respondents $i = 1, \dots, n$, and
$Z_{lj}$ be the ordinal vignette responses for individuals $l = 1,
\dots, L$ in the vignette subset for $j = 1, \dots, J$ vignette
questions, such that both $\{Y_i, Z_{lj} \}$ take integer values $k
= 1, \dots, K$ corresponding to the same ordinal assessment response categories.

\begin{itemize}
\item The \emph{stochastic components} are described by unobserved continuous
  variables, $Y_i^*$ and $Z_{lj}^*$, which follows normal
  distributions with mean $\mu_i$ and variance $\sigma^2$ in
  the case of $Y_i^*$, and mean $\theta_j$ and variance
  $\sigma_j^2$ in the case of each $Z_{lj}^*$.  Using the default
identification mechanism, the variance $\sigma^2$ for the
self-assessment is fixed to 1.  Thus, 
\begin{eqnarray*}
 Y_i^* & \sim & N(\mu_i, 1)\\
 Z_{lj}^* & \sim& N(\theta_j, \sigma_j^2)
\end{eqnarray*}
such that each vignette response $j$ has a scalar mean $\theta_j$ and
variance $\sigma_j^2$ that does not vary over observations
$l$.  In cases where more than one self-response was administered to the same subject, an additional random effect may be included in
the distribution of the latent $Y_i^*$ in the form
\begin{eqnarray*}
 Y_i^* & \sim & N(\mu_i, 1 + \omega^2)\\
\end{eqnarray*}
where the variance term is obtained via the proof described in
Appendix A of \cite{KinMurSal04}.

The observation mechanisms that divide the continuous $\{ Y_i^*,
Z_{lj}^* \}$ into the discrete $\{ Y_i, Z_{lj} \}$ are
\begin{eqnarray*}
 Y_i & = & k \quad {\rm if} \quad \tau_{i}^{k-1} \le Y_i^* \le \tau_{i}^{k}
    \; {\rm for} \; k = 1,\dots, K \\
 Z_{lj} & = & k \quad {\rm if} \quad \tau_{l}^{k-1} \le Z_{lj}^* \le \tau_{l}^{k}
    \; {\rm for} \; k = 1,\dots, K 
\end{eqnarray*}
where the threshold parameters $\tau$ vary over individuals $\{i, l
\}$, but are subject to the following constraints within each
individual: $\tau^p < \tau^q$ for all $p < q$ and $\tau_0=-\infty$ and
$\tau_K=\infty$.

\item There are three \emph{systematic components} in the model.  
\begin{itemize}

\item For the self-assessment component, let
\begin{eqnarray*}
\mu_i &=& x_i \beta 
\end{eqnarray*}
where $x_i$ is the vector of $q$ explanatory variables for observation $i$,
and $\beta$ is the associated vector of coefficients.

\item In addition, the threshold parameters also vary over individuals
in the self-assessment component as follows
\begin{eqnarray*}
\tau_{i}^{1} &=& v_i \gamma^1 \\
\tau_{i}^{k} &=& \tau_{i}^{k-1} + \exp(v_i \gamma^k) \; {\rm for} \;
k = 2, \dots, K
\end{eqnarray*}
where $v_i$ is the vector of $p$ explanatory variables for observation
$i$, and $\gamma^k$ for $k = 1, \dots, K$ are the vectors of
coefficients associated with each categorical response.  Thus, the
threshold parameters vary over individuals since $v_i$ vary, and over
response categories since the $\gamma^k$ vary over the threshold
parameters.  

\item Similarly, the threshold parameters vary over individuals in the
vignette component as follows
\begin{eqnarray*}
\tau_{l}^{1} &=& v_l \gamma^1 \\
\tau_{l}^{k} &=& \tau_{l}^{k-1} + \exp(v_l \gamma^k) \; {\rm for} \;
k = 2, \dots, K
\end{eqnarray*}
where $v_l$ is a vector of $p$ explanatory variables for observation
$l$ in the vignette subset, and $\gamma^k$ are restricted to be the
same $\gamma^k$ used to parameterize the threshold parameters for the
self-assessment component.
\end{itemize}
As \cite{KinMurSal04} note, the interpersonal comparability of
responses (or response consistency) is achieved by constraining
$\gamma^k$ to be the same in both the self-assessment and vignette
components of the model.  Note that the variables included in $v_i$ and
$v_l$ are the same, but the observed values of those variables differ across
the vignette and self-response samples.
\end{itemize}


\subsubsection{Quantities of Interest}

\begin{itemize}
\item The expected value ({\tt qi\$ev}) for the {\sc chopit} model
is the expected value of the posterior density for the systematic
component $\mu_i$,
\begin{equation*}
\textrm{EV} \; = \; E(\mu_i \mid x_i) \; = \; x_i \beta
\end{equation*}
given draws of $\beta$ from its sampling distribution.  

\item The first difference is the difference in the expected value of
the posterior density for the systematic component $\mu_i$ given $x_1$
and $x_0$:
\begin{equation*}
\textrm{FD} = E(\mu_i \mid x_1) - E(\mu_i \mid x_0).
\end{equation*}

\item In conditional prediction models, the conditional expected
values ({\tt qi\$cev}) are the expected value of the distribution of
$\mu_i$ conditional on the observed self-assessment response $Y_i$, 
%%This quantity is calculated in two steps.  First, using draws of
%%$\gamma^k$ for $k = 1, \dots, K$ from their posterior distributions,
%%we calculate simulations of the threshold parameters
%%\begin{eqnarray*}
%%\tau_i^1 &=& v_i \gamma^1 \\
%%\tau_i^k &=& \tau_i^{k-1} + \exp(v_i \gamma^k) \; {\rm for} \; k = 2, \dots, K.
%%\end{eqnarray*}
%%Next, we use adaptive integration to calculate the expected value of
%%the posterior distribution conditional on $Y_i$:  
%%\begin{eqnarray*}
%%E(\mu_i \mid x_i, Y_i) &=& \int \mu_i P(\mu_i \mid \tau_i, \beta, x_i,
%%Y_i) d \mu_i
%%\end{eqnarray*}
where
\begin{eqnarray*}
P(\mu_i \mid \tau_i, \beta, x_i, Y_i) &=& \prod_{k=1}^K [ \Phi(\tau_i^k -
\mu_i) - \Phi(\tau_i^{k-1} - \mu_i) ] \times N(x_i \beta, x_i
\widehat{V}(\widehat{\beta}) x_i^{\prime} + \widehat{\omega}^2)
\end{eqnarray*}
given the simulations of the threshold parameters calculated above,
draws of $\beta$ from its sampling distribution, and the
estimated variance-covariance matrix for $\widehat{\beta}$.

\end{itemize}

\subsubsection{Output Values}

The output of each Zelig command contains useful information which you
may view.  For example, if you run \texttt{z.out <- zelig(\dots,
  model = "chopit")}, then you may examine the available
information in \texttt{z.out} by using \texttt{names(z.out)},
see the estimated parameters by using {\tt z.out\$par}, and
a default summary of information through \texttt{summary(z.out)}.
Other elements available through the {\tt \$} operator are listed
below.

\begin{itemize}
\item From the {\tt zelig()} output object {\tt z.out}, you may
  extract:
   \begin{itemize}
     \item {\tt par}: the maximum likelihood parameter estimates for
$\widehat{\gamma}^k$ for $k = 1, \dots, K$ response categories,
log($\widehat{\omega}$) (if estimated), log($\widehat{\sigma}$) (if
estimated), log($\widehat{\sigma_j}$) for $j = 1, \dots, J$ vignette
questions, $\widehat{\theta_j}$, and $\widehat{\beta}$.
\item {\tt chopit.hessian}:  the estimated Hessian matrix, with rows
and columns corresponding to the elements in {\tt par}.    
\item {\tt value}: the value of the log-likelihood at its maximum
\item {\tt counts}:  the number of function and gradient calls to
reach the maximum.
\item {\tt formula}: the formula for {\tt self}, {\tt vign}, and {\tt
tau} selected by the user.
\item {\tt call}:  the call to {\tt zelig()}.  
\item {\tt \dots}:  additional outputs described in {\tt
help(chopit)}.  
   \end{itemize}

\item Typing {\tt summary(z.out)} provides a useful summary of the
output from {\tt zelig()}, but no items can be extracted.  

\item From the {\tt sim()} output object {\tt s.out}, you may extract
  quantities of interest arranged as matrices indexed by simulation
  $\times$ {\tt x}-observation (for more than one {\tt x}-observation).
  Available quantities are:

   \begin{itemize}
   \item {\tt qi\$ev}: the simulated expected values for the
     specified values of {\tt x}.
   \item {\tt qi\$fd}: the simulated first difference in the expected
     values for the values specified in {\tt x} and {\tt x1}.
   \item {\tt qi\$cev}: the simulated conditional expected value given
{\tt x}.  
   \end{itemize}
\end{itemize}

\subsection* {How to Cite} 

To cite the \emph{ chopit } Zelig model:
 \begin{verse}
 Kosuke Imai, Gary King, and Oliva Lau. 2007. "chopit: Compound Hierarchical Ordinal Probit Regression for Survey Vignettes" in Kosuke Imai, Gary King, and Olivia Lau, "Zelig: Everyone's Statistical Software,"\url{http://gking.harvard.edu/zelig} 
\end{verse}
To cite Zelig as a whole, please reference these two sources:
\begin{verse}
  Kosuke Imai, Gary King, and Olivia Lau. 2007. ``Zelig: Everyone's
  Statistical Software,'' \url{http://GKing.harvard.edu/zelig}.
\end{verse}
\begin{verse}
Imai, Kosuke, Gary King, and Olivia Lau. (2008). ``Toward A Common Framework for Statistical Analysis and Development.'' Journal of Computational and Graphical Statistics, Vol. 17, No. 4 (December), pp. 892-913. 
\end{verse}


\subsection* {See also}
The {\sc chopit} model is part of the anchors package by Jonathan
Wand, Gary King, and Olivia Lau \citep{WanKinLau07}. Advanced users
may wish to refer to \texttt{help(chopit)}, as well as
\cite{KinMurSal04} and \cite{KinWan07}. 
