\section{{\tt relogit}: Rare Events Logistic Regression for
Dichotomous Dependent Variables}
\label{relogit}

The {\tt relogit} procedure estimates the same model as standard
logistic regression (appropriate when you have a dichotomous dependent
variable and a set of explanatory variables; see \Sref{logit}), but
the estimates are corrected for the bias that occurs when the
sample is small or the observed events are rare (i.e., if the
dependent variable has many more 1s than 0s or the reverse).  The {\tt
  relogit} procedure also optionally uses prior correction for
case-control sampling designs.  

\subsubsection{Syntax}

\begin{verbatim}
> z.out <- zelig(Y ~ X1 + X2, model = "relogit", tau = NULL,
                 case.control = c("prior", "weighting"), 
                 bias.correct = TRUE, robust = FALSE, 
                 data = mydata, ...)
> x.out <- setx(z.out)
> s.out <- sim(z.out, x = x.out)
\end{verbatim}

\subsubsection{Arguments}

The {\tt relogit} procedure supports four optional arguments in
addition to the standard arguments for {\tt zelig()}.  You may
additionally use:  
\begin{itemize}
\item {\tt tau}: a vector containing either one or two values for
  $\tau$, the true population fraction of ones.  Use, for example,
  {\tt tau = c(0.05, 0.1)} to specify that the lower bound on {\tt
  tau} is 0.05 and the upper bound is 0.1.  If left unspecified, only
finite-sample bias correction is performed, not case-control correction.
\item {\tt case.control}: if {\tt tau} is specified, choose a method
to correct for case-control sampling design: {\tt "prior"} (default)
or {\tt "weighting"}.   
\item {\tt bias.correct}: a logical value of {\tt TRUE} (default) or
  {\tt FALSE} indicating whether the intercept should be corrected for
  finite sample (rare events) bias.
\item {\tt robust}:  defaults to {\tt FALSE} (except when {\tt
case.control = "weighting"}; the default in this case becomes {\tt
robust = TRUE}). If {\tt TRUE} is selected, {\tt zelig()} computes
robust standard errors via the {\tt sandwich} package (see
\cite{Zeileis04}).  The default type of robust standard error is
heteroskedastic and autocorrelation consistent (HAC), and assumes that
observations are ordered by time index.

In addition, {\tt robust} may be a list with the following options:  
\begin{itemize}
\item {\tt method}:  Choose from 
\begin{itemize}
\item {\tt "vcovHAC"}: (default if {\tt robust = TRUE}) HAC standard
errors. 
\item {\tt "kernHAC"}: HAC standard errors using the
weights given in \cite{Andrews91}. 
\item {\tt "weave"}: HAC standard errors using the
weights given in \cite{LumHea99}.  
\end{itemize}  
\item {\tt order.by}: defaults to {\tt NULL} (the observations are
chronologically ordered as in the original data).  Optionally, you may
specify a vector of weights (either as {\tt order.by = z}, where {\tt
z} exists outside the data frame; or as {\tt order.by = \~{}z}, where
{\tt z} is a variable in the data frame)  The observations are
chronologically ordered by the size of {\tt z}.
\item {\tt \dots}:  additional options passed to the functions 
specified in {\tt method}.   See the {\tt sandwich} library and
\cite{Zeileis04} for more options.   
\end{itemize}
\end{itemize}
Note that if {\tt tau = NULL, bias.correct = FALSE, robust = FALSE},
the {\tt relogit} procedure performs a standard logistic regression
without any correction.

\subsubsection*{Example 1: One Tau with Prior Correction and Bias Correction}

Due to memory and space considerations, the data used here are a
sample drawn from the full data set used in King and Zeng,
2001,\nocite{KinZen01b}  The proportion of militarized interstate
conflicts to the absence of disputes is $\tau = 1,042 / 303,772
\approx 0.00343$.  To estimate the model,
\begin{Schunk}
\begin{Sinput}
RRR>  data(mid)
\end{Sinput}
\end{Schunk}
\begin{Schunk}
\begin{Sinput}
RRR>  z.out1 <- zelig(conflict ~ major + contig + power + maxdem + mindem + years,
+                   data = mid, model = "relogit", tau = 1042/303772)
\end{Sinput}
\end{Schunk}
Summarize the model output:  
\begin{Schunk}
\begin{Sinput}
RRR>  summary(z.out1)
\end{Sinput}
\end{Schunk}
Set the explanatory variables to their means:  
\begin{Schunk}
\begin{Sinput}
RRR>  x.out1 <- setx(z.out1)
\end{Sinput}
\end{Schunk}
Simulate quantities of interest:
\begin{Schunk}
\begin{Sinput}
RRR>  s.out1 <- sim(z.out1, x = x.out1)
RRR>  summary(s.out1)
\end{Sinput}
\end{Schunk}
\begin{center} 
\begin{Schunk}
\begin{Sinput}
RRR>  plot(s.out1)
\end{Sinput}
\end{Schunk}
\includegraphics{vigpics/relogit-Example1Plot}
\end{center}

\subsubsection*{Example 2: One Tau with Weighting, Robust Standard
Errors, and Bias Correction} 

Suppose that we wish to perform case control correction using
weighting (rather than the default prior correction).  To
estimate the model:  
\begin{Schunk}
\begin{Sinput}
RRR>  z.out2 <- zelig(conflict ~ major + contig + power + maxdem + mindem + years,
+                   data = mid, model = "relogit", tau = 1042/303772, 
+                   case.control = "weighting", robust = TRUE)
\end{Sinput}
\end{Schunk}
Summarize the model output:  
\begin{Schunk}
\begin{Sinput}
RRR>  summary(z.out2)
\end{Sinput}
\end{Schunk}
Set the explanatory variables to their means:  
\begin{Schunk}
\begin{Sinput}
RRR>  x.out2 <- setx(z.out2)
\end{Sinput}
\end{Schunk}
Simulate quantities of interest:
\begin{Schunk}
\begin{Sinput}
RRR>  s.out2 <- sim(z.out2, x = x.out2)
RRR>  summary(s.out2)
\end{Sinput}
\end{Schunk}

\subsubsection*{Example 3: Two Taus with Bias Correction and Prior Correction}

Suppose that we did not know that $\tau \approx 0.00343$, but only
that it was somewhere between $(0.002, 0.005)$.  To estimate a model
with a range of feasible estimates for $\tau$ (using the default prior
correction method for case control correction):
\begin{Schunk}
\begin{Sinput}
RRR>  z.out2 <- zelig(conflict ~ major + contig + power + maxdem + mindem 
+                   + years, data = mid, model = "relogit", 
+                   tau = c(0.002, 0.005))
\end{Sinput}
\end{Schunk}
Summarize the model output:  
\begin{Schunk}
\begin{Sinput}
RRR>  summary(z.out2)
\end{Sinput}
\end{Schunk}
Set the explanatory variables to their means:  
\begin{Schunk}
\begin{Sinput}
RRR>  x.out2 <- setx(z.out2)
\end{Sinput}
\end{Schunk}
Simulate quantities of interest:
\begin{Schunk}
\begin{Sinput}
RRR>  s.out <- sim(z.out2, x = x.out2)
\end{Sinput}
\end{Schunk}
\begin{Schunk}
\begin{Sinput}
RRR> summary(s.out2)
\end{Sinput}
\end{Schunk}
\begin{center}
\begin{Schunk}
\begin{Sinput}
RRR>  plot(s.out2)
\end{Sinput}
\end{Schunk}
\includegraphics{vigpics/relogit-Example3Plot}
\end{center}
The cost of giving a range of values for $\tau$ is that point
estimates are not available for quantities of interest.  Instead,
quantities are presented as confidence intervals with significance
less than or equal to a specified level (e.g., at least 95\% of the
simulations are contained in the nominal 95\% confidence interval).  

\subsubsection{Model}

\begin{itemize}
\item Like the standard logistic regression, the \emph{stochastic
    component} for the rare events logistic regression is:
\begin{equation*}
  Y_i \; \sim \; \textrm{Bernoulli}(\pi_i), 
\end{equation*}
where $Y_i$ is the binary dependent variable, and takes a value of
either 0 or 1.

\item The \emph{systematic component} is: 
  \begin{equation*}
    \pi_i \; = \; \frac{1}{1 + \exp(-x_i \beta)}.
  \end{equation*}


\item If the sample is generated via a case-control (or choice-based)
  design, such as when drawing all events (or ``cases'') and a sample
  from the non-events (or ``controls'') and going backwards to collect
  the explanatory variables, you must correct for selecting on the
  dependent variable.  While the slope coefficients are approximately
  unbiased, the constant term may be significantly biased.  Zelig has
two methods for case control correction:  
\begin{enumerate}
\item The ``prior correction'' method 
adjusts the intercept term.  Let $\tau$ be the true population
fraction of events, $\bar{y}$ the fraction of events in the sample,
and $\hat{\beta_0}$ the uncorrected intercept term.  The corrected
intercept $\beta_0$ is:
\begin{equation*}
\beta =  \hat{\beta_0} - \ln \left[ \bigg( \frac{1 - \tau}{\tau}
  \bigg) \bigg( \frac{\bar{y}}{1 - \bar{y}} \bigg) \right].
\end{equation*}

\item The ``weighting'' method performs a weighted logistic regression to
correct for a case-control sampling design.  Let the 1 subscript
denote observations for which the dependent variable is observed as a
1, and the 0 subscript denote observations for which the dependent
variable is observed as a 0.  Then the vector of weights $w_i$
\begin{eqnarray*}
w_1 &=& \frac{\tau}{\bar{y}} \\
w_0 &=& \frac{(1 - \tau)}{(1 - \bar{y})} \\
w_i &=& w_1 Y_i + w_0 (1 - Y_i)
\end{eqnarray*} 
\end{enumerate}
  If $\tau$ is unknown, you may alternatively specify an upper and
  lower bound for the possible range of $\tau$.  In this case, the
  {\tt relogit} procedure uses ``robust Bayesian'' methods to generate
  a confidence interval (rather than a point estimate) for each
  quantity of interest.  The nominal coverage of the confidence
  interval is at least as great as the actual coverage.
  
\item By default, estimates of the the coefficients $\beta$ are
  bias-corrected to account for finite sample or rare events bias.  In
  addition, quantities of interest, such as predicted probabilities,
  are also corrected of rare-events bias.  If $\widehat{\beta}$ are
the uncorrected logit coefficients and bias($\widehat{\beta}$) is the
bias term, the corrected coefficients $\tilde{\beta}$ are
\begin{equation*}
\widehat{\beta} - \textrm{bias}(\widehat{\beta}) = \tilde{\beta}
\end{equation*}
The bias term is
\begin{equation*}
\textrm{bias}(\widehat{\beta}) = (X'WX)^{-1} X'W \xi
\end{equation*}
where
\begin{eqnarray*}
\xi_i &=& 0.5 Q_{ii} \Big( (1 + w-1)\widehat{\pi}_i - w_1 \Big) \\
Q &=& X(X'WX)^{-1} X' \\
W = \textrm{diag}\{\widehat{\pi}_i (1 - \widehat{\pi}_i) w_i\}
\end{eqnarray*}
where $w_i$ and $w_1$ are given in the ``weighting'' section above.  
\end{itemize}

\subsubsection{Quantities of Interest}

\begin{itemize}
\item For either one or no $\tau$:  
  \begin{itemize}
  \item The expected values ({\tt qi\$ev}) for the rare events logit
    are simulations of the predicted probability $$E(Y) = \pi_i =
    \frac{1}{1 + \exp(-x_i \beta)},$$
    given draws of $\beta$ from its posterior.
  \item The predicted value ({\tt qi\$pr}) is a draw from a binomial
    distribution with mean equal to the simulated $\pi_i$.  
  \item The first difference ({\tt qi\$fd}) is defined as
   \begin{equation*}
   \textrm{FD} = \Pr(Y = 1 \mid x_1, \tau) - \Pr(Y = 1 \mid x, \tau).
   \end{equation*}
  \item The risk ratio ({\tt qi\$rr}) is defined as
   \begin{equation*}
   \textrm{RR} = \Pr(Y = 1 \mid x_1, \tau) \ / \ \Pr(Y = 1 \mid x, \tau).
   \end{equation*}
   \end{itemize}
 \item For a range of $\tau$ defined by $[\tau_1, \tau_2]$, each of
   the quantities of interest are $n \times 2$ matrices, which report
   the lower and upper bounds, respectively, for a confidence interval
   with nominal coverage at least as great as the actual coverage.  At
   worst, these bounds are conservative estimates for the likely range
   for each quantity of interest.  Please refer to \hlink{King and
     Zeng
     (2002)}{http://gking.harvard.edu/files/1s.pdf}\nocite{KinZen02b}
   for the specific method of calculating bounded quantities of
   interest.

\item In conditional prediction models, the average expected treatment
  effect ({\tt att.ev}) for the treatment group is 
    \begin{equation*} \frac{1}{\sum_{i=1}^n t_i}\sum_{i:t_i=1}^n \left\{ Y_i(t_i=1) -
      E[Y_i(t_i=0)] \right\},
    \end{equation*} 
    where $t_i$ is a binary explanatory variable defining the treatment
    ($t_i=1$) and control ($t_i=0$) groups.  Variation in the
    simulations are due to uncertainty in simulating $E[Y_i(t_i=0)]$,
    the counterfactual expected value of $Y_i$ for observations in the
    treatment group, under the assumption that everything stays the
    same except that the treatment indicator is switched to $t_i=0$.

\item In conditional prediction models, the average predicted treatment
  effect ({\tt att.pr}) for the treatment group is 
    \begin{equation*} \frac{1}{\sum_{i=1}^n t_i}\sum_{i:t_i=1}^n \left\{ Y_i(t_i=1) -
      \widehat{Y_i(t_i=0)} \right\},
    \end{equation*} 
    where $t_i$ is a binary explanatory variable defining the
    treatment ($t_i=1$) and control ($t_i=0$) groups.  Variation in
    the simulations are due to uncertainty in simulating
    $\widehat{Y_i(t_i=0)}$, the counterfactual predicted value of
    $Y_i$ for observations in the treatment group, under the
    assumption that everything stays the same except that the
    treatment indicator is switched to $t_i=0$.
\end{itemize}

\subsubsection{Output Values}

The output of each Zelig command contains useful information which you
may view.  For example, if you run \texttt{z.out <- zelig(y \~\,
  x, model = "relogit", data)}, then you may examine the available
information in \texttt{z.out} by using \texttt{names(z.out)},
see the {\tt coefficients} by using {\tt z.out\$coefficients}, and
a default summary of information through \texttt{summary(z.out)}.
Other elements available through the {\tt \$} operator are listed
below.

\begin{itemize}
\item From the {\tt zelig()} output object {\tt z.out}, you may
  extract:
   \begin{itemize}
   \item {\tt coefficients}: parameter estimates for the explanatory
     variables.
   \item {\tt bias.correct}: {\tt TRUE} if bias correction was
selected, else {\tt FALSE}.  
    \item {\tt prior.correct}: {\tt TRUE} if prior correction was
selected, else {\tt FALSE}. 
    \item {\tt weighting}: {\tt TRUE} if weighting was selected, else
{\tt FALSE}.  
    \item {\tt tau}:  the value of {\tt tau} for which case control
correction was implemented.  
   \item {\tt residuals}: the working residuals in the final iteration
     of the IWLS fit.
   \item {\tt fitted.values}: the vector of fitted values for the
     systemic component, $\pi_i$.
   \item {\tt linear.predictors}: the vector of $x_{i} \beta$
   \item {\tt aic}: Akaike's Information Criterion (minus twice the
     maximized log-likelihood plus twice the number of coefficients).
   \item {\tt df.residual}: the residual degrees of freedom.
   \item {\tt df.null}: the residual degrees of freedom for the null
     model.
   \item {\tt zelig.data}: the input data frame if {\tt save.data = TRUE}.  
   \end{itemize}
   Note that for a range of $\tau$, each of the above items may be
   extracted from the {\tt "lower.estimate"} and {\tt
     "upper.estimate"} objects in your {\tt zelig} output.  Use {\tt
     lower <- z.out\$lower.estimate}, and then {\tt
     lower\$coefficients} to extract the coefficients for the
   empirical estimate generated for the smaller of the two $\tau$.

\item From {\tt summary(z.out)}, you may extract: 
   \begin{itemize}
   \item {\tt coefficients}: the parameter estimates with their
     associated standard errors, $p$-values, and $t$-statistics.
   \item{\tt cov.scaled}: a $k \times k$ matrix of scaled covariances.
   \item{\tt cov.unscaled}: a $k \times k$ matrix of unscaled
     covariances.  
   \end{itemize}

\item From the {\tt sim()} output object {\tt s.out}, you may extract
  quantities of interest arranged as matrices indexed by simulation
  $\times$ {\tt x}-observation (for more than one {\tt x}-observation).
  Available quantities are:

   \begin{itemize}
   \item {\tt qi\$ev}: the simulated expected values, or predicted
     probabilities, for the specified values of {\tt x}.
   \item {\tt qi\$pr}: the simulated predicted values drawn from Binomial
     distributions given the predicted probabilities.  
   \item {\tt qi\$fd}: the simulated first difference in the predicted
     probabilities for the values specified in {\tt x} and {\tt x1}.
   \item {\tt qi\$rr}: the simulated risk ratio for the predicted
     probabilities simulated from {\tt x} and {\tt x1}.
   \item {\tt qi\$att.ev}: the simulated average expected treatment
     effect for the treated from conditional prediction models.  
   \item {\tt qi\$att.pr}: the simulated average predicted treatment
     effect for the treated from conditional prediction models.  
   \end{itemize}
\end{itemize}

\subsubsection{Differences with Stata Version}
The Stata version of ReLogit and the R implementation differ slightly
in their coefficient estimates due to differences in the matrix
inversion routines implemented in R and Stata.  Zelig uses
orthogonal-triangular decomposition (through {\tt lm.influence()}) to
compute the bias term, which is more numerically stable than
standard matrix calculations.


\subsection* {How to Cite} 

To cite the \emph{ relogit } Zelig model:
 \begin{verse}
 Kosuke Imai, Gary King, and Oliva Lau. 2007. "relogit: Rare Events Logistic Regression for Dichotomous Dependent Variables" in Kosuke Imai, Gary King, and Olivia Lau, "Zelig: Everyone's Statistical Software,"\url{http://gking.harvard.edu/zelig} 
\end{verse}
To cite Zelig as a whole, please reference these two sources:
\begin{verse}
  Kosuke Imai, Gary King, and Olivia Lau. 2007. ``Zelig: Everyone's
  Statistical Software,'' \url{http://GKing.harvard.edu/zelig}.
\end{verse}
\begin{verse}
Imai, Kosuke, Gary King, and Olivia Lau. (2008). ``Toward A Common Framework for Statistical Analysis and Development.'' Journal of Computational and Graphical Statistics, Vol. 17, No. 4 (December), pp. 892-913. 
\end{verse}


\subsection* {See also}
For more information see \cite{KinZen01b},\cite{KinZen01},\cite{KinZen02b}.
Sample data are from \cite{KinZen01b}.
