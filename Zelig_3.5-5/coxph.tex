\section{\texttt{coxph}: Cox Proportional Hazards Regression for Duration Dependent Variables}
\label{coxph}

Choose the Cox proportional hazards regression model if the values in your dependent
variable are duration observations.  The advantage of the semi-parametric Cox proportional hazards model over fully parametric models such as the exponential or Weibull models is that it makes no assumptions about the shape of the baseline hazard.  The model only requires the proportional hazards assumption that the baseline hazard does not vary across observations.  The baseline hazard can be estimated from the model via post-hoc analysis.

\subsubsection{Syntax}

\begin{verbatim}
> z.out <- zelig(Surv(Y, C) ~ X1 + X2, model = "coxph", data = mydata)
> x.out <- setx(z.out)
> s.out <- sim(z.out, x = x.out)
\end{verbatim}
Cox proportional hazards models require that the dependent variable be in the form {\tt
  Surv(Y, C)}, where {\tt Y} and {\tt C} are vectors of length $n$.
For each observation $i$ in 1, \dots, $n$, the value $y_i$ is the
duration (lifetime, for example), and the associated $c_i$ is a binary
variable such that $c_i = 1$ if the duration is not censored ({\it
  e.g.}, the subject dies during the study) or $c_i = 0$ if the
duration is censored ({\it e.g.}, the subject is still alive at the
end of the study).  If $c_i$ is omitted, all Y are assumed to be
completed; that is, $c_i$ defaults to 1 for all observations.


\subsubsection{Additional Inputs}

In addition to the standard inputs, {\tt zelig()} takes the following additional options for Cox proportional hazards regression:

\begin{itemize}
\item {\tt robust}: defaults to {\tt FALSE}.  If {\tt TRUE, zelig()} computes robust standard errors based on sandwich estimators (see \citet{Huber81} and \citet{White80}) based on the options in {\tt cluster}.

\item {\tt cluster}: if {\tt robust = TRUE}, you may select a variable to define groups of correlated observations.  Let {\tt X3} be a variable that consists of either discrete numeric values, character strings, or factors that define the clusters.  Then 

\begin{verbatim}
> z.out <- zelig(Surv(Y,C) ~ X1 + X2, robust = TRUE, cluster = "X3", 
                 model = "coxph", data = mydata)
\end{verbatim}

means that the observations can be correlated within the clusters defined by the variable {\tt X3}, and that robust standard errors should be calculated according to those clusters.  If {\tt robust = TRUE} but {\tt cluster} is not specified, {\tt zelig()} assumes that each observation falls into its own cluster.

\item {\tt method}: defaults to {\tt "efron"}.  Use this argument to specify how to handle ties within event times.  The model assumes that no two event times should theoretically ever be the same, and any ties that occur are simply because the observation mechanism is not precise enough.  In practice, ties often exist in the data so the model commonly uses one of three methods to deal with ties. 

\begin{itemize}

\item \textbf{Breslow method} ({\tt method = "breslow"}): This method is the simplest computationally but also the least precise, especially as the number of tied events increases.  

%Because it is impossible to tell which of the tied events occurred first, the Breslow method simply treats the risk set as the same for all tied events in the risk set.  Suppose observations 1 and 3 are tied in a risk set of observations 1, 2, 3, and 4.  Theoretically, if the event occurred in 1 before in 3, then the risk set for observation 3 would have dropped observation 1.  However, since we cannot tell which event occurred first, in the partial likelihood, the risk set for observation 1 and observation 3 are the same, consisting of both observations 1 and 3 as well as 2 and 4.  For each risk set $R_j$, let $d_j$ equal the number of tied events in the $j$th risk set and let $D_j$ denote the set of $d_j$ tied events.  The approximate partial likelihood for the Breslow method is given by
%\begin{equation*}
%L(\beta | y) = \prod_{j=1}^J \frac{\prod_{i \in D_j}\exp(x_i \beta)}{\left[ \sum_{k \in R_j} \exp(x_k \beta)\right] ^{d_j}}
%\end{equation*}
%where $j$ indexes the $J$ number of risk sets, $k$ denotes the observations in the risk sets, and $i$ denotes the observations for which the events occur .  

\item \textbf{Efron method} ({\tt method = "efron"}): This is the default method and is more intensive computationally but also more precise than the Breslow method.

%The Efron method is more precise because it tries to account for how the risk set changes depending on the sequence of tied events.  For an intuition behind the Efron approximation, suppose as in the previous example that observations 1 and 3 are tied in a risk set of observations 1, 2, 3, and 4.  If the event occurred in 1 before 3, then the risk set for the second event would consist of observations $\{2, 3, 4\}$.  On the other hand, if the event occurred in 3 before 1, then the risk set for the second event would consist of observations $\{1,2,4\}$.  Since both cases are equally plausible with the tied event times, the Efron approximation suggests that the second risk set would consist of $\{2, 3, 4\}$ with $0.5$ probability and $\{1,2,4\}$ with $0.5$ probability.  The Efron approximate partial likelihood is then given by
%\begin{equation*}
%L(\beta | y) = \prod_{j=1}^J \left(\frac{\prod_{i \in D_j}\exp(x_i \beta)}{\prod_{r=1}^{d_j} \left[\sum_{k \in R_j} \exp(x_k \beta) - \frac{r-1}{d_j} \sum_{k \in D_j} \exp(x_k \beta)\right] } \right)  
%\end{equation*}
%where $r$ indexes $D_j$, which is the set of $d_j$ tied events for the $j$th risk set, and all other subscripts are the same as above.  In the example above, $D_j = \{1, 3\}$ with size $d_j = 2$ and observation 3 is the $r=2$nd observation in $D_j$.

\item \textbf{Exact discrete method} ({\tt method = "exact"}): This is the preferred method if the number of distinct events is rather small due to a large number of ties.  Although it can be very computationally intensive, the exact discrete method, which computes the exact partial likelihood, is the most precise method when there are many ties.  

%Unlike the Breslow and Efron methods, which assume a continuous time process, the exact discrete method assumes a discrete time process where the tied events actually do occur at exactly the same time.  The method begins by assuming that the data are grouped into risk sets $R_j$.  In each risk set, denote a binary dependent variable for each observation that takes on the value of 1 for each observation that experiences the event and 0 for each observation that does not experience the event.  Denote $d_j$ as the number of 1s in $R_j$ and $D_j$ the set of observations with 1s in $R_j$.  $D_j$ represents a specific pattern of 0s and 1s (in our previous example, the specific pattern of 0s and 1s is that observations 1 and 3 experienced an event while 2 and 4 did not, so $D_j$ represents the specific pattern of $\{1,3\}$).  Then for each $R_j$, we are interested in the conditional probability of getting the specific pattern of 0s and 1s given the total number of 1s in $R_j$.  Thus, the conditional probability for each risk set is given as
%\begin{eqnarray*}
%\mathrm{Pr}(D_j | d_j) = \frac{\prod_{i \in D_j} \exp(x_i \beta)}{\sum_{m=1}^M \left[\prod_{k \in A_{jm}}\exp(x_k \beta)\right] }
%\end{eqnarray*}  
%where $A_{jm}$ is a set of observations that represents one combination of $d_j$ number of 1s in $R_j$.  There are $M$ possible combinations for each risk set.  For example, in the case where there are two events among a risk set of four observations $\{1,2,3,4\}$, there are 6 combinations of choosing 2 events out of 4 observations.  Therefore, $M = 6$ and $A_{j1} = \{1,2\}, A_{j2} = \{1,3\}, A_{j3} = \{1,4\} ,A_{j4} = \{2,3\}, A_{j5} = \{2,4\}, A_{j6} = \{3,4\}$.  The partial likelihood then takes the conditional probability over each $j$ risk set.  Note that the exact discrete approximation method is equivalent to a conditional logit model.
\end{itemize}
\end{itemize}

\subsubsection{Stratified Cox Model}

In addition, {\tt zelig()} also supports the stratified Cox model, where the baseline hazards are assumed to be different across different strata but the coefficients are restricted to be the same across strata.  Let {\tt id} be a variable that consists of either discrete numeric values, character strings, or factors that define the strata.  Then the stratified Cox model can be estimated using {\tt strata()} in the formula.  The user can then find quantities of interest for a specific stratum by defining the stratum of choice in {\tt setx()}.  If no strata are defined, {\tt setx} takes the mode.  Strata on {\tt setx} are defined as followed:

\begin{itemize}
\item If strata were defined by a variable ({\tt strata(id)}), then strata should be defined as {\tt strata = "id=5"}.
\item If strata were defined by a mathematical expression ({\tt strata(id>10)}), then strata should be defined as {\tt strata = "id>10=TRUE"} or {\tt strata = "id>10=FALSE"}.
\end{itemize}

\begin{verbatim}
> z.out <- zelig(Surv(Y,C) ~ X1 + X2 + strata(id), model = "coxph", 
                 data = mydata)
> x.out <- setx(z.out, strata = "id=5")
> s.out <- sim(z.out, x = x.out)
\end{verbatim}


\subsubsection{Time-Varying Covariates}

{\tt zelig()} also supports the use of time-varying covariates for the Cox model, where some or all of the covariates change over time for each case.  Let ``case'' refer to each unit in the data.  Then each case can have one or more ``observations'', where each observation has a different value for one or more covariates for a specific case.  \\ 

\noindent Estimating a time-varying covariate model with {\tt zelig()} involves setting up the data differently to reflect a counting process.  In the typical non-time-varying covariate model, the cases include a duration time ($Y$), a censoring mechanism ($C$), and covariates ($X$).  A typical dataset would look like this:

\begin{table}[!htp]
 \begin{center}
\begin{tabular}{lllll}
\hline
Case & Y & C & X1 & X2 \\ 
\hline
1 & 35 & 0 & 4 & 7 \\ 
2 & 56 & 1 & 6 & 11\\
\hline
\end{tabular}
 \end{center}
\end{table}

\noindent The user would then estimate the model and find quantities of interest using the following syntax:

\begin{verbatim}
> z.out <- zelig(Surv(Y,C) ~ X1 + X2, model = "coxph", data = mydata)
> x.out <- setx(z.out)
> s.out <- sim(z.out, x = x.out)
\end{verbatim}

\noindent With time-varying covariates, each case is composed of multiple observations with start times, stop times, censoring (event) mechanisms, and covariates.  The covariates are assumed to be constant within the intervals defined by the start and stop times.  The covariates change only between intervals.  Thus, the covariates are constant at each observation.  The censoring mechanism equals 1 when an event occurs at the stop time and equals 0 if the observation is censored or if no event occurs at the stop time.  A typical time-varying dataset would look like this:

\begin{table}[!htp]
 \begin{center}
\begin{tabular}{llllll}
\hline
Case & Start & Stop & C & X1 & X2 \\ 
\hline
1 & 0 & 26 & 0 & 4 & 7 \\
1 & 26 & 35 & 0 & 4 & 10 \\
2 & 0 & 39 & 0 & 6 & 11 \\ 
2 & 39 & 56 & 1 & 9 & 5 \\
\hline
\end{tabular}
 \end{center}
\end{table}

\noindent The user would then estimate the model and find quantities of interest using the following syntax:

\begin{verbatim}
> z.out <- zelig(Surv(Start,Stop,C) ~ X1 + X2, model = "coxph", 
                 data = mydata)
> x.out <- setx(z.out)
> s.out <- sim(z.out, x = x.out)
\end{verbatim}


\subsubsection{Examples}
\begin{enumerate}
\item {Example 1: Basic Example}

Attaching the sample dataset:
\begin{Schunk}
\begin{Sinput}
RRR> data(coalition)
\end{Sinput}
\end{Schunk}
Estimating parameter values for the coxph regression:
\begin{Schunk}
\begin{Sinput}
RRR> z.out1 <- zelig(Surv(duration, ciep12) ~ invest + numst2 + crisis, robust = TRUE, cluster = "polar", model = "coxph", data = coalition)
\end{Sinput}
\end{Schunk}
Setting values for the explanatory variables:
\begin{Schunk}
\begin{Sinput}
RRR> x.low1<- setx(z.out1, numst2 = 0)
RRR> x.high1 <- setx(z.out1, numst2 = 1)
\end{Sinput}
\end{Schunk}
Simulating quantities of interest:
\begin{Schunk}
\begin{Sinput}
RRR> s.out1 <- sim(z.out1, x = x.low1, x1 = x.high1)
\end{Sinput}
\end{Schunk}
\begin{Schunk}
\begin{Sinput}
RRR> summary(s.out1)
\end{Sinput}
\end{Schunk}
\begin{center}
\begin{Schunk}
\begin{Sinput}
RRR> plot(s.out1)
\end{Sinput}
\end{Schunk}
\includegraphics{vigpics/coxph-ExamplePlot}
\end{center}

\item {Example 2: Example with Stratified Cox Model}

Estimating parameter values for the stratified coxph regression:
\begin{Schunk}
\begin{Sinput}
RRR> z.out2 <- zelig(Surv(duration, ciep12) ~ invest + strata(polar) + numst2 + crisis, model = "coxph", data = coalition)
\end{Sinput}
\end{Schunk}
Setting values for the explanatory variables:
\begin{Schunk}
\begin{Sinput}
RRR> x.low2 <- setx(z.out2, numst2 = 0, strata = "polar=3")
RRR> x.high2 <- setx(z.out2, numst2 = 1, strata = "polar=3")
\end{Sinput}
\end{Schunk}
Simulating quantities of interest:
\begin{Schunk}
\begin{Sinput}
RRR> s.out2 <- sim(z.out2, x = x.low2, x1 = x.high2)
\end{Sinput}
\end{Schunk}
\begin{Schunk}
\begin{Sinput}
RRR> summary(s.out2)
\end{Sinput}
\end{Schunk}
\begin{center}
\begin{Schunk}
\begin{Sinput}
RRR> plot(s.out2)
\end{Sinput}
\end{Schunk}
\includegraphics{vigpics/coxph-ExamplePlot2}
\end{center}

\item {Example 3: Example with Time-Varying Covariates}

Create sample toy dataset (from {\tt survival} package):
\begin{Schunk}
\begin{Sinput}
RRR> toy <- as.data.frame(list(start=c(1, 2, 5, 2, 1, 7, 3, 4, 8, 8),
+             stop=c(2, 3, 6, 7, 8, 9, 9, 9,14,17),
+             event=c(1, 1, 1, 1, 1, 1, 1, 0, 0, 0),
+             x=c(1, 0, 0, 1, 0, 1, 1, 1, 0, 0),
+ 	    x1=c(5, 5, 7, 4, 5, 6, 3, 2, 7, 4) ))
\end{Sinput}
\end{Schunk}
Estimating parameter values for the coxph regression:
\begin{Schunk}
\begin{Sinput}
RRR> z.out3 <- zelig(Surv(start, stop, event) ~ x + x1, model = "coxph", data = toy)
\end{Sinput}
\end{Schunk}
Setting values for the explanatory variables:
\begin{Schunk}
\begin{Sinput}
RRR> x.low3 <- setx(z.out3, x = 0)
RRR> x.high3 <- setx(z.out3, x = 1)
\end{Sinput}
\end{Schunk}
Simulating quantities of interest:
\begin{Schunk}
\begin{Sinput}
RRR> s.out3 <- sim(z.out3, x = x.low3, x1 = x.high3)
\end{Sinput}
\end{Schunk}
\begin{Schunk}
\begin{Sinput}
RRR> summary(s.out3)
\end{Sinput}
\end{Schunk}
\begin{center}
\begin{Schunk}
\begin{Sinput}
RRR> plot(s.out3)
\end{Sinput}
\end{Schunk}
\includegraphics{vigpics/coxph-ExamplePlot3}
\end{center}

\end{enumerate}

\subsubsection{The Model}
Let $Y_i^*$ be the survival time for observation $i$. This variable
might be censored for some observations at a fixed time $y_c$ such
that the fully observed dependent variable, $Y_i$, is defined as
\begin{equation*}
  Y_i = \left\{ \begin{array}{ll}
      Y_i^* & \textrm{if }Y_i^* \leq y_c \\
      y_c & \textrm{if }Y_i^* > y_c
    \end{array} \right.
\end{equation*}

\begin{itemize}
\item The \textit{stochastic component} is described by the distribution of the partially observed variable $Y^*$:
\begin{equation*}
Y_i^* \sim f(y_i^* | \mu_i, \alpha)
\end{equation*}
where $f$ is an unspecified distribution with some mean $\mu_i$ and shape $\alpha$.  In the Cox proportional hazards model, the distributional form of the duration times is unknown and left unparameterized.  Instead it uses the proportional hazards assumption to model the set of (ordered) event times on particular covariates.

An important component of all survival models is the hazard function $h(t)$, which measures the probability of an observation not surviving past time $t$ given survival up to $t$.  The hazard function is given by 
\begin{equation*}
h_i(t) = \lambda(t) \times \lambda_i
\end{equation*}
where $\lambda(t)$ is the baseline hazard (when all covariates equal 0), which varies over $t$ but not over $i$, and $\lambda_i$ is the parameterized part of the hazard function, which varies over $i$ but not over $t$ (the proportional hazards assumption).

The model estimates the parameters without a distributional assumption on the duration times by focusing on the occurrence of events and ignoring the time between events.  The data are reconceptualized from duration times to $K$ discrete event times such that each $y_i$ corresponds to exactly one event time $t_i$.  The model assumes that no two $y_i$ have the same event times.  

For each event time, denote $R(t_i)$ as the set of all observations $j$ that are at risk at $t_i$.  Given that an event occurred at $t_i$, we are interested in the conditional probability that the event occurred in observation $i$.  The conditional probability is given by 
\begin{eqnarray*}
\mathrm{Pr}(y_i = t_i \mid \mathrm{an \: event \: at \:}t_i) &=&  \frac{h_i(t_i)}{\sum_{j \in R(t_i)} h_j(t_i)}\\
&=& \frac{\lambda(t_i) \, \lambda_i}{\sum_{j \in R(t_i)} \lambda(t_i) \, \lambda_j}\\
&=& \frac{\lambda_i}{\sum_{j \in R(t_i)} \lambda_j}
\end{eqnarray*}
where the numerator denotes the probability of observation $i$ experiencing the event at $t_i$ and the denominator denotes the probability that an event occurred at $t_i$.  
%
%The model estimates the parameters without a distributional assumption on the duration times by focusing on the occurrence of events and ignoring the time between events.  The data are first sorted by $n$ discrete event times (indexed by $j$) such that $t_1 < t_2 < ... < t_n$, where $n$ corresponds to both the number of events and the number of uncensored observations.  The model assumes that there are no ties among the $n$ discrete event times.  See Additional Inputs for methods to deal with ties.  
%
%Each $y_i^*$ corresponds to one event time $t_j$ at which the event occurs for $i$ (for example, if the 5th observation experienced the shortest duration of 2, then $y_5^* = t_1 = 2$).  For each $j$ event time, denote $R_j$ as the set of all observations $k$ that are at risk at time $t_j$.  Then for each $t_j$, we are interested in the conditional probability of the event occurring in the \textbf{corresponding} observation $i$ given that an event occurred at $t_j$.  Note that for each event time, we are only interested in the conditional probability for the corresponding $i$ rather than for all $i$.  The conditional probability is given by
%\begin{eqnarray*}
%\mathrm{Pr}(y_i^* = t_j \mid \mathrm{1 \: event \: at \:}t_j) &=&  \frac{h_i(t_j)}{\sum_{k \in R_j} h_k(t_j)}\\
%&=& \frac{\lambda(t_j) \, \lambda_i}{\sum_{k \in R_j} \lambda(t_j) \, \lambda_k}\\
%&=& \frac{\lambda_i}{\sum_{k \in R_j} \lambda_k}
%\end{eqnarray*}
%where the numerator denotes the probability of observation $i$ experiencing the event at $t_j$ and the denominator denotes the probability that an event occurred at $t_j$.  Observations that are censored contribute to the denominator for event times prior to its censoring (as part of the risk set) but not the numerator since censored observations do not have corresponding event times.  Each $j$ event time contributes one conditional probability to the partial likelihood function, which is then maximized to give an estimate of the parameters.  For an example, see \citet[53]{BoxJon04}.  

\item The \textit{systematic component} $\lambda_i$ is modeled as
\begin{equation*}
\lambda_i = \exp(x_i \beta)
\end{equation*}
where $x_i$ is the vector of explanatory variables, and $\beta$ is the vector of coefficients.

\item Each risk set (and thus each event time) contributes one conditional probability to the partial likelihood function, given by 
\begin{equation*}
L(\beta | y) = \prod_{i=1}^K \left[ \frac{\exp(x_i \beta)}{\sum_{j \in R(t_i)} \exp(x_j \beta)} \right] ^{c_i}
\end{equation*}  
where $c_i$ is the binary censoring variable.  Note that event times corresponding to censored observations are not counted since their corresponding terms for the partial likelihood are exponentiated to 0.  However, all censored observations are considered part of the risk sets $R(t_i)$ for all event times prior to their censoring, but otherwise do not contribute to the partial likelihood.  For an example, see \citet[53]{BoxJon04}.  

\item In the case of the Cox model with time-varying covariates, the partial likelihood function is similarly given by  
\begin{equation*}
L(\beta | y) = \prod_{i=1}^K \left[ \frac{\exp(x_i(t_i) \; \beta)}{\sum_{j \in R(t_i)} \exp(x_j(t_i) \; \beta)} \right] ^{c_i}
\end{equation*}  
where $x_i(t_i)$ is the value of the covariates at time $t_i$.  Denote ``cases'' as the units in our data.  Each case is composed of one or more observations corresponding to different values in one or more covariates.  At each event time $t_i$, the partial likelihood evaluates the hazard of the case in which the event occurred in with its covariate values at $t_i$ (the numerator) and the hazard of all the other cases at risk at $t_i$ (risk set $R(t_i)$) with their covariate values at $t_i$ (the denominator).  See previous section for more information. 


\item Although the model assumes that there are no tied event times, in practice, data often have tied event times due to imprecise measurement.  There are three commonly used methods to deal with tied event times.  
\begin{itemize}
\item \textbf{Breslow method}: The Breslow method simply treats the risk set as the same for all tied events in the risk set.  Suppose observations 1 and 3 are tied in a risk set of observations 1, 2, 3, and 4.  Theoretically, if the event occurred in 1 before in 3, then the risk set for observation 3 would have dropped observation 1.  However, since we cannot tell which event occurred first, in the partial likelihood, the risk set for observation 1 and observation 3 are the same, consisting of both observations 1 and 3 as well as 2 and 4.  For each risk set $R(t_i)$, let $d_i$ equal the number of tied events in the $i$th risk set and let $D_i$ denote the set of $d_i$ tied events.  For risk sets with no tied events, $d_i = 1$.  The approximate partial likelihood for the Breslow method is given by
\begin{equation*}
L(\beta | y) = \prod_{i=1}^K \left[ \frac{\prod_{i \in D_i}\exp(x_i \beta)}{\left[ \sum_{j \in R(t_i)} \exp(x_j \beta)\right] ^{d_i}} \right] ^ {c_i}
\end{equation*} 

\item \textbf{Efron method}: The Efron method is more precise because it tries to account for how the risk set changes depending on the sequence of tied events.  For an intuition behind the Efron approximation, suppose as in the previous example that observations 1 and 3 are tied in a risk set of observations 1, 2, 3, and 4.  If the event occurred in 1 before 3, then the risk set for the second event would consist of observations $\{2, 3, 4\}$.  On the other hand, if the event occurred in 3 before 1, then the risk set for the second event would consist of observations $\{1,2,4\}$.  Since both cases are equally plausible with the tied event times, the Efron approximation suggests that the second risk set would consist of $\{2, 3, 4\}$ with $0.5$ probability and $\{1,2,4\}$ with $0.5$ probability.  The Efron approximate partial likelihood is then given by
\begin{equation*}
L(\beta | y) = \prod_{i=1}^K \left[\frac{\prod_{i \in D_i}\exp(x_i \beta)}{\prod_{r=1}^{d_i} \left[\sum_{j \in R(t_i)} \exp(x_j \beta) - \frac{r-1}{d_i} \sum_{j \in D_i} \exp(x_j \beta)\right] } \right] ^ {c_i}  
\end{equation*}
where $r$ indexes $D_i$, which is the set of $d_i$ tied events for the $i$th risk set.

\item \textbf{Exact discrete method}: Unlike the Breslow and Efron methods, which assume a continuous time process, the exact discrete method assumes a discrete time process where the tied events actually do occur at exactly the same time.  The method begins by assuming that the data are grouped into risk sets $R(t_i)$.  In each risk set and for each observation, denote a binary dependent variable which takes on the value of 1 for each observation that experiences the event and 0 for each observation that does not experience the event.  Denote $d_i$ as the number of 1s in $R(t_i)$ and $D_i$ as the set of observations with 1s in $R(t_i)$.  $D_i$ represents a specific pattern of 0s and 1s (in our previous example, the specific pattern of 0s and 1s is that observations 1 and 3 experienced an event while 2 and 4 did not, so $D_i$ is the set $\{1,3\}$).  Then for each $R(t_i)$, we are interested in the conditional probability of getting the specific pattern of 0s and 1s given the total number of 1s in $R(t_i)$.  Thus, the conditional probability for each risk set is given as
\begin{eqnarray*}
\mathrm{Pr}(D_i | d_i) = \frac{\prod_{i \in D_i} \exp(x_i \beta)}{\sum_{m=1}^M \left[\prod_{j \in A_{im}}\exp(x_j \beta)\right] }
\end{eqnarray*}  
where $A_{im}$ is a set of observations that represents one combination of $d_i$ number of 1s in $R(t_i)$.  There are $M$ possible combinations for each risk set.  The partial likelihood then takes the conditional probability over each $i$ risk set.  Note that the exact discrete approximation method is equivalent to a conditional logit model.
\end{itemize}
\end{itemize}

\subsubsection{Quantities of Interest}

\begin{itemize}

\item The hazard ratio ({\tt qi\$hr}) is defined as 
\begin{equation*}
\textrm{HR} = \frac{h(t \mid x_1)}{h(t \mid x)} = \frac{\lambda(t) \exp(x_1 \beta)}{\lambda(t) \exp(x \beta)} =  \frac{\exp(x_1 \beta)}{\exp(x \beta)}
\end{equation*}
given draws of $\beta$ from its sampling distribution, where $x$ and $x_1$ are values of the independent variables chosen by the user.  Typically, $x$ and $x_1$ should only differ over one independent variable to interpret the effect of that variable on the hazard rate.  In a stratified Cox model, the strata should be the same in both $x$ and $x_1$.

\item The survival function ({\tt qi\$survival}) is defined as the fraction of observations surviving past time $t$.  It is derived from the cumulative hazard function ({\tt exp(-cumhaz)}).  The confidence interval of the survival function is drawn on the {\tt log(survival)} scale.

\item The cumulative hazard function ({\tt qi\$cumhaz}) is defined as {\tt -log(survival)}.  Although there is no direct interpretation, the cumulative hazard function is estimated from the data and then other quantities of interest are derived from the cumulative hazard function.

\item The hazard function ({\tt qi\$hazard}) is defined as the probability of an observation not surviving past time $t$ given survival up to $t$.  It is derived directly from the cumulative hazard function.

\item For MI data, if survival times are multiply imputed, we suggest having a larger number of imputed datasets.  Because the quantities of interest are derived semi-parametrically, there may be instances in which survival times appear only in one or a small fraction of the multiply imputed datasets, which may bias the results.  
\end{itemize}

\subsubsection{Output Values}

The output of each Zelig command contains useful information which you
may view.  For example, if you run \texttt{z.out <- zelig(Surv(y,c) \~\, x,
  model = "coxph", data)}, then you may examine the available
information in \texttt{z.out} by using \texttt{names(z.out)},
see the {\tt coefficients} by using {\tt z.out\$coefficients}, and
a default summary of information through \texttt{summary(z.out)}.
Other elements available through the {\tt \$} operator are listed
below.

\begin{itemize}
\item From the {\tt zelig()} output object {\tt z.out}, you may
  extract:
   \begin{itemize}
   \item {\tt coefficients}: parameter estimates for the explanatory
     variables.
   \item {\tt var}: the variance-covariance matrix.
   \item {\tt residuals}: the working residuals of the fit.
   \item {\tt loglik}: the log-likelihood for the baseline and full models
   \item {\tt linear.predictors}: a mean-adjusted linear predictor $x_i \beta$, where $x_i = x_i - \mathrm{mean}(x)$.
   \end{itemize}

\item From {\tt summary(z.out)}, you may extract:
   \begin{itemize}
   \item {\tt coef}: the parameter estimates with their
     associated standard errors, $p$-values, and $z$-statistics.
   \item {\tt conf.int}: $\exp(\beta)$ and their associated confidence intervals.
   \end{itemize}

\item From the {\tt sim()} output object {\tt s.out}, you may extract
  quantities of interest arranged as matrices indexed by simulation
  $\times$ {\tt x}-observation (for more than one {\tt x}-observation).
  Available quantities are:

   \begin{itemize}
   \item {\tt qi\$hr}: the simulated hazard ratios for the
     specified values of {\tt x} and {\tt x1}.
   \item {\tt qi\$survival}: the estimated survival function for the values specified in {\tt x}.
   \item {\tt qi\$cumhaz}: the estimated cumulative hazard function for the values specified in {\tt x}.
   \item {\tt qi\$hazard}: the estimated hazard function for the values specified in {\tt x}.
   \end{itemize}
\end{itemize}

\subsection*{How To Cite}

%\section{\texttt{coxph}: Cox Proportional Hazards Regression for Duration Dependent Variables}
\label{coxph}

Choose the Cox proportional hazards regression model if the values in your dependent
variable are duration observations.  The advantage of the semi-parametric Cox proportional hazards model over fully parametric models such as the exponential or Weibull models is that it makes no assumptions about the shape of the baseline hazard.  The model only requires the proportional hazards assumption that the baseline hazard does not vary across observations.  The baseline hazard can be estimated from the model via post-hoc analysis.

\subsubsection{Syntax}

\begin{verbatim}
> z.out <- zelig(Surv(Y, C) ~ X1 + X2, model = "coxph", data = mydata)
> x.out <- setx(z.out)
> s.out <- sim(z.out, x = x.out)
\end{verbatim}
Cox proportional hazards models require that the dependent variable be in the form {\tt
  Surv(Y, C)}, where {\tt Y} and {\tt C} are vectors of length $n$.
For each observation $i$ in 1, \dots, $n$, the value $y_i$ is the
duration (lifetime, for example), and the associated $c_i$ is a binary
variable such that $c_i = 1$ if the duration is not censored ({\it
  e.g.}, the subject dies during the study) or $c_i = 0$ if the
duration is censored ({\it e.g.}, the subject is still alive at the
end of the study).  If $c_i$ is omitted, all Y are assumed to be
completed; that is, $c_i$ defaults to 1 for all observations.


\subsubsection{Additional Inputs}

In addition to the standard inputs, {\tt zelig()} takes the following additional options for Cox proportional hazards regression:

\begin{itemize}
\item {\tt robust}: defaults to {\tt FALSE}.  If {\tt TRUE, zelig()} computes robust standard errors based on sandwich estimators (see \citet{Huber81} and \citet{White80}) based on the options in {\tt cluster}.

\item {\tt cluster}: if {\tt robust = TRUE}, you may select a variable to define groups of correlated observations.  Let {\tt X3} be a variable that consists of either discrete numeric values, character strings, or factors that define the clusters.  Then 

\begin{verbatim}
> z.out <- zelig(Surv(Y,C) ~ X1 + X2, robust = TRUE, cluster = "X3", 
                 model = "coxph", data = mydata)
\end{verbatim}

means that the observations can be correlated within the clusters defined by the variable {\tt X3}, and that robust standard errors should be calculated according to those clusters.  If {\tt robust = TRUE} but {\tt cluster} is not specified, {\tt zelig()} assumes that each observation falls into its own cluster.

\item {\tt method}: defaults to {\tt "efron"}.  Use this argument to specify how to handle ties within event times.  The model assumes that no two event times should theoretically ever be the same, and any ties that occur are simply because the observation mechanism is not precise enough.  In practice, ties often exist in the data so the model commonly uses one of three methods to deal with ties. 

\begin{itemize}

\item \textbf{Breslow method} ({\tt method = "breslow"}): This method is the simplest computationally but also the least precise, especially as the number of tied events increases.  

%Because it is impossible to tell which of the tied events occurred first, the Breslow method simply treats the risk set as the same for all tied events in the risk set.  Suppose observations 1 and 3 are tied in a risk set of observations 1, 2, 3, and 4.  Theoretically, if the event occurred in 1 before in 3, then the risk set for observation 3 would have dropped observation 1.  However, since we cannot tell which event occurred first, in the partial likelihood, the risk set for observation 1 and observation 3 are the same, consisting of both observations 1 and 3 as well as 2 and 4.  For each risk set $R_j$, let $d_j$ equal the number of tied events in the $j$th risk set and let $D_j$ denote the set of $d_j$ tied events.  The approximate partial likelihood for the Breslow method is given by
%\begin{equation*}
%L(\beta | y) = \prod_{j=1}^J \frac{\prod_{i \in D_j}\exp(x_i \beta)}{\left[ \sum_{k \in R_j} \exp(x_k \beta)\right] ^{d_j}}
%\end{equation*}
%where $j$ indexes the $J$ number of risk sets, $k$ denotes the observations in the risk sets, and $i$ denotes the observations for which the events occur .  

\item \textbf{Efron method} ({\tt method = "efron"}): This is the default method and is more intensive computationally but also more precise than the Breslow method.

%The Efron method is more precise because it tries to account for how the risk set changes depending on the sequence of tied events.  For an intuition behind the Efron approximation, suppose as in the previous example that observations 1 and 3 are tied in a risk set of observations 1, 2, 3, and 4.  If the event occurred in 1 before 3, then the risk set for the second event would consist of observations $\{2, 3, 4\}$.  On the other hand, if the event occurred in 3 before 1, then the risk set for the second event would consist of observations $\{1,2,4\}$.  Since both cases are equally plausible with the tied event times, the Efron approximation suggests that the second risk set would consist of $\{2, 3, 4\}$ with $0.5$ probability and $\{1,2,4\}$ with $0.5$ probability.  The Efron approximate partial likelihood is then given by
%\begin{equation*}
%L(\beta | y) = \prod_{j=1}^J \left(\frac{\prod_{i \in D_j}\exp(x_i \beta)}{\prod_{r=1}^{d_j} \left[\sum_{k \in R_j} \exp(x_k \beta) - \frac{r-1}{d_j} \sum_{k \in D_j} \exp(x_k \beta)\right] } \right)  
%\end{equation*}
%where $r$ indexes $D_j$, which is the set of $d_j$ tied events for the $j$th risk set, and all other subscripts are the same as above.  In the example above, $D_j = \{1, 3\}$ with size $d_j = 2$ and observation 3 is the $r=2$nd observation in $D_j$.

\item \textbf{Exact discrete method} ({\tt method = "exact"}): This is the preferred method if the number of distinct events is rather small due to a large number of ties.  Although it can be very computationally intensive, the exact discrete method, which computes the exact partial likelihood, is the most precise method when there are many ties.  

%Unlike the Breslow and Efron methods, which assume a continuous time process, the exact discrete method assumes a discrete time process where the tied events actually do occur at exactly the same time.  The method begins by assuming that the data are grouped into risk sets $R_j$.  In each risk set, denote a binary dependent variable for each observation that takes on the value of 1 for each observation that experiences the event and 0 for each observation that does not experience the event.  Denote $d_j$ as the number of 1s in $R_j$ and $D_j$ the set of observations with 1s in $R_j$.  $D_j$ represents a specific pattern of 0s and 1s (in our previous example, the specific pattern of 0s and 1s is that observations 1 and 3 experienced an event while 2 and 4 did not, so $D_j$ represents the specific pattern of $\{1,3\}$).  Then for each $R_j$, we are interested in the conditional probability of getting the specific pattern of 0s and 1s given the total number of 1s in $R_j$.  Thus, the conditional probability for each risk set is given as
%\begin{eqnarray*}
%\mathrm{Pr}(D_j | d_j) = \frac{\prod_{i \in D_j} \exp(x_i \beta)}{\sum_{m=1}^M \left[\prod_{k \in A_{jm}}\exp(x_k \beta)\right] }
%\end{eqnarray*}  
%where $A_{jm}$ is a set of observations that represents one combination of $d_j$ number of 1s in $R_j$.  There are $M$ possible combinations for each risk set.  For example, in the case where there are two events among a risk set of four observations $\{1,2,3,4\}$, there are 6 combinations of choosing 2 events out of 4 observations.  Therefore, $M = 6$ and $A_{j1} = \{1,2\}, A_{j2} = \{1,3\}, A_{j3} = \{1,4\} ,A_{j4} = \{2,3\}, A_{j5} = \{2,4\}, A_{j6} = \{3,4\}$.  The partial likelihood then takes the conditional probability over each $j$ risk set.  Note that the exact discrete approximation method is equivalent to a conditional logit model.
\end{itemize}
\end{itemize}

\subsubsection{Stratified Cox Model}

In addition, {\tt zelig()} also supports the stratified Cox model, where the baseline hazards are assumed to be different across different strata but the coefficients are restricted to be the same across strata.  Let {\tt id} be a variable that consists of either discrete numeric values, character strings, or factors that define the strata.  Then the stratified Cox model can be estimated using {\tt strata()} in the formula.  The user can then find quantities of interest for a specific stratum by defining the stratum of choice in {\tt setx()}.  If no strata are defined, {\tt setx} takes the mode.  Strata on {\tt setx} are defined as followed:

\begin{itemize}
\item If strata were defined by a variable ({\tt strata(id)}), then strata should be defined as {\tt strata = "id=5"}.
\item If strata were defined by a mathematical expression ({\tt strata(id>10)}), then strata should be defined as {\tt strata = "id>10=TRUE"} or {\tt strata = "id>10=FALSE"}.
\end{itemize}

\begin{verbatim}
> z.out <- zelig(Surv(Y,C) ~ X1 + X2 + strata(id), model = "coxph", 
                 data = mydata)
> x.out <- setx(z.out, strata = "id=5")
> s.out <- sim(z.out, x = x.out)
\end{verbatim}


\subsubsection{Time-Varying Covariates}

{\tt zelig()} also supports the use of time-varying covariates for the Cox model, where some or all of the covariates change over time for each case.  Let ``case'' refer to each unit in the data.  Then each case can have one or more ``observations'', where each observation has a different value for one or more covariates for a specific case.  \\ 

\noindent Estimating a time-varying covariate model with {\tt zelig()} involves setting up the data differently to reflect a counting process.  In the typical non-time-varying covariate model, the cases include a duration time ($Y$), a censoring mechanism ($C$), and covariates ($X$).  A typical dataset would look like this:

\begin{table}[!htp]
 \begin{center}
\begin{tabular}{lllll}
\hline
Case & Y & C & X1 & X2 \\ 
\hline
1 & 35 & 0 & 4 & 7 \\ 
2 & 56 & 1 & 6 & 11\\
\hline
\end{tabular}
 \end{center}
\end{table}

\noindent The user would then estimate the model and find quantities of interest using the following syntax:

\begin{verbatim}
> z.out <- zelig(Surv(Y,C) ~ X1 + X2, model = "coxph", data = mydata)
> x.out <- setx(z.out)
> s.out <- sim(z.out, x = x.out)
\end{verbatim}

\noindent With time-varying covariates, each case is composed of multiple observations with start times, stop times, censoring (event) mechanisms, and covariates.  The covariates are assumed to be constant within the intervals defined by the start and stop times.  The covariates change only between intervals.  Thus, the covariates are constant at each observation.  The censoring mechanism equals 1 when an event occurs at the stop time and equals 0 if the observation is censored or if no event occurs at the stop time.  A typical time-varying dataset would look like this:

\begin{table}[!htp]
 \begin{center}
\begin{tabular}{llllll}
\hline
Case & Start & Stop & C & X1 & X2 \\ 
\hline
1 & 0 & 26 & 0 & 4 & 7 \\
1 & 26 & 35 & 0 & 4 & 10 \\
2 & 0 & 39 & 0 & 6 & 11 \\ 
2 & 39 & 56 & 1 & 9 & 5 \\
\hline
\end{tabular}
 \end{center}
\end{table}

\noindent The user would then estimate the model and find quantities of interest using the following syntax:

\begin{verbatim}
> z.out <- zelig(Surv(Start,Stop,C) ~ X1 + X2, model = "coxph", 
                 data = mydata)
> x.out <- setx(z.out)
> s.out <- sim(z.out, x = x.out)
\end{verbatim}


\subsubsection{Examples}
\begin{enumerate}
\item {Example 1: Basic Example}

Attaching the sample dataset:
\begin{Schunk}
\begin{Sinput}
RRR> data(coalition)
\end{Sinput}
\end{Schunk}
Estimating parameter values for the coxph regression:
\begin{Schunk}
\begin{Sinput}
RRR> z.out1 <- zelig(Surv(duration, ciep12) ~ invest + numst2 + crisis, robust = TRUE, cluster = "polar", model = "coxph", data = coalition)
\end{Sinput}
\end{Schunk}
Setting values for the explanatory variables:
\begin{Schunk}
\begin{Sinput}
RRR> x.low1<- setx(z.out1, numst2 = 0)
RRR> x.high1 <- setx(z.out1, numst2 = 1)
\end{Sinput}
\end{Schunk}
Simulating quantities of interest:
\begin{Schunk}
\begin{Sinput}
RRR> s.out1 <- sim(z.out1, x = x.low1, x1 = x.high1)
\end{Sinput}
\end{Schunk}
\begin{Schunk}
\begin{Sinput}
RRR> summary(s.out1)
\end{Sinput}
\end{Schunk}
\begin{center}
\begin{Schunk}
\begin{Sinput}
RRR> plot(s.out1)
\end{Sinput}
\end{Schunk}
\includegraphics{vigpics/coxph-ExamplePlot}
\end{center}

\item {Example 2: Example with Stratified Cox Model}

Estimating parameter values for the stratified coxph regression:
\begin{Schunk}
\begin{Sinput}
RRR> z.out2 <- zelig(Surv(duration, ciep12) ~ invest + strata(polar) + numst2 + crisis, model = "coxph", data = coalition)
\end{Sinput}
\end{Schunk}
Setting values for the explanatory variables:
\begin{Schunk}
\begin{Sinput}
RRR> x.low2 <- setx(z.out2, numst2 = 0, strata = "polar=3")
RRR> x.high2 <- setx(z.out2, numst2 = 1, strata = "polar=3")
\end{Sinput}
\end{Schunk}
Simulating quantities of interest:
\begin{Schunk}
\begin{Sinput}
RRR> s.out2 <- sim(z.out2, x = x.low2, x1 = x.high2)
\end{Sinput}
\end{Schunk}
\begin{Schunk}
\begin{Sinput}
RRR> summary(s.out2)
\end{Sinput}
\end{Schunk}
\begin{center}
\begin{Schunk}
\begin{Sinput}
RRR> plot(s.out2)
\end{Sinput}
\end{Schunk}
\includegraphics{vigpics/coxph-ExamplePlot2}
\end{center}

\item {Example 3: Example with Time-Varying Covariates}

Create sample toy dataset (from {\tt survival} package):
\begin{Schunk}
\begin{Sinput}
RRR> toy <- as.data.frame(list(start=c(1, 2, 5, 2, 1, 7, 3, 4, 8, 8),
+             stop=c(2, 3, 6, 7, 8, 9, 9, 9,14,17),
+             event=c(1, 1, 1, 1, 1, 1, 1, 0, 0, 0),
+             x=c(1, 0, 0, 1, 0, 1, 1, 1, 0, 0),
+ 	    x1=c(5, 5, 7, 4, 5, 6, 3, 2, 7, 4) ))
\end{Sinput}
\end{Schunk}
Estimating parameter values for the coxph regression:
\begin{Schunk}
\begin{Sinput}
RRR> z.out3 <- zelig(Surv(start, stop, event) ~ x + x1, model = "coxph", data = toy)
\end{Sinput}
\end{Schunk}
Setting values for the explanatory variables:
\begin{Schunk}
\begin{Sinput}
RRR> x.low3 <- setx(z.out3, x = 0)
RRR> x.high3 <- setx(z.out3, x = 1)
\end{Sinput}
\end{Schunk}
Simulating quantities of interest:
\begin{Schunk}
\begin{Sinput}
RRR> s.out3 <- sim(z.out3, x = x.low3, x1 = x.high3)
\end{Sinput}
\end{Schunk}
\begin{Schunk}
\begin{Sinput}
RRR> summary(s.out3)
\end{Sinput}
\end{Schunk}
\begin{center}
\begin{Schunk}
\begin{Sinput}
RRR> plot(s.out3)
\end{Sinput}
\end{Schunk}
\includegraphics{vigpics/coxph-ExamplePlot3}
\end{center}

\end{enumerate}

\subsubsection{The Model}
Let $Y_i^*$ be the survival time for observation $i$. This variable
might be censored for some observations at a fixed time $y_c$ such
that the fully observed dependent variable, $Y_i$, is defined as
\begin{equation*}
  Y_i = \left\{ \begin{array}{ll}
      Y_i^* & \textrm{if }Y_i^* \leq y_c \\
      y_c & \textrm{if }Y_i^* > y_c
    \end{array} \right.
\end{equation*}

\begin{itemize}
\item The \textit{stochastic component} is described by the distribution of the partially observed variable $Y^*$:
\begin{equation*}
Y_i^* \sim f(y_i^* | \mu_i, \alpha)
\end{equation*}
where $f$ is an unspecified distribution with some mean $\mu_i$ and shape $\alpha$.  In the Cox proportional hazards model, the distributional form of the duration times is unknown and left unparameterized.  Instead it uses the proportional hazards assumption to model the set of (ordered) event times on particular covariates.

An important component of all survival models is the hazard function $h(t)$, which measures the probability of an observation not surviving past time $t$ given survival up to $t$.  The hazard function is given by 
\begin{equation*}
h_i(t) = \lambda(t) \times \lambda_i
\end{equation*}
where $\lambda(t)$ is the baseline hazard (when all covariates equal 0), which varies over $t$ but not over $i$, and $\lambda_i$ is the parameterized part of the hazard function, which varies over $i$ but not over $t$ (the proportional hazards assumption).

The model estimates the parameters without a distributional assumption on the duration times by focusing on the occurrence of events and ignoring the time between events.  The data are reconceptualized from duration times to $K$ discrete event times such that each $y_i$ corresponds to exactly one event time $t_i$.  The model assumes that no two $y_i$ have the same event times.  

For each event time, denote $R(t_i)$ as the set of all observations $j$ that are at risk at $t_i$.  Given that an event occurred at $t_i$, we are interested in the conditional probability that the event occurred in observation $i$.  The conditional probability is given by 
\begin{eqnarray*}
\mathrm{Pr}(y_i = t_i \mid \mathrm{an \: event \: at \:}t_i) &=&  \frac{h_i(t_i)}{\sum_{j \in R(t_i)} h_j(t_i)}\\
&=& \frac{\lambda(t_i) \, \lambda_i}{\sum_{j \in R(t_i)} \lambda(t_i) \, \lambda_j}\\
&=& \frac{\lambda_i}{\sum_{j \in R(t_i)} \lambda_j}
\end{eqnarray*}
where the numerator denotes the probability of observation $i$ experiencing the event at $t_i$ and the denominator denotes the probability that an event occurred at $t_i$.  
%
%The model estimates the parameters without a distributional assumption on the duration times by focusing on the occurrence of events and ignoring the time between events.  The data are first sorted by $n$ discrete event times (indexed by $j$) such that $t_1 < t_2 < ... < t_n$, where $n$ corresponds to both the number of events and the number of uncensored observations.  The model assumes that there are no ties among the $n$ discrete event times.  See Additional Inputs for methods to deal with ties.  
%
%Each $y_i^*$ corresponds to one event time $t_j$ at which the event occurs for $i$ (for example, if the 5th observation experienced the shortest duration of 2, then $y_5^* = t_1 = 2$).  For each $j$ event time, denote $R_j$ as the set of all observations $k$ that are at risk at time $t_j$.  Then for each $t_j$, we are interested in the conditional probability of the event occurring in the \textbf{corresponding} observation $i$ given that an event occurred at $t_j$.  Note that for each event time, we are only interested in the conditional probability for the corresponding $i$ rather than for all $i$.  The conditional probability is given by
%\begin{eqnarray*}
%\mathrm{Pr}(y_i^* = t_j \mid \mathrm{1 \: event \: at \:}t_j) &=&  \frac{h_i(t_j)}{\sum_{k \in R_j} h_k(t_j)}\\
%&=& \frac{\lambda(t_j) \, \lambda_i}{\sum_{k \in R_j} \lambda(t_j) \, \lambda_k}\\
%&=& \frac{\lambda_i}{\sum_{k \in R_j} \lambda_k}
%\end{eqnarray*}
%where the numerator denotes the probability of observation $i$ experiencing the event at $t_j$ and the denominator denotes the probability that an event occurred at $t_j$.  Observations that are censored contribute to the denominator for event times prior to its censoring (as part of the risk set) but not the numerator since censored observations do not have corresponding event times.  Each $j$ event time contributes one conditional probability to the partial likelihood function, which is then maximized to give an estimate of the parameters.  For an example, see \citet[53]{BoxJon04}.  

\item The \textit{systematic component} $\lambda_i$ is modeled as
\begin{equation*}
\lambda_i = \exp(x_i \beta)
\end{equation*}
where $x_i$ is the vector of explanatory variables, and $\beta$ is the vector of coefficients.

\item Each risk set (and thus each event time) contributes one conditional probability to the partial likelihood function, given by 
\begin{equation*}
L(\beta | y) = \prod_{i=1}^K \left[ \frac{\exp(x_i \beta)}{\sum_{j \in R(t_i)} \exp(x_j \beta)} \right] ^{c_i}
\end{equation*}  
where $c_i$ is the binary censoring variable.  Note that event times corresponding to censored observations are not counted since their corresponding terms for the partial likelihood are exponentiated to 0.  However, all censored observations are considered part of the risk sets $R(t_i)$ for all event times prior to their censoring, but otherwise do not contribute to the partial likelihood.  For an example, see \citet[53]{BoxJon04}.  

\item In the case of the Cox model with time-varying covariates, the partial likelihood function is similarly given by  
\begin{equation*}
L(\beta | y) = \prod_{i=1}^K \left[ \frac{\exp(x_i(t_i) \; \beta)}{\sum_{j \in R(t_i)} \exp(x_j(t_i) \; \beta)} \right] ^{c_i}
\end{equation*}  
where $x_i(t_i)$ is the value of the covariates at time $t_i$.  Denote ``cases'' as the units in our data.  Each case is composed of one or more observations corresponding to different values in one or more covariates.  At each event time $t_i$, the partial likelihood evaluates the hazard of the case in which the event occurred in with its covariate values at $t_i$ (the numerator) and the hazard of all the other cases at risk at $t_i$ (risk set $R(t_i)$) with their covariate values at $t_i$ (the denominator).  See previous section for more information. 


\item Although the model assumes that there are no tied event times, in practice, data often have tied event times due to imprecise measurement.  There are three commonly used methods to deal with tied event times.  
\begin{itemize}
\item \textbf{Breslow method}: The Breslow method simply treats the risk set as the same for all tied events in the risk set.  Suppose observations 1 and 3 are tied in a risk set of observations 1, 2, 3, and 4.  Theoretically, if the event occurred in 1 before in 3, then the risk set for observation 3 would have dropped observation 1.  However, since we cannot tell which event occurred first, in the partial likelihood, the risk set for observation 1 and observation 3 are the same, consisting of both observations 1 and 3 as well as 2 and 4.  For each risk set $R(t_i)$, let $d_i$ equal the number of tied events in the $i$th risk set and let $D_i$ denote the set of $d_i$ tied events.  For risk sets with no tied events, $d_i = 1$.  The approximate partial likelihood for the Breslow method is given by
\begin{equation*}
L(\beta | y) = \prod_{i=1}^K \left[ \frac{\prod_{i \in D_i}\exp(x_i \beta)}{\left[ \sum_{j \in R(t_i)} \exp(x_j \beta)\right] ^{d_i}} \right] ^ {c_i}
\end{equation*} 

\item \textbf{Efron method}: The Efron method is more precise because it tries to account for how the risk set changes depending on the sequence of tied events.  For an intuition behind the Efron approximation, suppose as in the previous example that observations 1 and 3 are tied in a risk set of observations 1, 2, 3, and 4.  If the event occurred in 1 before 3, then the risk set for the second event would consist of observations $\{2, 3, 4\}$.  On the other hand, if the event occurred in 3 before 1, then the risk set for the second event would consist of observations $\{1,2,4\}$.  Since both cases are equally plausible with the tied event times, the Efron approximation suggests that the second risk set would consist of $\{2, 3, 4\}$ with $0.5$ probability and $\{1,2,4\}$ with $0.5$ probability.  The Efron approximate partial likelihood is then given by
\begin{equation*}
L(\beta | y) = \prod_{i=1}^K \left[\frac{\prod_{i \in D_i}\exp(x_i \beta)}{\prod_{r=1}^{d_i} \left[\sum_{j \in R(t_i)} \exp(x_j \beta) - \frac{r-1}{d_i} \sum_{j \in D_i} \exp(x_j \beta)\right] } \right] ^ {c_i}  
\end{equation*}
where $r$ indexes $D_i$, which is the set of $d_i$ tied events for the $i$th risk set.

\item \textbf{Exact discrete method}: Unlike the Breslow and Efron methods, which assume a continuous time process, the exact discrete method assumes a discrete time process where the tied events actually do occur at exactly the same time.  The method begins by assuming that the data are grouped into risk sets $R(t_i)$.  In each risk set and for each observation, denote a binary dependent variable which takes on the value of 1 for each observation that experiences the event and 0 for each observation that does not experience the event.  Denote $d_i$ as the number of 1s in $R(t_i)$ and $D_i$ as the set of observations with 1s in $R(t_i)$.  $D_i$ represents a specific pattern of 0s and 1s (in our previous example, the specific pattern of 0s and 1s is that observations 1 and 3 experienced an event while 2 and 4 did not, so $D_i$ is the set $\{1,3\}$).  Then for each $R(t_i)$, we are interested in the conditional probability of getting the specific pattern of 0s and 1s given the total number of 1s in $R(t_i)$.  Thus, the conditional probability for each risk set is given as
\begin{eqnarray*}
\mathrm{Pr}(D_i | d_i) = \frac{\prod_{i \in D_i} \exp(x_i \beta)}{\sum_{m=1}^M \left[\prod_{j \in A_{im}}\exp(x_j \beta)\right] }
\end{eqnarray*}  
where $A_{im}$ is a set of observations that represents one combination of $d_i$ number of 1s in $R(t_i)$.  There are $M$ possible combinations for each risk set.  The partial likelihood then takes the conditional probability over each $i$ risk set.  Note that the exact discrete approximation method is equivalent to a conditional logit model.
\end{itemize}
\end{itemize}

\subsubsection{Quantities of Interest}

\begin{itemize}

\item The hazard ratio ({\tt qi\$hr}) is defined as 
\begin{equation*}
\textrm{HR} = \frac{h(t \mid x_1)}{h(t \mid x)} = \frac{\lambda(t) \exp(x_1 \beta)}{\lambda(t) \exp(x \beta)} =  \frac{\exp(x_1 \beta)}{\exp(x \beta)}
\end{equation*}
given draws of $\beta$ from its sampling distribution, where $x$ and $x_1$ are values of the independent variables chosen by the user.  Typically, $x$ and $x_1$ should only differ over one independent variable to interpret the effect of that variable on the hazard rate.  In a stratified Cox model, the strata should be the same in both $x$ and $x_1$.

\item The survival function ({\tt qi\$survival}) is defined as the fraction of observations surviving past time $t$.  It is derived from the cumulative hazard function ({\tt exp(-cumhaz)}).  The confidence interval of the survival function is drawn on the {\tt log(survival)} scale.

\item The cumulative hazard function ({\tt qi\$cumhaz}) is defined as {\tt -log(survival)}.  Although there is no direct interpretation, the cumulative hazard function is estimated from the data and then other quantities of interest are derived from the cumulative hazard function.

\item The hazard function ({\tt qi\$hazard}) is defined as the probability of an observation not surviving past time $t$ given survival up to $t$.  It is derived directly from the cumulative hazard function.

\item For MI data, if survival times are multiply imputed, we suggest having a larger number of imputed datasets.  Because the quantities of interest are derived semi-parametrically, there may be instances in which survival times appear only in one or a small fraction of the multiply imputed datasets, which may bias the results.  
\end{itemize}

\subsubsection{Output Values}

The output of each Zelig command contains useful information which you
may view.  For example, if you run \texttt{z.out <- zelig(Surv(y,c) \~\, x,
  model = "coxph", data)}, then you may examine the available
information in \texttt{z.out} by using \texttt{names(z.out)},
see the {\tt coefficients} by using {\tt z.out\$coefficients}, and
a default summary of information through \texttt{summary(z.out)}.
Other elements available through the {\tt \$} operator are listed
below.

\begin{itemize}
\item From the {\tt zelig()} output object {\tt z.out}, you may
  extract:
   \begin{itemize}
   \item {\tt coefficients}: parameter estimates for the explanatory
     variables.
   \item {\tt var}: the variance-covariance matrix.
   \item {\tt residuals}: the working residuals of the fit.
   \item {\tt loglik}: the log-likelihood for the baseline and full models
   \item {\tt linear.predictors}: a mean-adjusted linear predictor $x_i \beta$, where $x_i = x_i - \mathrm{mean}(x)$.
   \end{itemize}

\item From {\tt summary(z.out)}, you may extract:
   \begin{itemize}
   \item {\tt coef}: the parameter estimates with their
     associated standard errors, $p$-values, and $z$-statistics.
   \item {\tt conf.int}: $\exp(\beta)$ and their associated confidence intervals.
   \end{itemize}

\item From the {\tt sim()} output object {\tt s.out}, you may extract
  quantities of interest arranged as matrices indexed by simulation
  $\times$ {\tt x}-observation (for more than one {\tt x}-observation).
  Available quantities are:

   \begin{itemize}
   \item {\tt qi\$hr}: the simulated hazard ratios for the
     specified values of {\tt x} and {\tt x1}.
   \item {\tt qi\$survival}: the estimated survival function for the values specified in {\tt x}.
   \item {\tt qi\$cumhaz}: the estimated cumulative hazard function for the values specified in {\tt x}.
   \item {\tt qi\$hazard}: the estimated hazard function for the values specified in {\tt x}.
   \end{itemize}
\end{itemize}

\subsection*{How To Cite}

%\section{\texttt{coxph}: Cox Proportional Hazards Regression for Duration Dependent Variables}
\label{coxph}

Choose the Cox proportional hazards regression model if the values in your dependent
variable are duration observations.  The advantage of the semi-parametric Cox proportional hazards model over fully parametric models such as the exponential or Weibull models is that it makes no assumptions about the shape of the baseline hazard.  The model only requires the proportional hazards assumption that the baseline hazard does not vary across observations.  The baseline hazard can be estimated from the model via post-hoc analysis.

\subsubsection{Syntax}

\begin{verbatim}
> z.out <- zelig(Surv(Y, C) ~ X1 + X2, model = "coxph", data = mydata)
> x.out <- setx(z.out)
> s.out <- sim(z.out, x = x.out)
\end{verbatim}
Cox proportional hazards models require that the dependent variable be in the form {\tt
  Surv(Y, C)}, where {\tt Y} and {\tt C} are vectors of length $n$.
For each observation $i$ in 1, \dots, $n$, the value $y_i$ is the
duration (lifetime, for example), and the associated $c_i$ is a binary
variable such that $c_i = 1$ if the duration is not censored ({\it
  e.g.}, the subject dies during the study) or $c_i = 0$ if the
duration is censored ({\it e.g.}, the subject is still alive at the
end of the study).  If $c_i$ is omitted, all Y are assumed to be
completed; that is, $c_i$ defaults to 1 for all observations.


\subsubsection{Additional Inputs}

In addition to the standard inputs, {\tt zelig()} takes the following additional options for Cox proportional hazards regression:

\begin{itemize}
\item {\tt robust}: defaults to {\tt FALSE}.  If {\tt TRUE, zelig()} computes robust standard errors based on sandwich estimators (see \citet{Huber81} and \citet{White80}) based on the options in {\tt cluster}.

\item {\tt cluster}: if {\tt robust = TRUE}, you may select a variable to define groups of correlated observations.  Let {\tt X3} be a variable that consists of either discrete numeric values, character strings, or factors that define the clusters.  Then 

\begin{verbatim}
> z.out <- zelig(Surv(Y,C) ~ X1 + X2, robust = TRUE, cluster = "X3", 
                 model = "coxph", data = mydata)
\end{verbatim}

means that the observations can be correlated within the clusters defined by the variable {\tt X3}, and that robust standard errors should be calculated according to those clusters.  If {\tt robust = TRUE} but {\tt cluster} is not specified, {\tt zelig()} assumes that each observation falls into its own cluster.

\item {\tt method}: defaults to {\tt "efron"}.  Use this argument to specify how to handle ties within event times.  The model assumes that no two event times should theoretically ever be the same, and any ties that occur are simply because the observation mechanism is not precise enough.  In practice, ties often exist in the data so the model commonly uses one of three methods to deal with ties. 

\begin{itemize}

\item \textbf{Breslow method} ({\tt method = "breslow"}): This method is the simplest computationally but also the least precise, especially as the number of tied events increases.  

%Because it is impossible to tell which of the tied events occurred first, the Breslow method simply treats the risk set as the same for all tied events in the risk set.  Suppose observations 1 and 3 are tied in a risk set of observations 1, 2, 3, and 4.  Theoretically, if the event occurred in 1 before in 3, then the risk set for observation 3 would have dropped observation 1.  However, since we cannot tell which event occurred first, in the partial likelihood, the risk set for observation 1 and observation 3 are the same, consisting of both observations 1 and 3 as well as 2 and 4.  For each risk set $R_j$, let $d_j$ equal the number of tied events in the $j$th risk set and let $D_j$ denote the set of $d_j$ tied events.  The approximate partial likelihood for the Breslow method is given by
%\begin{equation*}
%L(\beta | y) = \prod_{j=1}^J \frac{\prod_{i \in D_j}\exp(x_i \beta)}{\left[ \sum_{k \in R_j} \exp(x_k \beta)\right] ^{d_j}}
%\end{equation*}
%where $j$ indexes the $J$ number of risk sets, $k$ denotes the observations in the risk sets, and $i$ denotes the observations for which the events occur .  

\item \textbf{Efron method} ({\tt method = "efron"}): This is the default method and is more intensive computationally but also more precise than the Breslow method.

%The Efron method is more precise because it tries to account for how the risk set changes depending on the sequence of tied events.  For an intuition behind the Efron approximation, suppose as in the previous example that observations 1 and 3 are tied in a risk set of observations 1, 2, 3, and 4.  If the event occurred in 1 before 3, then the risk set for the second event would consist of observations $\{2, 3, 4\}$.  On the other hand, if the event occurred in 3 before 1, then the risk set for the second event would consist of observations $\{1,2,4\}$.  Since both cases are equally plausible with the tied event times, the Efron approximation suggests that the second risk set would consist of $\{2, 3, 4\}$ with $0.5$ probability and $\{1,2,4\}$ with $0.5$ probability.  The Efron approximate partial likelihood is then given by
%\begin{equation*}
%L(\beta | y) = \prod_{j=1}^J \left(\frac{\prod_{i \in D_j}\exp(x_i \beta)}{\prod_{r=1}^{d_j} \left[\sum_{k \in R_j} \exp(x_k \beta) - \frac{r-1}{d_j} \sum_{k \in D_j} \exp(x_k \beta)\right] } \right)  
%\end{equation*}
%where $r$ indexes $D_j$, which is the set of $d_j$ tied events for the $j$th risk set, and all other subscripts are the same as above.  In the example above, $D_j = \{1, 3\}$ with size $d_j = 2$ and observation 3 is the $r=2$nd observation in $D_j$.

\item \textbf{Exact discrete method} ({\tt method = "exact"}): This is the preferred method if the number of distinct events is rather small due to a large number of ties.  Although it can be very computationally intensive, the exact discrete method, which computes the exact partial likelihood, is the most precise method when there are many ties.  

%Unlike the Breslow and Efron methods, which assume a continuous time process, the exact discrete method assumes a discrete time process where the tied events actually do occur at exactly the same time.  The method begins by assuming that the data are grouped into risk sets $R_j$.  In each risk set, denote a binary dependent variable for each observation that takes on the value of 1 for each observation that experiences the event and 0 for each observation that does not experience the event.  Denote $d_j$ as the number of 1s in $R_j$ and $D_j$ the set of observations with 1s in $R_j$.  $D_j$ represents a specific pattern of 0s and 1s (in our previous example, the specific pattern of 0s and 1s is that observations 1 and 3 experienced an event while 2 and 4 did not, so $D_j$ represents the specific pattern of $\{1,3\}$).  Then for each $R_j$, we are interested in the conditional probability of getting the specific pattern of 0s and 1s given the total number of 1s in $R_j$.  Thus, the conditional probability for each risk set is given as
%\begin{eqnarray*}
%\mathrm{Pr}(D_j | d_j) = \frac{\prod_{i \in D_j} \exp(x_i \beta)}{\sum_{m=1}^M \left[\prod_{k \in A_{jm}}\exp(x_k \beta)\right] }
%\end{eqnarray*}  
%where $A_{jm}$ is a set of observations that represents one combination of $d_j$ number of 1s in $R_j$.  There are $M$ possible combinations for each risk set.  For example, in the case where there are two events among a risk set of four observations $\{1,2,3,4\}$, there are 6 combinations of choosing 2 events out of 4 observations.  Therefore, $M = 6$ and $A_{j1} = \{1,2\}, A_{j2} = \{1,3\}, A_{j3} = \{1,4\} ,A_{j4} = \{2,3\}, A_{j5} = \{2,4\}, A_{j6} = \{3,4\}$.  The partial likelihood then takes the conditional probability over each $j$ risk set.  Note that the exact discrete approximation method is equivalent to a conditional logit model.
\end{itemize}
\end{itemize}

\subsubsection{Stratified Cox Model}

In addition, {\tt zelig()} also supports the stratified Cox model, where the baseline hazards are assumed to be different across different strata but the coefficients are restricted to be the same across strata.  Let {\tt id} be a variable that consists of either discrete numeric values, character strings, or factors that define the strata.  Then the stratified Cox model can be estimated using {\tt strata()} in the formula.  The user can then find quantities of interest for a specific stratum by defining the stratum of choice in {\tt setx()}.  If no strata are defined, {\tt setx} takes the mode.  Strata on {\tt setx} are defined as followed:

\begin{itemize}
\item If strata were defined by a variable ({\tt strata(id)}), then strata should be defined as {\tt strata = "id=5"}.
\item If strata were defined by a mathematical expression ({\tt strata(id>10)}), then strata should be defined as {\tt strata = "id>10=TRUE"} or {\tt strata = "id>10=FALSE"}.
\end{itemize}

\begin{verbatim}
> z.out <- zelig(Surv(Y,C) ~ X1 + X2 + strata(id), model = "coxph", 
                 data = mydata)
> x.out <- setx(z.out, strata = "id=5")
> s.out <- sim(z.out, x = x.out)
\end{verbatim}


\subsubsection{Time-Varying Covariates}

{\tt zelig()} also supports the use of time-varying covariates for the Cox model, where some or all of the covariates change over time for each case.  Let ``case'' refer to each unit in the data.  Then each case can have one or more ``observations'', where each observation has a different value for one or more covariates for a specific case.  \\ 

\noindent Estimating a time-varying covariate model with {\tt zelig()} involves setting up the data differently to reflect a counting process.  In the typical non-time-varying covariate model, the cases include a duration time ($Y$), a censoring mechanism ($C$), and covariates ($X$).  A typical dataset would look like this:

\begin{table}[!htp]
 \begin{center}
\begin{tabular}{lllll}
\hline
Case & Y & C & X1 & X2 \\ 
\hline
1 & 35 & 0 & 4 & 7 \\ 
2 & 56 & 1 & 6 & 11\\
\hline
\end{tabular}
 \end{center}
\end{table}

\noindent The user would then estimate the model and find quantities of interest using the following syntax:

\begin{verbatim}
> z.out <- zelig(Surv(Y,C) ~ X1 + X2, model = "coxph", data = mydata)
> x.out <- setx(z.out)
> s.out <- sim(z.out, x = x.out)
\end{verbatim}

\noindent With time-varying covariates, each case is composed of multiple observations with start times, stop times, censoring (event) mechanisms, and covariates.  The covariates are assumed to be constant within the intervals defined by the start and stop times.  The covariates change only between intervals.  Thus, the covariates are constant at each observation.  The censoring mechanism equals 1 when an event occurs at the stop time and equals 0 if the observation is censored or if no event occurs at the stop time.  A typical time-varying dataset would look like this:

\begin{table}[!htp]
 \begin{center}
\begin{tabular}{llllll}
\hline
Case & Start & Stop & C & X1 & X2 \\ 
\hline
1 & 0 & 26 & 0 & 4 & 7 \\
1 & 26 & 35 & 0 & 4 & 10 \\
2 & 0 & 39 & 0 & 6 & 11 \\ 
2 & 39 & 56 & 1 & 9 & 5 \\
\hline
\end{tabular}
 \end{center}
\end{table}

\noindent The user would then estimate the model and find quantities of interest using the following syntax:

\begin{verbatim}
> z.out <- zelig(Surv(Start,Stop,C) ~ X1 + X2, model = "coxph", 
                 data = mydata)
> x.out <- setx(z.out)
> s.out <- sim(z.out, x = x.out)
\end{verbatim}


\subsubsection{Examples}
\begin{enumerate}
\item {Example 1: Basic Example}

Attaching the sample dataset:
\begin{Schunk}
\begin{Sinput}
RRR> data(coalition)
\end{Sinput}
\end{Schunk}
Estimating parameter values for the coxph regression:
\begin{Schunk}
\begin{Sinput}
RRR> z.out1 <- zelig(Surv(duration, ciep12) ~ invest + numst2 + crisis, robust = TRUE, cluster = "polar", model = "coxph", data = coalition)
\end{Sinput}
\end{Schunk}
Setting values for the explanatory variables:
\begin{Schunk}
\begin{Sinput}
RRR> x.low1<- setx(z.out1, numst2 = 0)
RRR> x.high1 <- setx(z.out1, numst2 = 1)
\end{Sinput}
\end{Schunk}
Simulating quantities of interest:
\begin{Schunk}
\begin{Sinput}
RRR> s.out1 <- sim(z.out1, x = x.low1, x1 = x.high1)
\end{Sinput}
\end{Schunk}
\begin{Schunk}
\begin{Sinput}
RRR> summary(s.out1)
\end{Sinput}
\end{Schunk}
\begin{center}
\begin{Schunk}
\begin{Sinput}
RRR> plot(s.out1)
\end{Sinput}
\end{Schunk}
\includegraphics{vigpics/coxph-ExamplePlot}
\end{center}

\item {Example 2: Example with Stratified Cox Model}

Estimating parameter values for the stratified coxph regression:
\begin{Schunk}
\begin{Sinput}
RRR> z.out2 <- zelig(Surv(duration, ciep12) ~ invest + strata(polar) + numst2 + crisis, model = "coxph", data = coalition)
\end{Sinput}
\end{Schunk}
Setting values for the explanatory variables:
\begin{Schunk}
\begin{Sinput}
RRR> x.low2 <- setx(z.out2, numst2 = 0, strata = "polar=3")
RRR> x.high2 <- setx(z.out2, numst2 = 1, strata = "polar=3")
\end{Sinput}
\end{Schunk}
Simulating quantities of interest:
\begin{Schunk}
\begin{Sinput}
RRR> s.out2 <- sim(z.out2, x = x.low2, x1 = x.high2)
\end{Sinput}
\end{Schunk}
\begin{Schunk}
\begin{Sinput}
RRR> summary(s.out2)
\end{Sinput}
\end{Schunk}
\begin{center}
\begin{Schunk}
\begin{Sinput}
RRR> plot(s.out2)
\end{Sinput}
\end{Schunk}
\includegraphics{vigpics/coxph-ExamplePlot2}
\end{center}

\item {Example 3: Example with Time-Varying Covariates}

Create sample toy dataset (from {\tt survival} package):
\begin{Schunk}
\begin{Sinput}
RRR> toy <- as.data.frame(list(start=c(1, 2, 5, 2, 1, 7, 3, 4, 8, 8),
+             stop=c(2, 3, 6, 7, 8, 9, 9, 9,14,17),
+             event=c(1, 1, 1, 1, 1, 1, 1, 0, 0, 0),
+             x=c(1, 0, 0, 1, 0, 1, 1, 1, 0, 0),
+ 	    x1=c(5, 5, 7, 4, 5, 6, 3, 2, 7, 4) ))
\end{Sinput}
\end{Schunk}
Estimating parameter values for the coxph regression:
\begin{Schunk}
\begin{Sinput}
RRR> z.out3 <- zelig(Surv(start, stop, event) ~ x + x1, model = "coxph", data = toy)
\end{Sinput}
\end{Schunk}
Setting values for the explanatory variables:
\begin{Schunk}
\begin{Sinput}
RRR> x.low3 <- setx(z.out3, x = 0)
RRR> x.high3 <- setx(z.out3, x = 1)
\end{Sinput}
\end{Schunk}
Simulating quantities of interest:
\begin{Schunk}
\begin{Sinput}
RRR> s.out3 <- sim(z.out3, x = x.low3, x1 = x.high3)
\end{Sinput}
\end{Schunk}
\begin{Schunk}
\begin{Sinput}
RRR> summary(s.out3)
\end{Sinput}
\end{Schunk}
\begin{center}
\begin{Schunk}
\begin{Sinput}
RRR> plot(s.out3)
\end{Sinput}
\end{Schunk}
\includegraphics{vigpics/coxph-ExamplePlot3}
\end{center}

\end{enumerate}

\subsubsection{The Model}
Let $Y_i^*$ be the survival time for observation $i$. This variable
might be censored for some observations at a fixed time $y_c$ such
that the fully observed dependent variable, $Y_i$, is defined as
\begin{equation*}
  Y_i = \left\{ \begin{array}{ll}
      Y_i^* & \textrm{if }Y_i^* \leq y_c \\
      y_c & \textrm{if }Y_i^* > y_c
    \end{array} \right.
\end{equation*}

\begin{itemize}
\item The \textit{stochastic component} is described by the distribution of the partially observed variable $Y^*$:
\begin{equation*}
Y_i^* \sim f(y_i^* | \mu_i, \alpha)
\end{equation*}
where $f$ is an unspecified distribution with some mean $\mu_i$ and shape $\alpha$.  In the Cox proportional hazards model, the distributional form of the duration times is unknown and left unparameterized.  Instead it uses the proportional hazards assumption to model the set of (ordered) event times on particular covariates.

An important component of all survival models is the hazard function $h(t)$, which measures the probability of an observation not surviving past time $t$ given survival up to $t$.  The hazard function is given by 
\begin{equation*}
h_i(t) = \lambda(t) \times \lambda_i
\end{equation*}
where $\lambda(t)$ is the baseline hazard (when all covariates equal 0), which varies over $t$ but not over $i$, and $\lambda_i$ is the parameterized part of the hazard function, which varies over $i$ but not over $t$ (the proportional hazards assumption).

The model estimates the parameters without a distributional assumption on the duration times by focusing on the occurrence of events and ignoring the time between events.  The data are reconceptualized from duration times to $K$ discrete event times such that each $y_i$ corresponds to exactly one event time $t_i$.  The model assumes that no two $y_i$ have the same event times.  

For each event time, denote $R(t_i)$ as the set of all observations $j$ that are at risk at $t_i$.  Given that an event occurred at $t_i$, we are interested in the conditional probability that the event occurred in observation $i$.  The conditional probability is given by 
\begin{eqnarray*}
\mathrm{Pr}(y_i = t_i \mid \mathrm{an \: event \: at \:}t_i) &=&  \frac{h_i(t_i)}{\sum_{j \in R(t_i)} h_j(t_i)}\\
&=& \frac{\lambda(t_i) \, \lambda_i}{\sum_{j \in R(t_i)} \lambda(t_i) \, \lambda_j}\\
&=& \frac{\lambda_i}{\sum_{j \in R(t_i)} \lambda_j}
\end{eqnarray*}
where the numerator denotes the probability of observation $i$ experiencing the event at $t_i$ and the denominator denotes the probability that an event occurred at $t_i$.  
%
%The model estimates the parameters without a distributional assumption on the duration times by focusing on the occurrence of events and ignoring the time between events.  The data are first sorted by $n$ discrete event times (indexed by $j$) such that $t_1 < t_2 < ... < t_n$, where $n$ corresponds to both the number of events and the number of uncensored observations.  The model assumes that there are no ties among the $n$ discrete event times.  See Additional Inputs for methods to deal with ties.  
%
%Each $y_i^*$ corresponds to one event time $t_j$ at which the event occurs for $i$ (for example, if the 5th observation experienced the shortest duration of 2, then $y_5^* = t_1 = 2$).  For each $j$ event time, denote $R_j$ as the set of all observations $k$ that are at risk at time $t_j$.  Then for each $t_j$, we are interested in the conditional probability of the event occurring in the \textbf{corresponding} observation $i$ given that an event occurred at $t_j$.  Note that for each event time, we are only interested in the conditional probability for the corresponding $i$ rather than for all $i$.  The conditional probability is given by
%\begin{eqnarray*}
%\mathrm{Pr}(y_i^* = t_j \mid \mathrm{1 \: event \: at \:}t_j) &=&  \frac{h_i(t_j)}{\sum_{k \in R_j} h_k(t_j)}\\
%&=& \frac{\lambda(t_j) \, \lambda_i}{\sum_{k \in R_j} \lambda(t_j) \, \lambda_k}\\
%&=& \frac{\lambda_i}{\sum_{k \in R_j} \lambda_k}
%\end{eqnarray*}
%where the numerator denotes the probability of observation $i$ experiencing the event at $t_j$ and the denominator denotes the probability that an event occurred at $t_j$.  Observations that are censored contribute to the denominator for event times prior to its censoring (as part of the risk set) but not the numerator since censored observations do not have corresponding event times.  Each $j$ event time contributes one conditional probability to the partial likelihood function, which is then maximized to give an estimate of the parameters.  For an example, see \citet[53]{BoxJon04}.  

\item The \textit{systematic component} $\lambda_i$ is modeled as
\begin{equation*}
\lambda_i = \exp(x_i \beta)
\end{equation*}
where $x_i$ is the vector of explanatory variables, and $\beta$ is the vector of coefficients.

\item Each risk set (and thus each event time) contributes one conditional probability to the partial likelihood function, given by 
\begin{equation*}
L(\beta | y) = \prod_{i=1}^K \left[ \frac{\exp(x_i \beta)}{\sum_{j \in R(t_i)} \exp(x_j \beta)} \right] ^{c_i}
\end{equation*}  
where $c_i$ is the binary censoring variable.  Note that event times corresponding to censored observations are not counted since their corresponding terms for the partial likelihood are exponentiated to 0.  However, all censored observations are considered part of the risk sets $R(t_i)$ for all event times prior to their censoring, but otherwise do not contribute to the partial likelihood.  For an example, see \citet[53]{BoxJon04}.  

\item In the case of the Cox model with time-varying covariates, the partial likelihood function is similarly given by  
\begin{equation*}
L(\beta | y) = \prod_{i=1}^K \left[ \frac{\exp(x_i(t_i) \; \beta)}{\sum_{j \in R(t_i)} \exp(x_j(t_i) \; \beta)} \right] ^{c_i}
\end{equation*}  
where $x_i(t_i)$ is the value of the covariates at time $t_i$.  Denote ``cases'' as the units in our data.  Each case is composed of one or more observations corresponding to different values in one or more covariates.  At each event time $t_i$, the partial likelihood evaluates the hazard of the case in which the event occurred in with its covariate values at $t_i$ (the numerator) and the hazard of all the other cases at risk at $t_i$ (risk set $R(t_i)$) with their covariate values at $t_i$ (the denominator).  See previous section for more information. 


\item Although the model assumes that there are no tied event times, in practice, data often have tied event times due to imprecise measurement.  There are three commonly used methods to deal with tied event times.  
\begin{itemize}
\item \textbf{Breslow method}: The Breslow method simply treats the risk set as the same for all tied events in the risk set.  Suppose observations 1 and 3 are tied in a risk set of observations 1, 2, 3, and 4.  Theoretically, if the event occurred in 1 before in 3, then the risk set for observation 3 would have dropped observation 1.  However, since we cannot tell which event occurred first, in the partial likelihood, the risk set for observation 1 and observation 3 are the same, consisting of both observations 1 and 3 as well as 2 and 4.  For each risk set $R(t_i)$, let $d_i$ equal the number of tied events in the $i$th risk set and let $D_i$ denote the set of $d_i$ tied events.  For risk sets with no tied events, $d_i = 1$.  The approximate partial likelihood for the Breslow method is given by
\begin{equation*}
L(\beta | y) = \prod_{i=1}^K \left[ \frac{\prod_{i \in D_i}\exp(x_i \beta)}{\left[ \sum_{j \in R(t_i)} \exp(x_j \beta)\right] ^{d_i}} \right] ^ {c_i}
\end{equation*} 

\item \textbf{Efron method}: The Efron method is more precise because it tries to account for how the risk set changes depending on the sequence of tied events.  For an intuition behind the Efron approximation, suppose as in the previous example that observations 1 and 3 are tied in a risk set of observations 1, 2, 3, and 4.  If the event occurred in 1 before 3, then the risk set for the second event would consist of observations $\{2, 3, 4\}$.  On the other hand, if the event occurred in 3 before 1, then the risk set for the second event would consist of observations $\{1,2,4\}$.  Since both cases are equally plausible with the tied event times, the Efron approximation suggests that the second risk set would consist of $\{2, 3, 4\}$ with $0.5$ probability and $\{1,2,4\}$ with $0.5$ probability.  The Efron approximate partial likelihood is then given by
\begin{equation*}
L(\beta | y) = \prod_{i=1}^K \left[\frac{\prod_{i \in D_i}\exp(x_i \beta)}{\prod_{r=1}^{d_i} \left[\sum_{j \in R(t_i)} \exp(x_j \beta) - \frac{r-1}{d_i} \sum_{j \in D_i} \exp(x_j \beta)\right] } \right] ^ {c_i}  
\end{equation*}
where $r$ indexes $D_i$, which is the set of $d_i$ tied events for the $i$th risk set.

\item \textbf{Exact discrete method}: Unlike the Breslow and Efron methods, which assume a continuous time process, the exact discrete method assumes a discrete time process where the tied events actually do occur at exactly the same time.  The method begins by assuming that the data are grouped into risk sets $R(t_i)$.  In each risk set and for each observation, denote a binary dependent variable which takes on the value of 1 for each observation that experiences the event and 0 for each observation that does not experience the event.  Denote $d_i$ as the number of 1s in $R(t_i)$ and $D_i$ as the set of observations with 1s in $R(t_i)$.  $D_i$ represents a specific pattern of 0s and 1s (in our previous example, the specific pattern of 0s and 1s is that observations 1 and 3 experienced an event while 2 and 4 did not, so $D_i$ is the set $\{1,3\}$).  Then for each $R(t_i)$, we are interested in the conditional probability of getting the specific pattern of 0s and 1s given the total number of 1s in $R(t_i)$.  Thus, the conditional probability for each risk set is given as
\begin{eqnarray*}
\mathrm{Pr}(D_i | d_i) = \frac{\prod_{i \in D_i} \exp(x_i \beta)}{\sum_{m=1}^M \left[\prod_{j \in A_{im}}\exp(x_j \beta)\right] }
\end{eqnarray*}  
where $A_{im}$ is a set of observations that represents one combination of $d_i$ number of 1s in $R(t_i)$.  There are $M$ possible combinations for each risk set.  The partial likelihood then takes the conditional probability over each $i$ risk set.  Note that the exact discrete approximation method is equivalent to a conditional logit model.
\end{itemize}
\end{itemize}

\subsubsection{Quantities of Interest}

\begin{itemize}

\item The hazard ratio ({\tt qi\$hr}) is defined as 
\begin{equation*}
\textrm{HR} = \frac{h(t \mid x_1)}{h(t \mid x)} = \frac{\lambda(t) \exp(x_1 \beta)}{\lambda(t) \exp(x \beta)} =  \frac{\exp(x_1 \beta)}{\exp(x \beta)}
\end{equation*}
given draws of $\beta$ from its sampling distribution, where $x$ and $x_1$ are values of the independent variables chosen by the user.  Typically, $x$ and $x_1$ should only differ over one independent variable to interpret the effect of that variable on the hazard rate.  In a stratified Cox model, the strata should be the same in both $x$ and $x_1$.

\item The survival function ({\tt qi\$survival}) is defined as the fraction of observations surviving past time $t$.  It is derived from the cumulative hazard function ({\tt exp(-cumhaz)}).  The confidence interval of the survival function is drawn on the {\tt log(survival)} scale.

\item The cumulative hazard function ({\tt qi\$cumhaz}) is defined as {\tt -log(survival)}.  Although there is no direct interpretation, the cumulative hazard function is estimated from the data and then other quantities of interest are derived from the cumulative hazard function.

\item The hazard function ({\tt qi\$hazard}) is defined as the probability of an observation not surviving past time $t$ given survival up to $t$.  It is derived directly from the cumulative hazard function.

\item For MI data, if survival times are multiply imputed, we suggest having a larger number of imputed datasets.  Because the quantities of interest are derived semi-parametrically, there may be instances in which survival times appear only in one or a small fraction of the multiply imputed datasets, which may bias the results.  
\end{itemize}

\subsubsection{Output Values}

The output of each Zelig command contains useful information which you
may view.  For example, if you run \texttt{z.out <- zelig(Surv(y,c) \~\, x,
  model = "coxph", data)}, then you may examine the available
information in \texttt{z.out} by using \texttt{names(z.out)},
see the {\tt coefficients} by using {\tt z.out\$coefficients}, and
a default summary of information through \texttt{summary(z.out)}.
Other elements available through the {\tt \$} operator are listed
below.

\begin{itemize}
\item From the {\tt zelig()} output object {\tt z.out}, you may
  extract:
   \begin{itemize}
   \item {\tt coefficients}: parameter estimates for the explanatory
     variables.
   \item {\tt var}: the variance-covariance matrix.
   \item {\tt residuals}: the working residuals of the fit.
   \item {\tt loglik}: the log-likelihood for the baseline and full models
   \item {\tt linear.predictors}: a mean-adjusted linear predictor $x_i \beta$, where $x_i = x_i - \mathrm{mean}(x)$.
   \end{itemize}

\item From {\tt summary(z.out)}, you may extract:
   \begin{itemize}
   \item {\tt coef}: the parameter estimates with their
     associated standard errors, $p$-values, and $z$-statistics.
   \item {\tt conf.int}: $\exp(\beta)$ and their associated confidence intervals.
   \end{itemize}

\item From the {\tt sim()} output object {\tt s.out}, you may extract
  quantities of interest arranged as matrices indexed by simulation
  $\times$ {\tt x}-observation (for more than one {\tt x}-observation).
  Available quantities are:

   \begin{itemize}
   \item {\tt qi\$hr}: the simulated hazard ratios for the
     specified values of {\tt x} and {\tt x1}.
   \item {\tt qi\$survival}: the estimated survival function for the values specified in {\tt x}.
   \item {\tt qi\$cumhaz}: the estimated cumulative hazard function for the values specified in {\tt x}.
   \item {\tt qi\$hazard}: the estimated hazard function for the values specified in {\tt x}.
   \end{itemize}
\end{itemize}

\subsection*{How To Cite}

%\section{\texttt{coxph}: Cox Proportional Hazards Regression for Duration Dependent Variables}
\label{coxph}

Choose the Cox proportional hazards regression model if the values in your dependent
variable are duration observations.  The advantage of the semi-parametric Cox proportional hazards model over fully parametric models such as the exponential or Weibull models is that it makes no assumptions about the shape of the baseline hazard.  The model only requires the proportional hazards assumption that the baseline hazard does not vary across observations.  The baseline hazard can be estimated from the model via post-hoc analysis.

\subsubsection{Syntax}

\begin{verbatim}
> z.out <- zelig(Surv(Y, C) ~ X1 + X2, model = "coxph", data = mydata)
> x.out <- setx(z.out)
> s.out <- sim(z.out, x = x.out)
\end{verbatim}
Cox proportional hazards models require that the dependent variable be in the form {\tt
  Surv(Y, C)}, where {\tt Y} and {\tt C} are vectors of length $n$.
For each observation $i$ in 1, \dots, $n$, the value $y_i$ is the
duration (lifetime, for example), and the associated $c_i$ is a binary
variable such that $c_i = 1$ if the duration is not censored ({\it
  e.g.}, the subject dies during the study) or $c_i = 0$ if the
duration is censored ({\it e.g.}, the subject is still alive at the
end of the study).  If $c_i$ is omitted, all Y are assumed to be
completed; that is, $c_i$ defaults to 1 for all observations.


\subsubsection{Additional Inputs}

In addition to the standard inputs, {\tt zelig()} takes the following additional options for Cox proportional hazards regression:

\begin{itemize}
\item {\tt robust}: defaults to {\tt FALSE}.  If {\tt TRUE, zelig()} computes robust standard errors based on sandwich estimators (see \citet{Huber81} and \citet{White80}) based on the options in {\tt cluster}.

\item {\tt cluster}: if {\tt robust = TRUE}, you may select a variable to define groups of correlated observations.  Let {\tt X3} be a variable that consists of either discrete numeric values, character strings, or factors that define the clusters.  Then 

\begin{verbatim}
> z.out <- zelig(Surv(Y,C) ~ X1 + X2, robust = TRUE, cluster = "X3", 
                 model = "coxph", data = mydata)
\end{verbatim}

means that the observations can be correlated within the clusters defined by the variable {\tt X3}, and that robust standard errors should be calculated according to those clusters.  If {\tt robust = TRUE} but {\tt cluster} is not specified, {\tt zelig()} assumes that each observation falls into its own cluster.

\item {\tt method}: defaults to {\tt "efron"}.  Use this argument to specify how to handle ties within event times.  The model assumes that no two event times should theoretically ever be the same, and any ties that occur are simply because the observation mechanism is not precise enough.  In practice, ties often exist in the data so the model commonly uses one of three methods to deal with ties. 

\begin{itemize}

\item \textbf{Breslow method} ({\tt method = "breslow"}): This method is the simplest computationally but also the least precise, especially as the number of tied events increases.  

%Because it is impossible to tell which of the tied events occurred first, the Breslow method simply treats the risk set as the same for all tied events in the risk set.  Suppose observations 1 and 3 are tied in a risk set of observations 1, 2, 3, and 4.  Theoretically, if the event occurred in 1 before in 3, then the risk set for observation 3 would have dropped observation 1.  However, since we cannot tell which event occurred first, in the partial likelihood, the risk set for observation 1 and observation 3 are the same, consisting of both observations 1 and 3 as well as 2 and 4.  For each risk set $R_j$, let $d_j$ equal the number of tied events in the $j$th risk set and let $D_j$ denote the set of $d_j$ tied events.  The approximate partial likelihood for the Breslow method is given by
%\begin{equation*}
%L(\beta | y) = \prod_{j=1}^J \frac{\prod_{i \in D_j}\exp(x_i \beta)}{\left[ \sum_{k \in R_j} \exp(x_k \beta)\right] ^{d_j}}
%\end{equation*}
%where $j$ indexes the $J$ number of risk sets, $k$ denotes the observations in the risk sets, and $i$ denotes the observations for which the events occur .  

\item \textbf{Efron method} ({\tt method = "efron"}): This is the default method and is more intensive computationally but also more precise than the Breslow method.

%The Efron method is more precise because it tries to account for how the risk set changes depending on the sequence of tied events.  For an intuition behind the Efron approximation, suppose as in the previous example that observations 1 and 3 are tied in a risk set of observations 1, 2, 3, and 4.  If the event occurred in 1 before 3, then the risk set for the second event would consist of observations $\{2, 3, 4\}$.  On the other hand, if the event occurred in 3 before 1, then the risk set for the second event would consist of observations $\{1,2,4\}$.  Since both cases are equally plausible with the tied event times, the Efron approximation suggests that the second risk set would consist of $\{2, 3, 4\}$ with $0.5$ probability and $\{1,2,4\}$ with $0.5$ probability.  The Efron approximate partial likelihood is then given by
%\begin{equation*}
%L(\beta | y) = \prod_{j=1}^J \left(\frac{\prod_{i \in D_j}\exp(x_i \beta)}{\prod_{r=1}^{d_j} \left[\sum_{k \in R_j} \exp(x_k \beta) - \frac{r-1}{d_j} \sum_{k \in D_j} \exp(x_k \beta)\right] } \right)  
%\end{equation*}
%where $r$ indexes $D_j$, which is the set of $d_j$ tied events for the $j$th risk set, and all other subscripts are the same as above.  In the example above, $D_j = \{1, 3\}$ with size $d_j = 2$ and observation 3 is the $r=2$nd observation in $D_j$.

\item \textbf{Exact discrete method} ({\tt method = "exact"}): This is the preferred method if the number of distinct events is rather small due to a large number of ties.  Although it can be very computationally intensive, the exact discrete method, which computes the exact partial likelihood, is the most precise method when there are many ties.  

%Unlike the Breslow and Efron methods, which assume a continuous time process, the exact discrete method assumes a discrete time process where the tied events actually do occur at exactly the same time.  The method begins by assuming that the data are grouped into risk sets $R_j$.  In each risk set, denote a binary dependent variable for each observation that takes on the value of 1 for each observation that experiences the event and 0 for each observation that does not experience the event.  Denote $d_j$ as the number of 1s in $R_j$ and $D_j$ the set of observations with 1s in $R_j$.  $D_j$ represents a specific pattern of 0s and 1s (in our previous example, the specific pattern of 0s and 1s is that observations 1 and 3 experienced an event while 2 and 4 did not, so $D_j$ represents the specific pattern of $\{1,3\}$).  Then for each $R_j$, we are interested in the conditional probability of getting the specific pattern of 0s and 1s given the total number of 1s in $R_j$.  Thus, the conditional probability for each risk set is given as
%\begin{eqnarray*}
%\mathrm{Pr}(D_j | d_j) = \frac{\prod_{i \in D_j} \exp(x_i \beta)}{\sum_{m=1}^M \left[\prod_{k \in A_{jm}}\exp(x_k \beta)\right] }
%\end{eqnarray*}  
%where $A_{jm}$ is a set of observations that represents one combination of $d_j$ number of 1s in $R_j$.  There are $M$ possible combinations for each risk set.  For example, in the case where there are two events among a risk set of four observations $\{1,2,3,4\}$, there are 6 combinations of choosing 2 events out of 4 observations.  Therefore, $M = 6$ and $A_{j1} = \{1,2\}, A_{j2} = \{1,3\}, A_{j3} = \{1,4\} ,A_{j4} = \{2,3\}, A_{j5} = \{2,4\}, A_{j6} = \{3,4\}$.  The partial likelihood then takes the conditional probability over each $j$ risk set.  Note that the exact discrete approximation method is equivalent to a conditional logit model.
\end{itemize}
\end{itemize}

\subsubsection{Stratified Cox Model}

In addition, {\tt zelig()} also supports the stratified Cox model, where the baseline hazards are assumed to be different across different strata but the coefficients are restricted to be the same across strata.  Let {\tt id} be a variable that consists of either discrete numeric values, character strings, or factors that define the strata.  Then the stratified Cox model can be estimated using {\tt strata()} in the formula.  The user can then find quantities of interest for a specific stratum by defining the stratum of choice in {\tt setx()}.  If no strata are defined, {\tt setx} takes the mode.  Strata on {\tt setx} are defined as followed:

\begin{itemize}
\item If strata were defined by a variable ({\tt strata(id)}), then strata should be defined as {\tt strata = "id=5"}.
\item If strata were defined by a mathematical expression ({\tt strata(id>10)}), then strata should be defined as {\tt strata = "id>10=TRUE"} or {\tt strata = "id>10=FALSE"}.
\end{itemize}

\begin{verbatim}
> z.out <- zelig(Surv(Y,C) ~ X1 + X2 + strata(id), model = "coxph", 
                 data = mydata)
> x.out <- setx(z.out, strata = "id=5")
> s.out <- sim(z.out, x = x.out)
\end{verbatim}


\subsubsection{Time-Varying Covariates}

{\tt zelig()} also supports the use of time-varying covariates for the Cox model, where some or all of the covariates change over time for each case.  Let ``case'' refer to each unit in the data.  Then each case can have one or more ``observations'', where each observation has a different value for one or more covariates for a specific case.  \\ 

\noindent Estimating a time-varying covariate model with {\tt zelig()} involves setting up the data differently to reflect a counting process.  In the typical non-time-varying covariate model, the cases include a duration time ($Y$), a censoring mechanism ($C$), and covariates ($X$).  A typical dataset would look like this:

\begin{table}[!htp]
 \begin{center}
\begin{tabular}{lllll}
\hline
Case & Y & C & X1 & X2 \\ 
\hline
1 & 35 & 0 & 4 & 7 \\ 
2 & 56 & 1 & 6 & 11\\
\hline
\end{tabular}
 \end{center}
\end{table}

\noindent The user would then estimate the model and find quantities of interest using the following syntax:

\begin{verbatim}
> z.out <- zelig(Surv(Y,C) ~ X1 + X2, model = "coxph", data = mydata)
> x.out <- setx(z.out)
> s.out <- sim(z.out, x = x.out)
\end{verbatim}

\noindent With time-varying covariates, each case is composed of multiple observations with start times, stop times, censoring (event) mechanisms, and covariates.  The covariates are assumed to be constant within the intervals defined by the start and stop times.  The covariates change only between intervals.  Thus, the covariates are constant at each observation.  The censoring mechanism equals 1 when an event occurs at the stop time and equals 0 if the observation is censored or if no event occurs at the stop time.  A typical time-varying dataset would look like this:

\begin{table}[!htp]
 \begin{center}
\begin{tabular}{llllll}
\hline
Case & Start & Stop & C & X1 & X2 \\ 
\hline
1 & 0 & 26 & 0 & 4 & 7 \\
1 & 26 & 35 & 0 & 4 & 10 \\
2 & 0 & 39 & 0 & 6 & 11 \\ 
2 & 39 & 56 & 1 & 9 & 5 \\
\hline
\end{tabular}
 \end{center}
\end{table}

\noindent The user would then estimate the model and find quantities of interest using the following syntax:

\begin{verbatim}
> z.out <- zelig(Surv(Start,Stop,C) ~ X1 + X2, model = "coxph", 
                 data = mydata)
> x.out <- setx(z.out)
> s.out <- sim(z.out, x = x.out)
\end{verbatim}


\subsubsection{Examples}
\begin{enumerate}
\item {Example 1: Basic Example}

Attaching the sample dataset:
\begin{Schunk}
\begin{Sinput}
RRR> data(coalition)
\end{Sinput}
\end{Schunk}
Estimating parameter values for the coxph regression:
\begin{Schunk}
\begin{Sinput}
RRR> z.out1 <- zelig(Surv(duration, ciep12) ~ invest + numst2 + crisis, robust = TRUE, cluster = "polar", model = "coxph", data = coalition)
\end{Sinput}
\end{Schunk}
Setting values for the explanatory variables:
\begin{Schunk}
\begin{Sinput}
RRR> x.low1<- setx(z.out1, numst2 = 0)
RRR> x.high1 <- setx(z.out1, numst2 = 1)
\end{Sinput}
\end{Schunk}
Simulating quantities of interest:
\begin{Schunk}
\begin{Sinput}
RRR> s.out1 <- sim(z.out1, x = x.low1, x1 = x.high1)
\end{Sinput}
\end{Schunk}
\begin{Schunk}
\begin{Sinput}
RRR> summary(s.out1)
\end{Sinput}
\end{Schunk}
\begin{center}
\begin{Schunk}
\begin{Sinput}
RRR> plot(s.out1)
\end{Sinput}
\end{Schunk}
\includegraphics{vigpics/coxph-ExamplePlot}
\end{center}

\item {Example 2: Example with Stratified Cox Model}

Estimating parameter values for the stratified coxph regression:
\begin{Schunk}
\begin{Sinput}
RRR> z.out2 <- zelig(Surv(duration, ciep12) ~ invest + strata(polar) + numst2 + crisis, model = "coxph", data = coalition)
\end{Sinput}
\end{Schunk}
Setting values for the explanatory variables:
\begin{Schunk}
\begin{Sinput}
RRR> x.low2 <- setx(z.out2, numst2 = 0, strata = "polar=3")
RRR> x.high2 <- setx(z.out2, numst2 = 1, strata = "polar=3")
\end{Sinput}
\end{Schunk}
Simulating quantities of interest:
\begin{Schunk}
\begin{Sinput}
RRR> s.out2 <- sim(z.out2, x = x.low2, x1 = x.high2)
\end{Sinput}
\end{Schunk}
\begin{Schunk}
\begin{Sinput}
RRR> summary(s.out2)
\end{Sinput}
\end{Schunk}
\begin{center}
\begin{Schunk}
\begin{Sinput}
RRR> plot(s.out2)
\end{Sinput}
\end{Schunk}
\includegraphics{vigpics/coxph-ExamplePlot2}
\end{center}

\item {Example 3: Example with Time-Varying Covariates}

Create sample toy dataset (from {\tt survival} package):
\begin{Schunk}
\begin{Sinput}
RRR> toy <- as.data.frame(list(start=c(1, 2, 5, 2, 1, 7, 3, 4, 8, 8),
+             stop=c(2, 3, 6, 7, 8, 9, 9, 9,14,17),
+             event=c(1, 1, 1, 1, 1, 1, 1, 0, 0, 0),
+             x=c(1, 0, 0, 1, 0, 1, 1, 1, 0, 0),
+ 	    x1=c(5, 5, 7, 4, 5, 6, 3, 2, 7, 4) ))
\end{Sinput}
\end{Schunk}
Estimating parameter values for the coxph regression:
\begin{Schunk}
\begin{Sinput}
RRR> z.out3 <- zelig(Surv(start, stop, event) ~ x + x1, model = "coxph", data = toy)
\end{Sinput}
\end{Schunk}
Setting values for the explanatory variables:
\begin{Schunk}
\begin{Sinput}
RRR> x.low3 <- setx(z.out3, x = 0)
RRR> x.high3 <- setx(z.out3, x = 1)
\end{Sinput}
\end{Schunk}
Simulating quantities of interest:
\begin{Schunk}
\begin{Sinput}
RRR> s.out3 <- sim(z.out3, x = x.low3, x1 = x.high3)
\end{Sinput}
\end{Schunk}
\begin{Schunk}
\begin{Sinput}
RRR> summary(s.out3)
\end{Sinput}
\end{Schunk}
\begin{center}
\begin{Schunk}
\begin{Sinput}
RRR> plot(s.out3)
\end{Sinput}
\end{Schunk}
\includegraphics{vigpics/coxph-ExamplePlot3}
\end{center}

\end{enumerate}

\subsubsection{The Model}
Let $Y_i^*$ be the survival time for observation $i$. This variable
might be censored for some observations at a fixed time $y_c$ such
that the fully observed dependent variable, $Y_i$, is defined as
\begin{equation*}
  Y_i = \left\{ \begin{array}{ll}
      Y_i^* & \textrm{if }Y_i^* \leq y_c \\
      y_c & \textrm{if }Y_i^* > y_c
    \end{array} \right.
\end{equation*}

\begin{itemize}
\item The \textit{stochastic component} is described by the distribution of the partially observed variable $Y^*$:
\begin{equation*}
Y_i^* \sim f(y_i^* | \mu_i, \alpha)
\end{equation*}
where $f$ is an unspecified distribution with some mean $\mu_i$ and shape $\alpha$.  In the Cox proportional hazards model, the distributional form of the duration times is unknown and left unparameterized.  Instead it uses the proportional hazards assumption to model the set of (ordered) event times on particular covariates.

An important component of all survival models is the hazard function $h(t)$, which measures the probability of an observation not surviving past time $t$ given survival up to $t$.  The hazard function is given by 
\begin{equation*}
h_i(t) = \lambda(t) \times \lambda_i
\end{equation*}
where $\lambda(t)$ is the baseline hazard (when all covariates equal 0), which varies over $t$ but not over $i$, and $\lambda_i$ is the parameterized part of the hazard function, which varies over $i$ but not over $t$ (the proportional hazards assumption).

The model estimates the parameters without a distributional assumption on the duration times by focusing on the occurrence of events and ignoring the time between events.  The data are reconceptualized from duration times to $K$ discrete event times such that each $y_i$ corresponds to exactly one event time $t_i$.  The model assumes that no two $y_i$ have the same event times.  

For each event time, denote $R(t_i)$ as the set of all observations $j$ that are at risk at $t_i$.  Given that an event occurred at $t_i$, we are interested in the conditional probability that the event occurred in observation $i$.  The conditional probability is given by 
\begin{eqnarray*}
\mathrm{Pr}(y_i = t_i \mid \mathrm{an \: event \: at \:}t_i) &=&  \frac{h_i(t_i)}{\sum_{j \in R(t_i)} h_j(t_i)}\\
&=& \frac{\lambda(t_i) \, \lambda_i}{\sum_{j \in R(t_i)} \lambda(t_i) \, \lambda_j}\\
&=& \frac{\lambda_i}{\sum_{j \in R(t_i)} \lambda_j}
\end{eqnarray*}
where the numerator denotes the probability of observation $i$ experiencing the event at $t_i$ and the denominator denotes the probability that an event occurred at $t_i$.  
%
%The model estimates the parameters without a distributional assumption on the duration times by focusing on the occurrence of events and ignoring the time between events.  The data are first sorted by $n$ discrete event times (indexed by $j$) such that $t_1 < t_2 < ... < t_n$, where $n$ corresponds to both the number of events and the number of uncensored observations.  The model assumes that there are no ties among the $n$ discrete event times.  See Additional Inputs for methods to deal with ties.  
%
%Each $y_i^*$ corresponds to one event time $t_j$ at which the event occurs for $i$ (for example, if the 5th observation experienced the shortest duration of 2, then $y_5^* = t_1 = 2$).  For each $j$ event time, denote $R_j$ as the set of all observations $k$ that are at risk at time $t_j$.  Then for each $t_j$, we are interested in the conditional probability of the event occurring in the \textbf{corresponding} observation $i$ given that an event occurred at $t_j$.  Note that for each event time, we are only interested in the conditional probability for the corresponding $i$ rather than for all $i$.  The conditional probability is given by
%\begin{eqnarray*}
%\mathrm{Pr}(y_i^* = t_j \mid \mathrm{1 \: event \: at \:}t_j) &=&  \frac{h_i(t_j)}{\sum_{k \in R_j} h_k(t_j)}\\
%&=& \frac{\lambda(t_j) \, \lambda_i}{\sum_{k \in R_j} \lambda(t_j) \, \lambda_k}\\
%&=& \frac{\lambda_i}{\sum_{k \in R_j} \lambda_k}
%\end{eqnarray*}
%where the numerator denotes the probability of observation $i$ experiencing the event at $t_j$ and the denominator denotes the probability that an event occurred at $t_j$.  Observations that are censored contribute to the denominator for event times prior to its censoring (as part of the risk set) but not the numerator since censored observations do not have corresponding event times.  Each $j$ event time contributes one conditional probability to the partial likelihood function, which is then maximized to give an estimate of the parameters.  For an example, see \citet[53]{BoxJon04}.  

\item The \textit{systematic component} $\lambda_i$ is modeled as
\begin{equation*}
\lambda_i = \exp(x_i \beta)
\end{equation*}
where $x_i$ is the vector of explanatory variables, and $\beta$ is the vector of coefficients.

\item Each risk set (and thus each event time) contributes one conditional probability to the partial likelihood function, given by 
\begin{equation*}
L(\beta | y) = \prod_{i=1}^K \left[ \frac{\exp(x_i \beta)}{\sum_{j \in R(t_i)} \exp(x_j \beta)} \right] ^{c_i}
\end{equation*}  
where $c_i$ is the binary censoring variable.  Note that event times corresponding to censored observations are not counted since their corresponding terms for the partial likelihood are exponentiated to 0.  However, all censored observations are considered part of the risk sets $R(t_i)$ for all event times prior to their censoring, but otherwise do not contribute to the partial likelihood.  For an example, see \citet[53]{BoxJon04}.  

\item In the case of the Cox model with time-varying covariates, the partial likelihood function is similarly given by  
\begin{equation*}
L(\beta | y) = \prod_{i=1}^K \left[ \frac{\exp(x_i(t_i) \; \beta)}{\sum_{j \in R(t_i)} \exp(x_j(t_i) \; \beta)} \right] ^{c_i}
\end{equation*}  
where $x_i(t_i)$ is the value of the covariates at time $t_i$.  Denote ``cases'' as the units in our data.  Each case is composed of one or more observations corresponding to different values in one or more covariates.  At each event time $t_i$, the partial likelihood evaluates the hazard of the case in which the event occurred in with its covariate values at $t_i$ (the numerator) and the hazard of all the other cases at risk at $t_i$ (risk set $R(t_i)$) with their covariate values at $t_i$ (the denominator).  See previous section for more information. 


\item Although the model assumes that there are no tied event times, in practice, data often have tied event times due to imprecise measurement.  There are three commonly used methods to deal with tied event times.  
\begin{itemize}
\item \textbf{Breslow method}: The Breslow method simply treats the risk set as the same for all tied events in the risk set.  Suppose observations 1 and 3 are tied in a risk set of observations 1, 2, 3, and 4.  Theoretically, if the event occurred in 1 before in 3, then the risk set for observation 3 would have dropped observation 1.  However, since we cannot tell which event occurred first, in the partial likelihood, the risk set for observation 1 and observation 3 are the same, consisting of both observations 1 and 3 as well as 2 and 4.  For each risk set $R(t_i)$, let $d_i$ equal the number of tied events in the $i$th risk set and let $D_i$ denote the set of $d_i$ tied events.  For risk sets with no tied events, $d_i = 1$.  The approximate partial likelihood for the Breslow method is given by
\begin{equation*}
L(\beta | y) = \prod_{i=1}^K \left[ \frac{\prod_{i \in D_i}\exp(x_i \beta)}{\left[ \sum_{j \in R(t_i)} \exp(x_j \beta)\right] ^{d_i}} \right] ^ {c_i}
\end{equation*} 

\item \textbf{Efron method}: The Efron method is more precise because it tries to account for how the risk set changes depending on the sequence of tied events.  For an intuition behind the Efron approximation, suppose as in the previous example that observations 1 and 3 are tied in a risk set of observations 1, 2, 3, and 4.  If the event occurred in 1 before 3, then the risk set for the second event would consist of observations $\{2, 3, 4\}$.  On the other hand, if the event occurred in 3 before 1, then the risk set for the second event would consist of observations $\{1,2,4\}$.  Since both cases are equally plausible with the tied event times, the Efron approximation suggests that the second risk set would consist of $\{2, 3, 4\}$ with $0.5$ probability and $\{1,2,4\}$ with $0.5$ probability.  The Efron approximate partial likelihood is then given by
\begin{equation*}
L(\beta | y) = \prod_{i=1}^K \left[\frac{\prod_{i \in D_i}\exp(x_i \beta)}{\prod_{r=1}^{d_i} \left[\sum_{j \in R(t_i)} \exp(x_j \beta) - \frac{r-1}{d_i} \sum_{j \in D_i} \exp(x_j \beta)\right] } \right] ^ {c_i}  
\end{equation*}
where $r$ indexes $D_i$, which is the set of $d_i$ tied events for the $i$th risk set.

\item \textbf{Exact discrete method}: Unlike the Breslow and Efron methods, which assume a continuous time process, the exact discrete method assumes a discrete time process where the tied events actually do occur at exactly the same time.  The method begins by assuming that the data are grouped into risk sets $R(t_i)$.  In each risk set and for each observation, denote a binary dependent variable which takes on the value of 1 for each observation that experiences the event and 0 for each observation that does not experience the event.  Denote $d_i$ as the number of 1s in $R(t_i)$ and $D_i$ as the set of observations with 1s in $R(t_i)$.  $D_i$ represents a specific pattern of 0s and 1s (in our previous example, the specific pattern of 0s and 1s is that observations 1 and 3 experienced an event while 2 and 4 did not, so $D_i$ is the set $\{1,3\}$).  Then for each $R(t_i)$, we are interested in the conditional probability of getting the specific pattern of 0s and 1s given the total number of 1s in $R(t_i)$.  Thus, the conditional probability for each risk set is given as
\begin{eqnarray*}
\mathrm{Pr}(D_i | d_i) = \frac{\prod_{i \in D_i} \exp(x_i \beta)}{\sum_{m=1}^M \left[\prod_{j \in A_{im}}\exp(x_j \beta)\right] }
\end{eqnarray*}  
where $A_{im}$ is a set of observations that represents one combination of $d_i$ number of 1s in $R(t_i)$.  There are $M$ possible combinations for each risk set.  The partial likelihood then takes the conditional probability over each $i$ risk set.  Note that the exact discrete approximation method is equivalent to a conditional logit model.
\end{itemize}
\end{itemize}

\subsubsection{Quantities of Interest}

\begin{itemize}

\item The hazard ratio ({\tt qi\$hr}) is defined as 
\begin{equation*}
\textrm{HR} = \frac{h(t \mid x_1)}{h(t \mid x)} = \frac{\lambda(t) \exp(x_1 \beta)}{\lambda(t) \exp(x \beta)} =  \frac{\exp(x_1 \beta)}{\exp(x \beta)}
\end{equation*}
given draws of $\beta$ from its sampling distribution, where $x$ and $x_1$ are values of the independent variables chosen by the user.  Typically, $x$ and $x_1$ should only differ over one independent variable to interpret the effect of that variable on the hazard rate.  In a stratified Cox model, the strata should be the same in both $x$ and $x_1$.

\item The survival function ({\tt qi\$survival}) is defined as the fraction of observations surviving past time $t$.  It is derived from the cumulative hazard function ({\tt exp(-cumhaz)}).  The confidence interval of the survival function is drawn on the {\tt log(survival)} scale.

\item The cumulative hazard function ({\tt qi\$cumhaz}) is defined as {\tt -log(survival)}.  Although there is no direct interpretation, the cumulative hazard function is estimated from the data and then other quantities of interest are derived from the cumulative hazard function.

\item The hazard function ({\tt qi\$hazard}) is defined as the probability of an observation not surviving past time $t$ given survival up to $t$.  It is derived directly from the cumulative hazard function.

\item For MI data, if survival times are multiply imputed, we suggest having a larger number of imputed datasets.  Because the quantities of interest are derived semi-parametrically, there may be instances in which survival times appear only in one or a small fraction of the multiply imputed datasets, which may bias the results.  
\end{itemize}

\subsubsection{Output Values}

The output of each Zelig command contains useful information which you
may view.  For example, if you run \texttt{z.out <- zelig(Surv(y,c) \~\, x,
  model = "coxph", data)}, then you may examine the available
information in \texttt{z.out} by using \texttt{names(z.out)},
see the {\tt coefficients} by using {\tt z.out\$coefficients}, and
a default summary of information through \texttt{summary(z.out)}.
Other elements available through the {\tt \$} operator are listed
below.

\begin{itemize}
\item From the {\tt zelig()} output object {\tt z.out}, you may
  extract:
   \begin{itemize}
   \item {\tt coefficients}: parameter estimates for the explanatory
     variables.
   \item {\tt var}: the variance-covariance matrix.
   \item {\tt residuals}: the working residuals of the fit.
   \item {\tt loglik}: the log-likelihood for the baseline and full models
   \item {\tt linear.predictors}: a mean-adjusted linear predictor $x_i \beta$, where $x_i = x_i - \mathrm{mean}(x)$.
   \end{itemize}

\item From {\tt summary(z.out)}, you may extract:
   \begin{itemize}
   \item {\tt coef}: the parameter estimates with their
     associated standard errors, $p$-values, and $z$-statistics.
   \item {\tt conf.int}: $\exp(\beta)$ and their associated confidence intervals.
   \end{itemize}

\item From the {\tt sim()} output object {\tt s.out}, you may extract
  quantities of interest arranged as matrices indexed by simulation
  $\times$ {\tt x}-observation (for more than one {\tt x}-observation).
  Available quantities are:

   \begin{itemize}
   \item {\tt qi\$hr}: the simulated hazard ratios for the
     specified values of {\tt x} and {\tt x1}.
   \item {\tt qi\$survival}: the estimated survival function for the values specified in {\tt x}.
   \item {\tt qi\$cumhaz}: the estimated cumulative hazard function for the values specified in {\tt x}.
   \item {\tt qi\$hazard}: the estimated hazard function for the values specified in {\tt x}.
   \end{itemize}
\end{itemize}

\subsection*{How To Cite}

%\input{cites/coxph}
\input{citeZelig}

\subsection*{See also}

The Cox proportional hazards model is part of the survival library by Terry Therneau \citep{TheGra00}, ported to R by Thomas Lumley.  Advanced users may wish to refer to {\tt help(coxph)} and {\tt help(survfit)} in the survival library.  Sample data are from \citet{KinAltBur90}

To cite Zelig as a whole, please reference these two sources:
\begin{verse}
  Kosuke Imai, Gary King, and Olivia Lau. 2007. ``Zelig: Everyone's
  Statistical Software,'' \url{http://GKing.harvard.edu/zelig}.
\end{verse}
\begin{verse}
Imai, Kosuke, Gary King, and Olivia Lau. (2008). ``Toward A Common Framework for Statistical Analysis and Development.'' Journal of Computational and Graphical Statistics, Vol. 17, No. 4 (December), pp. 892-913. 
\end{verse}


\subsection*{See also}

The Cox proportional hazards model is part of the survival library by Terry Therneau \citep{TheGra00}, ported to R by Thomas Lumley.  Advanced users may wish to refer to {\tt help(coxph)} and {\tt help(survfit)} in the survival library.  Sample data are from \citet{KinAltBur90}

To cite Zelig as a whole, please reference these two sources:
\begin{verse}
  Kosuke Imai, Gary King, and Olivia Lau. 2007. ``Zelig: Everyone's
  Statistical Software,'' \url{http://GKing.harvard.edu/zelig}.
\end{verse}
\begin{verse}
Imai, Kosuke, Gary King, and Olivia Lau. (2008). ``Toward A Common Framework for Statistical Analysis and Development.'' Journal of Computational and Graphical Statistics, Vol. 17, No. 4 (December), pp. 892-913. 
\end{verse}


\subsection*{See also}

The Cox proportional hazards model is part of the survival library by Terry Therneau \citep{TheGra00}, ported to R by Thomas Lumley.  Advanced users may wish to refer to {\tt help(coxph)} and {\tt help(survfit)} in the survival library.  Sample data are from \citet{KinAltBur90}

To cite Zelig as a whole, please reference these two sources:
\begin{verse}
  Kosuke Imai, Gary King, and Olivia Lau. 2007. ``Zelig: Everyone's
  Statistical Software,'' \url{http://GKing.harvard.edu/zelig}.
\end{verse}
\begin{verse}
Imai, Kosuke, Gary King, and Olivia Lau. (2008). ``Toward A Common Framework for Statistical Analysis and Development.'' Journal of Computational and Graphical Statistics, Vol. 17, No. 4 (December), pp. 892-913. 
\end{verse}


\subsection*{See also}

The Cox proportional hazards model is part of the survival library by Terry Therneau \citep{TheGra00}, ported to R by Thomas Lumley.  Advanced users may wish to refer to {\tt help(coxph)} and {\tt help(survfit)} in the survival library.  Sample data are from \citet{KinAltBur90}
