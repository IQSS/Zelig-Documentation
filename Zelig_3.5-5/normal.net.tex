\section{{\tt normal.net}: Network Normal Regression for Continuous Proximity Matrix Dependent Variables}

The Network Normal regression model is a close variant of the more standard least squares regression model (see {\tt netlm}). Both models specify a continuous proximity matrix (a.k.a. sociomatricies, adjacency matrices, or matrix representations of directed graphs) dependent variable as a linear function of a set of explanatory variables. The network Normal model reports maximum likelihood (rather than least squares) estimates. The two models differ only in their estimate for the stochastic parameter $\sigma$.

\subsubsection{Syntax}
\begin{verbatim}
> z.out <- zelig(y ~ x1 + x2, model = "normal.net", data = mydata) 
> x.out <- setx(z.out)
> s.out <- sim(z.out, x = x.out)
\end{verbatim}

\subsubsection{Additional Inputs}

In addition to the standard inputs, {\tt zelig()} takes the following additional options for network normal regression:

\begin{itemize}
\item {\tt LF}: specifies the link function to be used for the network normal regression. Default is {\tt LF="identity"}, but {\tt LF} can also be set to {\tt "log"} or {\tt "inverse"} by the user.
\end{itemize}

\subsubsection{Examples}
\begin{enumerate}
\item Basic Example

Load the sample data (see {\tt ?friendship} for details on the structure of the network dataframe):

\begin{Schunk}
\begin{Sinput}
RRR> data(friendship)
RRR> 
RRR> 
RRR> 
\end{Sinput}
\end{Schunk}
Estimate model:

\begin{Schunk}
\begin{Sinput}
RRR> z.out <- zelig(perpower ~ friends + advice + prestige, model = "normal.net", data = friendship)
RRR> summary(z.out)
RRR> 
\end{Sinput}
\end{Schunk}
Setting values for the explanatory variables to their default values:

\begin{Schunk}
\begin{Sinput}
RRR> x.out <- setx(z.out)
RRR> 
\end{Sinput}
\end{Schunk}
Simulate fitted values.
\begin{Schunk}
\begin{Sinput}
RRR> s.out <- sim(z.out, x = x.out) 
RRR> summary(s.out) 
RRR> plot(s.out) 
RRR> 
\end{Sinput}
\end{Schunk}

\begin{figure}[here]
\centering
\includegraphics{vigpics/normalnet-007}
\label{fig:plotgam}
\end{figure}

\end{enumerate}



\subsubsection{Model}
The {\tt normal.net} model performs a Normal regression of the proximity matrix $\mathbf{Y}$, a $m \times m$ matrix representing network ties, on a set of proximity matrices $\mathbf{X}$. This network regression model is directly analogous to standard Normal regression element-wise on the appropriately vectorized matrices. Proximity matrices are vectorized by creating $Y$, a $m^2 \times 1$ vector to represent the proximity matrix. The vectorization which produces the $Y$ vector from the $\mathbf{Y}$ matrix is performed by simple row-concatenation of $\mathbf{Y}$. For example, if $\mathbf{Y}$ is a $15 \times 15$ matrix, the $\mathbf{Y}_{1,1}$ element is the first element of $Y$, and the $\mathbf{Y}_{2,1}$ element is the second element of $Y$ and so on. Once the input matrices are vectorized, standard Normal regression is performed. 

Let $Y_{i}$ be the continuous dependent variable, produced by vectorizing a continuous proximity matrix, for observation $i$.  
\begin{itemize}
\item The \emph{stochastic component} is described by a univariate normal model with a vector of means $\mu_i$ and scalar variance $\sigma^2$: 
\begin{equation*}
Y_{i}  \sim \text{Normal} ( \mu_{i}, \sigma^2).\\
\end{equation*}
\item The \emph{systematic component} is given by:
\begin{equation*}
\mu_i = x_i\beta.
\end{equation*}
where $ x_i$ is the vector of $k$ explanatory variables and $\beta$ is the vector of coefficients.
\end{itemize}


\subsubsection{Quantities of Interest}
The quantities of interest for the network Normal regression are the same as those for the standard Normal regression. 
\begin{itemize}
\item The expected value ({\tt qi\$ev}) for the {\tt normal.net} model is the mean of simulations from the stochastic component,  
\begin{equation*}
E(Y) = \mu_{i} =  x_{i}\beta,
\end{equation*}
given a draw of $\beta$ from its posterior.

\item The predicted value ({\tt qi\$pr}) is a draw from the distribution defined by the set of parameters $(\mu_i, \sigma^2)$. 

\item The first difference ({\tt qi\$fd}) for the network Normal model is defined as 
\begin{equation*}
FD = \text{Pr}(Y| x_{1}) - \text{Pr}(Y| x)
\end{equation*}
\end{itemize}


\subsubsection{Output Values}

The output of each Zelig command contains useful information which you may view. For example, you run \verb{ z.out <- zelig(y ~ x, model = "normal.net", data){, then you may examine the available information in {\tt z.out} by using {\tt names(z.out)}, see the coefficients by using {\tt z.out\$coefficients}, and a default summary of information through {\tt summary(z.out)}. Other elements available through the {\tt \$} operator are listed below. 
\begin{itemize}
\item From the {\tt zelig()} output stored in {\tt  z.out}, you may extract:
\begin{itemize}
\item {\tt coefficients}: parameter estimates for the explanatory variables.
\item {\tt fitted.values}: the vector of fitted values for the systemic component $\lambda$.
\item {\tt residuals}: the working residuals in the final iteration of the IWLS fit. 
\item {\tt linear.predictors}: fitted values. For the normal model, these are identical to fitted values.
\item {\tt aic}: Akaike's Information Criterion (minus twice the maximized log-likelihood plus twice the number of coefficients).
\item {\tt bic}: the Bayesian Information Criterion (minus twice the maximized log-likelihood plus the number of coefficients times log $n$).
\item {\tt df.residual}: the residual degrees of freedom.
\item {\tt df.null}: the residual degrees of freedom for the null model. 
\item {\tt zelig.data}: the input data frame if {\tt save.data = TRUE} 

\end{itemize}
\item From {\tt summary(z.out)}(as well as from {\tt zelig()}), you may extract:
\begin{itemize}
\item {\tt mod.coefficients}: the parameter estimates with their associated standard errors, $p$-values, and $t$ statistics. 
\item {\tt cov.scaled}: a $k \times k$ matrix of scaled covariances.
\item {\tt cov.unscaled}: a $k \times k$ matrix of unscaled covariances. 
\end{itemize}
\item From the {\tt sim()} output stored in {\tt s.out}, you may extract:
\begin{itemize}
\item {\tt qi\$ev}: the simulated expected probabilities for the specified values of {\tt x}.
\item {\tt qi\$pr}: the simulated predicted values drawn from the distribution defined by $(\mu_i, \sigma^2)$.
\item {\tt qi\$fd}: the simulated first differences in the expected probabilities simulated from {\tt x} and {\tt x1}.
\end{itemize}
\end{itemize}


 \subsection* {How to Cite} 

To cite the \emph{ normal.net } Zelig model:
 \begin{verse}
 Skyler J. Cranmer. 2007. "normal.net: Social Network Normal Regression for Continuous Dependent Variables" in Kosuke Imai, Gary King, and Olivia Lau, "Zelig: Everyone's Statistical Software,"\url{http://gking.harvard.edu/zelig} 
\end{verse}
To cite Zelig as a whole, please reference these two sources:
\begin{verse}
  Kosuke Imai, Gary King, and Olivia Lau. 2007. ``Zelig: Everyone's
  Statistical Software,'' \url{http://GKing.harvard.edu/zelig}.
\end{verse}
\begin{verse}
Imai, Kosuke, Gary King, and Olivia Lau. (2008). ``Toward A Common Framework for Statistical Analysis and Development.'' Journal of Computational and Graphical Statistics, Vol. 17, No. 4 (December), pp. 892-913. 
\end{verse}

\subsection* {See also}
The network normal regression is part of the {\tt netglm} package by Skyler J. Cranmer and is built using some of the functionality of the  {\tt sna} package by Carter T. Butts \citep{ButCar01}.In addition, advanced users may wish to refer to {\tt help(normal.net)}. Sample data are fictional. 
