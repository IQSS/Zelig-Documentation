\section{{\tt probit}: Probit Regression for Dichotomous Dependent Variables}\label{probit}

Use probit regression to model binary dependent variables
specified as a function of a set of explanatory variables.  

\subsubsection{Syntax}
\begin{verbatim}
> z.out <- zelig(Y ~ X1 + X2, model = "probit", data = mydata)
> x.out <- setx(z.out)
> s.out <- sim(z.out, x = x.out, x1 = NULL)
\end{verbatim}

\subsubsection{Additional Inputs} 

In addition to the standard inputs, {\tt zelig()} takes the following
additional options for probit regression:  
\begin{itemize}
\item {\tt robust}: defaults to {\tt FALSE}.  If {\tt TRUE} is
selected, {\tt zelig()} computes robust standard errors via the {\tt
sandwich} package (see \cite{Zeileis04}).  The default type of robust
standard error is heteroskedastic and autocorrelation consistent (HAC),
and assumes that observations are ordered by time index.

In addition, {\tt robust} may be a list with the following options:  
\begin{itemize}
\item {\tt method}:  Choose from 
\begin{itemize}
\item {\tt "vcovHAC"}: (default if {\tt robust = TRUE}) HAC standard
errors. 
\item {\tt "kernHAC"}: HAC standard errors using the
weights given in \cite{Andrews91}. 
\item {\tt "weave"}: HAC standard errors using the
weights given in \cite{LumHea99}.  
\end{itemize}  
\item {\tt order.by}: defaults to {\tt NULL} (the observations are
chronologically ordered as in the original data).  Optionally, you may
specify a vector of weights (either as {\tt order.by = z}, where {\tt
z} exists outside the data frame; or as {\tt order.by = \~{}z}, where
{\tt z} is a variable in the data frame).  The observations are
chronologically ordered by the size of {\tt z}.
\item {\tt \dots}:  additional options passed to the functions 
specified in {\tt method}.   See the {\tt sandwich} library and
\cite{Zeileis04} for more options.   
\end{itemize}
\end{itemize}

\subsubsection{Examples}
Attach the sample turnout dataset:
\begin{Schunk}
\begin{Sinput}
>  data(turnout)
\end{Sinput}
\end{Schunk}
Estimate parameter values for the probit regression:
\begin{Schunk}
\begin{Sinput}
>  z.out <- zelig(vote ~ race + educate,  model = "probit", data = turnout) 
\end{Sinput}
\end{Schunk}
\begin{Schunk}
\begin{Sinput}
>  summary(z.out)
\end{Sinput}
\end{Schunk}
Set values for the explanatory variables to their default values.
\begin{Schunk}
\begin{Sinput}
>  x.out <- setx(z.out)
\end{Sinput}
\end{Schunk}
Simulate quantities of interest from the posterior distribution.
\begin{Schunk}
\begin{Sinput}
> s.out <- sim(z.out, x = x.out)
\end{Sinput}
\end{Schunk}
\begin{Schunk}
\begin{Sinput}
> summary(s.out)
\end{Sinput}
\end{Schunk}

\subsubsection{Model}
Let $Y_i$ be the observed binary dependent variable for observation
$i$ which takes the value of either 0 or 1.
\begin{itemize}
\item The \emph{stochastic component} is given by  
\begin{equation*}
Y_i \; \sim \; \textrm{Bernoulli}(\pi_i), 
\end{equation*}
where $\pi_i=\Pr(Y_i=1)$.

\item The \emph{systematic component} is 
\begin{equation*}
  \pi_i \; = \; \Phi (x_i \beta)
\end{equation*}
where $\Phi(\mu)$ is the cumulative distribution function of the
Normal distribution with mean 0 and unit variance.
\end{itemize}

\subsubsection{Quantities of Interest}

\begin{itemize}

\item The expected value ({\tt qi\$ev}) is a simulation of predicted
  probability of success $$E(Y) = \pi_i = \Phi(x_i
  \beta),$$ given a draw of $\beta$ from its sampling distribution.  

\item The predicted value ({\tt qi\$pr}) is a draw from a Bernoulli
  distribution with mean $\pi_i$.  
  
\item The first difference ({\tt qi\$fd}) in expected values is
  defined as
\begin{equation*}
\textrm{FD} = \Pr(Y = 1 \mid x_1) - \Pr(Y = 1 \mid x).
\end{equation*}

\item The risk ratio ({\tt qi\$rr}) is defined as
\begin{equation*}
\textrm{RR} = \Pr(Y = 1 \mid x_1) / \Pr(Y = 1 \mid x).
\end{equation*}

\item In conditional prediction models, the average expected treatment
  effect ({\tt att.ev}) for the treatment group is 
    \begin{equation*} \frac{1}{\sum_{i=1}^n t_i}\sum_{i:t_i=1}^n \left\{ Y_i(t_i=1) -
      E[Y_i(t_i=0)] \right\},
    \end{equation*} 
    where $t_i$ is a binary explanatory variable defining the treatment
    ($t_i=1$) and control ($t_i=0$) groups.  Variation in the
    simulations are due to uncertainty in simulating $E[Y_i(t_i=0)]$,
    the counterfactual expected value of $Y_i$ for observations in the
    treatment group, under the assumption that everything stays the
    same except that the treatment indicator is switched to $t_i=0$.

\item In conditional prediction models, the average predicted treatment
  effect ({\tt att.pr}) for the treatment group is 
    \begin{equation*} \frac{1}{\sum_{i=1}^n t_i}\sum_{i:t_i=1}^n \left\{ Y_i(t_i=1) -
      \widehat{Y_i(t_i=0)} \right\},
    \end{equation*} 
    where $t_i$ is a binary explanatory variable defining the
    treatment ($t_i=1$) and control ($t_i=0$) groups.  Variation in
    the simulations are due to uncertainty in simulating
    $\widehat{Y_i(t_i=0)}$, the counterfactual predicted value of
    $Y_i$ for observations in the treatment group, under the
    assumption that everything stays the same except that the
    treatment indicator is switched to $t_i=0$.
\end{itemize}

\subsubsection{Output Values}

The output of each Zelig command contains useful information which you
may view.  For example, if you run \texttt{z.out <- zelig(y \~\,
  x, model = "probit", data)}, then you may examine the available
information in \texttt{z.out} by using \texttt{names(z.out)},
see the {\tt coefficients} by using {\tt z.out\$coefficients}, and
a default summary of information through \texttt{summary(z.out)}.
Other elements available through the {\tt \$} operator are listed
below.

\begin{itemize}
\item From the {\tt zelig()} output object {\tt z.out}, you may extract:
   \begin{itemize}
   \item {\tt coefficients}: parameter estimates for the explanatory
     variables.
   \item {\tt residuals}: the working residuals in the final iteration
     of the IWLS fit.
   \item {\tt fitted.values}: a vector of the in-sample fitted values.
   \item {\tt linear.predictors}: a vector of $x_{i}\beta$.  
   \item {\tt aic}: Akaike's Information Criterion (minus twice the
     maximized log-likelihood plus twice the number of coefficients).
   \item {\tt df.residual}: the residual degrees of freedom.
   \item {\tt df.null}: the residual degrees of freedom for the null
     model.
   \item {\tt data}: the name of the input data frame.  
   \end{itemize}

\item From {\tt summary(z.out)}, you may extract: 
   \begin{itemize}
   \item {\tt coefficients}: the parameter estimates with their
     associated standard errors, $p$-values, and $t$-statistics.
   \item{\tt cov.scaled}: a $k \times k$ matrix of scaled covariances.
   \item{\tt cov.unscaled}: a $k \times k$ matrix of unscaled
     covariances.  
   \end{itemize}

\item From the {\tt sim()} output object {\tt s.out}, you may extract
  quantities of interest arranged as matrices indexed by simulation
  $\times$ {\tt x}-observation (for more than one {\tt x}-observation).
  Available quantities are:

   \begin{itemize}
   \item {\tt qi\$ev}: the simulated expected values, or predicted
     probabilities, for the specified values of {\tt x}.
   \item {\tt qi\$pr}: the simulated predicted values drawn from the
     distributions defined by the predicted probabilities.  
   \item {\tt qi\$fd}: the simulated first differences in the predicted
     probabilities for the values specified in {\tt x} and {\tt x1}.
   \item {\tt qi\$rr}: the simulated risk ratio for the predicted
     probabilities simulated from {\tt x} and {\tt x1}.
   \item {\tt qi\$att.ev}: the simulated average expected treatment
     effect for the treated from conditional prediction models.  
   \item {\tt qi\$att.pr}: the simulated average predicted treatment
     effect for the treated from conditional prediction models.  
   \end{itemize}
\end{itemize}

\subsection*{How to Cite the Logit Model}
\bibentry{ImaLauKin-probit11}

\subsection*{How to Cite the Zelig Software Package}
\CiteZelig


\subsection* {See also}
The probit model is part of the stats package by \citet{VenRip02}.
Advanced users may wish to refer to \texttt{help(glm)} and
\texttt{help(family)}, as well as \cite{McCNel89}. Robust standard
errors are implemented via the sandwich package by \citet{Zeileis04}.
Sample data are from \cite{KinTomWit00}.
