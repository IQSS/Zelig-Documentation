\section{{\tt ls.net}: Network Least Squares Regression for Continuous Proximity Matrix Dependent Variables}
\label{ls.net}

Use network least squares regression analysis to estimate the
best linear predictor when the dependent variable is a
continuously-valued proximity matrix (a.k.a.\ sociomatrices, adjacency
matrices, or matrix representations of directed graphs). 

\subsubsection{Syntax}
\begin{verbatim}
> z.out <- zelig(y ~  x1 + x2, model = "ls.net", data = mydata)
> x.out <- setx(z.out)
> s.out <- sim(z.out, x = x.out)
\end{verbatim}

\subsubsection{Examples}
\begin{enumerate}
\item Basic Example with First Differences 

Load sample data and format it for social networkx analysis:
%library sna required 
\begin{Schunk}
\begin{Sinput}
RRR>  data(sna.ex)
\end{Sinput}
\end{Schunk}
Estimate model:
\begin{Schunk}
\begin{Sinput}
RRR>  z.out <- zelig(Var1 ~ Var2 + Var3 + Var4, model = "ls.net", data = sna.ex)
RRR> 
\end{Sinput}
\end{Schunk}

Summarize regression results:
\begin{Schunk}
\begin{Sinput}
RRR>  summary(z.out)
\end{Sinput}
\end{Schunk}

Set explanatory variables to their default (mean/mode) values, with
high (80th percentile) and low (20th percentile) for the second
explanatory variable (Var3).
\begin{Schunk}
\begin{Sinput}
RRR>  x.high <- setx(z.out, Var3 = quantile(sna.ex$Var3, 0.8))
RRR>  x.low <- setx(z.out, Var3 = quantile(sna.ex$Var3, 0.2))
\end{Sinput}
\end{Schunk}
Generate first differences for the effect of high versus low values of
Var3 on the outcome variable.
\begin{Schunk}
\begin{Sinput}
RRR>  try(s.out <- sim(z.out, x = x.high, x1 = x.low))
RRR>  try(summary(s.out))
\end{Sinput}
\end{Schunk}
\begin{center}
\begin{Schunk}
\begin{Sinput}
RRR>  plot(s.out)
\end{Sinput}
\end{Schunk}
\includegraphics{vigpics/lsnet-ExamplesPlot}
\end{center}
\end{enumerate}

\subsubsection{Model}
The {\tt ls.net} model performs a least squares regression of the
sociomatrix $\mathbf{Y}$, a $m \times m$ matrix representing network
ties, on a set of sociomatrices $\mathbf{X}$. This network regression
model is a directly analogue to standard least squares regression
element-wise on the appropriately vectorized matrices. Sociomatrices
are vectorized by creating $Y$, an $m^{2} \times 1$ vector to
represent the sociomatrix. The vectorization which produces the $Y$
vector from the $\mathbf{Y}$ matrix is preformed by simple
row-concatenation of $\mathbf{Y}$. For example if $\mathbf{Y}$ is a
$15 \times 15$ matrix, the $\mathbf{Y}_{1,1}$ element is the first
element of $Y$, and the $\mathbf{Y}_{21}$ element is the second
element of $Y$ and so on. Once the input matrices are vectorized,
standard least squares regression is performed. As such:
\begin{itemize}
\item The \emph{stochastic component} is described by a density with
mean $\mu_{i}$ and the common variance $\sigma^{2}$ 
\begin{equation*}
Y_{i} \sim f(y_{i} | \mu_{i}, \sigma^{2}).
\end{equation*}

\item The \emph{systematic component} models the conditional mean as
\begin{equation*}
\mu_{i} = x_{i}\beta
\end{equation*}
where $x_{i}$ is the vector of covariates, and $\beta$ is the vector of coefficients.
\end{itemize}
The least squares estimator is the best linear predictor of a
dependent variable given $x_{i}$, and minimizes the sum of squared
errors $\sum_{i = 1}^{n} (Y_{i} - x_{i}\beta)^{2}$.

\subsubsection{Quantities of Interest}
The quantities of interest for the network least squares regression
are the same as those for the standard least squares regression.
\begin{itemize}
\item The expected value ({\tt qi\$ev}) is the mean of simulations from
the stochastic component, 
\begin{equation*}
E(Y) = x_{i}\beta,
\end{equation*}
given a draw of $\beta$ from its sampling distribution.

\item The first difference ({\tt qi\$fd}) is:
\begin{equation*}
FD = E(Y | x_{1}) - E(Y | x)
\end{equation*}
\end{itemize}

\subsubsection{Output Values}

The output of each Zelig command contains useful information which you
may view. For example, you run {\tt z.out <- zelig(y ~ x,
model="ls.net", data)}, then you may examine the available information
in {\tt z.out} by using {\tt names(z.out)}, see the coefficients by
using {\tt z.out\$coefficients}, and a default summary of information
through {\tt summary(z.out)}. Other elements available through the
{\tt \$} operator are listed below. 
\begin{itemize}
\item From the {\tt zelig()} output stored in {\tt z.out}, you may extract:
\begin{itemize}
\item {\tt coefficients}: parameter estimates for the explanatory variables.
\item {\tt fitted.values}: the vector of fitted values for the explanatory variables.
\item {\tt residuals}: the working residuals in the final iteration of the IWLS fit. 
\item {\tt df.residual}: the residual degrees of freedom.
\item {\tt zelig.data}: the input data frame if {\tt save.data = TRUE}
\end{itemize}

\item From {\tt summary(z.out)}, you may extract:
\begin{itemize}
\item {\tt mod.coefficients}: the parameter estimates with their associated standard errors, $p$-values, and $t$ statistics. 
\begin{equation*}
\hat{\beta} = \left( \sum_{i = 1}^{n} x'_{i}x_{i}  \right)^{-1} \sum x_{i}y_{i}
\end{equation*}
\item {\tt sigma}: the square root of the estimate variance of the
random error $\varepsilon$:
\begin{equation*}
\hat{\sigma} = \frac{\sum (Y_{i} - x_{i} \hat{\beta} ) ^{2}}{n - k}
\end{equation*}
\item {\tt r.squared}: the fraction of the variance explained by the model.
\begin{equation*}
R^{2} = 1 - \frac{\sum (Y_{i} - x_{i} \hat{\beta} ) ^{2}}{\sum (y_{i}
- \bar{y})^{2}}
\end{equation*}
\item {\tt adj.r.squared}: the above $R^{2}$ statistic, penalizing for
an increased number of explanatory variables.  
\item {\tt cov.unscaled}: a $k \times k$ matrix of unscaled covariances. 
\end{itemize}

\item From the {\tt sim()} output stored in {\tt s.out}, you may extract:
\begin{itemize}
\item {\tt qi\$ev}: the simulated expected values for the specified values of {\tt x}.
\item {\tt qi\$fd}: the simulated first differences (or differences in
expected values) for the specified values of {\tt x} and {\tt x1}.
\end{itemize}
\end{itemize}

\subsection* {How to Cite} 

To cite the \emph{ ls.net } Zelig model:
 \begin{verse}
 Skyler J. Cranmer. 2007. "ls.net: Social Network Least Squares Regression for Continuous Dependent Variables" in Kosuke Imai, Gary King, and Olivia Lau, "Zelig: Everyone's Statistical Software,"\url{http://gking.harvard.edu/zelig} 
\end{verse}
To cite Zelig as a whole, please reference these two sources:
\begin{verse}
  Kosuke Imai, Gary King, and Olivia Lau. 2007. ``Zelig: Everyone's
  Statistical Software,'' \url{http://GKing.harvard.edu/zelig}.
\end{verse}
\begin{verse}
Imai, Kosuke, Gary King, and Olivia Lau. (2008). ``Toward A Common Framework for Statistical Analysis and Development.'' Journal of Computational and Graphical Statistics, Vol. 17, No. 4 (December), pp. 892-913. 
\end{verse}

\subsection* {See also}
The network least squares regression is part of the sna package by
Carter T. Butts \citep{ButCar01}.In addition, advanced users may wish to refer to {\tt help(netlm)}.
