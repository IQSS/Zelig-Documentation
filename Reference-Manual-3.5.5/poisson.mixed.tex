\section{{\tt poisson.mixed}: Mixed effects poisson Regression}
\label{mixed}

Use generalized multi-level linear regression if you have covariates that are grouped according to one or more classification factors. Poisson regression applies to dependent variables that represent the number of independent events that occur during a fixed period of time.

While generally called multi-level models in the social sciences, this class of models is often referred to as mixed-effects models in the statistics literature and as hierarchical models in a Bayesian setting. This general class of models consists of linear models that are expressed as a function of both \emph{fixed effects}, parameters corresponding to an entire population or certain repeatable levels of experimental factors, and \emph{random effects}, parameters corresponding to individual experimental units drawn at random from a population.

\subsubsection{Syntax}

\begin{verbatim}
z.out <- zelig(formula= y ~ x1 + x2 + tag(z1 + z2 | g),
               data=mydata, model="poisson.mixed")

z.out <- zelig(formula= list(mu=y ~ xl + x2 + tag(z1, gamma | g),
               gamma= ~ tag(w1 + w2 | g)), data=mydata, model="poisson.mixed")
\end{verbatim}

\subsubsection{Inputs}

\noindent {\tt zelig()} takes the following arguments for {\tt mixed}:
\begin{itemize}
\item {\tt formula:} a two-sided linear formula object describing the systematic component of the model, with the response on the left of a {\tt $\tilde{}$} operator and the fixed effects terms, separated by {\tt +} operators, on the right. Any random effects terms are included with the notation {\tt tag(z1 + ... + zn | g)} with {\tt z1 + ... + zn} specifying the model for the random effects and {\tt g} the grouping structure. Random intercept terms are included with the notation {\tt tag(1 | g)}. \\
Alternatively, {\tt formula} may be a list where the first entry, {\tt mu}, is a two-sided linear formula object describing the systematic component of the model, with the repsonse on the left of a {\tt $\tilde{}$} operator and the fixed effects terms, separated by {\tt +} operators, on the right. Any random effects terms are included with the notation {\tt tag(z1, gamma | g)} with {\tt z1} specifying the individual level model for the random effects, {\tt g} the grouping structure and {\tt gamma} references the second equation in the list. The {\tt gamma} equation is one-sided linear formula object with the group level model for the random effects on the right side of a {\tt $\tilde{}$} operator. The model is specified with the notation {\tt tag(w1 + ... + wn | g)} with {\tt w1 + ... + wn} specifying the group level model and {\tt g} the grouping structure.
\end{itemize}

\subsubsection{Additional Inputs}

In addition, {\tt zelig()} accepts the following additional arguments for model specification:

\begin{itemize}
\item {\tt data:} An optional data frame containing the variables named in {\tt formula}. By default, the variables are taken from the environment from which {\tt zelig()} is called.
\item {\tt na.action:} A function that indicates what should happen when the data contain {\tt NAs}. The default action ({\tt na.fail}) causes {\tt zelig()} to print an error message and terminate if there are any incomplete observations.
\end{itemize}
Additionally, users may with to refer to {\tt lmer} in the package {\tt lme4} for more information, including control parameters for the estimation algorithm and their defaults.

\subsubsection{Examples}

\begin{enumerate}
\item Basic Example \\
\\
Attach sample data: \\
\begin{Schunk}
\begin{Sinput}
RRR> data(homerun)
\end{Sinput}
\end{Schunk}
Estimate model:
\begin{Schunk}
\begin{Sinput}
RRR> z.out1 <- zelig(homeruns ~ player + tag(player - 1 | month), data=homerun, model="poisson.mixed")
\end{Sinput}
\end{Schunk}

\noindent Summarize regression coefficients and estimated variance of random effects:\\
\begin{Schunk}
\begin{Sinput}
RRR> summary(z.out1)
\end{Sinput}
\end{Schunk}
Set explanatory variables to their default values:\\
\begin{Schunk}
\begin{Sinput}
RRR> x.out <- setx(z.out1)
\end{Sinput}
\end{Schunk}
Simulate draws using the default bootstrap method and view simulated quantities of interest: \\
\begin{Schunk}
\begin{Sinput}
RRR> s.out1 <- sim(z.out1, x=x.out)
RRR> summary(s.out1)
\end{Sinput}
\end{Schunk}

\end{enumerate}


\subsubsection{Mixed effects Poisson Regression Model}

Let $Y_{ij}$ be the number of independent events that occur during a fixed time period, realized for observation $j$ in group $i$ as $y_{ij}$, which takes any non-negative integer as its value, for $i = 1, \ldots, M$, $j = 1, \ldots, n_i$.

\begin{itemize}
\item The \emph{stochastic component} is described by a Poisson distribution with mean and variance parameter $\lambda_{ij}$.
\begin{equation*}
Y_{ij} \sim \mathrm{Poisson}(y_{ij} | \lambda_{ij}) = \frac{\exp(-\lambda_{ij}) \lambda_{ij}^{y_{ij}}}{y_{ij}!}
\end{equation*}
where
\begin{equation*}
y_{ij} = 0, 1, \ldots
\end{equation*}
\item The $q$-dimensional vector of \emph{random effects}, $b_i$, is restricted to be mean zero, and therefore is completely characterized by the variance covarance matrix $\Psi$, a $(q \times q)$ symmetric positive semi-definite matrix.
\begin{equation*}
b_i \sim Normal(0, \Psi)
\end{equation*}
\item The \emph{systematic component} is
\begin{equation*}
\lambda_{ij} \equiv \exp(X_{ij} \beta + Z_{ij} b_i)
\end{equation*}
where $X_{ij}$ is the $(n_i \times p \times M)$ array of known fixed effects explanatory variables, $\beta$ is the $p$-dimensional vector of fixed effects coefficients, $Z_{ij}$ is the $(n_i \times q \times M)$ array of known random effects explanatory variables and $b_i$ is the $q$-dimensional vector of random effects.
\end{itemize}

\subsubsection{Quantities of Interest}

\begin{itemize}
\item The predicted values ({\tt qi\$pr}) are draws from the poisson distribution defined by mean $ \lambda_{ij} $, for
\begin{equation*}
\lambda_{ij} = \exp(X_{ij} \beta + Z_{ij} b_i)
\end{equation*}
given $X_{ij}$ and $Z_{ij}$ and simulations of of $\beta$ and $b_i$ from their posterior distributions. The estimated variance covariance matrices are taken as correct and are themselves not simulated.

\item The expected values ({\tt qi\$ev}) is the mean of simulations of the stochastic component given draws of $\beta$ from its posterior:
\begin{equation*}
E(Y_{ij} | X_{ij}) = \lambda_{ij} = \exp(X_{ij} \beta).
\end{equation*}

\item The first difference ({\tt qi\$fd}) is given by the difference in expected values, conditional on $X_{ij}$ and $X_{ij}^\prime$, representing different values of the explanatory variables.
\begin{equation*}
FD(Y_{ij} | X_{ij}, X_{ij}^\prime) = E(Y_{ij} | X_{ij}) - E(Y_{ij} | X_{ij}^\prime)
\end{equation*}

\item In conditional prediction models, the average predicted treatment effect ({\tt qi\$att.pr}) for the treatment group is given by
\begin{equation*}
\frac{1}{\sum_{i = 1}^M \sum_{j = 1}^{n_i} t_{ij}} \sum_{i = 1}^M \sum_{j:t_{ij} = 1}^{n_i} \{ Y_{ij} (t_{ij} = 1) - \widehat{Y_{ij}(t_{ij} = 0)} \},
\end{equation*}
where $t_{ij}$ is a binary explanatory variable defining the treatment $(t_{ij} = 1)$ and control $(t_{ij} = 0)$ groups. Variation in the simulations is due to uncertainty in simulating $Y_{ij}(t_{ij} = 0)$, the counterfactual predicted value of $Y_{ij}$ for observations in the treatment group, under the assumption that everything stays the same except that the treatment indicator is switched to $t_{ij} = 0$.

\item In conditional prediction models, the average expected treatment effect ({\tt qi\$att.ev}) for the treatment group is given by
\begin{equation*}
\frac{1}{\sum_{i = 1}^M \sum_{j = 1}^{n_i} t_{ij}} \sum_{i = 1}^M \sum_{j:t_{ij} = 1}^{n_i} \{ Y_{ij} (t_{ij} = 1) - E[Y_{ij}(t_{ij} = 0)] \},
\end{equation*}
where $t_{ij}$ is a binary explanatory variable defining the treatment $(t_{ij} = 1)$ and control $(t_{ij} = 0)$ groups. Variation in the simulations is due to uncertainty in simulating $E[Y_{ij}(t_{ij} = 0)]$, the counterfactual expected value of $Y_{ij}$ for observations in the treatment group, under the assumption that everything stays the same except that the treatment indicator is switched to $t_{ij} = 0$.

\end{itemize}

\subsubsection{Output Values}

The output of each Zelig command contains useful information which you may view. You may examine the available information in {\tt z.out} by using {\tt slotNames(z.out)}, see the fixed effect coefficients by using {\tt summary(z.out)@coefs}, and a default summary of information through {\tt summary(z.out)}. Other elements available through the {\tt \@} operator are listed below.
\begin{itemize}
\item From the {\tt zelig()} output stored in {\tt summary(z.out)}, you may extract:
\begin{itemize}
\item[--] {\tt fixef}: numeric vector containing the conditional estimates of the fixed effects.
\item[--] {\tt ranef}: numeric vector containing the conditional modes of the random effects.
\item[--] {\tt frame}: the model frame for the model.
\end{itemize}
\item From the {\tt sim()} output stored in {\tt s.out}, you may extract quantities of interest stored in a data frame:
\begin{itemize}
\item {\tt qi\$pr}: the simulated predicted values drawn from the distributions defined by the expected values.
\item {\tt qi\$ev}: the simulated expected values for the specified values of x.
\item {\tt qi\$fd}: the simulated first differences in the expected values for the values specified in x and x1.
\item {\tt qi\$ate.pr}: the simulated average predicted treatment effect for the treated from conditional prediction models.
\item {\tt qi\$ate.ev}: the simulated average expected treatment effect for the treated from conditional prediction models.
\end{itemize}
\end{itemize}



\subsection* {How to Cite}

\section{{\tt poisson.mixed}: Mixed effects poisson Regression}
\label{mixed}

Use generalized multi-level linear regression if you have covariates that are grouped according to one or more classification factors. Poisson regression applies to dependent variables that represent the number of independent events that occur during a fixed period of time.

While generally called multi-level models in the social sciences, this class of models is often referred to as mixed-effects models in the statistics literature and as hierarchical models in a Bayesian setting. This general class of models consists of linear models that are expressed as a function of both \emph{fixed effects}, parameters corresponding to an entire population or certain repeatable levels of experimental factors, and \emph{random effects}, parameters corresponding to individual experimental units drawn at random from a population.

\subsubsection{Syntax}

\begin{verbatim}
z.out <- zelig(formula= y ~ x1 + x2 + tag(z1 + z2 | g),
               data=mydata, model="poisson.mixed")

z.out <- zelig(formula= list(mu=y ~ xl + x2 + tag(z1, gamma | g),
               gamma= ~ tag(w1 + w2 | g)), data=mydata, model="poisson.mixed")
\end{verbatim}

\subsubsection{Inputs}

\noindent {\tt zelig()} takes the following arguments for {\tt mixed}:
\begin{itemize}
\item {\tt formula:} a two-sided linear formula object describing the systematic component of the model, with the response on the left of a {\tt $\tilde{}$} operator and the fixed effects terms, separated by {\tt +} operators, on the right. Any random effects terms are included with the notation {\tt tag(z1 + ... + zn | g)} with {\tt z1 + ... + zn} specifying the model for the random effects and {\tt g} the grouping structure. Random intercept terms are included with the notation {\tt tag(1 | g)}. \\
Alternatively, {\tt formula} may be a list where the first entry, {\tt mu}, is a two-sided linear formula object describing the systematic component of the model, with the repsonse on the left of a {\tt $\tilde{}$} operator and the fixed effects terms, separated by {\tt +} operators, on the right. Any random effects terms are included with the notation {\tt tag(z1, gamma | g)} with {\tt z1} specifying the individual level model for the random effects, {\tt g} the grouping structure and {\tt gamma} references the second equation in the list. The {\tt gamma} equation is one-sided linear formula object with the group level model for the random effects on the right side of a {\tt $\tilde{}$} operator. The model is specified with the notation {\tt tag(w1 + ... + wn | g)} with {\tt w1 + ... + wn} specifying the group level model and {\tt g} the grouping structure.
\end{itemize}

\subsubsection{Additional Inputs}

In addition, {\tt zelig()} accepts the following additional arguments for model specification:

\begin{itemize}
\item {\tt data:} An optional data frame containing the variables named in {\tt formula}. By default, the variables are taken from the environment from which {\tt zelig()} is called.
\item {\tt na.action:} A function that indicates what should happen when the data contain {\tt NAs}. The default action ({\tt na.fail}) causes {\tt zelig()} to print an error message and terminate if there are any incomplete observations.
\end{itemize}
Additionally, users may with to refer to {\tt lmer} in the package {\tt lme4} for more information, including control parameters for the estimation algorithm and their defaults.

\subsubsection{Examples}

\begin{enumerate}
\item Basic Example \\
\\
Attach sample data: \\
\begin{Schunk}
\begin{Sinput}
RRR> data(homerun)
\end{Sinput}
\end{Schunk}
Estimate model:
\begin{Schunk}
\begin{Sinput}
RRR> z.out1 <- zelig(homeruns ~ player + tag(player - 1 | month), data=homerun, model="poisson.mixed")
\end{Sinput}
\end{Schunk}

\noindent Summarize regression coefficients and estimated variance of random effects:\\
\begin{Schunk}
\begin{Sinput}
RRR> summary(z.out1)
\end{Sinput}
\end{Schunk}
Set explanatory variables to their default values:\\
\begin{Schunk}
\begin{Sinput}
RRR> x.out <- setx(z.out1)
\end{Sinput}
\end{Schunk}
Simulate draws using the default bootstrap method and view simulated quantities of interest: \\
\begin{Schunk}
\begin{Sinput}
RRR> s.out1 <- sim(z.out1, x=x.out)
RRR> summary(s.out1)
\end{Sinput}
\end{Schunk}

\end{enumerate}


\subsubsection{Mixed effects Poisson Regression Model}

Let $Y_{ij}$ be the number of independent events that occur during a fixed time period, realized for observation $j$ in group $i$ as $y_{ij}$, which takes any non-negative integer as its value, for $i = 1, \ldots, M$, $j = 1, \ldots, n_i$.

\begin{itemize}
\item The \emph{stochastic component} is described by a Poisson distribution with mean and variance parameter $\lambda_{ij}$.
\begin{equation*}
Y_{ij} \sim \mathrm{Poisson}(y_{ij} | \lambda_{ij}) = \frac{\exp(-\lambda_{ij}) \lambda_{ij}^{y_{ij}}}{y_{ij}!}
\end{equation*}
where
\begin{equation*}
y_{ij} = 0, 1, \ldots
\end{equation*}
\item The $q$-dimensional vector of \emph{random effects}, $b_i$, is restricted to be mean zero, and therefore is completely characterized by the variance covarance matrix $\Psi$, a $(q \times q)$ symmetric positive semi-definite matrix.
\begin{equation*}
b_i \sim Normal(0, \Psi)
\end{equation*}
\item The \emph{systematic component} is
\begin{equation*}
\lambda_{ij} \equiv \exp(X_{ij} \beta + Z_{ij} b_i)
\end{equation*}
where $X_{ij}$ is the $(n_i \times p \times M)$ array of known fixed effects explanatory variables, $\beta$ is the $p$-dimensional vector of fixed effects coefficients, $Z_{ij}$ is the $(n_i \times q \times M)$ array of known random effects explanatory variables and $b_i$ is the $q$-dimensional vector of random effects.
\end{itemize}

\subsubsection{Quantities of Interest}

\begin{itemize}
\item The predicted values ({\tt qi\$pr}) are draws from the poisson distribution defined by mean $ \lambda_{ij} $, for
\begin{equation*}
\lambda_{ij} = \exp(X_{ij} \beta + Z_{ij} b_i)
\end{equation*}
given $X_{ij}$ and $Z_{ij}$ and simulations of of $\beta$ and $b_i$ from their posterior distributions. The estimated variance covariance matrices are taken as correct and are themselves not simulated.

\item The expected values ({\tt qi\$ev}) is the mean of simulations of the stochastic component given draws of $\beta$ from its posterior:
\begin{equation*}
E(Y_{ij} | X_{ij}) = \lambda_{ij} = \exp(X_{ij} \beta).
\end{equation*}

\item The first difference ({\tt qi\$fd}) is given by the difference in expected values, conditional on $X_{ij}$ and $X_{ij}^\prime$, representing different values of the explanatory variables.
\begin{equation*}
FD(Y_{ij} | X_{ij}, X_{ij}^\prime) = E(Y_{ij} | X_{ij}) - E(Y_{ij} | X_{ij}^\prime)
\end{equation*}

\item In conditional prediction models, the average predicted treatment effect ({\tt qi\$att.pr}) for the treatment group is given by
\begin{equation*}
\frac{1}{\sum_{i = 1}^M \sum_{j = 1}^{n_i} t_{ij}} \sum_{i = 1}^M \sum_{j:t_{ij} = 1}^{n_i} \{ Y_{ij} (t_{ij} = 1) - \widehat{Y_{ij}(t_{ij} = 0)} \},
\end{equation*}
where $t_{ij}$ is a binary explanatory variable defining the treatment $(t_{ij} = 1)$ and control $(t_{ij} = 0)$ groups. Variation in the simulations is due to uncertainty in simulating $Y_{ij}(t_{ij} = 0)$, the counterfactual predicted value of $Y_{ij}$ for observations in the treatment group, under the assumption that everything stays the same except that the treatment indicator is switched to $t_{ij} = 0$.

\item In conditional prediction models, the average expected treatment effect ({\tt qi\$att.ev}) for the treatment group is given by
\begin{equation*}
\frac{1}{\sum_{i = 1}^M \sum_{j = 1}^{n_i} t_{ij}} \sum_{i = 1}^M \sum_{j:t_{ij} = 1}^{n_i} \{ Y_{ij} (t_{ij} = 1) - E[Y_{ij}(t_{ij} = 0)] \},
\end{equation*}
where $t_{ij}$ is a binary explanatory variable defining the treatment $(t_{ij} = 1)$ and control $(t_{ij} = 0)$ groups. Variation in the simulations is due to uncertainty in simulating $E[Y_{ij}(t_{ij} = 0)]$, the counterfactual expected value of $Y_{ij}$ for observations in the treatment group, under the assumption that everything stays the same except that the treatment indicator is switched to $t_{ij} = 0$.

\end{itemize}

\subsubsection{Output Values}

The output of each Zelig command contains useful information which you may view. You may examine the available information in {\tt z.out} by using {\tt slotNames(z.out)}, see the fixed effect coefficients by using {\tt summary(z.out)@coefs}, and a default summary of information through {\tt summary(z.out)}. Other elements available through the {\tt \@} operator are listed below.
\begin{itemize}
\item From the {\tt zelig()} output stored in {\tt summary(z.out)}, you may extract:
\begin{itemize}
\item[--] {\tt fixef}: numeric vector containing the conditional estimates of the fixed effects.
\item[--] {\tt ranef}: numeric vector containing the conditional modes of the random effects.
\item[--] {\tt frame}: the model frame for the model.
\end{itemize}
\item From the {\tt sim()} output stored in {\tt s.out}, you may extract quantities of interest stored in a data frame:
\begin{itemize}
\item {\tt qi\$pr}: the simulated predicted values drawn from the distributions defined by the expected values.
\item {\tt qi\$ev}: the simulated expected values for the specified values of x.
\item {\tt qi\$fd}: the simulated first differences in the expected values for the values specified in x and x1.
\item {\tt qi\$ate.pr}: the simulated average predicted treatment effect for the treated from conditional prediction models.
\item {\tt qi\$ate.ev}: the simulated average expected treatment effect for the treated from conditional prediction models.
\end{itemize}
\end{itemize}



\subsection* {How to Cite}

\section{{\tt poisson.mixed}: Mixed effects poisson Regression}
\label{mixed}

Use generalized multi-level linear regression if you have covariates that are grouped according to one or more classification factors. Poisson regression applies to dependent variables that represent the number of independent events that occur during a fixed period of time.

While generally called multi-level models in the social sciences, this class of models is often referred to as mixed-effects models in the statistics literature and as hierarchical models in a Bayesian setting. This general class of models consists of linear models that are expressed as a function of both \emph{fixed effects}, parameters corresponding to an entire population or certain repeatable levels of experimental factors, and \emph{random effects}, parameters corresponding to individual experimental units drawn at random from a population.

\subsubsection{Syntax}

\begin{verbatim}
z.out <- zelig(formula= y ~ x1 + x2 + tag(z1 + z2 | g),
               data=mydata, model="poisson.mixed")

z.out <- zelig(formula= list(mu=y ~ xl + x2 + tag(z1, gamma | g),
               gamma= ~ tag(w1 + w2 | g)), data=mydata, model="poisson.mixed")
\end{verbatim}

\subsubsection{Inputs}

\noindent {\tt zelig()} takes the following arguments for {\tt mixed}:
\begin{itemize}
\item {\tt formula:} a two-sided linear formula object describing the systematic component of the model, with the response on the left of a {\tt $\tilde{}$} operator and the fixed effects terms, separated by {\tt +} operators, on the right. Any random effects terms are included with the notation {\tt tag(z1 + ... + zn | g)} with {\tt z1 + ... + zn} specifying the model for the random effects and {\tt g} the grouping structure. Random intercept terms are included with the notation {\tt tag(1 | g)}. \\
Alternatively, {\tt formula} may be a list where the first entry, {\tt mu}, is a two-sided linear formula object describing the systematic component of the model, with the repsonse on the left of a {\tt $\tilde{}$} operator and the fixed effects terms, separated by {\tt +} operators, on the right. Any random effects terms are included with the notation {\tt tag(z1, gamma | g)} with {\tt z1} specifying the individual level model for the random effects, {\tt g} the grouping structure and {\tt gamma} references the second equation in the list. The {\tt gamma} equation is one-sided linear formula object with the group level model for the random effects on the right side of a {\tt $\tilde{}$} operator. The model is specified with the notation {\tt tag(w1 + ... + wn | g)} with {\tt w1 + ... + wn} specifying the group level model and {\tt g} the grouping structure.
\end{itemize}

\subsubsection{Additional Inputs}

In addition, {\tt zelig()} accepts the following additional arguments for model specification:

\begin{itemize}
\item {\tt data:} An optional data frame containing the variables named in {\tt formula}. By default, the variables are taken from the environment from which {\tt zelig()} is called.
\item {\tt na.action:} A function that indicates what should happen when the data contain {\tt NAs}. The default action ({\tt na.fail}) causes {\tt zelig()} to print an error message and terminate if there are any incomplete observations.
\end{itemize}
Additionally, users may with to refer to {\tt lmer} in the package {\tt lme4} for more information, including control parameters for the estimation algorithm and their defaults.

\subsubsection{Examples}

\begin{enumerate}
\item Basic Example \\
\\
Attach sample data: \\
\begin{Schunk}
\begin{Sinput}
RRR> data(homerun)
\end{Sinput}
\end{Schunk}
Estimate model:
\begin{Schunk}
\begin{Sinput}
RRR> z.out1 <- zelig(homeruns ~ player + tag(player - 1 | month), data=homerun, model="poisson.mixed")
\end{Sinput}
\end{Schunk}

\noindent Summarize regression coefficients and estimated variance of random effects:\\
\begin{Schunk}
\begin{Sinput}
RRR> summary(z.out1)
\end{Sinput}
\end{Schunk}
Set explanatory variables to their default values:\\
\begin{Schunk}
\begin{Sinput}
RRR> x.out <- setx(z.out1)
\end{Sinput}
\end{Schunk}
Simulate draws using the default bootstrap method and view simulated quantities of interest: \\
\begin{Schunk}
\begin{Sinput}
RRR> s.out1 <- sim(z.out1, x=x.out)
RRR> summary(s.out1)
\end{Sinput}
\end{Schunk}

\end{enumerate}


\subsubsection{Mixed effects Poisson Regression Model}

Let $Y_{ij}$ be the number of independent events that occur during a fixed time period, realized for observation $j$ in group $i$ as $y_{ij}$, which takes any non-negative integer as its value, for $i = 1, \ldots, M$, $j = 1, \ldots, n_i$.

\begin{itemize}
\item The \emph{stochastic component} is described by a Poisson distribution with mean and variance parameter $\lambda_{ij}$.
\begin{equation*}
Y_{ij} \sim \mathrm{Poisson}(y_{ij} | \lambda_{ij}) = \frac{\exp(-\lambda_{ij}) \lambda_{ij}^{y_{ij}}}{y_{ij}!}
\end{equation*}
where
\begin{equation*}
y_{ij} = 0, 1, \ldots
\end{equation*}
\item The $q$-dimensional vector of \emph{random effects}, $b_i$, is restricted to be mean zero, and therefore is completely characterized by the variance covarance matrix $\Psi$, a $(q \times q)$ symmetric positive semi-definite matrix.
\begin{equation*}
b_i \sim Normal(0, \Psi)
\end{equation*}
\item The \emph{systematic component} is
\begin{equation*}
\lambda_{ij} \equiv \exp(X_{ij} \beta + Z_{ij} b_i)
\end{equation*}
where $X_{ij}$ is the $(n_i \times p \times M)$ array of known fixed effects explanatory variables, $\beta$ is the $p$-dimensional vector of fixed effects coefficients, $Z_{ij}$ is the $(n_i \times q \times M)$ array of known random effects explanatory variables and $b_i$ is the $q$-dimensional vector of random effects.
\end{itemize}

\subsubsection{Quantities of Interest}

\begin{itemize}
\item The predicted values ({\tt qi\$pr}) are draws from the poisson distribution defined by mean $ \lambda_{ij} $, for
\begin{equation*}
\lambda_{ij} = \exp(X_{ij} \beta + Z_{ij} b_i)
\end{equation*}
given $X_{ij}$ and $Z_{ij}$ and simulations of of $\beta$ and $b_i$ from their posterior distributions. The estimated variance covariance matrices are taken as correct and are themselves not simulated.

\item The expected values ({\tt qi\$ev}) is the mean of simulations of the stochastic component given draws of $\beta$ from its posterior:
\begin{equation*}
E(Y_{ij} | X_{ij}) = \lambda_{ij} = \exp(X_{ij} \beta).
\end{equation*}

\item The first difference ({\tt qi\$fd}) is given by the difference in expected values, conditional on $X_{ij}$ and $X_{ij}^\prime$, representing different values of the explanatory variables.
\begin{equation*}
FD(Y_{ij} | X_{ij}, X_{ij}^\prime) = E(Y_{ij} | X_{ij}) - E(Y_{ij} | X_{ij}^\prime)
\end{equation*}

\item In conditional prediction models, the average predicted treatment effect ({\tt qi\$att.pr}) for the treatment group is given by
\begin{equation*}
\frac{1}{\sum_{i = 1}^M \sum_{j = 1}^{n_i} t_{ij}} \sum_{i = 1}^M \sum_{j:t_{ij} = 1}^{n_i} \{ Y_{ij} (t_{ij} = 1) - \widehat{Y_{ij}(t_{ij} = 0)} \},
\end{equation*}
where $t_{ij}$ is a binary explanatory variable defining the treatment $(t_{ij} = 1)$ and control $(t_{ij} = 0)$ groups. Variation in the simulations is due to uncertainty in simulating $Y_{ij}(t_{ij} = 0)$, the counterfactual predicted value of $Y_{ij}$ for observations in the treatment group, under the assumption that everything stays the same except that the treatment indicator is switched to $t_{ij} = 0$.

\item In conditional prediction models, the average expected treatment effect ({\tt qi\$att.ev}) for the treatment group is given by
\begin{equation*}
\frac{1}{\sum_{i = 1}^M \sum_{j = 1}^{n_i} t_{ij}} \sum_{i = 1}^M \sum_{j:t_{ij} = 1}^{n_i} \{ Y_{ij} (t_{ij} = 1) - E[Y_{ij}(t_{ij} = 0)] \},
\end{equation*}
where $t_{ij}$ is a binary explanatory variable defining the treatment $(t_{ij} = 1)$ and control $(t_{ij} = 0)$ groups. Variation in the simulations is due to uncertainty in simulating $E[Y_{ij}(t_{ij} = 0)]$, the counterfactual expected value of $Y_{ij}$ for observations in the treatment group, under the assumption that everything stays the same except that the treatment indicator is switched to $t_{ij} = 0$.

\end{itemize}

\subsubsection{Output Values}

The output of each Zelig command contains useful information which you may view. You may examine the available information in {\tt z.out} by using {\tt slotNames(z.out)}, see the fixed effect coefficients by using {\tt summary(z.out)@coefs}, and a default summary of information through {\tt summary(z.out)}. Other elements available through the {\tt \@} operator are listed below.
\begin{itemize}
\item From the {\tt zelig()} output stored in {\tt summary(z.out)}, you may extract:
\begin{itemize}
\item[--] {\tt fixef}: numeric vector containing the conditional estimates of the fixed effects.
\item[--] {\tt ranef}: numeric vector containing the conditional modes of the random effects.
\item[--] {\tt frame}: the model frame for the model.
\end{itemize}
\item From the {\tt sim()} output stored in {\tt s.out}, you may extract quantities of interest stored in a data frame:
\begin{itemize}
\item {\tt qi\$pr}: the simulated predicted values drawn from the distributions defined by the expected values.
\item {\tt qi\$ev}: the simulated expected values for the specified values of x.
\item {\tt qi\$fd}: the simulated first differences in the expected values for the values specified in x and x1.
\item {\tt qi\$ate.pr}: the simulated average predicted treatment effect for the treated from conditional prediction models.
\item {\tt qi\$ate.ev}: the simulated average expected treatment effect for the treated from conditional prediction models.
\end{itemize}
\end{itemize}



\subsection* {How to Cite}

\section{{\tt poisson.mixed}: Mixed effects poisson Regression}
\label{mixed}

Use generalized multi-level linear regression if you have covariates that are grouped according to one or more classification factors. Poisson regression applies to dependent variables that represent the number of independent events that occur during a fixed period of time.

While generally called multi-level models in the social sciences, this class of models is often referred to as mixed-effects models in the statistics literature and as hierarchical models in a Bayesian setting. This general class of models consists of linear models that are expressed as a function of both \emph{fixed effects}, parameters corresponding to an entire population or certain repeatable levels of experimental factors, and \emph{random effects}, parameters corresponding to individual experimental units drawn at random from a population.

\subsubsection{Syntax}

\begin{verbatim}
z.out <- zelig(formula= y ~ x1 + x2 + tag(z1 + z2 | g),
               data=mydata, model="poisson.mixed")

z.out <- zelig(formula= list(mu=y ~ xl + x2 + tag(z1, gamma | g),
               gamma= ~ tag(w1 + w2 | g)), data=mydata, model="poisson.mixed")
\end{verbatim}

\subsubsection{Inputs}

\noindent {\tt zelig()} takes the following arguments for {\tt mixed}:
\begin{itemize}
\item {\tt formula:} a two-sided linear formula object describing the systematic component of the model, with the response on the left of a {\tt $\tilde{}$} operator and the fixed effects terms, separated by {\tt +} operators, on the right. Any random effects terms are included with the notation {\tt tag(z1 + ... + zn | g)} with {\tt z1 + ... + zn} specifying the model for the random effects and {\tt g} the grouping structure. Random intercept terms are included with the notation {\tt tag(1 | g)}. \\
Alternatively, {\tt formula} may be a list where the first entry, {\tt mu}, is a two-sided linear formula object describing the systematic component of the model, with the repsonse on the left of a {\tt $\tilde{}$} operator and the fixed effects terms, separated by {\tt +} operators, on the right. Any random effects terms are included with the notation {\tt tag(z1, gamma | g)} with {\tt z1} specifying the individual level model for the random effects, {\tt g} the grouping structure and {\tt gamma} references the second equation in the list. The {\tt gamma} equation is one-sided linear formula object with the group level model for the random effects on the right side of a {\tt $\tilde{}$} operator. The model is specified with the notation {\tt tag(w1 + ... + wn | g)} with {\tt w1 + ... + wn} specifying the group level model and {\tt g} the grouping structure.
\end{itemize}

\subsubsection{Additional Inputs}

In addition, {\tt zelig()} accepts the following additional arguments for model specification:

\begin{itemize}
\item {\tt data:} An optional data frame containing the variables named in {\tt formula}. By default, the variables are taken from the environment from which {\tt zelig()} is called.
\item {\tt na.action:} A function that indicates what should happen when the data contain {\tt NAs}. The default action ({\tt na.fail}) causes {\tt zelig()} to print an error message and terminate if there are any incomplete observations.
\end{itemize}
Additionally, users may with to refer to {\tt lmer} in the package {\tt lme4} for more information, including control parameters for the estimation algorithm and their defaults.

\subsubsection{Examples}

\begin{enumerate}
\item Basic Example \\
\\
Attach sample data: \\
\begin{Schunk}
\begin{Sinput}
RRR> data(homerun)
\end{Sinput}
\end{Schunk}
Estimate model:
\begin{Schunk}
\begin{Sinput}
RRR> z.out1 <- zelig(homeruns ~ player + tag(player - 1 | month), data=homerun, model="poisson.mixed")
\end{Sinput}
\end{Schunk}

\noindent Summarize regression coefficients and estimated variance of random effects:\\
\begin{Schunk}
\begin{Sinput}
RRR> summary(z.out1)
\end{Sinput}
\end{Schunk}
Set explanatory variables to their default values:\\
\begin{Schunk}
\begin{Sinput}
RRR> x.out <- setx(z.out1)
\end{Sinput}
\end{Schunk}
Simulate draws using the default bootstrap method and view simulated quantities of interest: \\
\begin{Schunk}
\begin{Sinput}
RRR> s.out1 <- sim(z.out1, x=x.out)
RRR> summary(s.out1)
\end{Sinput}
\end{Schunk}

\end{enumerate}


\subsubsection{Mixed effects Poisson Regression Model}

Let $Y_{ij}$ be the number of independent events that occur during a fixed time period, realized for observation $j$ in group $i$ as $y_{ij}$, which takes any non-negative integer as its value, for $i = 1, \ldots, M$, $j = 1, \ldots, n_i$.

\begin{itemize}
\item The \emph{stochastic component} is described by a Poisson distribution with mean and variance parameter $\lambda_{ij}$.
\begin{equation*}
Y_{ij} \sim \mathrm{Poisson}(y_{ij} | \lambda_{ij}) = \frac{\exp(-\lambda_{ij}) \lambda_{ij}^{y_{ij}}}{y_{ij}!}
\end{equation*}
where
\begin{equation*}
y_{ij} = 0, 1, \ldots
\end{equation*}
\item The $q$-dimensional vector of \emph{random effects}, $b_i$, is restricted to be mean zero, and therefore is completely characterized by the variance covarance matrix $\Psi$, a $(q \times q)$ symmetric positive semi-definite matrix.
\begin{equation*}
b_i \sim Normal(0, \Psi)
\end{equation*}
\item The \emph{systematic component} is
\begin{equation*}
\lambda_{ij} \equiv \exp(X_{ij} \beta + Z_{ij} b_i)
\end{equation*}
where $X_{ij}$ is the $(n_i \times p \times M)$ array of known fixed effects explanatory variables, $\beta$ is the $p$-dimensional vector of fixed effects coefficients, $Z_{ij}$ is the $(n_i \times q \times M)$ array of known random effects explanatory variables and $b_i$ is the $q$-dimensional vector of random effects.
\end{itemize}

\subsubsection{Quantities of Interest}

\begin{itemize}
\item The predicted values ({\tt qi\$pr}) are draws from the poisson distribution defined by mean $ \lambda_{ij} $, for
\begin{equation*}
\lambda_{ij} = \exp(X_{ij} \beta + Z_{ij} b_i)
\end{equation*}
given $X_{ij}$ and $Z_{ij}$ and simulations of of $\beta$ and $b_i$ from their posterior distributions. The estimated variance covariance matrices are taken as correct and are themselves not simulated.

\item The expected values ({\tt qi\$ev}) is the mean of simulations of the stochastic component given draws of $\beta$ from its posterior:
\begin{equation*}
E(Y_{ij} | X_{ij}) = \lambda_{ij} = \exp(X_{ij} \beta).
\end{equation*}

\item The first difference ({\tt qi\$fd}) is given by the difference in expected values, conditional on $X_{ij}$ and $X_{ij}^\prime$, representing different values of the explanatory variables.
\begin{equation*}
FD(Y_{ij} | X_{ij}, X_{ij}^\prime) = E(Y_{ij} | X_{ij}) - E(Y_{ij} | X_{ij}^\prime)
\end{equation*}

\item In conditional prediction models, the average predicted treatment effect ({\tt qi\$att.pr}) for the treatment group is given by
\begin{equation*}
\frac{1}{\sum_{i = 1}^M \sum_{j = 1}^{n_i} t_{ij}} \sum_{i = 1}^M \sum_{j:t_{ij} = 1}^{n_i} \{ Y_{ij} (t_{ij} = 1) - \widehat{Y_{ij}(t_{ij} = 0)} \},
\end{equation*}
where $t_{ij}$ is a binary explanatory variable defining the treatment $(t_{ij} = 1)$ and control $(t_{ij} = 0)$ groups. Variation in the simulations is due to uncertainty in simulating $Y_{ij}(t_{ij} = 0)$, the counterfactual predicted value of $Y_{ij}$ for observations in the treatment group, under the assumption that everything stays the same except that the treatment indicator is switched to $t_{ij} = 0$.

\item In conditional prediction models, the average expected treatment effect ({\tt qi\$att.ev}) for the treatment group is given by
\begin{equation*}
\frac{1}{\sum_{i = 1}^M \sum_{j = 1}^{n_i} t_{ij}} \sum_{i = 1}^M \sum_{j:t_{ij} = 1}^{n_i} \{ Y_{ij} (t_{ij} = 1) - E[Y_{ij}(t_{ij} = 0)] \},
\end{equation*}
where $t_{ij}$ is a binary explanatory variable defining the treatment $(t_{ij} = 1)$ and control $(t_{ij} = 0)$ groups. Variation in the simulations is due to uncertainty in simulating $E[Y_{ij}(t_{ij} = 0)]$, the counterfactual expected value of $Y_{ij}$ for observations in the treatment group, under the assumption that everything stays the same except that the treatment indicator is switched to $t_{ij} = 0$.

\end{itemize}

\subsubsection{Output Values}

The output of each Zelig command contains useful information which you may view. You may examine the available information in {\tt z.out} by using {\tt slotNames(z.out)}, see the fixed effect coefficients by using {\tt summary(z.out)@coefs}, and a default summary of information through {\tt summary(z.out)}. Other elements available through the {\tt \@} operator are listed below.
\begin{itemize}
\item From the {\tt zelig()} output stored in {\tt summary(z.out)}, you may extract:
\begin{itemize}
\item[--] {\tt fixef}: numeric vector containing the conditional estimates of the fixed effects.
\item[--] {\tt ranef}: numeric vector containing the conditional modes of the random effects.
\item[--] {\tt frame}: the model frame for the model.
\end{itemize}
\item From the {\tt sim()} output stored in {\tt s.out}, you may extract quantities of interest stored in a data frame:
\begin{itemize}
\item {\tt qi\$pr}: the simulated predicted values drawn from the distributions defined by the expected values.
\item {\tt qi\$ev}: the simulated expected values for the specified values of x.
\item {\tt qi\$fd}: the simulated first differences in the expected values for the values specified in x and x1.
\item {\tt qi\$ate.pr}: the simulated average predicted treatment effect for the treated from conditional prediction models.
\item {\tt qi\$ate.ev}: the simulated average expected treatment effect for the treated from conditional prediction models.
\end{itemize}
\end{itemize}



\subsection* {How to Cite}

\input{cites/poisson.mixed}
\input{citeZelig}

\subsection* {See also}
Mixed effects poisson regression is part of {\tt lme4} package by Douglas M. Bates \citep{Bates07}.

To cite Zelig as a whole, please reference these two sources:
\begin{verse}
  Kosuke Imai, Gary King, and Olivia Lau. 2007. ``Zelig: Everyone's
  Statistical Software,'' \url{http://GKing.harvard.edu/zelig}.
\end{verse}
\begin{verse}
Imai, Kosuke, Gary King, and Olivia Lau. (2008). ``Toward A Common Framework for Statistical Analysis and Development.'' Journal of Computational and Graphical Statistics, Vol. 17, No. 4 (December), pp. 892-913. 
\end{verse}


\subsection* {See also}
Mixed effects poisson regression is part of {\tt lme4} package by Douglas M. Bates \citep{Bates07}.

To cite Zelig as a whole, please reference these two sources:
\begin{verse}
  Kosuke Imai, Gary King, and Olivia Lau. 2007. ``Zelig: Everyone's
  Statistical Software,'' \url{http://GKing.harvard.edu/zelig}.
\end{verse}
\begin{verse}
Imai, Kosuke, Gary King, and Olivia Lau. (2008). ``Toward A Common Framework for Statistical Analysis and Development.'' Journal of Computational and Graphical Statistics, Vol. 17, No. 4 (December), pp. 892-913. 
\end{verse}


\subsection* {See also}
Mixed effects poisson regression is part of {\tt lme4} package by Douglas M. Bates \citep{Bates07}.

To cite Zelig as a whole, please reference these two sources:
\begin{verse}
  Kosuke Imai, Gary King, and Olivia Lau. 2007. ``Zelig: Everyone's
  Statistical Software,'' \url{http://GKing.harvard.edu/zelig}.
\end{verse}
\begin{verse}
Imai, Kosuke, Gary King, and Olivia Lau. (2008). ``Toward A Common Framework for Statistical Analysis and Development.'' Journal of Computational and Graphical Statistics, Vol. 17, No. 4 (December), pp. 892-913. 
\end{verse}


\subsection* {See also}
Mixed effects poisson regression is part of {\tt lme4} package by Douglas M. Bates \citep{Bates07}.
