\section{{\tt aov}: Analysis of Variance for Continuous Dependent Variables} 
\label{aov}

Model ``aov'' uses least squares regression to estimate the residual
sum of squares and degrees of freedom for each explanatory variable
around the best linear predictor for the specified dependent
variables. Model ``aov'' is particularly useful for the analysis of
randomized experiments with more than one strata or group (e.g.,
balanced incomplete block design).  For the model with only one
strata, ``aov'' reduces to ``ls''.

\subsubsection{Syntax}

\begin{verbatim}
> z.out <- zelig(Y ~ X1 + X2 + Error(Z), model = "aov", data = mydata)
> x.out <- setx(z.out)
> s.out <- sim(z.out, x = x.out)
\end{verbatim}
where the {\tt Error()} term is optional and takes strata formula.

\subsubsection{Examples}\begin{enumerate}

\item Basic Example of aov.

Attach sample data and set orthogonal contrasts:
\begin{Schunk}
\begin{Sinput}
RRR>  data(npk, package="MASS")
RRR>  op <- options(contrasts=c("contr.helmert", "contr.poly"))
\end{Sinput}
\end{Schunk}
Estimate the model \citep[p.165]{VenRip02}:
\begin{Schunk}
\begin{Sinput}
RRR>  z.out1 <- zelig(yield ~ block + N * P + K, model="aov", data=npk)
\end{Sinput}
\end{Schunk}
Summarize the fitted model:
\begin{Schunk}
\begin{Sinput}
RRR>  summary(z.out1)
\end{Sinput}
\end{Schunk}
Set explanatory variables to their default (mean/mode) values
\begin{Schunk}
\begin{Sinput}
RRR>  x <- setx(z.out1)
\end{Sinput}
\end{Schunk}
Simulate model at values explanatory variables as in x
\begin{Schunk}
\begin{Sinput}
RRR>  s.out1 <- sim(z.out1, x = x)
\end{Sinput}
\end{Schunk}
\begin{Schunk}
\begin{Sinput}
RRR> summary(s.out1)
\end{Sinput}
\end{Schunk}
\begin{center}
\begin{Schunk}
\begin{Sinput}
RRR>  plot(s.out1)
\end{Sinput}
\end{Schunk}
\includegraphics{vigpics/aov-ExamplesPlot1}
\end{center}

\item Example with {\tt Error()} term allowing for more than one
  source of random variation in an experiment (multistratum model). 

Estimate the model:
\begin{Schunk}
\begin{Sinput}
RRR>  z.out2 <- zelig(yield ~  N*P*K + Error(block), model="aov",data=npk)
\end{Sinput}
\end{Schunk}
Summarize regression coefficients:
\begin{Schunk}
\begin{Sinput}
RRR>  summary(z.out2)
\end{Sinput}
\end{Schunk}
Set explanatory variables to their default (mean/mode) values
\begin{Schunk}
\begin{Sinput}
RRR>  x <- setx(z.out2)
\end{Sinput}
\end{Schunk}
Simulate model at values explanatory variables as in x
\begin{Schunk}
\begin{Sinput}
RRR>  s.out2 <- sim(z.out2, x = x)
\end{Sinput}
\end{Schunk}
\begin{Schunk}
\begin{Sinput}
RRR>  summary(s.out2)
\end{Sinput}
\end{Schunk}
\begin{center}
\begin{Schunk}
\begin{Sinput}
RRR>  plot(s.out2)
\end{Sinput}
\end{Schunk}
\includegraphics{vigpics/aov-ExamplesPlot2}
\end{center}

\item Example with {\it Error()} term (multistratum model) and first
  differences. 

Reset to previous contrasts 
\begin{Schunk}
\begin{Sinput}
RRR> options(op)
\end{Sinput}
\end{Schunk}
Estimate model \citep[p.283]{VenRip02}:
\begin{Schunk}
\begin{Sinput}
RRR>  z.out3 <- zelig(Y ~ N*V + Error(B/V), model="aov", data=oats)
\end{Sinput}
\end{Schunk}
Summarize regression coefficients:
\begin{Schunk}
\begin{Sinput}
RRR>  summary(z.out3)
\end{Sinput}
\end{Schunk}
Set explanatory variables using mode
\begin{Schunk}
\begin{Sinput}
RRR>  x.out <- setx(z.out3, N="0.0cwt", V="Golden.rain")
RRR>  x.out1 <- setx(z.out3, N="0.0cwt", V="Victory")
\end{Sinput}
\end{Schunk}
Simulate model at values explanatory variables as in x
\begin{Schunk}
\begin{Sinput}
RRR>  s.out3 <- sim(z.out3, x = x.out,x1=x.out1)
\end{Sinput}
\end{Schunk}
\begin{Schunk}
\begin{Sinput}
RRR> summary(s.out3)
\end{Sinput}
\end{Schunk}
\begin{center}
\begin{Schunk}
\begin{Sinput}
RRR>  plot(s.out3)
\end{Sinput}
\end{Schunk}
\includegraphics{vigpics/aov-ExamplesPlot3}
\end{center}

\end{enumerate}

\subsubsection{Model}
\begin{itemize}
\item The \emph{stochastic component} is described by a normal density
  with mean $\mu_i$ and the common variance $\sigma^2$
  \begin{equation*}
    Y_i \; \sim \; f(y_i \mid \mu_i, \sigma^2).
  \end{equation*}
\item The \emph{systematic component} models the conditional mean as
  \begin{equation*}
     \mu_i =  x_i \beta
  \end{equation*} 
  where $x_i$ is the vector of covariates, and $\beta$ is the vector
  of coefficients.
  
  The least squares estimator is the best linear predictor of a
  dependent variable given $x_i$, and minimizes the sum of squared
  residuals, $\sum_{i=1}^n (Y_i-x_i \beta)^2$. The output of {\tt aov}
  model is the sum  of squared residuals. Note that {\tt aov} is the
  same model as {\tt ls} but the output values of function call   
  {\tt zelig} are different. You may check that {\tt name(z.out)}
  returns the same arguments for the two models.    
\end{itemize}

\subsubsection{Quantities of Interest} 
\begin{itemize}
\item The expected value ({\tt qi\$ev}) is the mean of simulations
  from the stochastic component,  
\begin{equation*}
E(Y) = x_i \beta,\end{equation*}
given a draw of $\beta$ from its sampling distribution.  

\item In conditional prediction models, the average expected treatment
  effect ({\tt att.ev}) for the treatment group is 
    \begin{equation*} \frac{1}{\sum_{i=1}^n t_i}\sum_{i:t_i=1}^n \left\{ Y_i(t_i=1) -
      E[Y_i(t_i=0)] \right\},
    \end{equation*} 
    where $t_i$ is a binary explanatory variable defining the treatment
    ($t_i=1$) and control ($t_i=0$) groups.  Variation in the
    simulations are due to uncertainty in simulating $E[Y_i(t_i=0)]$,
    the counterfactual expected value of $Y_i$ for observations in the
    treatment group, under the assumption that everything stays the
    same except that the treatment indicator is switched to $t_i=0$.

\end{itemize}

\subsubsection{Output Values}

The output of each Zelig command contains useful information which you
may view.  For example, if you run \texttt{z.out <- zelig(y \~\,
  x, model = "aov", data)}, then you may examine the available
information in \texttt{z.out} by using \texttt{names(z.out)},
see the {\tt coefficients} by using {\tt z.out\$coefficients}, and
a default summary of information through \texttt{summary(z.out)}.
Other elements available through the {\tt \$} operator are listed
below.

\begin{itemize}
  \item From the {\tt zelig()} output object {\tt z.out}, you may
  extract:
   \begin{itemize}
   \item {\tt coefficients}: parameter estimates for the explanatory
     variables.
   \item {\tt residuals}: the working residuals in the final iteration
     of the IWLS fit.
   \item {\tt fitted.values}: fitted values.
   \item {\tt df.residual}: the residual degrees of freedom.
   \item {\tt zelig.data}: the input data frame if {\tt save.data = TRUE}.  
   \end{itemize}
  
\item From {\tt summary(z.out)}, you may extract:
   \begin{itemize}
   \item {\tt coefficients}: the residuals sum of squares estimated
     with their associated degree of freedom, their mean squares,
     $F$-values, and $F$-statistics for all explanatory variables.
    
   \item {\tt residuals}: the sum of square, mean, degree of freedom,
     $F$-values, and $F$-statistics for the vector of residuals or
     standard errors that check the adequecy of the fit for the
     dependent variable versus the true values or data points.
   
   \end{itemize}
   
\item From the {\tt sim()} output object {\tt s.out}, you may extract
  quantities of interest arranged as matrices indexed by simulation
  $\times$ {\tt x}-observation (for more than one {\tt x}-observation).
  Available quantities are:

   \begin{itemize}
   \item {\tt qi\$ev}: the simulated expected values for the specified
     values of {\tt x}.
   \item {\tt qi\$fd}:  the simulated first differences (or
     differences in expected values) for the specified values of {\tt
       x} and {\tt x1}. 
   \item {\tt qi\$att.ev}: the simulated average expected treatment
     effect for the treated from conditional prediction models.  
   \end{itemize}
\end{itemize}


\subsection* {How to Cite} 
To cite the \emph{ aov } Zelig model:
 \begin{verse}
 Kosuke Imai, Gary King, and Oliva Lau. 2007. "aov: Fit an Analysis of Variance Model" in Kosuke Imai, Gary King, and Olivia Lau, "Zelig: Everyone's Statistical Software,"\url{http://gking.harvard.edu/zelig} 
\end{verse}
To cite Zelig as a whole, please reference these two sources:
\begin{verse}
  Kosuke Imai, Gary King, and Olivia Lau. 2007. ``Zelig: Everyone's
  Statistical Software,'' \url{http://GKing.harvard.edu/zelig}.
\end{verse}
\begin{verse}
Imai, Kosuke, Gary King, and Olivia Lau. (2008). ``Toward A Common Framework for Statistical Analysis and Development.'' Journal of Computational and Graphical Statistics, Vol. 17, No. 4 (December), pp. 892-913. 
\end{verse}


\subsection* {See also}
The analysis of variance model is part of the stats package by William
N. Venables and Brian D. Ripley \citep{VenRip02}.  In addition,
advanced users may wish to refer to \texttt{help(aov)} and
\texttt{help(lm)}. 
