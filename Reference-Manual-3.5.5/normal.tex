\section{{\tt normal}: Normal Regression for Continuous Dependent Variables}
\label{normal}

The Normal regression model is a close variant of the more standard
least squares regression model (see \Sref{ls}). Both models specify a
continuous dependent variable as a linear function of a set of
explanatory variables.  The Normal model reports maximum likelihood
(rather than least squares) estimates.  The two models differ only in
their estimate for the stochastic parameter $\sigma$.

\subsubsection{Syntax}

\begin{verbatim}
> z.out <- zelig(Y ~ X1 + X2, model = "normal", data = mydata)
> x.out <- setx(z.out)
> s.out <- sim(z.out, x = x.out)
\end{verbatim}

\subsubsection{Additional Inputs} 

In addition to the standard inputs, {\tt zelig()} takes the following
additional options for normal regression:  
\begin{itemize}
\item {\tt robust}: defaults to {\tt FALSE}.  If {\tt TRUE} is
selected, {\tt zelig()} computes robust standard errors via the {\tt
sandwich} package (see \cite{Zeileis04}).  The default type of robust
standard error is heteroskedastic and autocorrelation consistent (HAC),
and assumes that observations are ordered by time index.

In addition, {\tt robust} may be a list with the following options:  
\begin{itemize}
\item {\tt method}:  Choose from 
\begin{itemize}
\item {\tt "vcovHAC"}: (default if {\tt robust = TRUE}) HAC standard
errors. 
\item {\tt "kernHAC"}: HAC standard errors using the
weights given in \cite{Andrews91}. 
\item {\tt "weave"}: HAC standard errors using the
weights given in \cite{LumHea99}.  
\end{itemize}  
\item {\tt order.by}: defaults to {\tt NULL} (the observations are
chronologically ordered as in the original data).  Optionally, you may
specify a vector of weights (either as {\tt order.by = z}, where {\tt
z} exists outside the data frame; or as {\tt order.by = \~{}z}, where
{\tt z} is a variable in the data frame).  The observations are
chronologically ordered by the size of {\tt z}.
\item {\tt \dots}:  additional options passed to the functions 
specified in {\tt method}.   See the {\tt sandwich} library and
\cite{Zeileis04} for more options.   
\end{itemize}
\end{itemize}

\subsubsection{Examples}

\begin{enumerate}
\item Basic Example with First Differences

Attach sample data: 
\begin{Schunk}
\begin{Sinput}
RRR>  data(macro)
\end{Sinput}
\end{Schunk}
Estimate model:  
\begin{Schunk}
\begin{Sinput}
RRR>  z.out1 <- zelig(unem ~ gdp + capmob + trade, model = "normal", 
+                   data = macro)
\end{Sinput}
\end{Schunk}
Summarize of regression coefficients:  
\begin{Schunk}
\begin{Sinput}
RRR>  summary(z.out1)
\end{Sinput}
\end{Schunk}
Set explanatory variables to their default (mean/mode) values, with
high (80th percentile) and low (20th percentile) values for trade: 
\begin{Schunk}
\begin{Sinput}
RRR>  x.high <- setx(z.out1, trade = quantile(macro$trade, 0.8))
RRR>  x.low <- setx(z.out1, trade = quantile(macro$trade, 0.2))
\end{Sinput}
\end{Schunk}
Generate first differences for the effect of high versus low trade on
GDP: 
\begin{Schunk}
\begin{Sinput}
RRR>  s.out1 <- sim(z.out1, x = x.high, x1 = x.low)
\end{Sinput}
\end{Schunk}
\begin{Schunk}
\begin{Sinput}
RRR>  summary(s.out1)
\end{Sinput}
\end{Schunk}
%plot does not work 
A visual summary of quantities of interest:  
\begin{center}
\begin{Schunk}
\begin{Sinput}
RRR>  plot(s.out1)
\end{Sinput}
\end{Schunk}
\includegraphics{vigpics/normal-ExamplesPlot}
\end{center}

% \item Using Dummy Variables
% %the code in this section does not work well but there is no demo for this part either
%  
% Estimate a model with a dummy variable for each year and country (see
% \ref{factors} for help with dummy variables).  Note that you do not
% need to create dummy variables, as the program will automatically
% parse the unique values in the selected variables into dummy
% variables.    
% <<Dummy.zelig>>=
%  z.out2 <- zelig(unem ~ gdp + trade + capmob + as.factor(year) 
%                   + as.factor(country), model = "normal", data = macro)
% @ 
% Set values for the explanatory variables, using the default mean/mode
% variables, with country set to the United States and Japan,
% respectively: 
% <<Dummy.setx>>=
% ### x.US <- try(setx(z.out2, country = "United States"),silent=T)
% ### x.Japan <- try(setx(z.out2, country = "Japan"),silent=T)
% @ 
% Simulate quantities of interest:  
% <<Dummy.sim>>=
% ### s.out2 <- try(sim(z.out2, x = x.US, x1 = x.Japan), silent=T)
% @ 
% <<Dummy.summary>>= 
% ###try(summary(s.out2))
% @
% %plot does not work 
% \begin{center}
% <<label=DummyPlot,fig=true,echo=false>>= 
% plot(s.out2)
% @ 
%\end{center}
\end{enumerate}

\subsubsection{Model}
Let $Y_i$ be the continuous dependent variable for observation $i$.
\begin{itemize}
\item The \emph{stochastic component} is described by a univariate normal
  model with a vector of means $\mu_i$ and scalar variance $\sigma^2$:
  \begin{equation*}
    Y_i \; \sim \; \textrm{Normal}(\mu_i, \sigma^2). 
  \end{equation*}

\item The \emph{systematic component} is 
  \begin{equation*}
    \mu_i \;= \; x_i \beta,
  \end{equation*}
  where $x_i$ is the vector of $k$ explanatory variables and $\beta$ is
  the vector of coefficients.
\end{itemize}


\subsubsection{Quantities of Interest}

\begin{itemize}
\item The expected value ({\tt qi\$ev}) is the mean of simulations
  from the the stochastic component, $$E(Y) = \mu_i = x_i \beta,$$
  given a draw of $\beta$ from its posterior.  

\item The predicted value ({\tt qi\$pr}) is drawn from the distribution
  defined by the set of parameters $(\mu_i, \sigma)$.  

\item The first difference ({\tt qi\$fd}) is:
\begin{equation*}
\textrm{FD}\; = \;E(Y \mid x_1) -  E(Y \mid x)
\end{equation*}

\item In conditional prediction models, the average expected treatment
  effect ({\tt att.ev}) for the treatment group is 
    \begin{equation*} \frac{1}{\sum_{i=1}^n t_i}\sum_{i:t_i=1}^n \left\{ Y_i(t_i=1) -
      E[Y_i(t_i=0)] \right\},
    \end{equation*} 
    where $t_i$ is a binary explanatory variable defining the treatment
    ($t_i=1$) and control ($t_i=0$) groups.  Variation in the
    simulations are due to uncertainty in simulating $E[Y_i(t_i=0)]$,
    the counterfactual expected value of $Y_i$ for observations in the
    treatment group, under the assumption that everything stays the
    same except that the treatment indicator is switched to $t_i=0$.

\item In conditional prediction models, the average predicted treatment
  effect ({\tt att.pr}) for the treatment group is 
    \begin{equation*} \frac{1}{\sum_{i=1}^n t_i}\sum_{i:t_i=1}^n \left\{ Y_i(t_i=1) -
      \widehat{Y_i(t_i=0)} \right\},
    \end{equation*} 
    where $t_i$ is a binary explanatory variable defining the
    treatment ($t_i=1$) and control ($t_i=0$) groups.  Variation in
    the simulations are due to uncertainty in simulating
    $\widehat{Y_i(t_i=0)}$, the counterfactual predicted value of
    $Y_i$ for observations in the treatment group, under the
    assumption that everything stays the same except that the
    treatment indicator is switched to $t_i=0$.

\end{itemize}

\subsubsection{Output Values}

The output of each Zelig command contains useful information which you
may view.  For example, if you run \texttt{z.out <- zelig(y \~\,
  x, model = "normal", data)}, then you may examine the available
information in \texttt{z.out} by using \texttt{names(z.out)},
see the {\tt coefficients} by using {\tt z.out\$coefficients}, and
a default summary of information through \texttt{summary(z.out)}.
Other elements available through the {\tt \$} operator are listed
below.

\begin{itemize}
\item From the {\tt zelig()} output object {\tt z.out}, you may extract:
   \begin{itemize}
   \item {\tt coefficients}: parameter estimates for the explanatory
     variables.
   \item {\tt residuals}: the working residuals in the final iteration
     of the IWLS fit.
   \item {\tt fitted.values}: fitted values.  For the normal model,
     these are identical to the {\tt linear predictors}.
   \item {\tt linear.predictors}: fitted values.  For the normal
     model, these are identical to {\tt fitted.values}.
   \item {\tt aic}: Akaike's Information Criterion (minus twice the
     maximized log-likelihood plus twice the number of coefficients).
   \item {\tt df.residual}: the residual degrees of freedom.
   \item {\tt df.null}: the residual degrees of freedom for the null
     model.
   \item {\tt zelig.data}: the input data frame if {\tt save.data = TRUE}.  
   \end{itemize}

\item From {\tt summary(z.out)}, you may extract: 
   \begin{itemize}
   \item {\tt coefficients}: the parameter estimates with their
     associated standard errors, $p$-values, and $t$-statistics.
   \item{\tt cov.scaled}: a $k \times k$ matrix of scaled covariances.
   \item{\tt cov.unscaled}: a $k \times k$ matrix of unscaled
     covariances.  
   \end{itemize}

\item From the {\tt sim()} output object {\tt s.out}, you may extract
  quantities of interest arranged as matrices indexed by simulation
  $\times$ {\tt x}-observation (for more than one {\tt x}-observation).
  Available quantities are:

   \begin{itemize}
   \item {\tt qi\$ev}: the simulated expected values for the specified
     values of {\tt x}.
   \item {\tt qi\$pr}: the simulated predicted values drawn from the
     distribution defined by $(\mu_i, \sigma)$.
   \item {\tt qi\$fd}: the simulated first difference in the simulated
     expected values for the values specified in {\tt x} and {\tt x1}.
   \item {\tt qi\$att.ev}: the simulated average expected treatment
     effect for the treated from conditional prediction models.  
   \item {\tt qi\$att.pr}: the simulated average predicted treatment
     effect for the treated from conditional prediction models.  
   \end{itemize}
\end{itemize}

\subsection* {How to Cite} 

To cite the \emph{ normal } Zelig model:
 \begin{verse}
 Kosuke Imai, Gary King, and Oliva Lau. 2007. "normal: Normal Regression for Continuous Dependent Variables" in Kosuke Imai, Gary King, and Olivia Lau, "Zelig: Everyone's Statistical Software,"\url{http://gking.harvard.edu/zelig} 
\end{verse}
To cite Zelig as a whole, please reference these two sources:
\begin{verse}
  Kosuke Imai, Gary King, and Olivia Lau. 2007. ``Zelig: Everyone's
  Statistical Software,'' \url{http://GKing.harvard.edu/zelig}.
\end{verse}
\begin{verse}
Imai, Kosuke, Gary King, and Olivia Lau. (2008). ``Toward A Common Framework for Statistical Analysis and Development.'' Journal of Computational and Graphical Statistics, Vol. 17, No. 4 (December), pp. 892-913. 
\end{verse}


\subsection* {See also}

The normal model is part of the stats package by \citet{VenRip02}.
Advanced users may wish to refer to \texttt{help(glm)} and
\texttt{help(family)}, as well as \cite{McCNel89}. Robust standard
errors are implemented via the sandwich package by \citet{Zeileis04}.
Sample data are from \cite{KinTomWit00}.
