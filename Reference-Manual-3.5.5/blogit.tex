\section{{\tt blogit}: Bivariate Logistic Regression for Two
Dichotomous Dependent Variables}\label{blogit}

Use the bivariate logistic regression model if you have two binary
dependent variables $(Y_1, Y_2)$, and wish to model them jointly as a
function of some explanatory variables.  Each pair of dependent
variables $(Y_{i1}, Y_{i2})$ has four potential outcomes, $(Y_{i1}=1,
Y_{i2}=1)$, $(Y_{i1}=1, Y_{i2}=0)$, $(Y_{i1}=0, Y_{i2}=1)$, and
$(Y_{i1}=0, Y_{i2}=0)$.  The joint probability for each of these four
outcomes is modeled with three systematic components: the marginal
Pr$(Y_{i1} = 1)$ and Pr$(Y_{i2} = 1)$, and the odds ratio $\psi$,
which describes the dependence of one marginal on the other.  Each of
these systematic components may be modeled as functions of (possibly
different) sets of explanatory variables.

\subsubsection{Syntax}

\begin{verbatim}
> z.out <- zelig(list(mu1 = Y1 ~ X1 + X2 , 
                      mu2 = Y2 ~ X1 + X3), 
                 model = "blogit", data = mydata)
> x.out <- setx(z.out)
> s.out <- sim(z.out, x = x.out)
\end{verbatim}

\subsubsection{Input Values}

In every bivariate logit specification, there are three equations which
correspond to each dependent variable ($Y_1$, $Y_2$), and $\psi$, the
odds ratio. You should provide a list of formulas for each equation or, 
you may use {\tt cbind()} if the right hand side is the same for both equations
\begin{Schunk}
\begin{Sinput}
RRR> formulae <- list(cbind(Y1,Y2) ~ X1 + X2)
\end{Sinput}
\end{Schunk}
which means that all the explanatory variables in equations 1 and 2
(corresponding to $Y_1$ and $Y_2$) are included, but only an intercept
is estimated (all explanatory variables are omitted) for equation 3
($\psi$).  

You may use the function {\tt tag()} to constrain variables across
equations:
\begin{Schunk}
\begin{Sinput}
RRR> formulae <- list(mu1 = y1 ~ x1 + tag(x3, "x3"), 
+                  mu2 = y2 ~ x2 + tag(x3, "x3"))
\end{Sinput}
\end{Schunk}
where {\tt tag()} is a special function that constrains variables to
have the same effect across equations.  Thus, the coefficient for {\tt
x3} in equation {\tt mu1} is constrained to be equal to the
coefficient for {\tt x3} in equation {\tt mu2}.  

\subsubsection{Examples}

\begin{enumerate}

\item {Basic Example} \label{basic.bl}

Load the data and estimate the model:  
\begin{Schunk}
\begin{Sinput}
RRR>  data(sanction)
RRR> ## sanction
\end{Sinput}
\end{Schunk}
\begin{Schunk}
\begin{Sinput}
RRR>  z.out1 <- zelig(cbind(import, export) ~ coop + cost + target, 
+                   model = "blogit", data = sanction)
\end{Sinput}
\end{Schunk}
By default, {\tt zelig()} estimates two effect parameters
for each explanatory variable in addition to the odds ratio parameter;
this formulation is parametrically independent (estimating
unconstrained effects for each explanatory variable), but
stochastically dependent because the models share an odds ratio.
\newline \newline Generate baseline values for the explanatory
variables (with cost set to 1, net gain to sender) and alternative
values (with cost set to 4, major loss to sender):
\begin{Schunk}
\begin{Sinput}
RRR>  x.low <- setx(z.out1, cost = 1)
\end{Sinput}
\end{Schunk}
\begin{Schunk}
\begin{Sinput}
RRR> x.high <- setx(z.out1, cost = 4)
\end{Sinput}
\end{Schunk}
Simulate fitted values and first differences:  
\begin{Schunk}
\begin{Sinput}
RRR>  s.out1 <- sim(z.out1, x = x.low, x1 = x.high)
RRR>  summary(s.out1)
\end{Sinput}
\end{Schunk}
\begin{center}
\begin{Schunk}
\begin{Sinput}
RRR>  plot(s.out1)
\end{Sinput}
\end{Schunk}
\includegraphics{vigpics/blogit-BasicExamplePlot}
\end{center}

\item {Joint Estimation of a Model with Different Sets of Explanatory Variables}\label{sto.dep.logit}

Using sample data \texttt{sanction}, estimate the statistical model, 
with {\tt import} a function of {\tt coop} in the first equation and {\tt export} a 
function of {\tt cost} and {\tt target} in the second equation:
\begin{Schunk}
\begin{Sinput}
RRR>  z.out2 <- zelig(list(import ~ coop, export ~ cost + target), 
+                   model = "blogit", data = sanction)
RRR>  summary(z.out2)
\end{Sinput}
\end{Schunk}
Set the explanatory variables to their means:
\begin{Schunk}
\begin{Sinput}
RRR>  x.out2 <- setx(z.out2)
\end{Sinput}
\end{Schunk}
Simulate draws from the posterior distribution:
\begin{Schunk}
\begin{Sinput}
RRR>  s.out2 <- sim(z.out2, x = x.out2)
RRR>  summary(s.out2)
\end{Sinput}
\end{Schunk}
\begin{center}
\begin{Schunk}
\begin{Sinput}
RRR>  plot(s.out2)
\end{Sinput}
\end{Schunk}
\includegraphics{vigpics/blogit-JointExamplePlot}
\end{center}

\item Joint Estimation of a Parametrically and Stochastically
Dependent Model 
\label{pdep.l}
  
Using the sample data \texttt{sanction}
The bivariate model is parametrically dependent if $Y_1$ and $Y_2$ share
some or all explanatory variables, {\it and} the effects of the shared
explanatory variables are jointly estimated.  For example,
\begin{Schunk}
\begin{Sinput}
RRR>  z.out3 <- zelig(list(import ~ tag(coop,"coop") + tag(cost,"cost") + 
+                            tag(target,"target"), 
+                        export ~ tag(coop,"coop") + tag(cost,"cost") + 
+                            tag(target,"target")), 
+                        model = "blogit", data = sanction)
RRR>  summary(z.out3)
\end{Sinput}
\end{Schunk}
Note that this model only returns one parameter estimate for each of
{\tt coop}, {\tt cost}, and {\tt target}.  Contrast this to
Example~\ref{basic.bl} which returns two parameter estimates for each
of the explanatory variables.  \newline \newline Set values for the
explanatory variables:
\begin{Schunk}
\begin{Sinput}
RRR> x.out3 <- setx(z.out3, cost = 1:4)
\end{Sinput}
\end{Schunk}
Draw simulated expected values:  
\begin{Schunk}
\begin{Sinput}
RRR>  s.out3 <- sim(z.out3, x = x.out3)
RRR>  summary(s.out3)
\end{Sinput}
\end{Schunk}


\end{enumerate}

\subsubsection{Model}

For each observation, define two binary dependent variables, $Y_1$ and
$Y_2$, each of which take the value of either 0 or 1 (in the
following, we suppress the observation index).  We model the joint
outcome $(Y_1$, $Y_2)$ using a marginal probability for each dependent
variable, and the odds ratio, which parameterizes the relationship
between the two dependent variables. Define $Y_{rs}$ such that it is
equal to 1 when $Y_1=r$ and $Y_2=s$ and is 0 otherwise, where $r$ and
$s$ take a value of either 0 or 1. Then, the model is defined as follows,

\begin{itemize}
 
\item The \emph{stochastic component} is
\begin{eqnarray*}
  Y_{11} &\sim& \textrm{Bernoulli}(y_{11} \mid \pi_{11}) \\
  Y_{10} &\sim& \textrm{Bernoulli}(y_{10} \mid \pi_{10}) \\
  Y_{01} &\sim& \textrm{Bernoulli}(y_{01} \mid \pi_{01})
\end{eqnarray*}
where $\pi_{rs}=\Pr(Y_1=r, Y_2=s)$ is the joint probability, and
$\pi_{00}=1-\pi_{11}-\pi_{10}-\pi_{01}$.


\item The \emph{systematic components} model the marginal probabilities,
  $\pi_j=\Pr(Y_j=1)$, as well as the odds ratio.  The odds ratio
  is defined as $\psi = \pi_{00} \pi_{01}/\pi_{10}\pi_{11}$ and
  describes the relationship between the two outcomes.  Thus, for each
  observation we have
\begin{eqnarray*}
\pi_j & = & \frac{1}{1 + \exp(-x_j \beta_j)} \quad \textrm{ for} \quad
j=1,2, \\
\psi &= & \exp(x_3 \beta_3).
\end{eqnarray*}

\end{itemize}

\subsubsection{Quantities of Interest}
\begin{itemize}
\item The expected values ({\tt qi\$ev}) for the bivariate logit model
  are the predicted joint probabilities. Simulations of $\beta_1$,
  $\beta_2$, and $\beta_3$ (drawn from their sampling distributions)
  are substituted into the systematic components $(\pi_1, \pi_2,
  \psi)$ to find simulations of the predicted joint probabilities:
\begin{eqnarray*}
\pi_{11} & = & \left\{ \begin{array}{ll}
                 \frac{1}{2}(\psi - 1)^{-1} - {a - \sqrt{a^2 + b}} &
                 \textrm{for} \; \psi \ne 1 \\
                 \pi_1 \pi_2 & \textrm{for} \; \psi = 1 
                 \end{array} \right., \\
\pi_{10} &=& \pi_1 - \pi_{11}, \\
\pi_{01} &=& \pi_2 - \pi_{11}, \\
\pi_{00} &=& 1 - \pi_{10} - \pi_{01} - \pi_{11},
\end{eqnarray*}
where $a = 1 + (\pi_1 + \pi_2)(\psi - 1)$, $b = -4 \psi(\psi - 1)
\pi_1 \pi_2$, and the joint probabilities for each observation must sum
to one.  For $n$ simulations, the expected values form an $n \times 4$
matrix for each observation in {\tt x}.  

\item The predicted values ({\tt qi\$pr}) are draws from the
  multinomial distribution given the expected joint probabilities. 

\item The first differences ({\tt qi\$fd}) for each
  of the predicted joint probabilities are given by $$\textrm{FD}_{rs}
  = \Pr(Y_1=r, Y_2=s \mid x_1)-\Pr(Y_1=r, Y_2=s \mid x).$$  
  
\item The risk ratio ({\tt qi\$rr}) for each of the predicted joint
  probabilities are given by
\begin{equation*}
\textrm{RR}_{rs} = \frac{\Pr(Y_1=r, Y_2=s \mid x_1)}{\Pr(Y_1=r, Y_2=s \mid x)}
\end{equation*}

\item In conditional prediction models, the average expected treatment
  effect ({\tt att.ev}) for the treatment group is 
    \begin{equation*} \frac{1}{\sum_{i=1}^n t_i}\sum_{i:t_i=1}^n \left\{ Y_{ij}(t_i=1) -
      E[Y_{ij}(t_i=0)] \right\} \textrm{ for } j = 1,2,
    \end{equation*} 
    where $t_i$ is a binary explanatory variable defining the treatment
    ($t_i=1$) and control ($t_i=0$) groups.  Variation in the
    simulations are due to uncertainty in simulating $E[Y_{ij}(t_i=0)]$,
    the counterfactual expected value of $Y_{ij}$ for observations in the
    treatment group, under the assumption that everything stays the
    same except that the treatment indicator is switched to $t_i=0$.

\item In conditional prediction models, the average predicted treatment
  effect ({\tt att.pr}) for the treatment group is 
    \begin{equation*} \frac{1}{\sum_{i=1}^n t_i}\sum_{i:t_i=1}^n \left\{ Y_{ij}(t_i=1) -
      \widehat{Y_{ij}(t_i=0)} \right\} \textrm{ for } j = 1,2,
    \end{equation*} 
    where $t_i$ is a binary explanatory variable defining the treatment
    ($t_i=1$) and control ($t_i=0$) groups.  Variation in the
    simulations are due to uncertainty in simulating
    $\widehat{Y_{ij}(t_i=0)}$, the counterfactual predicted value of
    $Y_{ij}$ for observations in the treatment group, under the
    assumption that everything stays the same except that the
    treatment indicator is switched to $t_i=0$.
\end{itemize}

\subsubsection{Output Values}

The output of each Zelig command contains useful information which you
may view.  For example, if you run \texttt{z.out <- zelig(y \~\,
  x, model = "blogit", data)}, then you may examine the available
information in \texttt{z.out} by using \texttt{names(z.out)},
see the {\tt coefficients} by using {\tt z.out\$coefficients}, and
obtain a default summary of information through {\tt summary(z.out)}.
Other elements available through the {\tt \$} operator are listed
below.

\begin{itemize}
\item From the {\tt zelig()} output object {\tt z.out}, you may
  extract:
   \begin{itemize}
   \item {\tt coefficients}: the named vector of coefficients.   
   \item {\tt fitted.values}: an $n \times 4$ matrix of the in-sample
     fitted values.
   \item {\tt predictors}: an $n \times 3$ matrix of the linear
     predictors $x_j \beta_j$.
   \item {\tt residuals}: an $n \times 3$ matrix of the residuals.  
   \item {\tt df.residual}: the residual degrees of freedom.  
   \item {\tt df.total}: the total degrees of freedom.
   \item {\tt rss}: the residual sum of squares.  
   \item {\tt y}: an $n \times 2$ matrix of the dependent variables.
   \item {\tt zelig.data}: the input data frame if {\tt save.data = TRUE}.  
   \end{itemize}

\item From {\tt summary(z.out)}, you may extract:
  \begin{itemize}
  \item {\tt coef3}: a table of the coefficients with their associated
    standard errors and $t$-statistics.
  \item {\tt cov.unscaled}: the variance-covariance matrix. 
  \item {\tt pearson.resid}: an $n \times 3$ matrix of the Pearson residuals.  
  \end{itemize}

\item From the {\tt sim()} output object {\tt s.out}, you may extract
  quantities of interest arranged as arrays indexed by simulation
  $\times$ quantity $\times$ {\tt x}-observation (for more than one
  {\tt x}-observation; otherwise the quantities are matrices).
  Available quantities are:

   \begin{itemize}
   \item {\tt qi\$ev}: the simulated expected joint probabilities (or expected
     values) for the specified values of {\tt x}.  
   \item {\tt qi\$pr}: the simulated predicted outcomes drawn from a
     distribution defined by the expected joint probabilities.
   \item {\tt qi\$fd}: the simulated first difference in the
     expected joint probabilities for the values specified in {\tt x} and
     {\tt x1}.
   \item {\tt qi\$rr}: the simulated risk ratio in the predicted
     probabilities for given {\tt x} and {\tt x1}.
   \item {\tt qi\$att.ev}: the simulated average expected treatment
     effect for the treated from conditional prediction models.  
   \item {\tt qi\$att.pr}: the simulated average predicted treatment
     effect for the treated from conditional prediction models.  
   \end{itemize}
\end{itemize}

\subsection*{How to Cite}
\section{{\tt blogit}: Bivariate Logistic Regression for Two
Dichotomous Dependent Variables}\label{blogit}

Use the bivariate logistic regression model if you have two binary
dependent variables $(Y_1, Y_2)$, and wish to model them jointly as a
function of some explanatory variables.  Each pair of dependent
variables $(Y_{i1}, Y_{i2})$ has four potential outcomes, $(Y_{i1}=1,
Y_{i2}=1)$, $(Y_{i1}=1, Y_{i2}=0)$, $(Y_{i1}=0, Y_{i2}=1)$, and
$(Y_{i1}=0, Y_{i2}=0)$.  The joint probability for each of these four
outcomes is modeled with three systematic components: the marginal
Pr$(Y_{i1} = 1)$ and Pr$(Y_{i2} = 1)$, and the odds ratio $\psi$,
which describes the dependence of one marginal on the other.  Each of
these systematic components may be modeled as functions of (possibly
different) sets of explanatory variables.

\subsubsection{Syntax}

\begin{verbatim}
> z.out <- zelig(list(mu1 = Y1 ~ X1 + X2 , 
                      mu2 = Y2 ~ X1 + X3), 
                 model = "blogit", data = mydata)
> x.out <- setx(z.out)
> s.out <- sim(z.out, x = x.out)
\end{verbatim}

\subsubsection{Input Values}

In every bivariate logit specification, there are three equations which
correspond to each dependent variable ($Y_1$, $Y_2$), and $\psi$, the
odds ratio. You should provide a list of formulas for each equation or, 
you may use {\tt cbind()} if the right hand side is the same for both equations
\begin{Schunk}
\begin{Sinput}
RRR> formulae <- list(cbind(Y1,Y2) ~ X1 + X2)
\end{Sinput}
\end{Schunk}
which means that all the explanatory variables in equations 1 and 2
(corresponding to $Y_1$ and $Y_2$) are included, but only an intercept
is estimated (all explanatory variables are omitted) for equation 3
($\psi$).  

You may use the function {\tt tag()} to constrain variables across
equations:
\begin{Schunk}
\begin{Sinput}
RRR> formulae <- list(mu1 = y1 ~ x1 + tag(x3, "x3"), 
+                  mu2 = y2 ~ x2 + tag(x3, "x3"))
\end{Sinput}
\end{Schunk}
where {\tt tag()} is a special function that constrains variables to
have the same effect across equations.  Thus, the coefficient for {\tt
x3} in equation {\tt mu1} is constrained to be equal to the
coefficient for {\tt x3} in equation {\tt mu2}.  

\subsubsection{Examples}

\begin{enumerate}

\item {Basic Example} \label{basic.bl}

Load the data and estimate the model:  
\begin{Schunk}
\begin{Sinput}
RRR>  data(sanction)
RRR> ## sanction
\end{Sinput}
\end{Schunk}
\begin{Schunk}
\begin{Sinput}
RRR>  z.out1 <- zelig(cbind(import, export) ~ coop + cost + target, 
+                   model = "blogit", data = sanction)
\end{Sinput}
\end{Schunk}
By default, {\tt zelig()} estimates two effect parameters
for each explanatory variable in addition to the odds ratio parameter;
this formulation is parametrically independent (estimating
unconstrained effects for each explanatory variable), but
stochastically dependent because the models share an odds ratio.
\newline \newline Generate baseline values for the explanatory
variables (with cost set to 1, net gain to sender) and alternative
values (with cost set to 4, major loss to sender):
\begin{Schunk}
\begin{Sinput}
RRR>  x.low <- setx(z.out1, cost = 1)
\end{Sinput}
\end{Schunk}
\begin{Schunk}
\begin{Sinput}
RRR> x.high <- setx(z.out1, cost = 4)
\end{Sinput}
\end{Schunk}
Simulate fitted values and first differences:  
\begin{Schunk}
\begin{Sinput}
RRR>  s.out1 <- sim(z.out1, x = x.low, x1 = x.high)
RRR>  summary(s.out1)
\end{Sinput}
\end{Schunk}
\begin{center}
\begin{Schunk}
\begin{Sinput}
RRR>  plot(s.out1)
\end{Sinput}
\end{Schunk}
\includegraphics{vigpics/blogit-BasicExamplePlot}
\end{center}

\item {Joint Estimation of a Model with Different Sets of Explanatory Variables}\label{sto.dep.logit}

Using sample data \texttt{sanction}, estimate the statistical model, 
with {\tt import} a function of {\tt coop} in the first equation and {\tt export} a 
function of {\tt cost} and {\tt target} in the second equation:
\begin{Schunk}
\begin{Sinput}
RRR>  z.out2 <- zelig(list(import ~ coop, export ~ cost + target), 
+                   model = "blogit", data = sanction)
RRR>  summary(z.out2)
\end{Sinput}
\end{Schunk}
Set the explanatory variables to their means:
\begin{Schunk}
\begin{Sinput}
RRR>  x.out2 <- setx(z.out2)
\end{Sinput}
\end{Schunk}
Simulate draws from the posterior distribution:
\begin{Schunk}
\begin{Sinput}
RRR>  s.out2 <- sim(z.out2, x = x.out2)
RRR>  summary(s.out2)
\end{Sinput}
\end{Schunk}
\begin{center}
\begin{Schunk}
\begin{Sinput}
RRR>  plot(s.out2)
\end{Sinput}
\end{Schunk}
\includegraphics{vigpics/blogit-JointExamplePlot}
\end{center}

\item Joint Estimation of a Parametrically and Stochastically
Dependent Model 
\label{pdep.l}
  
Using the sample data \texttt{sanction}
The bivariate model is parametrically dependent if $Y_1$ and $Y_2$ share
some or all explanatory variables, {\it and} the effects of the shared
explanatory variables are jointly estimated.  For example,
\begin{Schunk}
\begin{Sinput}
RRR>  z.out3 <- zelig(list(import ~ tag(coop,"coop") + tag(cost,"cost") + 
+                            tag(target,"target"), 
+                        export ~ tag(coop,"coop") + tag(cost,"cost") + 
+                            tag(target,"target")), 
+                        model = "blogit", data = sanction)
RRR>  summary(z.out3)
\end{Sinput}
\end{Schunk}
Note that this model only returns one parameter estimate for each of
{\tt coop}, {\tt cost}, and {\tt target}.  Contrast this to
Example~\ref{basic.bl} which returns two parameter estimates for each
of the explanatory variables.  \newline \newline Set values for the
explanatory variables:
\begin{Schunk}
\begin{Sinput}
RRR> x.out3 <- setx(z.out3, cost = 1:4)
\end{Sinput}
\end{Schunk}
Draw simulated expected values:  
\begin{Schunk}
\begin{Sinput}
RRR>  s.out3 <- sim(z.out3, x = x.out3)
RRR>  summary(s.out3)
\end{Sinput}
\end{Schunk}


\end{enumerate}

\subsubsection{Model}

For each observation, define two binary dependent variables, $Y_1$ and
$Y_2$, each of which take the value of either 0 or 1 (in the
following, we suppress the observation index).  We model the joint
outcome $(Y_1$, $Y_2)$ using a marginal probability for each dependent
variable, and the odds ratio, which parameterizes the relationship
between the two dependent variables. Define $Y_{rs}$ such that it is
equal to 1 when $Y_1=r$ and $Y_2=s$ and is 0 otherwise, where $r$ and
$s$ take a value of either 0 or 1. Then, the model is defined as follows,

\begin{itemize}
 
\item The \emph{stochastic component} is
\begin{eqnarray*}
  Y_{11} &\sim& \textrm{Bernoulli}(y_{11} \mid \pi_{11}) \\
  Y_{10} &\sim& \textrm{Bernoulli}(y_{10} \mid \pi_{10}) \\
  Y_{01} &\sim& \textrm{Bernoulli}(y_{01} \mid \pi_{01})
\end{eqnarray*}
where $\pi_{rs}=\Pr(Y_1=r, Y_2=s)$ is the joint probability, and
$\pi_{00}=1-\pi_{11}-\pi_{10}-\pi_{01}$.


\item The \emph{systematic components} model the marginal probabilities,
  $\pi_j=\Pr(Y_j=1)$, as well as the odds ratio.  The odds ratio
  is defined as $\psi = \pi_{00} \pi_{01}/\pi_{10}\pi_{11}$ and
  describes the relationship between the two outcomes.  Thus, for each
  observation we have
\begin{eqnarray*}
\pi_j & = & \frac{1}{1 + \exp(-x_j \beta_j)} \quad \textrm{ for} \quad
j=1,2, \\
\psi &= & \exp(x_3 \beta_3).
\end{eqnarray*}

\end{itemize}

\subsubsection{Quantities of Interest}
\begin{itemize}
\item The expected values ({\tt qi\$ev}) for the bivariate logit model
  are the predicted joint probabilities. Simulations of $\beta_1$,
  $\beta_2$, and $\beta_3$ (drawn from their sampling distributions)
  are substituted into the systematic components $(\pi_1, \pi_2,
  \psi)$ to find simulations of the predicted joint probabilities:
\begin{eqnarray*}
\pi_{11} & = & \left\{ \begin{array}{ll}
                 \frac{1}{2}(\psi - 1)^{-1} - {a - \sqrt{a^2 + b}} &
                 \textrm{for} \; \psi \ne 1 \\
                 \pi_1 \pi_2 & \textrm{for} \; \psi = 1 
                 \end{array} \right., \\
\pi_{10} &=& \pi_1 - \pi_{11}, \\
\pi_{01} &=& \pi_2 - \pi_{11}, \\
\pi_{00} &=& 1 - \pi_{10} - \pi_{01} - \pi_{11},
\end{eqnarray*}
where $a = 1 + (\pi_1 + \pi_2)(\psi - 1)$, $b = -4 \psi(\psi - 1)
\pi_1 \pi_2$, and the joint probabilities for each observation must sum
to one.  For $n$ simulations, the expected values form an $n \times 4$
matrix for each observation in {\tt x}.  

\item The predicted values ({\tt qi\$pr}) are draws from the
  multinomial distribution given the expected joint probabilities. 

\item The first differences ({\tt qi\$fd}) for each
  of the predicted joint probabilities are given by $$\textrm{FD}_{rs}
  = \Pr(Y_1=r, Y_2=s \mid x_1)-\Pr(Y_1=r, Y_2=s \mid x).$$  
  
\item The risk ratio ({\tt qi\$rr}) for each of the predicted joint
  probabilities are given by
\begin{equation*}
\textrm{RR}_{rs} = \frac{\Pr(Y_1=r, Y_2=s \mid x_1)}{\Pr(Y_1=r, Y_2=s \mid x)}
\end{equation*}

\item In conditional prediction models, the average expected treatment
  effect ({\tt att.ev}) for the treatment group is 
    \begin{equation*} \frac{1}{\sum_{i=1}^n t_i}\sum_{i:t_i=1}^n \left\{ Y_{ij}(t_i=1) -
      E[Y_{ij}(t_i=0)] \right\} \textrm{ for } j = 1,2,
    \end{equation*} 
    where $t_i$ is a binary explanatory variable defining the treatment
    ($t_i=1$) and control ($t_i=0$) groups.  Variation in the
    simulations are due to uncertainty in simulating $E[Y_{ij}(t_i=0)]$,
    the counterfactual expected value of $Y_{ij}$ for observations in the
    treatment group, under the assumption that everything stays the
    same except that the treatment indicator is switched to $t_i=0$.

\item In conditional prediction models, the average predicted treatment
  effect ({\tt att.pr}) for the treatment group is 
    \begin{equation*} \frac{1}{\sum_{i=1}^n t_i}\sum_{i:t_i=1}^n \left\{ Y_{ij}(t_i=1) -
      \widehat{Y_{ij}(t_i=0)} \right\} \textrm{ for } j = 1,2,
    \end{equation*} 
    where $t_i$ is a binary explanatory variable defining the treatment
    ($t_i=1$) and control ($t_i=0$) groups.  Variation in the
    simulations are due to uncertainty in simulating
    $\widehat{Y_{ij}(t_i=0)}$, the counterfactual predicted value of
    $Y_{ij}$ for observations in the treatment group, under the
    assumption that everything stays the same except that the
    treatment indicator is switched to $t_i=0$.
\end{itemize}

\subsubsection{Output Values}

The output of each Zelig command contains useful information which you
may view.  For example, if you run \texttt{z.out <- zelig(y \~\,
  x, model = "blogit", data)}, then you may examine the available
information in \texttt{z.out} by using \texttt{names(z.out)},
see the {\tt coefficients} by using {\tt z.out\$coefficients}, and
obtain a default summary of information through {\tt summary(z.out)}.
Other elements available through the {\tt \$} operator are listed
below.

\begin{itemize}
\item From the {\tt zelig()} output object {\tt z.out}, you may
  extract:
   \begin{itemize}
   \item {\tt coefficients}: the named vector of coefficients.   
   \item {\tt fitted.values}: an $n \times 4$ matrix of the in-sample
     fitted values.
   \item {\tt predictors}: an $n \times 3$ matrix of the linear
     predictors $x_j \beta_j$.
   \item {\tt residuals}: an $n \times 3$ matrix of the residuals.  
   \item {\tt df.residual}: the residual degrees of freedom.  
   \item {\tt df.total}: the total degrees of freedom.
   \item {\tt rss}: the residual sum of squares.  
   \item {\tt y}: an $n \times 2$ matrix of the dependent variables.
   \item {\tt zelig.data}: the input data frame if {\tt save.data = TRUE}.  
   \end{itemize}

\item From {\tt summary(z.out)}, you may extract:
  \begin{itemize}
  \item {\tt coef3}: a table of the coefficients with their associated
    standard errors and $t$-statistics.
  \item {\tt cov.unscaled}: the variance-covariance matrix. 
  \item {\tt pearson.resid}: an $n \times 3$ matrix of the Pearson residuals.  
  \end{itemize}

\item From the {\tt sim()} output object {\tt s.out}, you may extract
  quantities of interest arranged as arrays indexed by simulation
  $\times$ quantity $\times$ {\tt x}-observation (for more than one
  {\tt x}-observation; otherwise the quantities are matrices).
  Available quantities are:

   \begin{itemize}
   \item {\tt qi\$ev}: the simulated expected joint probabilities (or expected
     values) for the specified values of {\tt x}.  
   \item {\tt qi\$pr}: the simulated predicted outcomes drawn from a
     distribution defined by the expected joint probabilities.
   \item {\tt qi\$fd}: the simulated first difference in the
     expected joint probabilities for the values specified in {\tt x} and
     {\tt x1}.
   \item {\tt qi\$rr}: the simulated risk ratio in the predicted
     probabilities for given {\tt x} and {\tt x1}.
   \item {\tt qi\$att.ev}: the simulated average expected treatment
     effect for the treated from conditional prediction models.  
   \item {\tt qi\$att.pr}: the simulated average predicted treatment
     effect for the treated from conditional prediction models.  
   \end{itemize}
\end{itemize}

\subsection*{How to Cite}
\section{{\tt blogit}: Bivariate Logistic Regression for Two
Dichotomous Dependent Variables}\label{blogit}

Use the bivariate logistic regression model if you have two binary
dependent variables $(Y_1, Y_2)$, and wish to model them jointly as a
function of some explanatory variables.  Each pair of dependent
variables $(Y_{i1}, Y_{i2})$ has four potential outcomes, $(Y_{i1}=1,
Y_{i2}=1)$, $(Y_{i1}=1, Y_{i2}=0)$, $(Y_{i1}=0, Y_{i2}=1)$, and
$(Y_{i1}=0, Y_{i2}=0)$.  The joint probability for each of these four
outcomes is modeled with three systematic components: the marginal
Pr$(Y_{i1} = 1)$ and Pr$(Y_{i2} = 1)$, and the odds ratio $\psi$,
which describes the dependence of one marginal on the other.  Each of
these systematic components may be modeled as functions of (possibly
different) sets of explanatory variables.

\subsubsection{Syntax}

\begin{verbatim}
> z.out <- zelig(list(mu1 = Y1 ~ X1 + X2 , 
                      mu2 = Y2 ~ X1 + X3), 
                 model = "blogit", data = mydata)
> x.out <- setx(z.out)
> s.out <- sim(z.out, x = x.out)
\end{verbatim}

\subsubsection{Input Values}

In every bivariate logit specification, there are three equations which
correspond to each dependent variable ($Y_1$, $Y_2$), and $\psi$, the
odds ratio. You should provide a list of formulas for each equation or, 
you may use {\tt cbind()} if the right hand side is the same for both equations
\begin{Schunk}
\begin{Sinput}
RRR> formulae <- list(cbind(Y1,Y2) ~ X1 + X2)
\end{Sinput}
\end{Schunk}
which means that all the explanatory variables in equations 1 and 2
(corresponding to $Y_1$ and $Y_2$) are included, but only an intercept
is estimated (all explanatory variables are omitted) for equation 3
($\psi$).  

You may use the function {\tt tag()} to constrain variables across
equations:
\begin{Schunk}
\begin{Sinput}
RRR> formulae <- list(mu1 = y1 ~ x1 + tag(x3, "x3"), 
+                  mu2 = y2 ~ x2 + tag(x3, "x3"))
\end{Sinput}
\end{Schunk}
where {\tt tag()} is a special function that constrains variables to
have the same effect across equations.  Thus, the coefficient for {\tt
x3} in equation {\tt mu1} is constrained to be equal to the
coefficient for {\tt x3} in equation {\tt mu2}.  

\subsubsection{Examples}

\begin{enumerate}

\item {Basic Example} \label{basic.bl}

Load the data and estimate the model:  
\begin{Schunk}
\begin{Sinput}
RRR>  data(sanction)
RRR> ## sanction
\end{Sinput}
\end{Schunk}
\begin{Schunk}
\begin{Sinput}
RRR>  z.out1 <- zelig(cbind(import, export) ~ coop + cost + target, 
+                   model = "blogit", data = sanction)
\end{Sinput}
\end{Schunk}
By default, {\tt zelig()} estimates two effect parameters
for each explanatory variable in addition to the odds ratio parameter;
this formulation is parametrically independent (estimating
unconstrained effects for each explanatory variable), but
stochastically dependent because the models share an odds ratio.
\newline \newline Generate baseline values for the explanatory
variables (with cost set to 1, net gain to sender) and alternative
values (with cost set to 4, major loss to sender):
\begin{Schunk}
\begin{Sinput}
RRR>  x.low <- setx(z.out1, cost = 1)
\end{Sinput}
\end{Schunk}
\begin{Schunk}
\begin{Sinput}
RRR> x.high <- setx(z.out1, cost = 4)
\end{Sinput}
\end{Schunk}
Simulate fitted values and first differences:  
\begin{Schunk}
\begin{Sinput}
RRR>  s.out1 <- sim(z.out1, x = x.low, x1 = x.high)
RRR>  summary(s.out1)
\end{Sinput}
\end{Schunk}
\begin{center}
\begin{Schunk}
\begin{Sinput}
RRR>  plot(s.out1)
\end{Sinput}
\end{Schunk}
\includegraphics{vigpics/blogit-BasicExamplePlot}
\end{center}

\item {Joint Estimation of a Model with Different Sets of Explanatory Variables}\label{sto.dep.logit}

Using sample data \texttt{sanction}, estimate the statistical model, 
with {\tt import} a function of {\tt coop} in the first equation and {\tt export} a 
function of {\tt cost} and {\tt target} in the second equation:
\begin{Schunk}
\begin{Sinput}
RRR>  z.out2 <- zelig(list(import ~ coop, export ~ cost + target), 
+                   model = "blogit", data = sanction)
RRR>  summary(z.out2)
\end{Sinput}
\end{Schunk}
Set the explanatory variables to their means:
\begin{Schunk}
\begin{Sinput}
RRR>  x.out2 <- setx(z.out2)
\end{Sinput}
\end{Schunk}
Simulate draws from the posterior distribution:
\begin{Schunk}
\begin{Sinput}
RRR>  s.out2 <- sim(z.out2, x = x.out2)
RRR>  summary(s.out2)
\end{Sinput}
\end{Schunk}
\begin{center}
\begin{Schunk}
\begin{Sinput}
RRR>  plot(s.out2)
\end{Sinput}
\end{Schunk}
\includegraphics{vigpics/blogit-JointExamplePlot}
\end{center}

\item Joint Estimation of a Parametrically and Stochastically
Dependent Model 
\label{pdep.l}
  
Using the sample data \texttt{sanction}
The bivariate model is parametrically dependent if $Y_1$ and $Y_2$ share
some or all explanatory variables, {\it and} the effects of the shared
explanatory variables are jointly estimated.  For example,
\begin{Schunk}
\begin{Sinput}
RRR>  z.out3 <- zelig(list(import ~ tag(coop,"coop") + tag(cost,"cost") + 
+                            tag(target,"target"), 
+                        export ~ tag(coop,"coop") + tag(cost,"cost") + 
+                            tag(target,"target")), 
+                        model = "blogit", data = sanction)
RRR>  summary(z.out3)
\end{Sinput}
\end{Schunk}
Note that this model only returns one parameter estimate for each of
{\tt coop}, {\tt cost}, and {\tt target}.  Contrast this to
Example~\ref{basic.bl} which returns two parameter estimates for each
of the explanatory variables.  \newline \newline Set values for the
explanatory variables:
\begin{Schunk}
\begin{Sinput}
RRR> x.out3 <- setx(z.out3, cost = 1:4)
\end{Sinput}
\end{Schunk}
Draw simulated expected values:  
\begin{Schunk}
\begin{Sinput}
RRR>  s.out3 <- sim(z.out3, x = x.out3)
RRR>  summary(s.out3)
\end{Sinput}
\end{Schunk}


\end{enumerate}

\subsubsection{Model}

For each observation, define two binary dependent variables, $Y_1$ and
$Y_2$, each of which take the value of either 0 or 1 (in the
following, we suppress the observation index).  We model the joint
outcome $(Y_1$, $Y_2)$ using a marginal probability for each dependent
variable, and the odds ratio, which parameterizes the relationship
between the two dependent variables. Define $Y_{rs}$ such that it is
equal to 1 when $Y_1=r$ and $Y_2=s$ and is 0 otherwise, where $r$ and
$s$ take a value of either 0 or 1. Then, the model is defined as follows,

\begin{itemize}
 
\item The \emph{stochastic component} is
\begin{eqnarray*}
  Y_{11} &\sim& \textrm{Bernoulli}(y_{11} \mid \pi_{11}) \\
  Y_{10} &\sim& \textrm{Bernoulli}(y_{10} \mid \pi_{10}) \\
  Y_{01} &\sim& \textrm{Bernoulli}(y_{01} \mid \pi_{01})
\end{eqnarray*}
where $\pi_{rs}=\Pr(Y_1=r, Y_2=s)$ is the joint probability, and
$\pi_{00}=1-\pi_{11}-\pi_{10}-\pi_{01}$.


\item The \emph{systematic components} model the marginal probabilities,
  $\pi_j=\Pr(Y_j=1)$, as well as the odds ratio.  The odds ratio
  is defined as $\psi = \pi_{00} \pi_{01}/\pi_{10}\pi_{11}$ and
  describes the relationship between the two outcomes.  Thus, for each
  observation we have
\begin{eqnarray*}
\pi_j & = & \frac{1}{1 + \exp(-x_j \beta_j)} \quad \textrm{ for} \quad
j=1,2, \\
\psi &= & \exp(x_3 \beta_3).
\end{eqnarray*}

\end{itemize}

\subsubsection{Quantities of Interest}
\begin{itemize}
\item The expected values ({\tt qi\$ev}) for the bivariate logit model
  are the predicted joint probabilities. Simulations of $\beta_1$,
  $\beta_2$, and $\beta_3$ (drawn from their sampling distributions)
  are substituted into the systematic components $(\pi_1, \pi_2,
  \psi)$ to find simulations of the predicted joint probabilities:
\begin{eqnarray*}
\pi_{11} & = & \left\{ \begin{array}{ll}
                 \frac{1}{2}(\psi - 1)^{-1} - {a - \sqrt{a^2 + b}} &
                 \textrm{for} \; \psi \ne 1 \\
                 \pi_1 \pi_2 & \textrm{for} \; \psi = 1 
                 \end{array} \right., \\
\pi_{10} &=& \pi_1 - \pi_{11}, \\
\pi_{01} &=& \pi_2 - \pi_{11}, \\
\pi_{00} &=& 1 - \pi_{10} - \pi_{01} - \pi_{11},
\end{eqnarray*}
where $a = 1 + (\pi_1 + \pi_2)(\psi - 1)$, $b = -4 \psi(\psi - 1)
\pi_1 \pi_2$, and the joint probabilities for each observation must sum
to one.  For $n$ simulations, the expected values form an $n \times 4$
matrix for each observation in {\tt x}.  

\item The predicted values ({\tt qi\$pr}) are draws from the
  multinomial distribution given the expected joint probabilities. 

\item The first differences ({\tt qi\$fd}) for each
  of the predicted joint probabilities are given by $$\textrm{FD}_{rs}
  = \Pr(Y_1=r, Y_2=s \mid x_1)-\Pr(Y_1=r, Y_2=s \mid x).$$  
  
\item The risk ratio ({\tt qi\$rr}) for each of the predicted joint
  probabilities are given by
\begin{equation*}
\textrm{RR}_{rs} = \frac{\Pr(Y_1=r, Y_2=s \mid x_1)}{\Pr(Y_1=r, Y_2=s \mid x)}
\end{equation*}

\item In conditional prediction models, the average expected treatment
  effect ({\tt att.ev}) for the treatment group is 
    \begin{equation*} \frac{1}{\sum_{i=1}^n t_i}\sum_{i:t_i=1}^n \left\{ Y_{ij}(t_i=1) -
      E[Y_{ij}(t_i=0)] \right\} \textrm{ for } j = 1,2,
    \end{equation*} 
    where $t_i$ is a binary explanatory variable defining the treatment
    ($t_i=1$) and control ($t_i=0$) groups.  Variation in the
    simulations are due to uncertainty in simulating $E[Y_{ij}(t_i=0)]$,
    the counterfactual expected value of $Y_{ij}$ for observations in the
    treatment group, under the assumption that everything stays the
    same except that the treatment indicator is switched to $t_i=0$.

\item In conditional prediction models, the average predicted treatment
  effect ({\tt att.pr}) for the treatment group is 
    \begin{equation*} \frac{1}{\sum_{i=1}^n t_i}\sum_{i:t_i=1}^n \left\{ Y_{ij}(t_i=1) -
      \widehat{Y_{ij}(t_i=0)} \right\} \textrm{ for } j = 1,2,
    \end{equation*} 
    where $t_i$ is a binary explanatory variable defining the treatment
    ($t_i=1$) and control ($t_i=0$) groups.  Variation in the
    simulations are due to uncertainty in simulating
    $\widehat{Y_{ij}(t_i=0)}$, the counterfactual predicted value of
    $Y_{ij}$ for observations in the treatment group, under the
    assumption that everything stays the same except that the
    treatment indicator is switched to $t_i=0$.
\end{itemize}

\subsubsection{Output Values}

The output of each Zelig command contains useful information which you
may view.  For example, if you run \texttt{z.out <- zelig(y \~\,
  x, model = "blogit", data)}, then you may examine the available
information in \texttt{z.out} by using \texttt{names(z.out)},
see the {\tt coefficients} by using {\tt z.out\$coefficients}, and
obtain a default summary of information through {\tt summary(z.out)}.
Other elements available through the {\tt \$} operator are listed
below.

\begin{itemize}
\item From the {\tt zelig()} output object {\tt z.out}, you may
  extract:
   \begin{itemize}
   \item {\tt coefficients}: the named vector of coefficients.   
   \item {\tt fitted.values}: an $n \times 4$ matrix of the in-sample
     fitted values.
   \item {\tt predictors}: an $n \times 3$ matrix of the linear
     predictors $x_j \beta_j$.
   \item {\tt residuals}: an $n \times 3$ matrix of the residuals.  
   \item {\tt df.residual}: the residual degrees of freedom.  
   \item {\tt df.total}: the total degrees of freedom.
   \item {\tt rss}: the residual sum of squares.  
   \item {\tt y}: an $n \times 2$ matrix of the dependent variables.
   \item {\tt zelig.data}: the input data frame if {\tt save.data = TRUE}.  
   \end{itemize}

\item From {\tt summary(z.out)}, you may extract:
  \begin{itemize}
  \item {\tt coef3}: a table of the coefficients with their associated
    standard errors and $t$-statistics.
  \item {\tt cov.unscaled}: the variance-covariance matrix. 
  \item {\tt pearson.resid}: an $n \times 3$ matrix of the Pearson residuals.  
  \end{itemize}

\item From the {\tt sim()} output object {\tt s.out}, you may extract
  quantities of interest arranged as arrays indexed by simulation
  $\times$ quantity $\times$ {\tt x}-observation (for more than one
  {\tt x}-observation; otherwise the quantities are matrices).
  Available quantities are:

   \begin{itemize}
   \item {\tt qi\$ev}: the simulated expected joint probabilities (or expected
     values) for the specified values of {\tt x}.  
   \item {\tt qi\$pr}: the simulated predicted outcomes drawn from a
     distribution defined by the expected joint probabilities.
   \item {\tt qi\$fd}: the simulated first difference in the
     expected joint probabilities for the values specified in {\tt x} and
     {\tt x1}.
   \item {\tt qi\$rr}: the simulated risk ratio in the predicted
     probabilities for given {\tt x} and {\tt x1}.
   \item {\tt qi\$att.ev}: the simulated average expected treatment
     effect for the treated from conditional prediction models.  
   \item {\tt qi\$att.pr}: the simulated average predicted treatment
     effect for the treated from conditional prediction models.  
   \end{itemize}
\end{itemize}

\subsection*{How to Cite}
\section{{\tt blogit}: Bivariate Logistic Regression for Two
Dichotomous Dependent Variables}\label{blogit}

Use the bivariate logistic regression model if you have two binary
dependent variables $(Y_1, Y_2)$, and wish to model them jointly as a
function of some explanatory variables.  Each pair of dependent
variables $(Y_{i1}, Y_{i2})$ has four potential outcomes, $(Y_{i1}=1,
Y_{i2}=1)$, $(Y_{i1}=1, Y_{i2}=0)$, $(Y_{i1}=0, Y_{i2}=1)$, and
$(Y_{i1}=0, Y_{i2}=0)$.  The joint probability for each of these four
outcomes is modeled with three systematic components: the marginal
Pr$(Y_{i1} = 1)$ and Pr$(Y_{i2} = 1)$, and the odds ratio $\psi$,
which describes the dependence of one marginal on the other.  Each of
these systematic components may be modeled as functions of (possibly
different) sets of explanatory variables.

\subsubsection{Syntax}

\begin{verbatim}
> z.out <- zelig(list(mu1 = Y1 ~ X1 + X2 , 
                      mu2 = Y2 ~ X1 + X3), 
                 model = "blogit", data = mydata)
> x.out <- setx(z.out)
> s.out <- sim(z.out, x = x.out)
\end{verbatim}

\subsubsection{Input Values}

In every bivariate logit specification, there are three equations which
correspond to each dependent variable ($Y_1$, $Y_2$), and $\psi$, the
odds ratio. You should provide a list of formulas for each equation or, 
you may use {\tt cbind()} if the right hand side is the same for both equations
\begin{Schunk}
\begin{Sinput}
RRR> formulae <- list(cbind(Y1,Y2) ~ X1 + X2)
\end{Sinput}
\end{Schunk}
which means that all the explanatory variables in equations 1 and 2
(corresponding to $Y_1$ and $Y_2$) are included, but only an intercept
is estimated (all explanatory variables are omitted) for equation 3
($\psi$).  

You may use the function {\tt tag()} to constrain variables across
equations:
\begin{Schunk}
\begin{Sinput}
RRR> formulae <- list(mu1 = y1 ~ x1 + tag(x3, "x3"), 
+                  mu2 = y2 ~ x2 + tag(x3, "x3"))
\end{Sinput}
\end{Schunk}
where {\tt tag()} is a special function that constrains variables to
have the same effect across equations.  Thus, the coefficient for {\tt
x3} in equation {\tt mu1} is constrained to be equal to the
coefficient for {\tt x3} in equation {\tt mu2}.  

\subsubsection{Examples}

\begin{enumerate}

\item {Basic Example} \label{basic.bl}

Load the data and estimate the model:  
\begin{Schunk}
\begin{Sinput}
RRR>  data(sanction)
RRR> ## sanction
\end{Sinput}
\end{Schunk}
\begin{Schunk}
\begin{Sinput}
RRR>  z.out1 <- zelig(cbind(import, export) ~ coop + cost + target, 
+                   model = "blogit", data = sanction)
\end{Sinput}
\end{Schunk}
By default, {\tt zelig()} estimates two effect parameters
for each explanatory variable in addition to the odds ratio parameter;
this formulation is parametrically independent (estimating
unconstrained effects for each explanatory variable), but
stochastically dependent because the models share an odds ratio.
\newline \newline Generate baseline values for the explanatory
variables (with cost set to 1, net gain to sender) and alternative
values (with cost set to 4, major loss to sender):
\begin{Schunk}
\begin{Sinput}
RRR>  x.low <- setx(z.out1, cost = 1)
\end{Sinput}
\end{Schunk}
\begin{Schunk}
\begin{Sinput}
RRR> x.high <- setx(z.out1, cost = 4)
\end{Sinput}
\end{Schunk}
Simulate fitted values and first differences:  
\begin{Schunk}
\begin{Sinput}
RRR>  s.out1 <- sim(z.out1, x = x.low, x1 = x.high)
RRR>  summary(s.out1)
\end{Sinput}
\end{Schunk}
\begin{center}
\begin{Schunk}
\begin{Sinput}
RRR>  plot(s.out1)
\end{Sinput}
\end{Schunk}
\includegraphics{vigpics/blogit-BasicExamplePlot}
\end{center}

\item {Joint Estimation of a Model with Different Sets of Explanatory Variables}\label{sto.dep.logit}

Using sample data \texttt{sanction}, estimate the statistical model, 
with {\tt import} a function of {\tt coop} in the first equation and {\tt export} a 
function of {\tt cost} and {\tt target} in the second equation:
\begin{Schunk}
\begin{Sinput}
RRR>  z.out2 <- zelig(list(import ~ coop, export ~ cost + target), 
+                   model = "blogit", data = sanction)
RRR>  summary(z.out2)
\end{Sinput}
\end{Schunk}
Set the explanatory variables to their means:
\begin{Schunk}
\begin{Sinput}
RRR>  x.out2 <- setx(z.out2)
\end{Sinput}
\end{Schunk}
Simulate draws from the posterior distribution:
\begin{Schunk}
\begin{Sinput}
RRR>  s.out2 <- sim(z.out2, x = x.out2)
RRR>  summary(s.out2)
\end{Sinput}
\end{Schunk}
\begin{center}
\begin{Schunk}
\begin{Sinput}
RRR>  plot(s.out2)
\end{Sinput}
\end{Schunk}
\includegraphics{vigpics/blogit-JointExamplePlot}
\end{center}

\item Joint Estimation of a Parametrically and Stochastically
Dependent Model 
\label{pdep.l}
  
Using the sample data \texttt{sanction}
The bivariate model is parametrically dependent if $Y_1$ and $Y_2$ share
some or all explanatory variables, {\it and} the effects of the shared
explanatory variables are jointly estimated.  For example,
\begin{Schunk}
\begin{Sinput}
RRR>  z.out3 <- zelig(list(import ~ tag(coop,"coop") + tag(cost,"cost") + 
+                            tag(target,"target"), 
+                        export ~ tag(coop,"coop") + tag(cost,"cost") + 
+                            tag(target,"target")), 
+                        model = "blogit", data = sanction)
RRR>  summary(z.out3)
\end{Sinput}
\end{Schunk}
Note that this model only returns one parameter estimate for each of
{\tt coop}, {\tt cost}, and {\tt target}.  Contrast this to
Example~\ref{basic.bl} which returns two parameter estimates for each
of the explanatory variables.  \newline \newline Set values for the
explanatory variables:
\begin{Schunk}
\begin{Sinput}
RRR> x.out3 <- setx(z.out3, cost = 1:4)
\end{Sinput}
\end{Schunk}
Draw simulated expected values:  
\begin{Schunk}
\begin{Sinput}
RRR>  s.out3 <- sim(z.out3, x = x.out3)
RRR>  summary(s.out3)
\end{Sinput}
\end{Schunk}


\end{enumerate}

\subsubsection{Model}

For each observation, define two binary dependent variables, $Y_1$ and
$Y_2$, each of which take the value of either 0 or 1 (in the
following, we suppress the observation index).  We model the joint
outcome $(Y_1$, $Y_2)$ using a marginal probability for each dependent
variable, and the odds ratio, which parameterizes the relationship
between the two dependent variables. Define $Y_{rs}$ such that it is
equal to 1 when $Y_1=r$ and $Y_2=s$ and is 0 otherwise, where $r$ and
$s$ take a value of either 0 or 1. Then, the model is defined as follows,

\begin{itemize}
 
\item The \emph{stochastic component} is
\begin{eqnarray*}
  Y_{11} &\sim& \textrm{Bernoulli}(y_{11} \mid \pi_{11}) \\
  Y_{10} &\sim& \textrm{Bernoulli}(y_{10} \mid \pi_{10}) \\
  Y_{01} &\sim& \textrm{Bernoulli}(y_{01} \mid \pi_{01})
\end{eqnarray*}
where $\pi_{rs}=\Pr(Y_1=r, Y_2=s)$ is the joint probability, and
$\pi_{00}=1-\pi_{11}-\pi_{10}-\pi_{01}$.


\item The \emph{systematic components} model the marginal probabilities,
  $\pi_j=\Pr(Y_j=1)$, as well as the odds ratio.  The odds ratio
  is defined as $\psi = \pi_{00} \pi_{01}/\pi_{10}\pi_{11}$ and
  describes the relationship between the two outcomes.  Thus, for each
  observation we have
\begin{eqnarray*}
\pi_j & = & \frac{1}{1 + \exp(-x_j \beta_j)} \quad \textrm{ for} \quad
j=1,2, \\
\psi &= & \exp(x_3 \beta_3).
\end{eqnarray*}

\end{itemize}

\subsubsection{Quantities of Interest}
\begin{itemize}
\item The expected values ({\tt qi\$ev}) for the bivariate logit model
  are the predicted joint probabilities. Simulations of $\beta_1$,
  $\beta_2$, and $\beta_3$ (drawn from their sampling distributions)
  are substituted into the systematic components $(\pi_1, \pi_2,
  \psi)$ to find simulations of the predicted joint probabilities:
\begin{eqnarray*}
\pi_{11} & = & \left\{ \begin{array}{ll}
                 \frac{1}{2}(\psi - 1)^{-1} - {a - \sqrt{a^2 + b}} &
                 \textrm{for} \; \psi \ne 1 \\
                 \pi_1 \pi_2 & \textrm{for} \; \psi = 1 
                 \end{array} \right., \\
\pi_{10} &=& \pi_1 - \pi_{11}, \\
\pi_{01} &=& \pi_2 - \pi_{11}, \\
\pi_{00} &=& 1 - \pi_{10} - \pi_{01} - \pi_{11},
\end{eqnarray*}
where $a = 1 + (\pi_1 + \pi_2)(\psi - 1)$, $b = -4 \psi(\psi - 1)
\pi_1 \pi_2$, and the joint probabilities for each observation must sum
to one.  For $n$ simulations, the expected values form an $n \times 4$
matrix for each observation in {\tt x}.  

\item The predicted values ({\tt qi\$pr}) are draws from the
  multinomial distribution given the expected joint probabilities. 

\item The first differences ({\tt qi\$fd}) for each
  of the predicted joint probabilities are given by $$\textrm{FD}_{rs}
  = \Pr(Y_1=r, Y_2=s \mid x_1)-\Pr(Y_1=r, Y_2=s \mid x).$$  
  
\item The risk ratio ({\tt qi\$rr}) for each of the predicted joint
  probabilities are given by
\begin{equation*}
\textrm{RR}_{rs} = \frac{\Pr(Y_1=r, Y_2=s \mid x_1)}{\Pr(Y_1=r, Y_2=s \mid x)}
\end{equation*}

\item In conditional prediction models, the average expected treatment
  effect ({\tt att.ev}) for the treatment group is 
    \begin{equation*} \frac{1}{\sum_{i=1}^n t_i}\sum_{i:t_i=1}^n \left\{ Y_{ij}(t_i=1) -
      E[Y_{ij}(t_i=0)] \right\} \textrm{ for } j = 1,2,
    \end{equation*} 
    where $t_i$ is a binary explanatory variable defining the treatment
    ($t_i=1$) and control ($t_i=0$) groups.  Variation in the
    simulations are due to uncertainty in simulating $E[Y_{ij}(t_i=0)]$,
    the counterfactual expected value of $Y_{ij}$ for observations in the
    treatment group, under the assumption that everything stays the
    same except that the treatment indicator is switched to $t_i=0$.

\item In conditional prediction models, the average predicted treatment
  effect ({\tt att.pr}) for the treatment group is 
    \begin{equation*} \frac{1}{\sum_{i=1}^n t_i}\sum_{i:t_i=1}^n \left\{ Y_{ij}(t_i=1) -
      \widehat{Y_{ij}(t_i=0)} \right\} \textrm{ for } j = 1,2,
    \end{equation*} 
    where $t_i$ is a binary explanatory variable defining the treatment
    ($t_i=1$) and control ($t_i=0$) groups.  Variation in the
    simulations are due to uncertainty in simulating
    $\widehat{Y_{ij}(t_i=0)}$, the counterfactual predicted value of
    $Y_{ij}$ for observations in the treatment group, under the
    assumption that everything stays the same except that the
    treatment indicator is switched to $t_i=0$.
\end{itemize}

\subsubsection{Output Values}

The output of each Zelig command contains useful information which you
may view.  For example, if you run \texttt{z.out <- zelig(y \~\,
  x, model = "blogit", data)}, then you may examine the available
information in \texttt{z.out} by using \texttt{names(z.out)},
see the {\tt coefficients} by using {\tt z.out\$coefficients}, and
obtain a default summary of information through {\tt summary(z.out)}.
Other elements available through the {\tt \$} operator are listed
below.

\begin{itemize}
\item From the {\tt zelig()} output object {\tt z.out}, you may
  extract:
   \begin{itemize}
   \item {\tt coefficients}: the named vector of coefficients.   
   \item {\tt fitted.values}: an $n \times 4$ matrix of the in-sample
     fitted values.
   \item {\tt predictors}: an $n \times 3$ matrix of the linear
     predictors $x_j \beta_j$.
   \item {\tt residuals}: an $n \times 3$ matrix of the residuals.  
   \item {\tt df.residual}: the residual degrees of freedom.  
   \item {\tt df.total}: the total degrees of freedom.
   \item {\tt rss}: the residual sum of squares.  
   \item {\tt y}: an $n \times 2$ matrix of the dependent variables.
   \item {\tt zelig.data}: the input data frame if {\tt save.data = TRUE}.  
   \end{itemize}

\item From {\tt summary(z.out)}, you may extract:
  \begin{itemize}
  \item {\tt coef3}: a table of the coefficients with their associated
    standard errors and $t$-statistics.
  \item {\tt cov.unscaled}: the variance-covariance matrix. 
  \item {\tt pearson.resid}: an $n \times 3$ matrix of the Pearson residuals.  
  \end{itemize}

\item From the {\tt sim()} output object {\tt s.out}, you may extract
  quantities of interest arranged as arrays indexed by simulation
  $\times$ quantity $\times$ {\tt x}-observation (for more than one
  {\tt x}-observation; otherwise the quantities are matrices).
  Available quantities are:

   \begin{itemize}
   \item {\tt qi\$ev}: the simulated expected joint probabilities (or expected
     values) for the specified values of {\tt x}.  
   \item {\tt qi\$pr}: the simulated predicted outcomes drawn from a
     distribution defined by the expected joint probabilities.
   \item {\tt qi\$fd}: the simulated first difference in the
     expected joint probabilities for the values specified in {\tt x} and
     {\tt x1}.
   \item {\tt qi\$rr}: the simulated risk ratio in the predicted
     probabilities for given {\tt x} and {\tt x1}.
   \item {\tt qi\$att.ev}: the simulated average expected treatment
     effect for the treated from conditional prediction models.  
   \item {\tt qi\$att.pr}: the simulated average predicted treatment
     effect for the treated from conditional prediction models.  
   \end{itemize}
\end{itemize}

\subsection*{How to Cite}
\input{cites/blogit}
\input{citeZelig}
\subsection*{See also}
The bivariate logit function is part of the VGAM package by Thomas Yee \citep{YeeHas03}. In addition, advanced users may wish to refer to \texttt{help(vglm)} 
in the VGAM library.  Additional documentation is available at
\hlink{http://www.stat.auckland.ac.nz/\~\,yee}{http://www.stat.auckland.ac.nz/~yee}.Sample data are from \cite{Martin92}

To cite Zelig as a whole, please reference these two sources:
\begin{verse}
  Kosuke Imai, Gary King, and Olivia Lau. 2007. ``Zelig: Everyone's
  Statistical Software,'' \url{http://GKing.harvard.edu/zelig}.
\end{verse}
\begin{verse}
Imai, Kosuke, Gary King, and Olivia Lau. (2008). ``Toward A Common Framework for Statistical Analysis and Development.'' Journal of Computational and Graphical Statistics, Vol. 17, No. 4 (December), pp. 892-913. 
\end{verse}

\subsection*{See also}
The bivariate logit function is part of the VGAM package by Thomas Yee \citep{YeeHas03}. In addition, advanced users may wish to refer to \texttt{help(vglm)} 
in the VGAM library.  Additional documentation is available at
\hlink{http://www.stat.auckland.ac.nz/\~\,yee}{http://www.stat.auckland.ac.nz/~yee}.Sample data are from \cite{Martin92}

To cite Zelig as a whole, please reference these two sources:
\begin{verse}
  Kosuke Imai, Gary King, and Olivia Lau. 2007. ``Zelig: Everyone's
  Statistical Software,'' \url{http://GKing.harvard.edu/zelig}.
\end{verse}
\begin{verse}
Imai, Kosuke, Gary King, and Olivia Lau. (2008). ``Toward A Common Framework for Statistical Analysis and Development.'' Journal of Computational and Graphical Statistics, Vol. 17, No. 4 (December), pp. 892-913. 
\end{verse}

\subsection*{See also}
The bivariate logit function is part of the VGAM package by Thomas Yee \citep{YeeHas03}. In addition, advanced users may wish to refer to \texttt{help(vglm)} 
in the VGAM library.  Additional documentation is available at
\hlink{http://www.stat.auckland.ac.nz/\~\,yee}{http://www.stat.auckland.ac.nz/~yee}.Sample data are from \cite{Martin92}

To cite Zelig as a whole, please reference these two sources:
\begin{verse}
  Kosuke Imai, Gary King, and Olivia Lau. 2007. ``Zelig: Everyone's
  Statistical Software,'' \url{http://GKing.harvard.edu/zelig}.
\end{verse}
\begin{verse}
Imai, Kosuke, Gary King, and Olivia Lau. (2008). ``Toward A Common Framework for Statistical Analysis and Development.'' Journal of Computational and Graphical Statistics, Vol. 17, No. 4 (December), pp. 892-913. 
\end{verse}

\subsection*{See also}
The bivariate logit function is part of the VGAM package by Thomas Yee \citep{YeeHas03}. In addition, advanced users may wish to refer to \texttt{help(vglm)} 
in the VGAM library.  Additional documentation is available at
\hlink{http://www.stat.auckland.ac.nz/\~\,yee}{http://www.stat.auckland.ac.nz/~yee}.Sample data are from \cite{Martin92}
