\section{Introduction}

The use of function ``hooks\footnote{\url{http://en.wikipedia.org/wiki/Hooking}}''
in computer programming and software engineering is common. The technique is
useful when:

\begin{itemize}
  \item Data needs to be manipulated between function calls
  \item The developer needs to edit an object in the middle of a sequence of
    events
  \item Extending a feature of a model-fitting function
  \item Correcting quantities of interest that are computer incorrectly
\end{itemize}

\section{Simple Example}

\begin{verbatim}
robust.glm.hook <- function (obj, zcall, call, robust = FALSE, ...) {

  # If "robust" is a list, 
  if (is.list(robust)) {

    # if none of the entries of robust belong to the vector below
    if (!any(rob$method %in% c("vcovHAC", "kernHAC", "weave")))
      stop("robust contains elements that are not supported.")

    # Acquire the value of the robust parameter
    obj$robust <- robust
  }
  else if (!is.logical(robust))
    stop("Invalid input for robust: choose either TRUE or a list of options.")

  # Set as a robust generalized linear model model (in addition to other types)
  class(obj) <- c("glm.robust", class(obj))

  # Return...
  obj
}
\end{verbatim}

\section{Specification}



\subsection{Function Signature}

\subsection{Function Return Value}
