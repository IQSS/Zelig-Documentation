\section{Introduction}
% Introduction Material

For any Zelig module, the \emph{qi} function is ultimately the most
important piece of code that must be written; it describes the actual
process which simulates the \emph{quantities of interest}.  Because of
the nature of this process - and the gamut of statistical packages and
their underlying statistical model - it is rare that the simulation
process can be generalized for arbitrary fitted models.  Despite this,
it is possible to break down the simulation process into smaller steps.

\section{\code{qi} Method Signature}
% quick intro
The \emph{qi} function's signature accepts 4 parameters:
%
%
\begin{description}

	\item[obj:] An object of type \code{zelig}.  This wraps the fitted
		model in the slot ``result''
		
	\item[x:] An object of type \code{setx}.  This object is used to compute
		important coefficients, parameters, and features of the data.frame passed to
		the function call

	\item[x1:] Also an object of type ``\emph{setx}''.  This object is used in a
		similar fashion, however its presence allows a variety of \emph{quantities
		of interest} to be computed.  Notably, this is a necessary parameter to
		compute first-differences
	
	\item[num:] The number of simulations to compute

	\item[param:] An object of type \code{param}. This is the resulting object from
		the \code{param} function, typically containing a variety of important quantities
		- \code{simulations}, the \code{inverse link function}, \code{}

\end{description}


% code example
%
%\subsection{Code Example: \emph{qi} Function Signature}
\begin{verbatim}
qi.your_model_name <- function(z, x=NULL, x1=NULL, num=1000) {
	# start typing your code here
	# ...
	# ...
\end{verbatim}



% RETURN VALUE
\section{\code{qi} Method Return Value}
In order for Zelig to process the simulations, they must be returned in one of several formats:

\begin{itemize}
	% First Example
	\item{\begin{verbatim}
		list(
		     "TITLE OF QI 1" = val1,
		     "TITLE OF QI 2" = val2,
		     # any number of title-val pairs
		     # ...
		     "TITLE OF QI N" = val.n
		     )
	\end{verbatim}}

	% Second Example
	\item{\begin{verbatim}
		make.qi(
		        titles = list(title1, title2),
		        stats  = list(val1, val2)
		        )
	\end{verbatim}}
\end{itemize}


In the above example,\emph{val1, val2}are data.frames, matrices, or lists representing
the simulations of the \emph{quantities of interests}, and \emph{title1, title2} - and
any number of titles - are character-strings that will act as human-readable descriptions
of the \emph{quantities of interest}.  Once results are returned in this format, Zelig
will convert the results into a machine-readable format and summarize the simulations
into a comprehensible format.

NOTE: Because of its readability, it is suggested that the first method is used when
returning \emph{quantities of interest}.



%
%
\section{Notable Features of \emph{qi} Function}

The typical \emph{qi} function has several basic procedures:

\begin{enumerate}

	\item \emph{Call the param function}:  This is entirely optional but
		sometimes important for the clarity of your algorithm.  This step
		typically consists of taking random draws from the fitted model's
		underlying probability distribution.
		
	\item \emph{Compute the Quantity of Interest}: Depending on your model,
		there are several ways to compute necessary quantities of interest.
		Typical methods for computing quantities of interest include:
		\begin{enumerate}
			
			\item Using the sample provided by `param' to generate simulations
				of the \emph{quantities of interest}
			
			\item Using a Maximum-likelihood estimate on the fitted model
			
		\end{enumerate}
		
	\item \emph{Create a list of titles for your Quantities of Interest}: 
	
	\item \emph{Generate the Quantity of Interest Object}: Finally, with the
		computed Quantities of Interest, you must
		
\end{enumerate}


% Section: Function Body

\section{Details in Coding the \code{qi} Method}
\label{section:qi-details}

The function body of \emph{qi} function varies largely from model to model.  As a
result, it is impossible to create general guidelines to simulate \emph{quantities of
interest} - or even determine what the \emph{quantity of interest} is.  Typical methods
for computing \emph{quantities of interest} include:

\begin{itemize}

	\item Implementing sampling algorithms based on the underlying fitted model, or

	\item ``Predicting'' a large number of values from the fitted model

\end{itemize}



% find better way to output this data
\section{Simple Example \code{qi} function (\code{qi.logit.R})}


\begin{Code}
#' simulate quantities of interest for the logit models
qi.logit <- function(z, x=NULL, x1=NULL, y=NULL, num=1000, param=NULL) {

  # compute expected values using the function ".compute.ev"
  ev1 <- .compute.ev(obj, x, num, param)
  ev2 <- .compute.ev(obj, x1, num, param)

  # return simulations of quantities of interest
  list(
       "Expected Values: E(Y|X)"  = ev1,
       "Expected Values (for X1): E(Y|X1)" = ev2,
       "First Differences: E(Y|X1) - E(Y|X)" = ev2 - ev1
       )
}
\end{Code}