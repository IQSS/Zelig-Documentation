\documentclass{book}

\usepackage{fancyvrb}
\usepackage{ZeligDoc}
\usepackage{hyperref}

\title{Zelig Developers' Manual \code{v4.0-10}}

\begin{document}

% title
\maketitle


% make that toc, yo
\tableofcontents


% Quick Start Guide
\chapter[Quick Start Guide]{Quick Start Guide to Making a Zelig Model}
\label{chapter:quickstart}

% Section: introduction
% Modified:7/6/2011
% By: Matt Owen
% This section gives an introduction to Zelig

\section{Introduction}\label{section:introduction}

Programming a Zelig module is a simple procedure. By following several simple steps, any statistical model can be implemented in the Zelig software suite. The following document places emphasis on speed and practicality, rather than the numerous, technical details involved in developing statistical models. That is, this guide will explain how to quickly and most simply include existing statistical software in the Zelig suite.


% Section: overview
% Modified:7/6/2011
% By: Matt Owen
% This section gives an overview of Zelig's necessary components

\section{Overview}\label{section:overview}

In order for a Zelig model to function correctly, four components need to exist:

\begin{description}

	\item[\emph{a statistical model}:] This can be any statistical model of the developer's choosing, though it is suggested that it be written in R. Examples of statistical models already implemented in Zelig include: Brian Ripley's \code{glm} and Kosuke Imai's \code{MNP} models.

	\item[\code{zelig2}\emph{model}] This method acts as a bridge between the external statistical model and the Zelig software suite

	\item[\code{param.}\emph{model}] This method specifies the simulated parameters used to compute quantities of interest

	\item[\code{qi.}\emph{model}] This method computes - using the fitted statistical model, simulated parameters, and explanatory data - the \emph{quantities of interest}. Compared with the \code{zelig2} and \code{param} methods, 

\end{description}

In the above description, replace the italicized \emph{model} text with the name of the developer's model. For example, if the model's name is ``logit'', then the corresponding methods will be titled \code{zelig2logit}, \code{param.logit}, and \code{qi.logit}.


% Section: zelig.skeleton
% Modified:7/6/2011
% By: Matt Owen
% This section gives details concerning zelig.skeleton

\section{\code{zelig.skeleton}: Automating Zelig Model Creation}\label{zelig.skeleton}

The fastest way to setup and begin programming a Zelig model is the use the \code{zelig.skeleton} function, available within the \code{Zelig} package. This function allows a fast, simple way to create the \code{zelig2}, \code{describe}, \code{param}, and \code{qi} methods with the necessary boilerplate. As a result, \code{zelig.skeleton} closely mirrors the \code{package.skeleton} method included in core R.

\subsection{A Demonstrative Example}

\begin{verbatim}
library(Zelig)  # [1]

zelig.skeleton(
               "my.zelig.package",                # [2]
               models = c("gamma", "logit"),      # [3]
               author = "Your Name",              # [4]
               email = "your.email@someplace.com" # [5]
               )
\end{verbatim}

\subsection{Explanation of the \code{zelig.skeleton} Example}

The above numbered comments correspond to the following:

\begin{description}

	\item[{[1]}] The Zelig package must be imported when using \code{zelig.skeleton}.

	\item[{[2]}] The first parameter of \code{zelig.skeleton} specifies the name of the package

	\item[{[3]}] The \code{models} parameter specifies the titles of the Zelig models to be included in the package. In the above example, all necessary files and methods for building the ``gamma'' and ``logit'' models will be included in Zelig package.

	\item[{[4]}] Specify the author's name

	\item[{[5]}] Specify the email address of the software maintainer

\end{description}

\subsection{Conclusion}

The \code{zelig.skeleton} function provides a way to automatically generate the necessary methods and file to create an arbitrary Zelig package. The method body, however, will be incomplete, save for some light documentation additions and programming boilerplate. For a detailed specification of the \code{zelig.skeleton} method, refer to Zelig help file by typing:

\begin{verbatim}
library(Zelig)

?zelig.skeleton
\end{verbatim}

{\noindent}in an interactive R-session.
 

\section{\emph{zelig2}: Interacting with Existing Statistical Models in Zelig}
\label{section:zelig2}

The \code{zelig2} function acts as the bridge between the Zelig module and the existing statistical model. That is, the results of this function specify the parameters to be passed to a \emph{previously completed} statistical model-fitting function. In this sense, there is nothing tricky about the \code{zelig2} function. Simply construct a list with key-value pairs in the following fashion:

\begin{itemize}

  \item {\bf Keys} (names on the lefthand-side of an equal sign) represent
        parameters that are submitted to the existing model function

  \item {\bf Values} (variables, etc. on the righthand-side of an equal sign)
        represent values to set the corresponding the parameter to.
        
  \item {\bf Keys with leading periods} are typically reserved for specific
        \code{zelig2} purposes. In particular, the key \code{.function}
        specifies the name of the function that calls the existing statistical
        model.
	\item[an ellipsis (\dots)] specifies that all additional, optional parameters not specified in the signature of the \code{zelig2model\_function} method, will be included in the external method's call, despite not being specifically set.


\end{itemize}

\subsection{A Simple Example}

\noindent For example, if a developer wanted to call an existing model
\code{"SomeModel"} with the parameter \code{weights} set to \code{1},
the appropriate return-value (a list) for the \code{zelig2} function would be:


% SHORT EXAMPLE
\begin{verbatim}
zelig2some.model <- function(formula, data) {
    list(.function = "SomeModel",
         formula   = formula,
         weights   = 1
         )
}
\end{verbatim}


% ....
\subsection{A More Detailed Example}

\noindent A more typical example would be the case of fitting a basic logistic
regression. The following code, already implemented in Zelig, acts as an
interface between Zelig packages and R's built-in \code{glm} function:


% LONG EXAMPLE
\begin{verbatim}
zelig2logit <- function (formula, weights = NULL, ..., data) {
  list(.function = "glm",    # [1]
       
       formula = formula,    # [2]
       weights = weights,    # ...
       data    = data,       # ...

       family  =             # [3]
                 binomial(link="logit"),
       model   = FALSE       # ...
       )
}
\end{verbatim}

\noindent The comments in the above code correspond to the following:

\begin{description}
	\item[{[1]}] Pass all parameters to the \code{glm} function

	\item[{[2]}] Specify that the parameters \code{formula}, \code{weights}, and \code{data} be given the same values as those passed into the \code{zelig2} function itself. That is, whichever values the end-user passes to \code{zelig} will be passed to the \code{glm} function

	\item[{[3]}] Specify that the parameters \code{family} and \code{model} \emph{always} be given the corresponding values - \code{binomial(link="logit")} and \code{FALSE} - regardless of what the end-user passes as a parameter.
	
\end{description}

Note that the parameters - \code{formula}, \code{weights}, \code{data}, \code{family}, \code{model} - correspond to those of the \code{glm} function. In general, this will be the case for any \code{zelig2} method. That is, every \code{zelig2} method should return a list containing the parameters belonging to the external model, as well as, the reserved keyword \code{.function}.

If you are unsure about the parameters that are passed to an existing statistical model, simply use the \code{args} or \code{formals} functions (included in R). For example, to get a list of acceptable parameters to the \code{glm} function, simply type:

\begin{verbatim}
args(glm)
\end{verbatim}


\subsection{An Even-More Detailed Example}

{\noindent}Occasionally the statistical model and the standard style of Zelig input differ. In these instances, it may be necessary to manipulate information about the \code{formula} and \code{constraints}. This additional step in building the \code{zelig2} method is common only amongst multivariate models, as seen below in the \code{bprobit} model (bivariate probit regression for Zelig).

% Complex example of a Zelig 2 Function
%

\begin{verbatim}
zelig2bprobit <- function(formula, ..., data) {

  # [1]
  formula <- parse.formula(formula, "bprobit")
  
  # [2]
  tmp <- cmvglm(formula, "bprobit", 3)

  
  # return list
  list(
       .function = "vglm",    # [3]
       
       formula = tmp$formula, # [4]
       family  = bprobit,     # [5]
       data = data,
       # [6]
       constraints = tmp$constraints
       )
}
\end{verbatim}

{\noindent \bf The following is an explanation of the above code:}

% Describe the above code
%

\begin{description}

	\item[{[1]}] Convert Zelig-style \code{formula} data-types into the style that the \code{vglm} function understands

	\item[{[2]}] Extract constraint information from the \code{formula} object, as is the style commonly supported by Zelig

	\item[{[3]}] Specify the \code{vglm} as the statistical model fitting function

	\item[{[4]}] Specify the formula to be used by the \code{vglm} function when performing the model fitting. Note that this object is created by using both the \code{parse\.formula} and \code{cmvglm} functions 

	\item[{[5]}] Specify the \code{family} of the model

	\item[{[6]}] Specify the constraints to be used by the \code{vglm} function when performing the model fitting. Note that this object is created by using both the \code{parse\.formula} and \code{cmvglm} functions

\end{description}

% Explain how to look up information on the cmvglm and parse.formula functions
%

\noindent Note that the functions \code{parse.formula} and \code{cmvglm} are included in the core Zelig software package. Information concerning these functions can be found by typing:

\begin{verbatim}
library(Zelig)

?parase.formula
?cmvglm
\end{verbatim}

in an interactive R-session.


\subsection{Summary and More Information aboput \code{zelig2} Methods}

\code{zelig2} functions can be of varying difficulty - from simple parameter passing to reformatting and creating new data objects to be used by the external model-fitting function. To see more examples of this usage, please refer to the \code{survey.zelig} and \code{multinomial.zelig} packages. Regardless of the model's complexity, it ends with a simple list specifying which parameters to pass to a preexisting statistical model.

For more information on the \code{zelig2} function's full features, see
the \emph{Advanced zelig2 Manual}, or type:

\begin{verbatim}
library(Zelig)

?zelig2
\end{verbatim}

within an interactive R-session.


% Section:  param
% Modified: 6/29/2011
% By: Matt Owen
% This section explains how to use the param API

\section{\emph{param}: Simulating Parameters}\label{section:param}


The \code{param} function simulates and specifies parameters necessary for computing
\emph{quantities of interest}. That is, the \code{param} function is the ideal place
to specify information necessary for the \code{qi} method. This includes:

\begin{description}

	\item[Ancillary parameters] These parameters specifying information about the
		underlying probability distribution. For example, in the case of the Normal
		Distribution, $\sigma$ (standard deviation) and $\mu$ (mean) would be considered ancillary parameters.
	
	\item[Link function] That is, the function providing the relationship between
		the predictors and the mean of the distribution function. This is typically of
		very little importance (compared to the inverse link function), but is frequently included for completeness. For Gamma distribution, the link function is the inverse function: $ f(x) = \frac{1}{x} $
	
	\item[Inverse link function] Typically crucial for simulating \emph{quantities of
		interest} of \emph{Generalized Linear Models}. For the binomial distribution, the inverse-link function is the logit function: $ f(x) = \frac{e^x}{1+e^x} $
	
	\item[Simulated Parameters] These random draws simulate the parameters of the fitted statistical model. Typically, the \code{qi} method uses these to simulate \emph{quantities of interest} for the given model. As a result, these are of paramount importance.

\end{description}

\noindent The following sections describe how these ideas correspond to the structure of a well-written
\code{param} function.

\subsection{The Function Signature}

The \code{param} function takes only two parameters, but outputs a wealth of information important in computing \emph{quantities of interest}. The following is the function signature:

\begin{verbatim}
  param.logit <- function (obj, num)
\end{verbatim}

\noindent The above parameters are:

\begin{description}
	\item[obj] An object of class \code{zelig}
		\footnote{
		%
		%
		For a detailed specification of the \code{zelig} class, type: \code{?zelig} within a interactive Zelig-session.
		}. This contains the fitted statistical model and associated information.	
			
	\item[num] An integer specifying the number of simulations to be drawn. This value is specified by the end-user, and defaults to \code{1000} if no value is specified.

\end{description}

% For full technical documen

\subsection{The Function Return Value}

In similar fashion to the \code{zelig2} method, the \code{param} method takes return values as a list of key-value pairs. However, the options are not as diverse. That is, the list can only be given a set of specific values: \code{ancillary}, \code{coef}, \code{simulations}, \code{link}, \code{linkinv}, and \code{family}.

In most cases, however, the parameters \code{ancillary}, \code{simulations}, and \code{linkinv} are sufficient. The following is an example take from Zelig's \code{gamma} model:

\pagebreak
\begin{verbatim}
# Simulate Parameters for the gamma Model
param.gamma <- function(obj, num) {

# NOTE: gamma.shape is a method belonging to the
#       GLM class, specifying maximum likelihood
#       estimates of the distribution's shape
#       parameter. It is a list containing two
#       values: 'alpha' and 'SE'

  shape <- gamma.shape(obj)
	
  # simulate ancillary parameters
  alpha <- rnorm(n=num, mean=shape$alpha, sd=shape$SE)
  
  # simulate maximum
  sims <- mvrnorm(n = num, mu = coef(obj), Sigma = vcov(obj))

	# return results  
  list(
       alpha = alpha,       # [1]
       simulations  = sims, # [2]
                            # ...
                               
                            # [3]
       linkinv = function (x) 1/x
       )
}

\end{verbatim}

The above code does the following:

\begin{description}

	\item[{[1]}] Specify the ancillary parameters, typically referred to as the greek letter $\alpha$. In the above example, \code{alpha} is the \emph{shape} of the model's underlying gamma distribution.
	
	\item[{[2]}] Specify the parameter simulations, typically referred to as the greek letter $\beta$, to be used in the \code{qi} function.
	
	\item[{[3]}] Specify the inverse-link function
		\footnote{The ``inverse-link'' function is also commonly referred to as the ``mean'' function. Typically, this function specifies the relationship between linear predictors and the mean of a distribution function. As a result, it is only used in describing \emph{generalized linear models} },
		used to compute \emph{expected values} and a variety of other \emph{quantities of interest}, once samples are extracted from the model's statistical distribution.

\end{description}


\subsection{Summary and More Information \code{param} Methods}

The \code{param} method's basic purpose is to describe the statistical and systematic variables of the Zelig model's underlying distribution. Defining this method is an important step towards simplifying the \code{sim} method. That is, by specifying features of the model - coefficients, systematic components, inverse link functions, etc. - and simulating specific parameters, the \code{sim} method can focus entirely on simulating \emph{quantities of interest}.
	


% Section:  qi
% Modified: 6/29/2011
% By: Matt Owen
% This section explains how to use the qi API

\section{qi: Simulating Quantities of Interest}\label{section:qi}

The \code{qi} function of any Zelig model simulates \emph{quantities of interest}
using the fitted statistical model, taken from the \code{zelig2} function,
and the simulated parameters, taken from the \code{param} function. As a result,
the \code{qi} function is the most important component of a Zelig model.

\subsection{The \code{qi} Function Signature}

While the implementation of the \code{qi} function can differ greatly from one
model to another, the signature always remains the same and closely parallels the 
signature of the \code{sim} function.


\begin{verbatim}
qi.logit <- function(obj, x=NULL, x1=NULL, y=NULL, param=NULL)
\end{verbatim}



\subsection{The \code{qi} Function Return Values}

Similar to the return values of both the \code{zelig2} and \code{param} function,
the \code{qi} function takes an list of key-value pairs as a return value. The keys,
however, follow a much simpler convention, and a single rule: the key (left-side
of the equal sign) is a \emph{quoted} character-string naming the \emph{quantity of
interest} and the value (right-side of the equal sign) are the actual simulations.

The following is a short example:

\begin{verbatim}
  list(
       "Expected Value"  = ev,
       "Predicted Value" = pv
       )
\end{verbatim}

\noindent where \code{ev} and \code{pv} are respectively simulations of the model's
\emph{expected values} and \emph{predicted values}.

\subsection{Coding Conventions for the \code{qi} Function}

While the following is unnecessary, it provides a few simple guidelines to simplifying
and improving readability of a model's \code{qi} function:

\begin{itemize}

	\item Divide repetitive work amongst other functions. For example, if you simulate
		an \emph{expected value} for both the \code{x} and \code{x1}, it is better to 
		write a \code{.compute.ev} function and simply call it twice
		
	\item Always compute an \emph{expected values} and \emph{predicted values} independently
		and before writing code to create \emph{first differences}, \emph{risk ratios}, and
		\emph{average treatment effects}
		
	\item Write code for \emph{average treatment effects} only after all the other code has
		been debugged and completed

\end{itemize}

\pagebreak
\subsection{A Simplified Example}

The following is a simplified example of the \code{qi} function for the logit model. Note that the example is divided into two sections: one specifying the return values and titles of the \emph{quantities of interest} (see Section~\ref{example:qi.logit}) and one computing the simulated \emph{expected values} of the model (see Section~\ref{example:.compute.ev}).


\subsubsection{\code{qi.logit} Function}\label{example:qi.logit}
\begin{verbatim}
#' simulate quantities of interest for the logit models
qi.logit <- function(obj, x=NULL, x1=NULL, y=NULL, num=1000,
                     param=NULL) {
  # [1]
  ev1 <- .compute.ev(obj, x, num, param)
  ev2 <- .compute.ev(obj, x1, num, param)

  # [2]
  list(
       "Expected Values: E(Y|X)"  = ev1,
       "Expected Values (for X1)" = ev2,
       
  # [3]
       "First Differences: E(Y|X1) - E(Y|X)" = ev2 - ev1
       )
}
\end{verbatim}


\subsubsection{\code{.compute.ev} Function}\label{example:.compute.ev}
\begin{verbatim}
.compute.ev <- function(obj, x=NULL, num=1000, param=NULL) {
  # values of NA are ignored by the summary function
  if (is.null(x))
    return(NA)

  # extract simulations
  coef <- coef(param)
  
  link.inverse <- linkinv(param)

  eta <- coef %*% t(x)

  # invert link function
  theta <- matrix(link.inverse(eta), nrow = nrow(coef))
  ev <- matrix(theta, ncol=ncol(theta))

  ev
}  
\end{verbatim}

\noindent The above code illustrates a few of the ideas:

\begin{description}

	\item[{[1]}] Compute \emph{quantities of interest} using re-usable functions that express the idea clearly. This both reduces the amount of code necessary to produce the simulations, and improves readability of the source code.
	
	\item[{[2]}] Return \emph{quantities of interest} as a list. Note: titles of
		\emph{quantities of interest} are on the left of the equal signs, while
		simulated values are on the right.
		
	\item[{[3]}] Simulate \emph{first differences} by using two previous computed \emph{quantities of interest}.
	  
	\item[{[4]}] Define an additional function that simulates \emph{expected values}, rather than placing such code in the actual \code{qi} method.

\end{description}

\noindent In addition, this function two \emph{generic functions} that are
defined in the Zelig software suite, and are particularly used with the \code{param} class:

\begin{description}

	\item[coef] Extract the simulations of the parameters. Specifically, this returns the simulations produced in the \code{param} function
	
	\item[linkinv] Return the inverse of the link function. Specifically, this returns the inverse-link functions specified in the \code{param} function

\end{description}

\subsection{Summary and More Information about \code{qi} Methods}

The \code{qi} function offers a simple template for computing \emph{quantities of interest}. Particularly, if a few a coding conventions are followed, the \code{qi} function can provide transparent, easy-to-read simulation methods.

\section{Conclusion}

The above sections detail the fastest way to develop Zelig models. For the vast majority of applications and external statistical packages, this should suffice. However, at times, more elaborate measures may need to be taken. If this is the case, the API specifications for each particular methods should be read, since a wealth of information has been omitted in order to simplify this tutorial.

For more detailed information, consult the \code{zelig2}, \code{param}, and \code{qi} sections of the Zelig Development manual.



% sample chapter
%  delete this chapter afterwards.
%  This is kept for reference and not intended to be in
%  the final book
% \chapter[sample]{Sample Chapter}
% \label{chapter:sample}
% 

\section{Introduction}
\label{section:sampintro}
Express the purpose and significance of this method


\section{Method Signature}
\label{section:sampsig}


\section{Method Return Values}
\label{section:sampreturn}


\section{Notable Features}
\label{section:sampfeatures}


\section{Details in Coding}
\label{section:sampdetails}


\section{Example}
\label{section:example}


\section{Conclusion}
\label{section:conclusion}





% describe method
\chapter[\code{describe}]{\code{describe}: Describing a Zelig Model}
\label{chapter:describe}
\documentclass[a4paper,10pt]{article}

\begin{document}

\title{The \emph{describe} Function}
\author{Matt Owen}
\maketitle

\section{Introduction}
The {\tt describe} function serves two purposes:

\begin{enumerate}
	\item{to give correct citation information for the Zelig model}
	\item{to declare the type of data-sets that can be processed by the Zelig model}
\end{enumerate}

The developer can accomplish these two things simply by writing the {\tt describe} function for their model.

\section{Form of a {\tt describe} Function}
The {\tt describe} function should - in almost all cases - simply return a list or character-vector specifying the author, year or publication, and description of the developed model.  That is, 

\section{Example of a {\tt describe} Function}
\begin{verbatim}
describe.logit <- function(zelig.obj)
  list(
       author   = c("Kosuke Imai", "Gary King"),
       year     = 2008,
       describe = "Logistic Regression for Dichotomous Dependent Variables"
       )
\end{verbatim}

\section{Resulting Citation from the Above Example}
\begin{verbatim}
How to cite this model in Zelig:
  Kosuke Imai, Gary King, and Olivia Lau. 2008.
  "logit: Logistic Regression for Dichotomous Dependent Variables"
  in Kosuke Imai, Gary King, and Olivia Lau,
  "Zelig: Everyone's Statistical Software,"
  http://gking.harvard.edu/zelig
\end{verbatim}

\section{Explanation of the Above Example}
The above example is an actual copy of the {\tt describe} function for the ``logit'' model in Zelig's core package.  It specifies the author, the year of publication, and the description text in a clear and concise manner.  All Zelig models can have citation information generated for the exact same fashion that the logit model does.
\end{document}


% zelig2 method
\chapter[\code{zelig2}]{\code{The zelig2 Method API}}
\label{chapter:zelig2}
\documentclass[11pt]{article}
\usepackage{ZeligDoc}
\begin{document}

% TITLE INFORMATION
\title{Making the Model Compatible with Zelig: Writing the \emph{zelig2} Function}
\author{Matthew Owen}
\maketitle


% INTRODUCTION
\section{Introduction}
Developers can develop a model, write the model-fitting function, and test it
within the Zelig framework without explicit intervention from the Zelig team. 
This modularity relies on two R programming conventions:


\begin{enumerate}

	\item {\bf wrappers}, which pass arguments from R functions to other R functions
		or foreign function calls (such as in C, C++, or Fortran).  This step is
		facilitated by - as will be explained in detail in the upcoming chapter -
		the {\tt zelig2} function.
		
	\item {\bf classes}, which tell generic functions how to handle objects of a given
		class.  For a statistical model to be compliant with Zelig, the model-fitting
		function \emph{must} return a classed object.
		
\end{enumerate}

Zelig implements a unique and simple method for incorporating existing statistical
models which lets developers test \emph{within} the Zelig framework \emph{without} any
modification of both their own code or the {\tt zelig} function itself.  The heart of
this procedure is the {\tt zelig2} function, which acts as an interface between the
{\tt zelig} function and the existing statistical model.  That is, the {\tt zelig2}
function maps the user-input from the {\tt zelig} function into input for the existing
statistical model's constructor function.  Specifically, a Zelig model requires:

% !!
%
\begin{enumerate}

	\item An existing statistical model, which is invoked through a function call and
		returns an object

	\item A {\tt zelig2} function which maps user-input from the {\tt zelig} function to
		the existing statistical model
		
	\item A name for the {\tt \bf zelig} model, which can differ from the original name of
		the statistical model.
		
\end{enumerate}


% THE ZELIG2 FUNCTION
\section{The \emph{zelig2} Function}
The following sections explain how to write a {\tt zelig2} function, given an arbitrary
statistical model.  In the illustrative examples, the following conventions are used:



\begin{description}

	\item[model] will refer to the name of the \emph{Zelig} model, not the name of the
		existing model - though these two names are not necessarily different.  If the developer
		names his model ``logit'' then model refers to ``logit''.

	\item[model\_function] will refer to the name of function that produces the existing
		statistical model.  If the developer is writing a wrapper for R's built-in logit function,
		then \emph{model\_function} refers to ``glm''.
		
	\item[zelig2model] will refer to the name of the {\tt zelig2} function.  If the developer
		names his model ``logit'', then \emph{zelig2model} refers to ``zelig2logit''.
		
\end{description}


% WRITING THE ZELIG2 FUNCTION
\subsection{Writing the \emph{zelig2} Function}

The {\tt zelig2} function should follow several specific conventions:

\begin{enumerate}

	\item The {\tt zelig2model} function should be simply named \emph{zelig2model}, where
		\emph{model} is the chosen name for the zelig package

	\item The {\tt zelig2model} function itself should have arguments that list entirety of
		possible inputs to the {\tt model\_function}

	\item The {\tt zelig2model} function should return a list of key-value pairs that represent
		the map from {\tt zelig} input to {\tt model\_function} input

\end{enumerate}


% EXAMPLE USING zelig2logit
\pagebreak
\subsection{Example of a \emph{zelig2} Function}

\begin{verbatim}
zelig2logit <- function(formula, ..., data, weights=NULL)
  list(
       .function = "glm",
        
       formula = formula,
       data    = data,
       weights = weights,
        
       family = binomial(link="logit"),
       model  = FALSE
       )
\end{verbatim}


% EXPLANATION
\subsection{Explanation of \code{zelig2} Return Values}

A {\tt zelig2model} function must always return a list as its return value.

The entries of the returned list have the following format:

\begin{description}

	\item[\code{.function}] specifies, with a character string, {\tt model\_function} used to fit the data
	
	\item[key-value pairs] represent an explicit mapping specified by the developer for the
		parameter that matches ``key''. Value are specified typically in one of two ways:

		\begin{itemize}
		
			\item As a variable based on some information from the user. In the above example, this
				corresponds to the \code{formula}, \code{data}, and \code{weights} keys in the returned
				list. \emph{Note how all the parameters make use of variables specified in the function
				signature}

			\item As a variable that is statically set. This is useful for parameters that do not require
				user-input. In the above example, this corresponds to the \code{family} and \code{model}
				parameters, as their values are specified to specific values regardless of user-input.
				\emph{Note how neither parameter make no use of the parameters in the function signature}
						
		\end{itemize}
	
	\item[an ellipsis (\dots)] specifies that all additional, optional parameters not specified in the signature of the \code{zelig2model\_function} method, will be included in the external method's call, despite not being specifically set.
				
	
\end{description}


\end{document}

























% param method
\chapter[\code{param}]{\code{The param Method API}}
\label{chapter:param}
\section{Introduction}
\label{section:param-intro}

Several general features - sampling distribution, link function,
systematic component, ancillary parameters, etc. - comprise
statistical models.  These features, while vastly differing between
any two given specific models, share features that are easily
classifiable, and usually necessary in the simulation of
\emph{quantities of interest}.  That is, all statistical models have
similar traits, and can be simulated using similar methods.  Using
this fact, the \emph{parameters} class provides a set of functions
and data-structures to aid in the planning and implementation of 
the statistical model.

\section{Method Signature of \code{param}}

The signature of the \code{param} method is straightforward and does not vary between differ Zelig models.

\begin{Code}
param.logit <- function (obj, num, ...) {
  # ...
}
\end{Code}

\section{Return Value of \code{param}}

The return value of a \code{param} method is simply a list containing several entries:

\begin{description}
	\item[simulations] A vector or matrix of random draws taken from
		the model's distribution function.  For example, a logit model
		will take random draws from a Multivariate Normal distribution.
		
	\item[alpha] A vector specifying parameters to be passed into
		the distribution function.  Values for this range from scaling
		factors to statistical means.
	
	\item[fam] An optional parameter.  \emph{fam} must be an object
		of class ``family''.  This allows for the implicit specification
		of the link and link-inverse function.  It is recommend that the
		developer set either this, the link, or the linkinv parameter
		explicitly.  Setting the family object implicitly defines
		\emph{link} and \emph{linkinv}.
		
	\item[link] An optional parameter.  \emph{link} must be a function.
		Setting the link function explicitly is useful for defining
		arbitrary statistical models.  \emph{link} is used primarily to
		numerically approximate its inverse - a necessary step for
		simulating \emph{quantities of interest}.
		
	\item[linkinv] An optional parameter.  \emph{linkinv} must be a
		function.  Setting the link's inverse explicitly allows for faster
		computations than a numerical approximation provides.  If the
		inverse function is known, it is recommended that this function
		is explicitly defined.
		
\end{description}

%\section{Methods of the \code{parameters} Object}
%\begin{description}
%	\item[alpha] Extracts the contents of alpha
%	\item[coef] Extracts the simulated parameters specified in the key \code{simulations}
%	\item[link] Extracts the link function from the parameter.  This value exists as long as \emph{fam} or \emph{link} are explicitly set
%	\item[linkinv] Extracts the link-inverse function from the parameters. This value exists as long as \emph{fam}, \emph{link} or \emph{linkinv} are explicitly set.  If \emph{linkinv} is not explicitly set, then a numerical approximation is used based on \emph{fam} or \emph{link}
%\end{description}


\section{Writing the \emph{param} Method}

The ``param'' function of an arbitrary Zelig model draws samples from the
model, and describes the statistical model.  In practice, this may be done
in a variety of fashions, depending upon the complexity of the model


% LIST METHODS
% ------------
% LOGIT EXAMPLE
\subsection{List Method: Returning an Indexed List of Parameters}

While the simple method of returning a vector or matrix from a \emph{param} function is extremely simple, it has no method for setting link or link-inverse functions for use within the actual simulation process.  That is, it does not provide a clear, easy-to-read method for simulating \emph{quantities of interest}.  By returning an indexed list - or a parameters object - the developer can provide clearly labeled and stored link and link-inverse functions, as well as, ancillary parameters.


\subsubsection{Example of Indexed List Method with \emph{fam} Object Set}

\begin{verbatim}
param.logit <- function(z, x, x1=NULL, num=num)
  list(
       coef  = mvrnorm(n=num, mu=coef(z), Sigma=vcov(z)),
       alpha = NULL,
       fam   = binomial(link="logit")
       )
\end{verbatim}

% warum kommst du nicht herueber?

\subsubsection{Explanation of Indexed List with \emph{fam} Object Set Example}

The above example shows how link and link-inverse functions (for a ``logit'' model) can be set using a ``family'' object.  Family objects exist for most statistical models - logit, probit, normal, Gaussian, et cetera - and come preset with values for link and link-inverses.  This method does not differ immensely from the simple, vector-only method; however, it allows for the use of several API functions - \emph{link}, \emph{linkinv}, \emph{coef}, \emph{alpha} - that improve the readability and simplicity of the model's implementation.

The \emph{param} function and the \emph{parameters} class offer methods for automating and simplifying a large amount of repetitive and cumbersome code that may come with building the arbitrary statistical model.  While both are in principle entirely optional - so long as the \emph{qi} function is well-written - they serve as a means to quickly and elegantly implement Zelig models.


% POISSON EXAMPLE
\subsubsection{Example of Indexed List Method (with \emph{link} Function) Set}

\begin{verbatim}
param.poisson <- function(z, x, x1=NULL, num=num) {
  list(
       coef = mvrnorm(n=num, mu=coef(z), Sigma=vcov(z)),
       link = log,
             
       # because ``link'' is set,
       # the next line is purely optional
       linkinv = exp
       )
}
\end{verbatim}

\subsubsection{Explanation of Indexed List (with \emph{link} Function) Example}

The above example shows how a \emph{parameters} object can be created with by explicitly setting the statistical model's link function.  The \emph{linkinv} parameter is purely optional, since Zelig will create a numerical inverse if it is undefined.  However, the computation of the inverse is typically slower than non-iterative methods.  As a result of this, if the link-inverse is known, it should be set, using the \emph{linkinv} parameter.

The above example can also contain an \emph{alpha} parameter, in order to store important ancillary parameters - mean, standard deviation, gamma-scale, etc. - that would be necessary in the computation of \emph{quantities of interest}.


%
\section{Using a \emph{parameters} Object}

Typically, a \emph{parameters} object is used within a model's \emph{qi} function.  While the developer can typically omit the \emph{param} function and the \emph{parameters} object, it is not recommended.  This is because making use of this function can vastly improve readability and functionality of a Zelig model.  That is, \emph{param} and \emph{parameters} automate a large amount of repetitive, cumbersome code, and offer allow access to an easy-to-use API.

\subsection{Example \emph{param} Function}

\begin{Code}
qi.logit <- function(z, x, x1=NULL, sim.param=NULL, num=1000) {
  coef <- coef(sim.param)
  inverse <- linkinv(sim.param)

  eta <- coef %*% t(x)
  theta <- link.inverse(eta)

  # et cetera...
}

\end{Code}


\subsection{Explanation of Above \emph{qi} Code}

The above is a portion of the actual code used to simulate \emph{quantities of interest} for a ``logit'' model.  By using the sim.par object, which is automatically passed into the function if a \emph{param} function is written, \emph{quantities of interest} can be computed extremely generically.  The step-by-step process of the above function is as follows:

\begin{itemize}
	\item{Assign the simulations from \emph{param.logit} to the variable ``coef''}
	\item{Assign the link-inverse from \emph{param.logit} to the variable ``inverse''}
	\item{Compute $\eta$ (eta) by matrix-multiplying our simulations with our explanatory results}
	\item{Conclude ``simulating'' the \emph{quantities of interest} by applying the inverse of the link function.  The result is a vector whose median is an approximate value of the \emph{quantity of interest} and has a standard deviation that will define the confidence interval around this value}
	
\end{itemize}


\section{Future Improvements}

In future releases of Zelig, \emph{parameters} will have more API functions to facilitate common operations - sample drawing, matrix-multiplication, et cetera - so that the developer's focus can be exclusively on implementing important components of the model.


% qi method
\chapter[\code{qi}]{\code{The qi Method API}}
\label{chapter:qi}
\section{Introduction}
% Introduction Material

For any Zelig module, the \emph{qi} function is ultimately the most
important piece of code that must be written; it describes the actual
process which simulates the \emph{quantities of interest}.  Because of
the nature of this process - and the gamut of statistical packages and
their underlying statistical model - it is rare that the simulation
process can be generalized for arbitrary fitted models.  Despite this,
it is possible to break down the simulation process into smaller steps.


%
%
\section{Notable Features of \emph{qi} Function}

The typical \emph{qi} function has several basic procedures:

\begin{enumerate}

	\item \emph{Call the param function}:  This is entirely optional but
		sometimes important for the clarity of your algorithm.  This step
		typically consists of taking random draws from the fitted model's
		underlying probability distribution.
		
	\item \emph{Compute the Quantity of Interest}: Depending on your model,
		there are several ways to compute necessary quantities of interest.
		Typical methods for computing quantities of interest include:
		\begin{enumerate}
			
			\item Using the sample provided by `param' to generate simulations
				of the \emph{quantities of interest}
			
			\item Using a Maximum-likelihood estimate on the fitted model
			
		\end{enumerate}
		
	\item \emph{Create a list of titles for your Quantities of Interest}: 
	
	\item \emph{Generate the Quantity of Interest Object}: Finally, with the
		computed Quantities of Interest, you must
		
\end{enumerate}


%
\section{Basic Layout of a \emph{qi} Function}
Now with the general outline of a \emph{qi} function defined, it is
important to discuss the expected procedures and specifics of
implementation.


\subsection{The Function's Signature}
% quick intro
The \emph{qi} function's signature accepts 4 parameters:


%
%
\begin{description}

	\item[obj:] An object of type \code{zelig}.  This wraps the fitted
		model in the slot ``result''
		
	\item[x:] An object of type \code{setx}.  This object is used to compute
		important coefficients, parameters, and features of the data.frame passed to
		the function call

	\item[x1:] Also an object of type ``\emph{setx}''.  This object is used in a
		similar fashion, however its presence allows a variety of \emph{quantities
		of interest} to be computed.  Notably, this is a necessary parameter to
		compute first-differences
	
	\item[num:] The number of simulations to compute

	\item[param:] An object of type \code{param}. This is the resulting object from
		the \code{param} function, typically containing a variety of important quantities
		- \code{simulations}, the \code{inverse link function}, \code{}

\end{description}


% code example
%
\subsection{Code Example: \emph{qi} Function Signature}
\begin{verbatim}
qi.your_model_name <- function(z, x=NULL, x1=NULL, num=1000) {
	# start typing your code here
	# ...
	# ...
\end{verbatim}


\noindent Note: In the above example, the function name ``qi.your\_model\_name'' is
merely a placeholder.  In order to register a \emph{qi} function with zelig, the
developer must follow the naming convention qi.\emph{your mode name}, where
\emph{your\_model\_name} is the name of the developer's module.  For example, if a
developer titled his or her zelig module ``logit'', then the corresponding \emph{qi}
function is titled ``\emph{qi.logit}''.

%
\subsection{The Function Body}
The function body of \emph{qi} function varies largely from model to model.  As a
result, it is impossible to create general guidelines to simulate \emph{quantities of
interest} - or even determine what the \emph{quantity of interest} is.  Typical methods
for computing \emph{quantities of interest} include:

\begin{itemize}

	\item Implementing sampling algorithms based on the underlying fitted model, or

	\item ``Predicting'' a large number of values from the fitted model

\end{itemize}


% return values
\subsection{The Return Value}
In order for Zelig to process the simulations, they must be returned in one of several formats:

\begin{itemize}
	% First Example
	\item{\begin{verbatim}
		list(
		     "TITLE OF QI 1" = val1,
		     "TITLE OF QI 2" = val2,
		     # any number of title-val pairs
		     # ...
		     "TITLE OF QI N" = val.n
		     )
	\end{verbatim}}

	% Second Example
	\item{\begin{verbatim}
		make.qi(
		        titles = list(title1, title2),
		        stats  = list(val1, val2)
		        )
	\end{verbatim}}
\end{itemize}


In the above example,\emph{val1, val2}are data.frames, matrices, or lists representing
the simulations of the \emph{quantities of interests}, and \emph{title1, title2} - and
any number of titles - are character-strings that will act as human-readable descriptions
of the \emph{quantities of interest}.  Once results are returned in this format, Zelig
will convert the results into a machine-readable format and summarize the simulations
into a comprehensible format.

NOTE: Because of its readability, it is suggested that the first method is used when
returning \emph{quantities of interest}.

% break
\pagebreak


% find better way to output this data
\section{Simple Example \code{qi} function (\code{qi.logit.R})}


\begin{verbatim}
#' simulate quantities of interest for the logit models
qi.logit <- function(z, x=NULL, x1=NULL, y=NULL, num=1000, param=NULL) {

  # compute expected values using the function ".compute.ev"
  ev1 <- .compute.ev(obj, x, num, param)
  ev2 <- .compute.ev(obj, x1, num, param)

  # return simulations of quantities of interest
  list(
       "Expected Values: E(Y|X)"  = ev1,
       "Expected Values (for X1): E(Y|X1)" = ev2,
       "First Differences: E(Y|X1) - E(Y|X)" = ev2 - ev1
       )
}
\end{verbatim}


% style guide
\chapter[Zelig Style Guide]{Zelig Style Guide}
\label{chapter:styleguide}
\documentclass[a4paper,10pt]{article}

\begin{document}

% title
%
\title{Zelig Naming Conventions}
\author{Matt Owen}
\maketitle


\section{Mandatory Naming Conventions}
When developing a Zelig package, it is important to write code in a style that is consistent with existing code and clearly understandable.  The following specifies guidelines that are suggested for the any Zelig package, and strictly enforced for commits to Zelig's main trunk.


\subsection{Function and Method Naming}
The Zelig developer writes Functions and Class Methods as verbs in camel-case verbs with leading capital letters.  That is, a function name must:

% list
{\tt\begin{enumerate}
	\item{be a verb}
	\item{have a leading capital letter}
	\item{contain only alphanumeric characters}
	\item{use a capital letter to denote the beginning of a new word within the function name}
	\item{treat all abbreviations as being lowercase}
\end{enumerate}}

For example, if the developer is writing a function that converts results to HTML, then the function should be named: {\tt ResultsToHtml}.



\subsection{Generic Function Naming}
Generic functions, by Zelig's convention, are preferably single-word lowercase verbs.  If it is impossible to phrase the generic function as a single-word, then it should be named as a verb in camel case with a leading lowercase letter.  It is highly recommended that careful consideration be taken when naming generic functions.  That is, a generic function name must:

{\tt\begin{enumerate}
	\item{be a verb}
	\item{have a leading lowercase letter}
	\item{contain only alphanumeric characters}
	\item{use a capital letter to denote the presence of a space between words}
	\item{treat all abbreviations as being lowercase}
\end{enumerate}}


For example, if the developer is writing a function that stores class information in a log-file for a variety of datatypes, then the function be named: {\tt store} or {\tt writeToLog}.



\subsection{Class Naming}
Class names are preferably single-word, alphanumeric, and camel-case nouns.  This convention may be loosely followed.  The name of the constructor function and the class name must be identical.  That is, a class name must:
{\tt\begin{enumerate}
	\item{be a noun}
	\item{contain only alphanumeric characters}
	\item{match their function/constructors name}
\end{enumerate}}

For example, if the developer writes a class that represents a polygon, then it should be named: {\tt polygon} or {\tt Polygon}.  The constructor function should be named to match this.  Furthermore, classes should be kept in their own R file which is similarly named.



\subsection{Variables}
Variables follow very different conventions that functions, methods, classes, and generics.  Few rules govern their naming, except that they are must be all lower-case and descriptive.  That is, a variable name must:
{\tt\begin{itemize}
	\item{contain only lowercase letters and dots}
	\item{be descriptive}
	\item{be longer than three letters long, unless it is an iterator}
\end{itemize}}

For example, 



\subsection{Private Functions and Variables}
Functions and variables that are intended to be hidden from the user follow all the rules of their visible counterparts, except they must begin with a leading dot.  This is so that the export command will ignore them.



\subsection{Operator Overloading}
While in many circumstances operator overloading is a useful tool, it is highly discouraged in Zelig packages.  Please store all overloaded operators in a file named {\tt ZELIG\_zzz\_\emph{model\_class}\_operators.R}, where \emph{model\_class} is the model's class.  Note that, because operators do not begin with an alphabet character, they will be ignored by the standard construction of a {\tt NAMESPACE} file.  As a result, operators must be explicitly exported in the {\tt NAMESPACE} file.



\section{Suggested Conventions}


\subsection{S3 Object Orientation}
While S4 objects offer a stricter style of object-oriented programming, it is still standard to write functions as S3 objects due to better support and consistency with core packages (all of which as S3 objects).  Zelig can interface with S4 objects, however the results at times are unpredictable, and might only be suitable for truly savvy R developers.



\subsection{Package Naming}
It is suggested that packages depending on the Zelig library have ".zelig" appended to them.  This will make clear that the syntax and structure of the package is compatible with that of other Zelig objects.


\section{Synopsis}

\begin{description}
	\item[function]{verb, camel-case, leading capital letter}
	\item[method]{\emph{same as function}}
	\item[generic function]{verb, alphanumeric, lowercase leading letter, preferably single word}
	\item[variable]{noun with lowercase letters, numbers, and dots only}
	\item[class object]{alphanumeric characters only.  name must match constructor name and filename}
	\item[hidden variable or function]{typical naming conventions but with leading dot}
	\item[overloaded operators]{stored in {\tt ZELIG\_zzz\_\emph{model\_name}\_operators.R}}
\end{description}


\end{document}

% FAQ
\chapter[FAQ]{Frequently Asked Questions}
\label{chapter:faq}
\documentclass{article}

\title{Zelig Cookbook}
\author{Matt Owen}

%
\usepackage{hyperref}
\usepackage{xifthen}
\usepackage{ifthen}

% Newly defined commands
\newcommand{\code}[2]{

  :::#2;;;
}



% Problem/Cause/Solution Commands
\newcommand{\problem}[1] {

  % Problem: #1

}


\begin{document}

\maketitle

\code{dead}




\section{Introduction}

\section{FAQ's}
\section{Problem-Cause-Solution}



\subsection{I cannot install a specific Zelig package on my system}
\label{subsec:install-specific-package}

\subsubsection{Problem}

\problem{A particular package cannot be installed on my system, using the}
\code{install.packages} function.

\subsubsection{Cause}

\subsubsection{Solution}


% % Section
% \section{How do I install an Zelig package \code{<X>}?}
% 
% Several methods exist for installing an individual zelig package.
% 
% \begin{enumerate}
%   \item If your package is officially supported by the Zelig team, you can
%     selectively install packages from an the install wizard provided by
%     Harvard. Simply copy-and-paste the following in an active R terminal, and
%     \code{source("\url{http://r.iq.harvard.edu/install\_live.R}")}
% \end{enumerate}
% 
% 
% \subsection{Issue}
% 
% \begin{itemize}
%   \item I know the name of the package that I want to install
%   \item I get an error when I install via \verb+install.packages+ 
% \end{itemize}
% 
% 
% \subsection{Resolution}
% 
% Typically the command:
% \begin{verbatim}
% install.packages("<X>", repos="http://r.iq.harvard.edu/", type="source")
% \end{verbatim}
% 
% {\noindent}will correctly install the Zelig package perfectly. However, this
% can produce an error if:
% 
% \begin{itemize}
%   \item The package \code{<X>} does not exist (or the package name is a typo).
%   \item The package \code{<X>} depends on a package \emph{not} installed on
%     the user's system.
% \end{itemize}

The solution


Problem: Using the \code{install.packages} function, I cannot install a specific Zelig package on my system.


Causes:

\begin{enumerate}
  \item The package name contains a type
  \item The package is not available on the Zelig repository
  \item One or more dependency of the package is not installed on the user's system
\end{enumerate}

Solutions:





\end{document}


\end{document}
