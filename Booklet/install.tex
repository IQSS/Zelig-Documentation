\tableofcontents
\newcommand{\InstallInteractive}[0]{
  \href{http://r.iq.harvard.edu/install_live.R}{http://r.iq.harvard.edu/install_live.R}
}

% COMMANDS: Useful for ensuring the links don't break between versions
\newcommand{\CranMirror}[0]{
  \href{http://cran.opensourceresources.org/}{http://cran.opensourceresources.org/}
}

% The entire R install guide
\newcommand{\InstallInstructionsHref}[0]{
  \href{http://cran.r-project.org/doc/manuals/R-admin.html}{http://cran.r-project.org/doc/manuals/R-admin.html}
}

% Individual sections in the install guide

% #Mac
\newcommand{\MacInstallHref}[0]{
\href{http://cran.r-project.org/doc/manuals/R-admin.html\#Installing-R-under-_0028Mac_0029-OS-X}{Mac}
}

% #Windows
\newcommand{\WindowsInstallHref}[0]{
\href{http://cran.r-project.org/doc/manuals/R-admin.html\#Installing-R-under-Windows}{Windows}
}

% #Unix
\newcommand{\UnixInstallHref}[0]{
\href{http://cran.r-project.org/doc/manuals/R-admin.html\#Installing-R-under-Unix_002dalikes}{Unix-alike}
}


% Introduction Installation
\section{Introduction}

Zelig is a Statistical software suite written in the R programming language.
As a result, installing any Zelig add-on is a matter of three steps:

\begin{enumerate}
  \item Download and install R (section \ref{sec:install-r}),
  \item Install Zelig (section \ref{sec:install-zelig}), and
  \item Install optional Zelig add-ons (section \ref{sec:install-add-ons})
\end{enumerate}

The following guide is intended to quickly and easily explain each of these steps.


% Requirements for Installation: list the two requirements
%
%
\section{Requirements for Installing R}

The Zelig software suite has only two requirements: 

\begin{enumerate}

  \item {\bf R version 2.15+}, which can be downloaded at \href{http://www.r-project.org/}{\tt http://r-project.org/}

  \item A major operating system, either:

    \begin{itemize}
      \item Mac OS X 10.4+,
      \item Windows or
      \item Linux
    \end{itemize}

\end{enumerate}

Installation instructions for R can be found on the R-project website. Simply
visit the download page, and select any mirror link, though this one is
recommended:

\CranMirror



% Installing R
\section{Installing R}
\label{sec:install-r}

Installing R is typically straightforward, regardless of which operating
system is being used. Several useful documents exist on CRAN
(The Comprehensive R Archive Network) for explanation and troubleshooting of R
installation. These documents can be found on any CRAN mirror. Specifically,
the complete guide to install R can be found in section
\ref{sec:install-wizard}.

\InstallInstructionsHref

This document contains specific documents for installing R on \MacInstallHref,
\WindowsInstallHref, and \UnixInstallHref systems.



% Easy Installation Instructions: explain how to install Zelig via install.R
\section{Installing Zelig}
\label{sec:install-zelig}

Once R version 2.15 or greater has been installed on the client's machine,
setting up Zelig is simple, and can be completed in one of two ways. The first
method is via an installation script (section \ref{subsec:install-wizard}) which
guides the user through the Zelig installation, and subsequently lets the user
select exactly which add-on packages he or she would like to install. The second
method is via R's built-in package-installer (section
\ref{subsec:install-manual}). If, at a later time, the user wishes to install
additional Zelig add-ons, they can be manually installed separately from the
core package (see section \ref{sec:install-add-ons}.

\subsection{Install Wizard: Automating a Zelig Installation}
\label{subsec:install-wizard}

Once the R software is installed, the entire installation procedure of Zelig and
its add-on packages can be handled automatically via an installation script.

To install Zelig interactively, as well as its add-on packages, simply:

\begin{enumerate}

  \item {\bf Launch R}. This program can be found wherever the computer stores
    its applications (e.g. ``Program Files'' on a Windows machine or
    ``Applications'' on a MacOS X Computer)

  \item {\bf Run the Setup Wizard}. At the R command-prompt, type:
    \begin{verbatim}
    source("http://r.iq.harvard.edu/install_live.R")
    \end{verbatim}

  \item {\bf View the Option Menu}.
    A text-menu with options - ``Install Everything'' and ``Custom Install'' -
    will be displayed. 

  \item {\bf Choose Whether to Install Everything or Perform a Custom Install}.
    If you would like to install every package, simply choose
    ``Install Everything'' (Option 1). Otherwise, ``Custom Install'' will
    guide you through the process of package selection.

\end{enumerate}

\subsection{Custom Install: Customizing Installation via the Installation Wizard}
\label{subsec:custominstall}

The option for a ``Custom Install'' of the Zelig software suite allows for
users to select only a particular set of software packages to install.
This is appropriate for the user who wishes to use Zelig to run only specific
sets of models. Running the custom install requires the interaction of the user
to pick-and-choose the specific software packages.

When ``Custom Install'' is selected, the install script will display the
complete list of Zelig packages along with numbers corresponding to each
package. Users will choose which package to download by entering the number this
associated number.

After each selection, the install script will display a listing of each package
that will be installed. Additionally, selections that have already been made
will have an asterisk placed to their left within the menu. To remove a package,
simply re-enter the same number that was used to select it (this number is
always to the left of the package's name in the menu).

When all the selections have been made, enter ``{\tt 0}'' to begin installing
the selected packages.

\subsection{Manual Install: Installing Zelig via {\tt install.packages}}
\label{subsec:install-manual}

The following installation procedure will install Zelig without any add-on
packages. That is, Zelig will only download files necessarily for developing
\emph{new} Zelig packages and basic generalized linear model regressions -
logit, gamma, gaussian, etc.

To install this "core" package, simply type the following from the R command
prompt:

\begin{verbatim}
install.packages(
                 "Zelig",
                 repos = "http://r.iq.harvard.edu",
                 type  = "source"
                 )
\end{verbatim}

{\noindent}{\bf NOTE: }This method requires that Zelig, and all its dependent
packages be installed manually.


\section{List of Available Packages}
\label{sec:availablepackages}

These add-on packages include:

\begin{itemize}
  \item {\tt ZeligOrdinal}: Ordinal Models for Logit and Probit Regressions

  \item {\tt ZeligMultivariate}: Bivariate Models for Logit and Probit Regressions

  \item {\tt ZeligMultinomial}: Multinomial Models Logit and Probit Regressions

  \item {\tt ZeligGAM}: Generalized Additive Models for Logit, Gaussian, Poisson and Probit Regressions

  \item {\tt ZeligGEE}: Generalized Estimating Equation Models for Gamma, Logit, Gaussian, Poisson and Probit Regressions

  \item {\tt ZeligMixed}: Mixed Effect Models (Multi-level) for Gamma, Logit, Least-squares, Poisson and Probit Regressions

  \item {\tt ZeligSurvey}: Survey-weighted Models for Gamma, Logit, Normal, Poisson and Probit Regressions

  \item {\tt ZeligNetwork}: Social Network Regression Models for Gamma, Logit, Normal, Poisson and Probit Regressions

  \item {\tt ZeligBayesian}: Bayesian Regressions for Logit, Multionomial Logot, Normal, Poisson and Probit Models

  \item {\tt ZeligLeastSquares}: Least Squares Regressions for Seeminly Unrelated, Two-Stage, and Three-Stage Methods

  \item {\tt ZeligCommon}: Common Statistical Models

  \item {\tt ZeligMisc}: Miscellaneous and Uncategorized Models

\end{itemize}

\section{Installing Add-on packages}
\label{sec:install-add-ons}

\subsection{Using {\tt source("http://r.iq.harvard.edu/install\_live.R")}}

In most situations, it is simpler and easier to re-run the install script,
rather than use the common {\tt install.packages} function that comes packaged
with the R software. If you wish to employ this method, simply follow along
with the instructions found on page \pageref{subsec:custominstall}. That is, by
entering:

{\indent}\verb|source("http://r.iq.harvard.edu/install.zelig.R")}|

\subsection{Using {\tt install.packages}}

\begin{minipage}{\linewidth}

To download and install any of these packages individually, simply type the
following from an R command prompt:

\begin{verbatim}
install.packages(
                 "PACKAGE_NAME",
                 repos = "http://r.iq.harvard.edu/",
                 type  = "source"
                 )
\end{verbatim}

Where {\tt "PACKAGE\_NAME"} is replaced with the title of the Add-on packages
in the above itemized list. For example, to download "Generalized Estimating
Equation Models...", note that its package name is {\tt ZeligGEE}, and type
from the R command prompt:

\begin{verbatim}
install.packages(
                 "ZeligGAM",
                 repos = "http://r.iq.harvard.edu/",
                 type  = "source"
                 )
\end{verbatim}

The complete list of statistical packages for Zelig can be found on page
\pageref{sec:availablepackages}.


\end{minipage}

% Archived Installation: How to install Zelig via tarballs, etc.
%%\section{Install from Archived Zelig File}
%
%In addition to the interactive installation procedures - sections 3 and 4 - Zelig can be %installed 



\section{Post-Installation}

Barring any installation errors, Zelig and any add-on packages that were manually installed, should now be available from an R-session. Simply type from an R command prompt:

\begin{verbatim}
library(Zelig)
?Zelig
\end{verbatim}

To begin interacting and using the Zelig software package. Additionally, demo files can be listed via the command:

\begin{verbatim}
demo(package="Zelig")
\end{verbatim}
