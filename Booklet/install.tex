
% COMMANDS: useful commands for formatting this particular document
%
\newcommand{\CranMirror}[0]{
  \href{http://cran.opensourceresources.org/}{http://cran.opensourceresources.org/}
}

\newcommand{\InstallInstructionsHref}[0]{
  \href{http://cran.r-project.org/doc/manuals/R-admin.html}{http://cran.r-project.org/doc/manuals/R-admin.html}
}

\newcommand{\MacInstallHref}[0]{
\href{http://cran.r-project.org/doc/manuals/R-admin.html\#Installing-R-under-_0028Mac_0029-OS-X}{Mac}
}
\newcommand{\WindowsInstallHref}[0]{
\href{http://cran.r-project.org/doc/manuals/R-admin.html\#Installing-R-under-Windows}{Windows}
}
\newcommand{\UnixInstallHref}[0]{
\href{http://cran.r-project.org/doc/manuals/R-admin.html\#Installing-R-under-Unix_002dalikes}{Unix-alike}
}


% Introduction Installation
\section{Introduction}

Zelig's installation procedure is straightforward, though the package itself is
not standalone, and requires the installation of R version 2.13 (or greater).
That is, because Zelig is written in the R statistical programming language, it
cannot be installed without support from the R programming language. As a
result of this, installing Zelig and Zelig-compliant packages can be divided
into three tasks:

\begin{enumerate}
	\item Download and Install R,
	\item Install Zelig, and
	\item Install Optional Zelig Add-ons
\end{enumerate}

The following guide is intended to quickly and easily explain each of these steps.


% Requirements for Installation: list the two requirements
%
%
\section{Requirements for Installing R}

The Zelig software suite has only two requirements: 

\begin{enumerate}
	\item {\bf R version 2.13+}, which can be downloaded at \href{http://www.r-project.org/}{\tt http://r-project.org/}
	\item A major operating system, either:
		\begin{itemize}
			\item Mac OS X 10.4+,
			\item Windows or
			\item Linux
		\end{itemize}
\end{enumerate}

Installation instructions for R can be found on the R-project website. Simply visit the download page, and select any mirror link, though this one is recommended:

\CranMirror



% Installing R
%
%
\section{Installing R}

Installing R is typically straightforward, regardless of which operating system is being used. Several useful documents exist on CRAN (The Comprehensive R Archive Network) for explanation and troubleshooting of R installation. These documents can be found on any CRAN mirror. Specifically, the complete guide to install R can be found at:

\InstallInstructionsHref

This document contains specific documents for installing R on \MacInstallHref, \WindowsInstallHref, and \UnixInstallHref systems.

% Easy Installation Instructions: explain how to install Zelig via install.R
%
%
\section{Simple Zelig Installation}

Once R version 2.13 or greater has been installed on the client's machine, setting up Zelig is a breeze. R has built-in facilities for managing the installation of statistical packages. This provides a simple mechanism for installing Zelig, regardless of the operating system that is being used.

To install Zelig, as well as its add-on packages, simply:

\begin{enumerate}
	\item {\bf Install R version 2.13 or greater}. Download R from the R project's website, which can be found at \CranMirror
  \item {\bf Launch R}. Once R is installed, this program can be found wherever the computer stores applications (e.g. ``Program Files" on a Windows machine)
  \item At the R command-prompt, type:
    \begin{verbatim}
    source("http://r.iq.harvard.edu/install_live.R")
    \end{verbatim}
    This launches an interactive install script that behaves much like a  ``setup-wizard'', which directs R to download all the appropriate statistical packages associated with the Zelig software suite. 

    Users will be given a choice of installing every model in the Zelig software suit or selecting several particular packages geared for specific types of analysis. The packages will be displayed in an ordered list, and to choose
    a particular package to install requires that the user enter the number corresponding to the listed software package.
\end{enumerate}






% Advanced Installation
%
%
\section{Advanced Installation}

For users familiar with R and Zelig, it may be useful to selectively install packages. In order to do this, users simply need to use the {\tt install.packages} function built into R's functionality.

\subsection{Install Zelig without Additional Packages}
This installation procedure will install Zelig without any add-on packages. That is, Zelig will only download files necessarily for developing \emph{new} Zelig packages and basic generalized linear model regressions - logit, gamma, gaussian, etc.

To install this "core" package, simply type the following from the R command prompt:
\begin{verbatim}
install.packages(
                 "Zelig",
                 repos = "http://r.iq.harvard.edu",
                 type  = "source"
                 )
\end{verbatim}

\subsection{Install Add-on Packages}

In addition to Zelig's core package, which exclusively contains simple regression models and a Developers' API for producing novel R packages, a myriad of Zelig add-on packages are available. These packages supplement Zelig's features, and add specialized, advanced models to Zelig.

\section{List of Available Packages}

These add-on packages include:

\begin{itemize}

	\item {\tt ZeligOrdinal}: Ordinal Models for Logit and Probit Regressions

	\item {\tt ZeligMultivariate}: Bivariate Models for Logit and Probit Regressions

	\item {\tt ZeligMultinomial}: Multinomial Models Logit and Probit Regressions
	
	\item {\tt ZeligGAM}: Generalized Additive Models for Logit, Gaussian, Poisson and Probit Regressions

	\item {\tt ZeligGEE}: Generalized Estimating Equation Models for Gamma, Logit, Gaussian, Poisson and Probit Regressions
	
	\item {\tt ZeligMixed}: Mixed Effect Models (Multi-level) for Gamma, Logit, Least-squares, Poisson and Probit Regressions
	
	\item {\tt ZeligSurvey}: Survey-weighted Models for Gamma, Logit, Normal, Poisson and Probit Regressions

	\item {\tt ZeligNetwork}: Social Network Regression Models for Gamma, Logit, Normal, Poisson and Probit Regressions

	\item {\tt ZeligBayesian}: Bayesian Regressions for Logit, Multionomial Logot, Normal, Poisson and Probit Models

	\item {\tt ZeligLeastSquares}: Least Squares Regressions for Seeminly Unrelated, Two-Stage, and Three-Stage Methods

	\item {\tt ZeligCommon}: Common Statistical Models

	\item {\tt ZeligMisc}: Miscellaneous and Uncategorized Models

\end{itemize}


\subsection{Using {\tt source("http://r.iq.harvard.edu/install\_live.R")}}







\subsection{Using {\tt install.packages}}
\begin{minipage}{\linewidth}
To download any of these packages independently, simply type the following from an R command prompt:
\begin{verbatim}
install.packages(
                 "MODEL NAME",
                 repos = "http://r.iq.harvard.edu/",
                 type  = "source"
                 )
\end{verbatim}

Where {\tt "MODEL NAME"} is replaced with the title of the Add-on packages in the above itemized list. For example, to download "Generalized Estimating Equation Models...", note that its package name is {\tt ZeligGEE}, and type from the R command prompt:
\begin{verbatim}
install.packages(
                 "ZeligGAM",
                 repos = "http://r.iq.harvard.edu/",
                 type  = "source"
                 )
\end{verbatim}
\end{minipage}

% Archived Installation: How to install Zelig via tarballs, etc.
%%\section{Install from Archived Zelig File}
%
%In addition to the interactive installation procedures - sections 3 and 4 - Zelig can be %installed 



\section{Post-Installation}

Barring any installation errors, Zelig and any add-on packages that were manually installed, should now be available from an R-session. Simply type from an R command prompt:

\begin{verbatim}
library(Zelig)
?Zelig
\end{verbatim}

To begin interacting and using the Zelig software package. Additionally, demo files can be listed via the command:

\begin{verbatim}
demo(package="Zelig")
\end{verbatim}








